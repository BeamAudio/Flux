\documentclass[twoside]{book}

% Packages required by doxygen
\usepackage{doxygen}
\usepackage{graphicx}
\usepackage[utf8]{inputenc}
\usepackage{makeidx}
\usepackage{multicol}
\usepackage{multirow}
\usepackage{textcomp}
\usepackage[table]{xcolor}
\usepackage{ifpdf}
\usepackage{geometry}
\geometry{a4paper,top=2.5cm,bottom=2.5cm,left=2.5cm,right=2.5cm}

% Font selection
\usepackage[T1]{fontenc}
\usepackage[scaled=.90]{helvet}
\usepackage{courier}
\renewcommand{\familydefault}{\sfdefault}

% Hyperlinks
\usepackage[pdftex, pagebackref=true]{hyperref}
\hypersetup{
    colorlinks=true,
    linkcolor=blue,
    citecolor=blue,
    unicode
}

% Headers & footers
\usepackage{fancyhdr}
\pagestyle{fancyplain}
\fancyhead[LE, RO]{\bfseries\thepage}
\fancyhead[LO]{\bfseries\rightmark}
\fancyhead[RE]{\bfseries\leftmark}
\fancyfoot[LO, RE]{\bfseries\scriptsize Generated by Doxygen}

\makeindex

\begin{document}

% --- TITLE PAGE ---
\begin{titlepage}
\vspace*{7cm}
\begin{center}
{\Huge \bfseries Beam Audio Flux}\
\vspace{1cm}
{\Large Technical Report & API Reference}\
\vspace{0.5cm}
{\large Version 0.1.0}\
\vspace{2cm}
{\large \today}\
\vspace{4cm}
{\large Generated by Doxygen & Gemini CLI}
\end{center}
\end{titlepage}

\clearemptydoublepage
\tableofcontents
\clearemptydoublepage

% --- CHAPTER 1: INTRODUCTION ---
\chapter{Introduction}
Beam Audio Flux is a modern Digital Audio Workstation (DAW) prototype focused on seamless integration between creative audio flux and precise splicing. This report documents the technical architecture, implemented features, and the roadmap for the system's evolution into a production-grade audio environment.

The project is built using C++20 and aims to provide a high-performance, low-latency audio engine with a custom, lightweight UI framework.

% --- CHAPTER 2: ARCHITECTURE ---
\chapter{System Architecture}

\section{Overview}
Beam Audio Flux is a hybrid DAW designed for real-time audio manipulation. It uses a modular architecture where the DSP (Digital Signal Processing) logic is decoupled from the UI through a graph-based abstraction layer.

\section{Audio Engine & DSP Abstraction}
The core audio processing is managed by the \texttt{FluxGraph}.

\subsection{The FluxNode}
All audio processing entities inherit from \texttt{FluxNode}. A node defines:
\begin{itemize}
    \item \textbf{Input Ports}: Buffers where incoming audio is summed.
    \item \textbf{Output Ports}: Buffers where processed audio is stored.
    \item \textbf{Process Method}: The core DSP loop.
\end{itemize}

\subsection{FluxGraph}
The \texttt{FluxGraph} manages the lifecycle and connectivity of nodes.
\begin{itemize}
    \item \textbf{Topological Sorting}: On every connection change, the graph rebuilds a "schedule" using Kahn's algorithm to ensure nodes are processed in the correct order.
    \item \textbf{Propagation}: The graph handles the transfer of data from a source node's output buffer to the destination node's input buffer.
\end{itemize}

\subsection{AudioEngine}
The system uses \textbf{SDL3} for low-latency audio I/O. It holds a reference to the \texttt{FluxGraph} and triggers the \texttt{process()} chain inside the SDL audio callback.

\section{Rendering Engine (QuadBatcher)}
To achieve high performance for complex UI and waveforms, the project uses a custom OpenGL batch renderer.
\begin{itemize}
    \item \textbf{Batching}: Instead of one draw call per rectangle, \texttt{QuadBatcher} accumulates vertices into a large buffer and issues a single \texttt{glDrawElements} call.
    \item \textbf{Shaders}: A simple GLSL vertex/fragment shader pair handles orthographic projection and vertex colors.
    \item \textbf{Coordinate System}: Uses pixel-perfect screen space coordinates (Top-Left is 0,0).
\end{itemize}

\section{UI Component System}
\subsection{Component Base Class}
Every UI element inherits from \texttt{Component}. It provides:
\begin{itemize}
    \item \textbf{Bounds}: A \texttt{Rect} defining the position and size.
    \item \textbf{Event Callbacks}: \texttt{onMouseDown}, \texttt{onMouseUp}, \texttt{onMouseMove}.
\end{itemize}

\subsection{InputHandler}
The \texttt{InputHandler} translates SDL events into component-specific events, handling focus and Z-order hit testing.

\section{Project Management}
\texttt{FluxProject} is the central state container. It holds the \texttt{FluxGraph} and is responsible for serialization/deserialization via \texttt{nlohmann::json}.

% --- CHAPTER 3: IMPLEMENTATION (DOXYGEN) ---
\chapter{API Reference}
The following sections detail the Classes and Files implemented in the project.

\section{Directory Hierarchy}
\doxysection{Directories}
\begin{DoxyCompactList}
\item \contentsline{section}{core }{\pageref{dir_aebb8dcc11953d78e620bbef0b9e2183}}{}
\begin{DoxyCompactList}
\item \contentsline{section}{beam\+\_\+host.\+cpp }{\pageref{beam__host_8cpp}}{}
\item \contentsline{section}{beam\+\_\+host.\+hpp }{\pageref{beam__host_8hpp}}{}
\item \contentsline{section}{flux\+\_\+project.\+hpp }{\pageref{flux__project_8hpp}}{}
\item \contentsline{section}{parameter.\+hpp }{\pageref{parameter_8hpp}}{}
\item \contentsline{section}{project\+\_\+manager.\+hpp }{\pageref{project__manager_8hpp}}{}
\end{DoxyCompactList}
\item \contentsline{section}{dsp }{\pageref{dir_01d2cdbd82b1e18dc073ab3ebd5ada7e}}{}
\begin{DoxyCompactList}
\item \contentsline{section}{audio\+\_\+engine.\+cpp }{\pageref{audio__engine_8cpp}}{}
\item \contentsline{section}{audio\+\_\+engine.\+hpp }{\pageref{audio__engine_8hpp}}{}
\item \contentsline{section}{audio\+\_\+node.\+hpp }{\pageref{audio__node_8hpp}}{}
\item \contentsline{section}{biquad\+\_\+filter\+\_\+node.\+hpp }{\pageref{biquad__filter__node_8hpp}}{}
\item \contentsline{section}{custom\+\_\+filter.\+hpp }{\pageref{custom__filter_8hpp}}{}
\item \contentsline{section}{delay\+\_\+node.\+hpp }{\pageref{delay__node_8hpp}}{}
\item \contentsline{section}{disk\+\_\+streamer.\+cpp }{\pageref{disk__streamer_8cpp}}{}
\item \contentsline{section}{disk\+\_\+streamer.\+hpp }{\pageref{disk__streamer_8hpp}}{}
\item \contentsline{section}{flux\+\_\+fx\+\_\+nodes.\+hpp }{\pageref{flux__fx__nodes_8hpp}}{}
\item \contentsline{section}{flux\+\_\+graph.\+hpp }{\pageref{flux__graph_8hpp}}{}
\item \contentsline{section}{flux\+\_\+node.\+hpp }{\pageref{flux__node_8hpp}}{}
\item \contentsline{section}{flux\+\_\+plugin.\+hpp }{\pageref{flux__plugin_8hpp}}{}
\item \contentsline{section}{flux\+\_\+track\+\_\+node.\+hpp }{\pageref{flux__track__node_8hpp}}{}
\item \contentsline{section}{gain\+\_\+node.\+hpp }{\pageref{gain__node_8hpp}}{}
\item \contentsline{section}{master\+\_\+node.\+hpp }{\pageref{master__node_8hpp}}{}
\item \contentsline{section}{sine\+\_\+oscillator.\+hpp }{\pageref{sine__oscillator_8hpp}}{}
\item \contentsline{section}{track\+\_\+node.\+hpp }{\pageref{track__node_8hpp}}{}
\item \contentsline{section}{wav\+\_\+reader.\+hpp }{\pageref{wav__reader_8hpp}}{}
\item \contentsline{section}{wav\+\_\+writer.\+hpp }{\pageref{wav__writer_8hpp}}{}
\end{DoxyCompactList}
\item \contentsline{section}{graphics }{\pageref{dir_560415a5d2bc4999842279f4fc1debef}}{}
\begin{DoxyCompactList}
\item \contentsline{section}{quad\+\_\+batcher.\+cpp }{\pageref{quad__batcher_8cpp}}{}
\item \contentsline{section}{quad\+\_\+batcher.\+hpp }{\pageref{quad__batcher_8hpp}}{}
\item \contentsline{section}{shader.\+cpp }{\pageref{shader_8cpp}}{}
\item \contentsline{section}{shader.\+hpp }{\pageref{shader_8hpp}}{}
\item \contentsline{section}{ui\+\_\+shaders.\+hpp }{\pageref{ui__shaders_8hpp}}{}
\end{DoxyCompactList}
\item \contentsline{section}{src }{\pageref{dir_68267d1309a1af8e8297ef4c3efbcdba}}{}
\begin{DoxyCompactList}
\item \contentsline{section}{core }{\pageref{dir_aebb8dcc11953d78e620bbef0b9e2183}}{}
\begin{DoxyCompactList}
\item \contentsline{section}{beam\+\_\+host.\+cpp }{\pageref{beam__host_8cpp}}{}
\item \contentsline{section}{beam\+\_\+host.\+hpp }{\pageref{beam__host_8hpp}}{}
\item \contentsline{section}{flux\+\_\+project.\+hpp }{\pageref{flux__project_8hpp}}{}
\item \contentsline{section}{parameter.\+hpp }{\pageref{parameter_8hpp}}{}
\item \contentsline{section}{project\+\_\+manager.\+hpp }{\pageref{project__manager_8hpp}}{}
\end{DoxyCompactList}
\item \contentsline{section}{dsp }{\pageref{dir_01d2cdbd82b1e18dc073ab3ebd5ada7e}}{}
\begin{DoxyCompactList}
\item \contentsline{section}{audio\+\_\+engine.\+cpp }{\pageref{audio__engine_8cpp}}{}
\item \contentsline{section}{audio\+\_\+engine.\+hpp }{\pageref{audio__engine_8hpp}}{}
\item \contentsline{section}{audio\+\_\+node.\+hpp }{\pageref{audio__node_8hpp}}{}
\item \contentsline{section}{biquad\+\_\+filter\+\_\+node.\+hpp }{\pageref{biquad__filter__node_8hpp}}{}
\item \contentsline{section}{custom\+\_\+filter.\+hpp }{\pageref{custom__filter_8hpp}}{}
\item \contentsline{section}{delay\+\_\+node.\+hpp }{\pageref{delay__node_8hpp}}{}
\item \contentsline{section}{disk\+\_\+streamer.\+cpp }{\pageref{disk__streamer_8cpp}}{}
\item \contentsline{section}{disk\+\_\+streamer.\+hpp }{\pageref{disk__streamer_8hpp}}{}
\item \contentsline{section}{flux\+\_\+fx\+\_\+nodes.\+hpp }{\pageref{flux__fx__nodes_8hpp}}{}
\item \contentsline{section}{flux\+\_\+graph.\+hpp }{\pageref{flux__graph_8hpp}}{}
\item \contentsline{section}{flux\+\_\+node.\+hpp }{\pageref{flux__node_8hpp}}{}
\item \contentsline{section}{flux\+\_\+plugin.\+hpp }{\pageref{flux__plugin_8hpp}}{}
\item \contentsline{section}{flux\+\_\+track\+\_\+node.\+hpp }{\pageref{flux__track__node_8hpp}}{}
\item \contentsline{section}{gain\+\_\+node.\+hpp }{\pageref{gain__node_8hpp}}{}
\item \contentsline{section}{master\+\_\+node.\+hpp }{\pageref{master__node_8hpp}}{}
\item \contentsline{section}{sine\+\_\+oscillator.\+hpp }{\pageref{sine__oscillator_8hpp}}{}
\item \contentsline{section}{track\+\_\+node.\+hpp }{\pageref{track__node_8hpp}}{}
\item \contentsline{section}{wav\+\_\+reader.\+hpp }{\pageref{wav__reader_8hpp}}{}
\item \contentsline{section}{wav\+\_\+writer.\+hpp }{\pageref{wav__writer_8hpp}}{}
\end{DoxyCompactList}
\item \contentsline{section}{graphics }{\pageref{dir_560415a5d2bc4999842279f4fc1debef}}{}
\begin{DoxyCompactList}
\item \contentsline{section}{quad\+\_\+batcher.\+cpp }{\pageref{quad__batcher_8cpp}}{}
\item \contentsline{section}{quad\+\_\+batcher.\+hpp }{\pageref{quad__batcher_8hpp}}{}
\item \contentsline{section}{shader.\+cpp }{\pageref{shader_8cpp}}{}
\item \contentsline{section}{shader.\+hpp }{\pageref{shader_8hpp}}{}
\item \contentsline{section}{ui\+\_\+shaders.\+hpp }{\pageref{ui__shaders_8hpp}}{}
\end{DoxyCompactList}
\item \contentsline{section}{ui }{\pageref{dir_da5c6b39c0a2f54e57df6799511cd3ab}}{}
\begin{DoxyCompactList}
\item \contentsline{section}{audio\+\_\+module.\+hpp }{\pageref{audio__module_8hpp}}{}
\item \contentsline{section}{cable.\+hpp }{\pageref{cable_8hpp}}{}
\item \contentsline{section}{component.\+hpp }{\pageref{component_8hpp}}{}
\item \contentsline{section}{input\+\_\+handler.\+cpp }{\pageref{input__handler_8cpp}}{}
\item \contentsline{section}{input\+\_\+handler.\+hpp }{\pageref{input__handler_8hpp}}{}
\item \contentsline{section}{knob.\+hpp }{\pageref{knob_8hpp}}{}
\item \contentsline{section}{master\+\_\+strip.\+hpp }{\pageref{master__strip_8hpp}}{}
\item \contentsline{section}{port.\+hpp }{\pageref{port_8hpp}}{}
\item \contentsline{section}{sidebar.\+hpp }{\pageref{sidebar_8hpp}}{}
\item \contentsline{section}{tape\+\_\+reel.\+hpp }{\pageref{tape__reel_8hpp}}{}
\item \contentsline{section}{timeline.\+hpp }{\pageref{timeline_8hpp}}{}
\item \contentsline{section}{top\+\_\+bar.\+hpp }{\pageref{top__bar_8hpp}}{}
\item \contentsline{section}{vu\+\_\+meter.\+hpp }{\pageref{vu__meter_8hpp}}{}
\item \contentsline{section}{workspace.\+hpp }{\pageref{workspace_8hpp}}{}
\end{DoxyCompactList}
\end{DoxyCompactList}
\begin{DoxyCompactList}
\item \contentsline{section}{main.\+cpp }{\pageref{main_8cpp}}{}
\end{DoxyCompactList}
\item \contentsline{section}{ui }{\pageref{dir_da5c6b39c0a2f54e57df6799511cd3ab}}{}
\begin{DoxyCompactList}
\item \contentsline{section}{audio\+\_\+module.\+hpp }{\pageref{audio__module_8hpp}}{}
\item \contentsline{section}{cable.\+hpp }{\pageref{cable_8hpp}}{}
\item \contentsline{section}{component.\+hpp }{\pageref{component_8hpp}}{}
\item \contentsline{section}{input\+\_\+handler.\+cpp }{\pageref{input__handler_8cpp}}{}
\item \contentsline{section}{input\+\_\+handler.\+hpp }{\pageref{input__handler_8hpp}}{}
\item \contentsline{section}{knob.\+hpp }{\pageref{knob_8hpp}}{}
\item \contentsline{section}{master\+\_\+strip.\+hpp }{\pageref{master__strip_8hpp}}{}
\item \contentsline{section}{port.\+hpp }{\pageref{port_8hpp}}{}
\item \contentsline{section}{sidebar.\+hpp }{\pageref{sidebar_8hpp}}{}
\item \contentsline{section}{tape\+\_\+reel.\+hpp }{\pageref{tape__reel_8hpp}}{}
\item \contentsline{section}{timeline.\+hpp }{\pageref{timeline_8hpp}}{}
\item \contentsline{section}{top\+\_\+bar.\+hpp }{\pageref{top__bar_8hpp}}{}
\item \contentsline{section}{vu\+\_\+meter.\+hpp }{\pageref{vu__meter_8hpp}}{}
\item \contentsline{section}{workspace.\+hpp }{\pageref{workspace_8hpp}}{}
\end{DoxyCompactList}
\end{DoxyCompactList}


\section{Namespace Index}
\doxysection{Namespace List}
Here is a list of all namespaces with brief descriptions:\+\begin{DoxyCompactList}
\item\contentsline{section}{\doxymbox{\hyperlink{namespace_beam}{Beam}} }{\pageref{namespace_beam}}{}
\item\contentsline{section}{\doxymbox{\hyperlink{namespace_beam_1_1_s_i_m_d}{Beam::\+\+SIMD}} }{\pageref{namespace_beam_1_1_s_i_m_d}}{}
\end{DoxyCompactList}


\section{Hierarchical Index}
\doxysection{Class Hierarchy}
This inheritance list is sorted roughly, but not completely, alphabetically:\+\begin{DoxyCompactList}
\item \contentsline{section}{Beam::\+\+Audio\+Engine }{\pageref{class_beam_1_1_audio_engine}}{}
\item \contentsline{section}{Beam::\+\+Audio\+Node }{\pageref{class_beam_1_1_audio_node}}{}
\begin{DoxyCompactList}
\item \contentsline{section}{Beam::\+\+Biquad\+Filter\+Node }{\pageref{class_beam_1_1_biquad_filter_node}}{}
\item \contentsline{section}{Beam::\+\+Delay\+Node }{\pageref{class_beam_1_1_delay_node}}{}
\item \contentsline{section}{Beam::\+\+Gain\+Node }{\pageref{class_beam_1_1_gain_node}}{}
\item \contentsline{section}{Beam::\+\+Sine\+Oscillator }{\pageref{class_beam_1_1_sine_oscillator}}{}
\item \contentsline{section}{Beam::\+\+Track\+Node }{\pageref{class_beam_1_1_track_node}}{}
\end{DoxyCompactList}
\item \contentsline{section}{Beam::\+\+Beam\+Host }{\pageref{class_beam_1_1_beam_host}}{}
\item \contentsline{section}{Beam::\+\+Cable }{\pageref{struct_beam_1_1_cable}}{}
\item \contentsline{section}{Beam::\+\+Component }{\pageref{class_beam_1_1_component}}{}
\begin{DoxyCompactList}
\item \contentsline{section}{Beam::\+\+Audio\+Module }{\pageref{class_beam_1_1_audio_module}}{}
\begin{DoxyCompactList}
\item \contentsline{section}{Beam::\+\+Tape\+Reel }{\pageref{class_beam_1_1_tape_reel}}{}
\end{DoxyCompactList}
\item \contentsline{section}{Beam::\+\+Knob }{\pageref{class_beam_1_1_knob}}{}
\item \contentsline{section}{Beam::\+\+Master\+Strip }{\pageref{class_beam_1_1_master_strip}}{}
\item \contentsline{section}{Beam::\+\+Port }{\pageref{class_beam_1_1_port}}{}
\item \contentsline{section}{Beam::\+\+Sidebar }{\pageref{class_beam_1_1_sidebar}}{}
\item \contentsline{section}{Beam::\+\+Timeline }{\pageref{class_beam_1_1_timeline}}{}
\item \contentsline{section}{Beam::\+\+Top\+Bar }{\pageref{class_beam_1_1_top_bar}}{}
\item \contentsline{section}{Beam::\+\+VUMeter }{\pageref{class_beam_1_1_v_u_meter}}{}
\item \contentsline{section}{Beam::\+\+Workspace }{\pageref{class_beam_1_1_workspace}}{}
\end{DoxyCompactList}
\item \contentsline{section}{Beam::\+\+Disk\+Streamer }{\pageref{class_beam_1_1_disk_streamer}}{}
\item \contentsline{section}{Beam::\+\+Flux\+Connection }{\pageref{struct_beam_1_1_flux_connection}}{}
\item \contentsline{section}{Beam::\+\+Flux\+Graph }{\pageref{class_beam_1_1_flux_graph}}{}
\item \contentsline{section}{Beam::\+\+Flux\+Node }{\pageref{class_beam_1_1_flux_node}}{}
\begin{DoxyCompactList}
\item \contentsline{section}{Beam::\+\+Flux\+Delay\+Node }{\pageref{class_beam_1_1_flux_delay_node}}{}
\item \contentsline{section}{Beam::\+\+Flux\+Filter\+Node }{\pageref{class_beam_1_1_flux_filter_node}}{}
\item \contentsline{section}{Beam::\+\+Flux\+Gain\+Node }{\pageref{class_beam_1_1_flux_gain_node}}{}
\item \contentsline{section}{Beam::\+\+Flux\+Plugin }{\pageref{class_beam_1_1_flux_plugin}}{}
\begin{DoxyCompactList}
\item \contentsline{section}{Beam::\+\+Custom\+Filter }{\pageref{class_beam_1_1_custom_filter}}{}
\end{DoxyCompactList}
\item \contentsline{section}{Beam::\+\+Flux\+Track\+Node }{\pageref{class_beam_1_1_flux_track_node}}{}
\item \contentsline{section}{Beam::\+\+Master\+Node }{\pageref{class_beam_1_1_master_node}}{}
\end{DoxyCompactList}
\item \contentsline{section}{Beam::\+\+Flux\+Project }{\pageref{class_beam_1_1_flux_project}}{}
\item \contentsline{section}{Beam::\+\+Input\+Handler }{\pageref{class_beam_1_1_input_handler}}{}
\item \contentsline{section}{Beam::\+\+Parameter }{\pageref{class_beam_1_1_parameter}}{}
\item \contentsline{section}{Beam::\+\+Flux\+Node::\+\+Port }{\pageref{struct_beam_1_1_flux_node_1_1_port}}{}
\item \contentsline{section}{Beam::\+\+Project\+Manager }{\pageref{class_beam_1_1_project_manager}}{}
\item \contentsline{section}{Beam::\+\+Quad\+Batcher }{\pageref{class_beam_1_1_quad_batcher}}{}
\item \contentsline{section}{Beam::\+\+Rect }{\pageref{struct_beam_1_1_rect}}{}
\item \contentsline{section}{Beam::\+\+Shader }{\pageref{class_beam_1_1_shader}}{}
\item \contentsline{section}{Beam::\+\+Vertex }{\pageref{struct_beam_1_1_vertex}}{}
\item \contentsline{section}{Beam::\+\+Wav\+Reader }{\pageref{class_beam_1_1_wav_reader}}{}
\item \contentsline{section}{Beam::\+\+Wav\+Writer }{\pageref{class_beam_1_1_wav_writer}}{}
\end{DoxyCompactList}


\section{Class Index}
\doxysection{Class List}
Here are the classes, structs, unions and interfaces with brief descriptions:\+\begin{DoxyCompactList}
\item\contentsline{section}{\doxymbox{\hyperlink{class_beam_1_1_audio_engine}{Beam::\+\+Audio\+Engine}} }{\pageref{class_beam_1_1_audio_engine}}{}
\item\contentsline{section}{\doxymbox{\hyperlink{class_beam_1_1_audio_module}{Beam::\+\+Audio\+Module}} }{\pageref{class_beam_1_1_audio_module}}{}
\item\contentsline{section}{\doxymbox{\hyperlink{class_beam_1_1_audio_node}{Beam::\+\+Audio\+Node}} }{\pageref{class_beam_1_1_audio_node}}{}
\item\contentsline{section}{\doxymbox{\hyperlink{class_beam_1_1_beam_host}{Beam::\+\+Beam\+Host}} }{\pageref{class_beam_1_1_beam_host}}{}
\item\contentsline{section}{\doxymbox{\hyperlink{class_beam_1_1_biquad_filter_node}{Beam::\+\+Biquad\+Filter\+Node}} }{\pageref{class_beam_1_1_biquad_filter_node}}{}
\item\contentsline{section}{\doxymbox{\hyperlink{struct_beam_1_1_cable}{Beam::\+\+Cable}} }{\pageref{struct_beam_1_1_cable}}{}
\item\contentsline{section}{\doxymbox{\hyperlink{class_beam_1_1_component}{Beam::\+\+Component}} }{\pageref{class_beam_1_1_component}}{}
\item\contentsline{section}{\doxymbox{\hyperlink{class_beam_1_1_custom_filter}{Beam::\+\+Custom\+Filter}} }{\pageref{class_beam_1_1_custom_filter}}{}
\item\contentsline{section}{\doxymbox{\hyperlink{class_beam_1_1_delay_node}{Beam::\+\+Delay\+Node}} }{\pageref{class_beam_1_1_delay_node}}{}
\item\contentsline{section}{\doxymbox{\hyperlink{class_beam_1_1_disk_streamer}{Beam::\+\+Disk\+Streamer}} }{\pageref{class_beam_1_1_disk_streamer}}{}
\item\contentsline{section}{\doxymbox{\hyperlink{struct_beam_1_1_flux_connection}{Beam::\+\+Flux\+Connection}} }{\pageref{struct_beam_1_1_flux_connection}}{}
\item\contentsline{section}{\doxymbox{\hyperlink{class_beam_1_1_flux_delay_node}{Beam::\+\+Flux\+Delay\+Node}} }{\pageref{class_beam_1_1_flux_delay_node}}{}
\item\contentsline{section}{\doxymbox{\hyperlink{class_beam_1_1_flux_filter_node}{Beam::\+\+Flux\+Filter\+Node}} }{\pageref{class_beam_1_1_flux_filter_node}}{}
\item\contentsline{section}{\doxymbox{\hyperlink{class_beam_1_1_flux_gain_node}{Beam::\+\+Flux\+Gain\+Node}} }{\pageref{class_beam_1_1_flux_gain_node}}{}
\item\contentsline{section}{\doxymbox{\hyperlink{class_beam_1_1_flux_graph}{Beam::\+\+Flux\+Graph}} }{\pageref{class_beam_1_1_flux_graph}}{}
\item\contentsline{section}{\doxymbox{\hyperlink{class_beam_1_1_flux_node}{Beam::\+\+Flux\+Node}} }{\pageref{class_beam_1_1_flux_node}}{}
\item\contentsline{section}{\doxymbox{\hyperlink{class_beam_1_1_flux_plugin}{Beam::\+\+Flux\+Plugin}} }{\pageref{class_beam_1_1_flux_plugin}}{}
\item\contentsline{section}{\doxymbox{\hyperlink{class_beam_1_1_flux_project}{Beam::\+\+Flux\+Project}} }{\pageref{class_beam_1_1_flux_project}}{}
\item\contentsline{section}{\doxymbox{\hyperlink{class_beam_1_1_flux_track_node}{Beam::\+\+Flux\+Track\+Node}} }{\pageref{class_beam_1_1_flux_track_node}}{}
\item\contentsline{section}{\doxymbox{\hyperlink{class_beam_1_1_gain_node}{Beam::\+\+Gain\+Node}} }{\pageref{class_beam_1_1_gain_node}}{}
\item\contentsline{section}{\doxymbox{\hyperlink{class_beam_1_1_input_handler}{Beam::\+\+Input\+Handler}} }{\pageref{class_beam_1_1_input_handler}}{}
\item\contentsline{section}{\doxymbox{\hyperlink{class_beam_1_1_knob}{Beam::\+\+Knob}} }{\pageref{class_beam_1_1_knob}}{}
\item\contentsline{section}{\doxymbox{\hyperlink{class_beam_1_1_master_node}{Beam::\+\+Master\+Node}} }{\pageref{class_beam_1_1_master_node}}{}
\item\contentsline{section}{\doxymbox{\hyperlink{class_beam_1_1_master_strip}{Beam::\+\+Master\+Strip}} }{\pageref{class_beam_1_1_master_strip}}{}
\item\contentsline{section}{\doxymbox{\hyperlink{class_beam_1_1_parameter}{Beam::\+\+Parameter}} }{\pageref{class_beam_1_1_parameter}}{}
\item\contentsline{section}{\doxymbox{\hyperlink{struct_beam_1_1_flux_node_1_1_port}{Beam::\+\+Flux\+Node::\+\+Port}} }{\pageref{struct_beam_1_1_flux_node_1_1_port}}{}
\item\contentsline{section}{\doxymbox{\hyperlink{class_beam_1_1_port}{Beam::\+\+Port}} }{\pageref{class_beam_1_1_port}}{}
\item\contentsline{section}{\doxymbox{\hyperlink{class_beam_1_1_project_manager}{Beam::\+\+Project\+Manager}} }{\pageref{class_beam_1_1_project_manager}}{}
\item\contentsline{section}{\doxymbox{\hyperlink{class_beam_1_1_quad_batcher}{Beam::\+\+Quad\+Batcher}} }{\pageref{class_beam_1_1_quad_batcher}}{}
\item\contentsline{section}{\doxymbox{\hyperlink{struct_beam_1_1_rect}{Beam::\+\+Rect}} }{\pageref{struct_beam_1_1_rect}}{}
\item\contentsline{section}{\doxymbox{\hyperlink{class_beam_1_1_shader}{Beam::\+\+Shader}} }{\pageref{class_beam_1_1_shader}}{}
\item\contentsline{section}{\doxymbox{\hyperlink{class_beam_1_1_sidebar}{Beam::\+\+Sidebar}} }{\pageref{class_beam_1_1_sidebar}}{}
\item\contentsline{section}{\doxymbox{\hyperlink{class_beam_1_1_sine_oscillator}{Beam::\+\+Sine\+Oscillator}} }{\pageref{class_beam_1_1_sine_oscillator}}{}
\item\contentsline{section}{\doxymbox{\hyperlink{class_beam_1_1_tape_reel}{Beam::\+\+Tape\+Reel}} }{\pageref{class_beam_1_1_tape_reel}}{}
\item\contentsline{section}{\doxymbox{\hyperlink{class_beam_1_1_timeline}{Beam::\+\+Timeline}} }{\pageref{class_beam_1_1_timeline}}{}
\item\contentsline{section}{\doxymbox{\hyperlink{class_beam_1_1_top_bar}{Beam::\+\+Top\+Bar}} }{\pageref{class_beam_1_1_top_bar}}{}
\item\contentsline{section}{\doxymbox{\hyperlink{class_beam_1_1_track_node}{Beam::\+\+Track\+Node}} }{\pageref{class_beam_1_1_track_node}}{}
\item\contentsline{section}{\doxymbox{\hyperlink{struct_beam_1_1_vertex}{Beam::\+\+Vertex}} }{\pageref{struct_beam_1_1_vertex}}{}
\item\contentsline{section}{\doxymbox{\hyperlink{class_beam_1_1_v_u_meter}{Beam::\+\+VUMeter}} }{\pageref{class_beam_1_1_v_u_meter}}{}
\item\contentsline{section}{\doxymbox{\hyperlink{class_beam_1_1_wav_reader}{Beam::\+\+Wav\+Reader}} }{\pageref{class_beam_1_1_wav_reader}}{}
\item\contentsline{section}{\doxymbox{\hyperlink{class_beam_1_1_wav_writer}{Beam::\+\+Wav\+Writer}} }{\pageref{class_beam_1_1_wav_writer}}{}
\item\contentsline{section}{\doxymbox{\hyperlink{class_beam_1_1_workspace}{Beam::\+\+Workspace}} }{\pageref{class_beam_1_1_workspace}}{}
\end{DoxyCompactList}


\section{File Index}
\doxysection{File List}
Here is a list of all files with brief descriptions:\+\begin{DoxyCompactList}
\item\contentsline{section}{src/\+\doxymbox{\hyperlink{main_8cpp}{main.\+cpp}} }{\pageref{main_8cpp}}{}
\item\contentsline{section}{src/\+application/\+\doxymbox{\hyperlink{application__base_8cpp}{application\+\_\+base.\+cpp}} }{\pageref{application__base_8cpp}}{}
\item\contentsline{section}{src/\+application/\+\doxymbox{\hyperlink{application__base_8hpp}{application\+\_\+base.\+hpp}} }{\pageref{application__base_8hpp}}{}
\item\contentsline{section}{src/\+engine/\+\doxymbox{\hyperlink{analog__base_8hpp}{analog\+\_\+base.\+hpp}} }{\pageref{analog__base_8hpp}}{}
\item\contentsline{section}{src/\+engine/\+\doxymbox{\hyperlink{analog__suite_8hpp}{analog\+\_\+suite.\+hpp}} }{\pageref{analog__suite_8hpp}}{}
\item\contentsline{section}{src/\+engine/\+\doxymbox{\hyperlink{audio__buffer_8hpp}{audio\+\_\+buffer.\+hpp}} }{\pageref{audio__buffer_8hpp}}{}
\item\contentsline{section}{src/\+engine/\+\doxymbox{\hyperlink{audio__device__manager_8cpp}{audio\+\_\+device\+\_\+manager.\+cpp}} }{\pageref{audio__device__manager_8cpp}}{}
\item\contentsline{section}{src/\+engine/\+\doxymbox{\hyperlink{audio__device__manager_8hpp}{audio\+\_\+device\+\_\+manager.\+hpp}} }{\pageref{audio__device__manager_8hpp}}{}
\item\contentsline{section}{src/\+engine/\+\doxymbox{\hyperlink{audio__engine_8cpp}{audio\+\_\+engine.\+cpp}} }{\pageref{audio__engine_8cpp}}{}
\item\contentsline{section}{src/\+engine/\+\doxymbox{\hyperlink{audio__engine_8hpp}{audio\+\_\+engine.\+hpp}} }{\pageref{audio__engine_8hpp}}{}
\item\contentsline{section}{src/\+engine/\+\doxymbox{\hyperlink{audio__node_8hpp}{audio\+\_\+node.\+hpp}} }{\pageref{audio__node_8hpp}}{}
\item\contentsline{section}{src/\+engine/\+\doxymbox{\hyperlink{audio__processor_8cpp}{audio\+\_\+processor.\+cpp}} }{\pageref{audio__processor_8cpp}}{}
\item\contentsline{section}{src/\+engine/\+\doxymbox{\hyperlink{audio__processor_8hpp}{audio\+\_\+processor.\+hpp}} }{\pageref{audio__processor_8hpp}}{}
\item\contentsline{section}{src/\+engine/\+\doxymbox{\hyperlink{audio__processor__value__tree__state_8cpp}{audio\+\_\+processor\+\_\+value\+\_\+tree\+\_\+state.\+cpp}} }{\pageref{audio__processor__value__tree__state_8cpp}}{}
\item\contentsline{section}{src/\+engine/\+\doxymbox{\hyperlink{audio__processor__value__tree__state_8hpp}{audio\+\_\+processor\+\_\+value\+\_\+tree\+\_\+state.\+hpp}} }{\pageref{audio__processor__value__tree__state_8hpp}}{}
\item\contentsline{section}{src/\+engine/\+\doxymbox{\hyperlink{audio__reader_8hpp}{audio\+\_\+reader.\+hpp}} }{\pageref{audio__reader_8hpp}}{}
\item\contentsline{section}{src/\+engine/\+\doxymbox{\hyperlink{biquad__filter__node_8hpp}{biquad\+\_\+filter\+\_\+node.\+hpp}} }{\pageref{biquad__filter__node_8hpp}}{}
\item\contentsline{section}{src/\+engine/\+\doxymbox{\hyperlink{custom__filter_8hpp}{custom\+\_\+filter.\+hpp}} }{\pageref{custom__filter_8hpp}}{}
\item\contentsline{section}{src/\+engine/\+\doxymbox{\hyperlink{delay__node_8hpp}{delay\+\_\+node.\+hpp}} }{\pageref{delay__node_8hpp}}{}
\item\contentsline{section}{src/\+engine/\+\doxymbox{\hyperlink{disk__streamer_8cpp}{disk\+\_\+streamer.\+cpp}} }{\pageref{disk__streamer_8cpp}}{}
\item\contentsline{section}{src/\+engine/\+\doxymbox{\hyperlink{disk__streamer_8hpp}{disk\+\_\+streamer.\+hpp}} }{\pageref{disk__streamer_8hpp}}{}
\item\contentsline{section}{src/\+engine/\+\doxymbox{\hyperlink{dsp__utils_8hpp}{dsp\+\_\+utils.\+hpp}} }{\pageref{dsp__utils_8hpp}}{}
\item\contentsline{section}{src/\+engine/\+\doxymbox{\hyperlink{flux__fx__nodes_8hpp}{flux\+\_\+fx\+\_\+nodes.\+hpp}} }{\pageref{flux__fx__nodes_8hpp}}{}
\item\contentsline{section}{src/\+engine/\+\doxymbox{\hyperlink{flux__graph_8hpp}{flux\+\_\+graph.\+hpp}} }{\pageref{flux__graph_8hpp}}{}
\item\contentsline{section}{src/\+engine/\+\doxymbox{\hyperlink{flux__node_8hpp}{flux\+\_\+node.\+hpp}} }{\pageref{flux__node_8hpp}}{}
\item\contentsline{section}{src/\+engine/\+\doxymbox{\hyperlink{flux__node__audio__processor__wrapper_8hpp}{flux\+\_\+node\+\_\+audio\+\_\+processor\+\_\+wrapper.\+hpp}} }{\pageref{flux__node__audio__processor__wrapper_8hpp}}{}
\item\contentsline{section}{src/\+engine/\+\doxymbox{\hyperlink{flux__plugin_8hpp}{flux\+\_\+plugin.\+hpp}} }{\pageref{flux__plugin_8hpp}}{}
\item\contentsline{section}{src/\+engine/\+\doxymbox{\hyperlink{flux__track__node_8hpp}{flux\+\_\+track\+\_\+node.\+hpp}} }{\pageref{flux__track__node_8hpp}}{}
\item\contentsline{section}{src/\+engine/\+\doxymbox{\hyperlink{input__node_8hpp}{input\+\_\+node.\+hpp}} }{\pageref{input__node_8hpp}}{}
\item\contentsline{section}{src/\+engine/\+\doxymbox{\hyperlink{master__node_8hpp}{master\+\_\+node.\+hpp}} }{\pageref{master__node_8hpp}}{}
\item\contentsline{section}{src/\+engine/\+\doxymbox{\hyperlink{midi__event_8hpp}{midi\+\_\+event.\+hpp}} }{\pageref{midi__event_8hpp}}{}
\item\contentsline{section}{src/\+engine/\+\doxymbox{\hyperlink{offline__renderer_8hpp}{offline\+\_\+renderer.\+hpp}} }{\pageref{offline__renderer_8hpp}}{}
\item\contentsline{section}{src/\+engine/\+\doxymbox{\hyperlink{render__plan_8hpp}{render\+\_\+plan.\+hpp}} }{\pageref{render__plan_8hpp}}{}
\item\contentsline{section}{src/\+engine/\+\doxymbox{\hyperlink{simd__utils_8hpp}{simd\+\_\+utils.\+hpp}} }{\pageref{simd__utils_8hpp}}{}
\item\contentsline{section}{src/\+engine/\+\doxymbox{\hyperlink{simple__gain__processor_8cpp}{simple\+\_\+gain\+\_\+processor.\+cpp}} }{\pageref{simple__gain__processor_8cpp}}{}
\item\contentsline{section}{src/\+engine/\+\doxymbox{\hyperlink{simple__gain__processor_8hpp}{simple\+\_\+gain\+\_\+processor.\+hpp}} }{\pageref{simple__gain__processor_8hpp}}{}
\item\contentsline{section}{src/\+engine/\+\doxymbox{\hyperlink{sine__synth__node_8hpp}{sine\+\_\+synth\+\_\+node.\+hpp}} }{\pageref{sine__synth__node_8hpp}}{}
\item\contentsline{section}{src/\+engine/\+\doxymbox{\hyperlink{track__node_8hpp}{track\+\_\+node.\+hpp}} }{\pageref{track__node_8hpp}}{}
\item\contentsline{section}{src/\+engine/\+\doxymbox{\hyperlink{tube__compressor__node_8hpp}{tube\+\_\+compressor\+\_\+node.\+hpp}} }{\pageref{tube__compressor__node_8hpp}}{}
\item\contentsline{section}{src/\+engine/\+\doxymbox{\hyperlink{wav__reader_8hpp}{wav\+\_\+reader.\+hpp}} }{\pageref{wav__reader_8hpp}}{}
\item\contentsline{section}{src/\+engine/\+\doxymbox{\hyperlink{wav__writer_8hpp}{wav\+\_\+writer.\+hpp}} }{\pageref{wav__writer_8hpp}}{}
\item\contentsline{section}{src/\+interface/\+\doxymbox{\hyperlink{audio__config__view_8hpp}{audio\+\_\+config\+\_\+view.\+hpp}} }{\pageref{audio__config__view_8hpp}}{}
\item\contentsline{section}{src/\+interface/\+\doxymbox{\hyperlink{audio__module_8hpp}{audio\+\_\+module.\+hpp}} }{\pageref{audio__module_8hpp}}{}
\item\contentsline{section}{src/\+interface/\+\doxymbox{\hyperlink{button_8cpp}{button.\+cpp}} }{\pageref{button_8cpp}}{}
\item\contentsline{section}{src/\+interface/\+\doxymbox{\hyperlink{button_8hpp}{button.\+hpp}} }{\pageref{button_8hpp}}{}
\item\contentsline{section}{src/\+interface/\+\doxymbox{\hyperlink{cable_8hpp}{cable.\+hpp}} }{\pageref{cable_8hpp}}{}
\item\contentsline{section}{src/\+interface/\+\doxymbox{\hyperlink{component_8hpp}{component.\+hpp}} }{\pageref{component_8hpp}}{}
\item\contentsline{section}{src/\+interface/\+\doxymbox{\hyperlink{dynamics__module_8hpp}{dynamics\+\_\+module.\+hpp}} }{\pageref{dynamics__module_8hpp}}{}
\item\contentsline{section}{src/\+interface/\+\doxymbox{\hyperlink{filter__graph_8hpp}{filter\+\_\+graph.\+hpp}} }{\pageref{filter__graph_8hpp}}{}
\item\contentsline{section}{src/\+interface/\+\doxymbox{\hyperlink{filter__module_8hpp}{filter\+\_\+module.\+hpp}} }{\pageref{filter__module_8hpp}}{}
\item\contentsline{section}{src/\+interface/\+\doxymbox{\hyperlink{gui__component_8cpp}{gui\+\_\+component.\+cpp}} }{\pageref{gui__component_8cpp}}{}
\item\contentsline{section}{src/\+interface/\+\doxymbox{\hyperlink{gui__component_8hpp}{gui\+\_\+component.\+hpp}} }{\pageref{gui__component_8hpp}}{}
\item\contentsline{section}{src/\+interface/\+\doxymbox{\hyperlink{input__handler_8cpp}{input\+\_\+handler.\+cpp}} }{\pageref{input__handler_8cpp}}{}
\item\contentsline{section}{src/\+interface/\+\doxymbox{\hyperlink{input__handler_8hpp}{input\+\_\+handler.\+hpp}} }{\pageref{input__handler_8hpp}}{}
\item\contentsline{section}{src/\+interface/\+\doxymbox{\hyperlink{knob_8hpp}{knob.\+hpp}} }{\pageref{knob_8hpp}}{}
\item\contentsline{section}{src/\+interface/\+\doxymbox{\hyperlink{master__strip_8hpp}{master\+\_\+strip.\+hpp}} }{\pageref{master__strip_8hpp}}{}
\item\contentsline{section}{src/\+interface/\+\doxymbox{\hyperlink{port_8hpp}{port.\+hpp}} }{\pageref{port_8hpp}}{}
\item\contentsline{section}{src/\+interface/\+\doxymbox{\hyperlink{sidebar_8hpp}{sidebar.\+hpp}} }{\pageref{sidebar_8hpp}}{}
\item\contentsline{section}{src/\+interface/\+\doxymbox{\hyperlink{slider_8cpp}{slider.\+cpp}} }{\pageref{slider_8cpp}}{}
\item\contentsline{section}{src/\+interface/\+\doxymbox{\hyperlink{slider_8hpp}{slider.\+hpp}} }{\pageref{slider_8hpp}}{}
\item\contentsline{section}{src/\+interface/\+\doxymbox{\hyperlink{tape__reel_8hpp}{tape\+\_\+reel.\+hpp}} }{\pageref{tape__reel_8hpp}}{}
\item\contentsline{section}{src/\+interface/\+\doxymbox{\hyperlink{timeline_8hpp}{timeline.\+hpp}} }{\pageref{timeline_8hpp}}{}
\item\contentsline{section}{src/\+interface/\+\doxymbox{\hyperlink{top__bar_8hpp}{top\+\_\+bar.\+hpp}} }{\pageref{top__bar_8hpp}}{}
\item\contentsline{section}{src/\+interface/\+\doxymbox{\hyperlink{tube__compressor__ui_8hpp}{tube\+\_\+compressor\+\_\+ui.\+hpp}} }{\pageref{tube__compressor__ui_8hpp}}{}
\item\contentsline{section}{src/\+interface/\+\doxymbox{\hyperlink{vu__meter_8hpp}{vu\+\_\+meter.\+hpp}} }{\pageref{vu__meter_8hpp}}{}
\item\contentsline{section}{src/\+interface/\+\doxymbox{\hyperlink{workspace_8hpp}{workspace.\+hpp}} }{\pageref{workspace_8hpp}}{}
\item\contentsline{section}{src/\+render/\+\doxymbox{\hyperlink{quad__batcher_8cpp}{quad\+\_\+batcher.\+cpp}} }{\pageref{quad__batcher_8cpp}}{}
\item\contentsline{section}{src/\+render/\+\doxymbox{\hyperlink{quad__batcher_8hpp}{quad\+\_\+batcher.\+hpp}} }{\pageref{quad__batcher_8hpp}}{}
\item\contentsline{section}{src/\+render/\+\doxymbox{\hyperlink{shader_8cpp}{shader.\+cpp}} }{\pageref{shader_8cpp}}{}
\item\contentsline{section}{src/\+render/\+\doxymbox{\hyperlink{shader_8hpp}{shader.\+hpp}} }{\pageref{shader_8hpp}}{}
\item\contentsline{section}{src/\+render/\+\doxymbox{\hyperlink{texture_8cpp}{texture.\+cpp}} }{\pageref{texture_8cpp}}{}
\item\contentsline{section}{src/\+render/\+\doxymbox{\hyperlink{texture_8hpp}{texture.\+hpp}} }{\pageref{texture_8hpp}}{}
\item\contentsline{section}{src/\+render/\+\doxymbox{\hyperlink{ui__shaders_8hpp}{ui\+\_\+shaders.\+hpp}} }{\pageref{ui__shaders_8hpp}}{}
\item\contentsline{section}{src/\+session/\+\doxymbox{\hyperlink{asset__manager_8cpp}{asset\+\_\+manager.\+cpp}} }{\pageref{asset__manager_8cpp}}{}
\item\contentsline{section}{src/\+session/\+\doxymbox{\hyperlink{asset__manager_8hpp}{asset\+\_\+manager.\+hpp}} }{\pageref{asset__manager_8hpp}}{}
\item\contentsline{section}{src/\+session/\+\doxymbox{\hyperlink{automation_8hpp}{automation.\+hpp}} }{\pageref{automation_8hpp}}{}
\item\contentsline{section}{src/\+session/\+\doxymbox{\hyperlink{beam__host_8cpp}{beam\+\_\+host.\+cpp}} }{\pageref{beam__host_8cpp}}{}
\item\contentsline{section}{src/\+session/\+\doxymbox{\hyperlink{beam__host_8hpp}{beam\+\_\+host.\+hpp}} }{\pageref{beam__host_8hpp}}{}
\item\contentsline{section}{src/\+session/\+\doxymbox{\hyperlink{flux__project_8hpp}{flux\+\_\+project.\+hpp}} }{\pageref{flux__project_8hpp}}{}
\item\contentsline{section}{src/\+session/\+\doxymbox{\hyperlink{parameter_8hpp}{parameter.\+hpp}} }{\pageref{parameter_8hpp}}{}
\item\contentsline{section}{src/\+session/\+\doxymbox{\hyperlink{project__manager_8hpp}{project\+\_\+manager.\+hpp}} }{\pageref{project__manager_8hpp}}{}
\item\contentsline{section}{src/\+session/\+\doxymbox{\hyperlink{region_8hpp}{region.\+hpp}} }{\pageref{region_8hpp}}{}
\item\contentsline{section}{src/\+utilities/\+\doxymbox{\hyperlink{flux__audio__utils_8hpp}{flux\+\_\+audio\+\_\+utils.\+hpp}} }{\pageref{flux__audio__utils_8hpp}}{}
\item\contentsline{section}{src/\+utilities/\+\doxymbox{\hyperlink{third__party__impl_8cpp}{third\+\_\+party\+\_\+impl.\+cpp}} }{\pageref{third__party__impl_8cpp}}{}
\end{DoxyCompactList}


% --- DOXYGEN DOCUMENTATION ---
\chapter{Namespace Documentation}
\doxysection{Beam Namespace Reference}
\hypertarget{namespace_beam}{}\label{namespace_beam}\index{Beam@{Beam}}
\doxysubsubsection*{Classes}
\begin{DoxyCompactItemize}
\item 
class \doxymbox{\hyperlink{class_beam_1_1_beam_host}{Beam\+Host}}
\item 
class \doxymbox{\hyperlink{class_beam_1_1_flux_project}{Flux\+Project}}
\item 
class \doxymbox{\hyperlink{class_beam_1_1_parameter}{Parameter}}
\item 
class \doxymbox{\hyperlink{class_beam_1_1_project_manager}{Project\+Manager}}
\item 
class \doxymbox{\hyperlink{class_beam_1_1_audio_engine}{Audio\+Engine}}
\item 
class \doxymbox{\hyperlink{class_beam_1_1_audio_node}{Audio\+Node}}
\item 
class \doxymbox{\hyperlink{class_beam_1_1_biquad_filter_node}{Biquad\+Filter\+Node}}
\item 
class \doxymbox{\hyperlink{class_beam_1_1_custom_filter}{Custom\+Filter}}
\item 
class \doxymbox{\hyperlink{class_beam_1_1_delay_node}{Delay\+Node}}
\item 
class \doxymbox{\hyperlink{class_beam_1_1_disk_streamer}{Disk\+Streamer}}
\item 
class \doxymbox{\hyperlink{class_beam_1_1_flux_gain_node}{Flux\+Gain\+Node}}
\item 
class \doxymbox{\hyperlink{class_beam_1_1_flux_filter_node}{Flux\+Filter\+Node}}
\item 
class \doxymbox{\hyperlink{class_beam_1_1_flux_delay_node}{Flux\+Delay\+Node}}
\item 
struct \doxymbox{\hyperlink{struct_beam_1_1_flux_connection}{Flux\+Connection}}
\item 
class \doxymbox{\hyperlink{class_beam_1_1_flux_graph}{Flux\+Graph}}
\item 
class \doxymbox{\hyperlink{class_beam_1_1_flux_node}{Flux\+Node}}
\item 
class \doxymbox{\hyperlink{class_beam_1_1_flux_plugin}{Flux\+Plugin}}
\item 
class \doxymbox{\hyperlink{class_beam_1_1_flux_track_node}{Flux\+Track\+Node}}
\item 
class \doxymbox{\hyperlink{class_beam_1_1_gain_node}{Gain\+Node}}
\item 
class \doxymbox{\hyperlink{class_beam_1_1_master_node}{Master\+Node}}
\item 
class \doxymbox{\hyperlink{class_beam_1_1_sine_oscillator}{Sine\+Oscillator}}
\item 
class \doxymbox{\hyperlink{class_beam_1_1_track_node}{Track\+Node}}
\item 
class \doxymbox{\hyperlink{class_beam_1_1_wav_reader}{Wav\+Reader}}
\item 
class \doxymbox{\hyperlink{class_beam_1_1_wav_writer}{Wav\+Writer}}
\item 
struct \doxymbox{\hyperlink{struct_beam_1_1_vertex}{Vertex}}
\item 
class \doxymbox{\hyperlink{class_beam_1_1_quad_batcher}{Quad\+Batcher}}
\item 
class \doxymbox{\hyperlink{class_beam_1_1_shader}{Shader}}
\item 
class \doxymbox{\hyperlink{class_beam_1_1_audio_module}{Audio\+Module}}
\item 
struct \doxymbox{\hyperlink{struct_beam_1_1_cable}{Cable}}
\item 
struct \doxymbox{\hyperlink{struct_beam_1_1_rect}{Rect}}
\item 
class \doxymbox{\hyperlink{class_beam_1_1_component}{Component}}
\item 
class \doxymbox{\hyperlink{class_beam_1_1_input_handler}{Input\+Handler}}
\item 
class \doxymbox{\hyperlink{class_beam_1_1_knob}{Knob}}
\item 
class \doxymbox{\hyperlink{class_beam_1_1_master_strip}{Master\+Strip}}
\item 
class \doxymbox{\hyperlink{class_beam_1_1_port}{Port}}
\item 
class \doxymbox{\hyperlink{class_beam_1_1_sidebar}{Sidebar}}
\item 
class \doxymbox{\hyperlink{class_beam_1_1_tape_reel}{Tape\+Reel}}
\item 
class \doxymbox{\hyperlink{class_beam_1_1_timeline}{Timeline}}
\item 
class \doxymbox{\hyperlink{class_beam_1_1_top_bar}{Top\+Bar}}
\item 
class \doxymbox{\hyperlink{class_beam_1_1_v_u_meter}{VUMeter}}
\item 
class \doxymbox{\hyperlink{class_beam_1_1_workspace}{Workspace}}
\end{DoxyCompactItemize}
\doxysubsubsection*{Enumerations}
\begin{DoxyCompactItemize}
\item 
enum class \doxymbox{\hyperlink{namespace_beam_a857abf9a9358b461e5e0b3a25b0c3d88}{DAWMode}} \{ \doxymbox{\hyperlink{namespace_beam_a857abf9a9358b461e5e0b3a25b0c3d88a79cf326cd40869983c7c685989e6dde6}{Splicing}}
, \doxymbox{\hyperlink{namespace_beam_a857abf9a9358b461e5e0b3a25b0c3d88a1c30b35b12895df175ccd44dbb6f5ace}{Flux}}
 \}
\item 
enum class \doxymbox{\hyperlink{namespace_beam_aa79c53bfd2295b09750b964fa2a326d6}{Filter\+Type}} \{ \doxymbox{\hyperlink{namespace_beam_aa79c53bfd2295b09750b964fa2a326d6a835f2dc4fd545ddcd275198d9fbadf63}{Low\+Pass}}
, \doxymbox{\hyperlink{namespace_beam_aa79c53bfd2295b09750b964fa2a326d6ae9456dc865a19e434997ea20bb5373f6}{High\+Pass}}
 \}
\item 
enum class \doxymbox{\hyperlink{namespace_beam_aaab526c0becfd931c9f29361daaf7e9f}{Track\+State}} \{ \doxymbox{\hyperlink{namespace_beam_aaab526c0becfd931c9f29361daaf7e9fae599161956d626eda4cb0a5ffb85271c}{Idle}}
, \doxymbox{\hyperlink{namespace_beam_aaab526c0becfd931c9f29361daaf7e9fac9dbb2b7c84159b632d71e512eba8428}{Playing}}
, \doxymbox{\hyperlink{namespace_beam_aaab526c0becfd931c9f29361daaf7e9fac5564d2e8b8e0ae08bf4363f2b947166}{Recording}}
 \}
\item 
enum class \doxymbox{\hyperlink{namespace_beam_ae4c392a66fa6e2b6c86c6d5356ffe846}{Port\+Type}} \{ \doxymbox{\hyperlink{namespace_beam_ae4c392a66fa6e2b6c86c6d5356ffe846a324118a6721dd6b8a9b9f4e327df2bf5}{Input}}
, \doxymbox{\hyperlink{namespace_beam_ae4c392a66fa6e2b6c86c6d5356ffe846a29c2c02a361c9d7028472e5d92cd4a54}{Output}}
 \}
\end{DoxyCompactItemize}
\doxysubsubsection*{Variables}
\begin{DoxyCompactItemize}
\item 
static const unsigned char \doxymbox{\hyperlink{namespace_beam_a6a3b07fc2741415e7291bcc8a4c08f45}{FONT\+\_\+\+DATA}} \+[95]\+\+[8]\+
\item 
const char \texorpdfstring{$\ast$}{*} \doxymbox{\hyperlink{namespace_beam_a7373b74547a3122b4d22d2969b46c6eb}{UI\+\_\+\+VERTEX\+\_\+\+SHADER}}
\item 
const char \texorpdfstring{$\ast$}{*} \doxymbox{\hyperlink{namespace_beam_a4f7be035693c2c29ba4dfa3150e8fc20}{UI\+\_\+\+FRAGMENT\+\_\+\+SHADER}}
\end{DoxyCompactItemize}


\label{doc-enum-members}
\Hypertarget{namespace_beam_doc-enum-members}
\doxysubsection{Enumeration Type Documentation}
\Hypertarget{namespace_beam_a857abf9a9358b461e5e0b3a25b0c3d88}\index{Beam@{Beam}!DAWMode@{DAWMode}}
\index{DAWMode@{DAWMode}!Beam@{Beam}}
\doxysubsubsection{\texorpdfstring{DAWMode}{DAWMode}}
{\footnotesize\ttfamily \label{namespace_beam_a857abf9a9358b461e5e0b3a25b0c3d88} 
enum class \doxymbox{\hyperlink{namespace_beam_a857abf9a9358b461e5e0b3a25b0c3d88}{Beam::\+\+DAWMode}}\hspace{0.3cm}{\ttfamily [strong]}}

\begin{DoxyEnumFields}[2]{Enumerator}
\raisebox{\heightof{T}}[0pt][0pt]{\index{Splicing@{Splicing}!Beam@{Beam}}
\index{Beam@{Beam}!Splicing@{Splicing}}
}\Hypertarget{namespace_beam_a857abf9a9358b461e5e0b3a25b0c3d88a79cf326cd40869983c7c685989e6dde6}\label{namespace_beam_a857abf9a9358b461e5e0b3a25b0c3d88a79cf326cd40869983c7c685989e6dde6} 
Splicing&\\
\hline

\raisebox{\heightof{T}}[0pt][0pt]{\index{Flux@{Flux}!Beam@{Beam}}
\index{Beam@{Beam}!Flux@{Flux}}
}\Hypertarget{namespace_beam_a857abf9a9358b461e5e0b3a25b0c3d88a1c30b35b12895df175ccd44dbb6f5ace}\label{namespace_beam_a857abf9a9358b461e5e0b3a25b0c3d88a1c30b35b12895df175ccd44dbb6f5ace} 
Flux&\\
\hline

\end{DoxyEnumFields}
\Hypertarget{namespace_beam_aa79c53bfd2295b09750b964fa2a326d6}\index{Beam@{Beam}!FilterType@{FilterType}}
\index{FilterType@{FilterType}!Beam@{Beam}}
\doxysubsubsection{\texorpdfstring{FilterType}{FilterType}}
{\footnotesize\ttfamily \label{namespace_beam_aa79c53bfd2295b09750b964fa2a326d6} 
enum class \doxymbox{\hyperlink{namespace_beam_aa79c53bfd2295b09750b964fa2a326d6}{Beam::\+\+Filter\+Type}}\hspace{0.3cm}{\ttfamily [strong]}}

\begin{DoxyEnumFields}[2]{Enumerator}
\raisebox{\heightof{T}}[0pt][0pt]{\index{LowPass@{LowPass}!Beam@{Beam}}
\index{Beam@{Beam}!LowPass@{LowPass}}
}\Hypertarget{namespace_beam_aa79c53bfd2295b09750b964fa2a326d6a835f2dc4fd545ddcd275198d9fbadf63}\label{namespace_beam_aa79c53bfd2295b09750b964fa2a326d6a835f2dc4fd545ddcd275198d9fbadf63} 
Low\+Pass&\\
\hline

\raisebox{\heightof{T}}[0pt][0pt]{\index{HighPass@{HighPass}!Beam@{Beam}}
\index{Beam@{Beam}!HighPass@{HighPass}}
}\Hypertarget{namespace_beam_aa79c53bfd2295b09750b964fa2a326d6ae9456dc865a19e434997ea20bb5373f6}\label{namespace_beam_aa79c53bfd2295b09750b964fa2a326d6ae9456dc865a19e434997ea20bb5373f6} 
High\+Pass&\\
\hline

\end{DoxyEnumFields}
\Hypertarget{namespace_beam_ae4c392a66fa6e2b6c86c6d5356ffe846}\index{Beam@{Beam}!PortType@{PortType}}
\index{PortType@{PortType}!Beam@{Beam}}
\doxysubsubsection{\texorpdfstring{PortType}{PortType}}
{\footnotesize\ttfamily \label{namespace_beam_ae4c392a66fa6e2b6c86c6d5356ffe846} 
enum class \doxymbox{\hyperlink{namespace_beam_ae4c392a66fa6e2b6c86c6d5356ffe846}{Beam::\+\+Port\+Type}}\hspace{0.3cm}{\ttfamily [strong]}}

\begin{DoxyEnumFields}[2]{Enumerator}
\raisebox{\heightof{T}}[0pt][0pt]{\index{Input@{Input}!Beam@{Beam}}
\index{Beam@{Beam}!Input@{Input}}
}\Hypertarget{namespace_beam_ae4c392a66fa6e2b6c86c6d5356ffe846a324118a6721dd6b8a9b9f4e327df2bf5}\label{namespace_beam_ae4c392a66fa6e2b6c86c6d5356ffe846a324118a6721dd6b8a9b9f4e327df2bf5} 
Input&\\
\hline

\raisebox{\heightof{T}}[0pt][0pt]{\index{Output@{Output}!Beam@{Beam}}
\index{Beam@{Beam}!Output@{Output}}
}\Hypertarget{namespace_beam_ae4c392a66fa6e2b6c86c6d5356ffe846a29c2c02a361c9d7028472e5d92cd4a54}\label{namespace_beam_ae4c392a66fa6e2b6c86c6d5356ffe846a29c2c02a361c9d7028472e5d92cd4a54} 
Output&\\
\hline

\end{DoxyEnumFields}
\Hypertarget{namespace_beam_aaab526c0becfd931c9f29361daaf7e9f}\index{Beam@{Beam}!TrackState@{TrackState}}
\index{TrackState@{TrackState}!Beam@{Beam}}
\doxysubsubsection{\texorpdfstring{TrackState}{TrackState}}
{\footnotesize\ttfamily \label{namespace_beam_aaab526c0becfd931c9f29361daaf7e9f} 
enum class \doxymbox{\hyperlink{namespace_beam_aaab526c0becfd931c9f29361daaf7e9f}{Beam::\+\+Track\+State}}\hspace{0.3cm}{\ttfamily [strong]}}

\begin{DoxyEnumFields}[2]{Enumerator}
\raisebox{\heightof{T}}[0pt][0pt]{\index{Idle@{Idle}!Beam@{Beam}}
\index{Beam@{Beam}!Idle@{Idle}}
}\Hypertarget{namespace_beam_aaab526c0becfd931c9f29361daaf7e9fae599161956d626eda4cb0a5ffb85271c}\label{namespace_beam_aaab526c0becfd931c9f29361daaf7e9fae599161956d626eda4cb0a5ffb85271c} 
Idle&\\
\hline

\raisebox{\heightof{T}}[0pt][0pt]{\index{Playing@{Playing}!Beam@{Beam}}
\index{Beam@{Beam}!Playing@{Playing}}
}\Hypertarget{namespace_beam_aaab526c0becfd931c9f29361daaf7e9fac9dbb2b7c84159b632d71e512eba8428}\label{namespace_beam_aaab526c0becfd931c9f29361daaf7e9fac9dbb2b7c84159b632d71e512eba8428} 
Playing&\\
\hline

\raisebox{\heightof{T}}[0pt][0pt]{\index{Recording@{Recording}!Beam@{Beam}}
\index{Beam@{Beam}!Recording@{Recording}}
}\Hypertarget{namespace_beam_aaab526c0becfd931c9f29361daaf7e9fac5564d2e8b8e0ae08bf4363f2b947166}\label{namespace_beam_aaab526c0becfd931c9f29361daaf7e9fac5564d2e8b8e0ae08bf4363f2b947166} 
Recording&\\
\hline

\end{DoxyEnumFields}


\label{doc-var-members}
\Hypertarget{namespace_beam_doc-var-members}
\doxysubsection{Variable Documentation}
\Hypertarget{namespace_beam_a6a3b07fc2741415e7291bcc8a4c08f45}\index{Beam@{Beam}!FONT\_DATA@{FONT\_DATA}}
\index{FONT\_DATA@{FONT\_DATA}!Beam@{Beam}}
\doxysubsubsection{\texorpdfstring{FONT\_DATA}{FONT\_DATA}}
{\footnotesize\ttfamily \label{namespace_beam_a6a3b07fc2741415e7291bcc8a4c08f45} 
const unsigned char Beam::\+\+FONT\+\_\+\+DATA\+[95]\+\+[8]\+\hspace{0.3cm}{\ttfamily [static]}}

\Hypertarget{namespace_beam_a4f7be035693c2c29ba4dfa3150e8fc20}\index{Beam@{Beam}!UI\_FRAGMENT\_SHADER@{UI\_FRAGMENT\_SHADER}}
\index{UI\_FRAGMENT\_SHADER@{UI\_FRAGMENT\_SHADER}!Beam@{Beam}}
\doxysubsubsection{\texorpdfstring{UI\_FRAGMENT\_SHADER}{UI\_FRAGMENT\_SHADER}}
{\footnotesize\ttfamily \label{namespace_beam_a4f7be035693c2c29ba4dfa3150e8fc20} 
const char\texorpdfstring{$\ast$}{*} Beam::\+\+UI\+\_\+\+FRAGMENT\+\_\+\+SHADER}

\Hypertarget{namespace_beam_a7373b74547a3122b4d22d2969b46c6eb}\index{Beam@{Beam}!UI\_VERTEX\_SHADER@{UI\_VERTEX\_SHADER}}
\index{UI\_VERTEX\_SHADER@{UI\_VERTEX\_SHADER}!Beam@{Beam}}
\doxysubsubsection{\texorpdfstring{UI\_VERTEX\_SHADER}{UI\_VERTEX\_SHADER}}
{\footnotesize\ttfamily \label{namespace_beam_a7373b74547a3122b4d22d2969b46c6eb} 
const char\texorpdfstring{$\ast$}{*} Beam::\+\+UI\+\_\+\+VERTEX\+\_\+\+SHADER}

{\bfseries Initial value:\+}
\begin{DoxyCode}{0}
\DoxyCodeLine{\ \ \ \ \ \ \ \ \ \ \ \ \ \ \ \ \ \ \ \ \ \ \ \ \ \ \ \ =\ R\textcolor{stringliteral}{"{}(}}
\DoxyCodeLine{\textcolor{stringliteral}{\#version\ 330\ core}}
\DoxyCodeLine{\textcolor{stringliteral}{layout\ (location\ =\ 0)\ in\ vec2\ aPos;}}
\DoxyCodeLine{\textcolor{stringliteral}{layout\ (location\ =\ 1)\ in\ vec2\ aTexCoord;}}
\DoxyCodeLine{\textcolor{stringliteral}{layout\ (location\ =\ 2)\ in\ vec4\ aColor;}}
\DoxyCodeLine{\textcolor{stringliteral}{}}
\DoxyCodeLine{\textcolor{stringliteral}{out\ vec2\ TexCoord;}}
\DoxyCodeLine{\textcolor{stringliteral}{out\ vec4\ Color;}}
\DoxyCodeLine{\textcolor{stringliteral}{}}
\DoxyCodeLine{\textcolor{stringliteral}{uniform\ mat4\ projection;}}
\DoxyCodeLine{\textcolor{stringliteral}{}}
\DoxyCodeLine{\textcolor{stringliteral}{void\ main()\ \{}}
\DoxyCodeLine{\textcolor{stringliteral}{\ \ \ \ gl\_Position\ =\ projection\ *\ vec4(aPos,\ 0.0,\ 1.0);}}
\DoxyCodeLine{\textcolor{stringliteral}{\ \ \ \ TexCoord\ =\ aTexCoord;}}
\DoxyCodeLine{\textcolor{stringliteral}{\ \ \ \ Color\ =\ aColor;}}
\DoxyCodeLine{\textcolor{stringliteral}{\}}}
\DoxyCodeLine{\textcolor{stringliteral}{)"{}}}

\end{DoxyCode}

\doxysection{Beam::\+SIMD Namespace Reference}
\hypertarget{namespace_beam_1_1_s_i_m_d}{}\label{namespace_beam_1_1_s_i_m_d}\index{Beam::SIMD@{Beam::SIMD}}
\doxysubsubsection*{Functions}
\begin{DoxyCompactItemize}
\item 
void \doxymbox{\hyperlink{namespace_beam_1_1_s_i_m_d_ade8c1c6b1b45abf8d66ee11c8fb936a1}{add}} (const float \texorpdfstring{$\ast$}{*}src, float \texorpdfstring{$\ast$}{*}dst, int count)
\begin{DoxyCompactList}\small\item\em Adds two buffers using SSE (4 floats at a time). \end{DoxyCompactList}\item 
void \doxymbox{\hyperlink{namespace_beam_1_1_s_i_m_d_acda0a1ac1a531cf8e16b2d9b4f3b77df}{copy}} (const float \texorpdfstring{$\ast$}{*}src, float \texorpdfstring{$\ast$}{*}dst, int count)
\begin{DoxyCompactList}\small\item\em Copies a buffer using SSE. \end{DoxyCompactList}\end{DoxyCompactItemize}


\label{doc-func-members}
\Hypertarget{namespace_beam_1_1_s_i_m_d_doc-func-members}
\doxysubsection{Function Documentation}
\Hypertarget{namespace_beam_1_1_s_i_m_d_ade8c1c6b1b45abf8d66ee11c8fb936a1}\index{Beam::SIMD@{Beam::SIMD}!add@{add}}
\index{add@{add}!Beam::SIMD@{Beam::SIMD}}
\doxysubsubsection{\texorpdfstring{add()}{add()}}
{\footnotesize\ttfamily \label{namespace_beam_1_1_s_i_m_d_ade8c1c6b1b45abf8d66ee11c8fb936a1} 
void Beam::\+\+SIMD::\+add (\begin{DoxyParamCaption}\item[{const float \texorpdfstring{$\ast$}{*}}]{src}{, }\item[{float \texorpdfstring{$\ast$}{*}}]{dst}{, }\item[{int}]{count}{}\end{DoxyParamCaption})\hspace{0.3cm}{\ttfamily [inline]}}



Adds two buffers using SSE (4 floats at a time). 


\begin{DoxyParams}{Parameters}
{\em src} & Source buffer. \\
\hline
{\em dst} & Destination buffer (result is summed into this). \\
\hline
{\em count} & Total number of samples (must be multiple of 4 for full optimization). \\
\hline
\end{DoxyParams}
\Hypertarget{namespace_beam_1_1_s_i_m_d_acda0a1ac1a531cf8e16b2d9b4f3b77df}\index{Beam::SIMD@{Beam::SIMD}!copy@{copy}}
\index{copy@{copy}!Beam::SIMD@{Beam::SIMD}}
\doxysubsubsection{\texorpdfstring{copy()}{copy()}}
{\footnotesize\ttfamily \label{namespace_beam_1_1_s_i_m_d_acda0a1ac1a531cf8e16b2d9b4f3b77df} 
void Beam::\+\+SIMD::\+copy (\begin{DoxyParamCaption}\item[{const float \texorpdfstring{$\ast$}{*}}]{src}{, }\item[{float \texorpdfstring{$\ast$}{*}}]{dst}{, }\item[{int}]{count}{}\end{DoxyParamCaption})\hspace{0.3cm}{\ttfamily [inline]}}



Copies a buffer using SSE. 



\chapter{Class Documentation}
\doxysection{Beam::\+Beam\+Host Class Reference}
\hypertarget{class_beam_1_1_beam_host}{}\label{class_beam_1_1_beam_host}\index{Beam::BeamHost@{Beam::BeamHost}}


{\ttfamily \+\#include $<$beam\+\_\+host.\+hpp$>$}

\doxysubsubsection*{Public Member Functions}
\begin{DoxyCompactItemize}
\item 
\doxymbox{\hyperlink{class_beam_1_1_beam_host_a144a92978b5e61326146a3a8f29d761e}{Beam\+Host}} (const std::\+string \&title, int \doxymbox{\hyperlink{texture_8cpp_a8710f3c5c66c09e158c8619b3fca614a}{width}}, int \doxymbox{\hyperlink{texture_8cpp_a1055637f17e35a0ca82b396bb94914e5}{height}})
\item 
\doxymbox{\hyperlink{class_beam_1_1_beam_host_a69049b6b6862a23f94223d4f596ee594}{\texorpdfstring{$\sim$}{\string~}\+Beam\+Host}} ()
\item 
bool \doxymbox{\hyperlink{class_beam_1_1_beam_host_a624244ab7bcf572b1b931aa41608b168}{init}} ()
\item 
void \doxymbox{\hyperlink{class_beam_1_1_beam_host_a10355c107fb7450b2a0640232bbeb81e}{run}} ()
\item 
void \doxymbox{\hyperlink{class_beam_1_1_beam_host_affc3b3fb2b16eaea708a8b07a1438c2a}{stop}} ()
\item 
void \doxymbox{\hyperlink{class_beam_1_1_beam_host_aaeeaebd43fe7b6bdff057c5cf13e293c}{set\+Mode}} (\doxymbox{\hyperlink{namespace_beam_a857abf9a9358b461e5e0b3a25b0c3d88}{DAWMode}} mode)
\item 
\doxymbox{\hyperlink{namespace_beam_a857abf9a9358b461e5e0b3a25b0c3d88}{DAWMode}} \doxymbox{\hyperlink{class_beam_1_1_beam_host_abe5013f58a2048c6f6d344ff5a04cf7c}{get\+Mode}} () const
\end{DoxyCompactItemize}
\doxysubsubsection*{Static Public Member Functions}
\begin{DoxyCompactItemize}
\item 
static void \doxymbox{\hyperlink{class_beam_1_1_beam_host_a34046f705ed8e44849fa27838d64af16}{on\+File\+Selected}} (void \texorpdfstring{$\ast$}{*}userdata, const char \texorpdfstring{$\ast$}{*}const \texorpdfstring{$\ast$}{*}filelist, int filter)
\end{DoxyCompactItemize}
\doxysubsubsection*{Private Member Functions}
\begin{DoxyCompactItemize}
\item 
void \doxymbox{\hyperlink{class_beam_1_1_beam_host_ab14ad5484149d76f707b81af188eede5}{handle\+Events}} ()
\item 
void \doxymbox{\hyperlink{class_beam_1_1_beam_host_a4c1ebd2058246d175889db7de8513a8b}{update}} ()
\item 
void \doxymbox{\hyperlink{class_beam_1_1_beam_host_abde90804501af271a7f377bd12cfc58d}{render}} (float dt)
\item 
void \doxymbox{\hyperlink{class_beam_1_1_beam_host_af7bba52e0b33db552233c3197a8b39c7}{perform\+Layout}} ()
\end{DoxyCompactItemize}
\doxysubsubsection*{Private Attributes}
\begin{DoxyCompactItemize}
\item 
std::\+string \doxymbox{\hyperlink{class_beam_1_1_beam_host_a96e1bd8f9a31b93da7ad3401b1dc778f}{m\+\_\+title}}
\item 
int \doxymbox{\hyperlink{class_beam_1_1_beam_host_a8588100db6dadbf9f3d3bb5faff93a61}{m\+\_\+width}}
\item 
int \doxymbox{\hyperlink{class_beam_1_1_beam_host_a8a402533c41936ec6b0554a03716fd1a}{m\+\_\+height}}
\item 
bool \doxymbox{\hyperlink{class_beam_1_1_beam_host_ac7abda34255ce3126887a0fe44df7a65}{m\+\_\+is\+Running}}
\item 
\doxymbox{\hyperlink{namespace_beam_a857abf9a9358b461e5e0b3a25b0c3d88}{DAWMode}} \doxymbox{\hyperlink{class_beam_1_1_beam_host_a2aaeeb49d79e64d1b5bbe6844b67ee7a}{m\+\_\+mode}} = \doxymbox{\hyperlink{namespace_beam_a857abf9a9358b461e5e0b3a25b0c3d88a1c30b35b12895df175ccd44dbb6f5ace}{DAWMode::\+\+Flux}}
\item 
SDL\+\_\+\+Window \texorpdfstring{$\ast$}{*} \doxymbox{\hyperlink{class_beam_1_1_beam_host_afea4f3dffefd6d2e09e30d3e0744d82e}{m\+\_\+window}}
\item 
SDL\+\_\+\+GLContext \doxymbox{\hyperlink{class_beam_1_1_beam_host_a25a31d24eb87a8254522491ad671b1d4}{m\+\_\+gl\+Context}}
\item 
std::\+shared\+\_\+ptr$<$ \doxymbox{\hyperlink{class_beam_1_1_flux_project}{Flux\+Project}} $>$ \doxymbox{\hyperlink{class_beam_1_1_beam_host_a8c40df6cc8ed1c3fa6e3751526dea8d5}{m\+\_\+project}}
\item 
std::\+unique\+\_\+ptr$<$ \doxymbox{\hyperlink{class_beam_1_1_audio_engine}{Audio\+Engine}} $>$ \doxymbox{\hyperlink{class_beam_1_1_beam_host_aefb093db282894cda12f36206a7ded34}{m\+\_\+audio\+Engine}}
\item 
std::\+unique\+\_\+ptr$<$ class \doxymbox{\hyperlink{class_beam_1_1_audio_device_manager}{Audio\+Device\+Manager}} $>$ \doxymbox{\hyperlink{class_beam_1_1_beam_host_aac26727eb841244dc6e063eb3c39f935}{m\+\_\+audio\+Device\+Manager}}
\item 
std::\+unique\+\_\+ptr$<$ \doxymbox{\hyperlink{class_beam_1_1_quad_batcher}{Quad\+Batcher}} $>$ \doxymbox{\hyperlink{class_beam_1_1_beam_host_a5a7952cd0cecc2865975a796ba2a5523}{m\+\_\+batcher}}
\item 
std::\+unique\+\_\+ptr$<$ \doxymbox{\hyperlink{class_beam_1_1_shader}{Shader}} $>$ \doxymbox{\hyperlink{class_beam_1_1_beam_host_aeed02c8e1d960b2b0dbdee48c62c671a}{m\+\_\+ui\+Shader}}
\item 
std::\+unique\+\_\+ptr$<$ \doxymbox{\hyperlink{class_beam_1_1_input_handler}{Input\+Handler}} $>$ \doxymbox{\hyperlink{class_beam_1_1_beam_host_a5bd4a6f62b57908f5607e0fe7ef8166d}{m\+\_\+ui\+Handler}}
\item 
std::\+shared\+\_\+ptr$<$ class \doxymbox{\hyperlink{class_beam_1_1_workspace}{Workspace}} $>$ \doxymbox{\hyperlink{class_beam_1_1_beam_host_acb133910e5fa3a9f78e4b9f13625d0c9}{m\+\_\+workspace}}
\item 
std::\+shared\+\_\+ptr$<$ class \doxymbox{\hyperlink{class_beam_1_1_timeline}{Timeline}} $>$ \doxymbox{\hyperlink{class_beam_1_1_beam_host_ad49d80c9ee8b743f20073756f1805ec3}{m\+\_\+timeline}}
\item 
std::\+shared\+\_\+ptr$<$ class \doxymbox{\hyperlink{class_beam_1_1_top_bar}{Top\+Bar}} $>$ \doxymbox{\hyperlink{class_beam_1_1_beam_host_a9d7779bbadd94af671a40877e0118b8b}{m\+\_\+top\+Bar}}
\item 
std::\+shared\+\_\+ptr$<$ class \doxymbox{\hyperlink{class_beam_1_1_sidebar}{Sidebar}} $>$ \doxymbox{\hyperlink{class_beam_1_1_beam_host_a29b3d9b64a195b17b3bb4d854e2854e8}{m\+\_\+browser}}
\item 
std::\+shared\+\_\+ptr$<$ class \doxymbox{\hyperlink{class_beam_1_1_master_strip}{Master\+Strip}} $>$ \doxymbox{\hyperlink{class_beam_1_1_beam_host_a4c6405471172009a7ee7509e93152ff8}{m\+\_\+master\+Strip}}
\item 
std::\+shared\+\_\+ptr$<$ class \doxymbox{\hyperlink{class_beam_1_1_audio_config_view}{Audio\+Config\+View}} $>$ \doxymbox{\hyperlink{class_beam_1_1_beam_host_a38fd61099ad675f9d816366de3ef852e}{m\+\_\+config\+View}}
\end{DoxyCompactItemize}


\label{doc-constructors}
\Hypertarget{class_beam_1_1_beam_host_doc-constructors}
\doxysubsection{Constructor \& Destructor Documentation}
\Hypertarget{class_beam_1_1_beam_host_a144a92978b5e61326146a3a8f29d761e}\index{Beam::BeamHost@{Beam::BeamHost}!BeamHost@{BeamHost}}
\index{BeamHost@{BeamHost}!Beam::BeamHost@{Beam::BeamHost}}
\doxysubsubsection{\texorpdfstring{BeamHost()}{BeamHost()}}
{\footnotesize\ttfamily \label{class_beam_1_1_beam_host_a144a92978b5e61326146a3a8f29d761e} 
Beam::\+\+Beam\+Host::\+\+Beam\+Host (\begin{DoxyParamCaption}\item[{const std::\+string \&}]{title}{, }\item[{int}]{width}{, }\item[{int}]{height}{}\end{DoxyParamCaption})}

\Hypertarget{class_beam_1_1_beam_host_a69049b6b6862a23f94223d4f596ee594}\index{Beam::BeamHost@{Beam::BeamHost}!````~BeamHost@{\texorpdfstring{$\sim$}{\string~}BeamHost}}
\index{````~BeamHost@{\texorpdfstring{$\sim$}{\string~}BeamHost}!Beam::BeamHost@{Beam::BeamHost}}
\doxysubsubsection{\texorpdfstring{\texorpdfstring{$\sim$}{\string~}BeamHost()}{\string~BeamHost()}}
{\footnotesize\ttfamily \label{class_beam_1_1_beam_host_a69049b6b6862a23f94223d4f596ee594} 
Beam::\+\+Beam\+Host::\+\texorpdfstring{$\sim$}{\string~}\+Beam\+Host (\begin{DoxyParamCaption}{}{}\end{DoxyParamCaption})}



\label{doc-func-members}
\Hypertarget{class_beam_1_1_beam_host_doc-func-members}
\doxysubsection{Member Function Documentation}
\Hypertarget{class_beam_1_1_beam_host_abe5013f58a2048c6f6d344ff5a04cf7c}\index{Beam::BeamHost@{Beam::BeamHost}!getMode@{getMode}}
\index{getMode@{getMode}!Beam::BeamHost@{Beam::BeamHost}}
\doxysubsubsection{\texorpdfstring{getMode()}{getMode()}}
{\footnotesize\ttfamily \label{class_beam_1_1_beam_host_abe5013f58a2048c6f6d344ff5a04cf7c} 
\doxymbox{\hyperlink{namespace_beam_a857abf9a9358b461e5e0b3a25b0c3d88}{DAWMode}} Beam::\+\+Beam\+Host::\+get\+Mode (\begin{DoxyParamCaption}{}{}\end{DoxyParamCaption}) const\hspace{0.3cm}{\ttfamily [inline]}}

\Hypertarget{class_beam_1_1_beam_host_ab14ad5484149d76f707b81af188eede5}\index{Beam::BeamHost@{Beam::BeamHost}!handleEvents@{handleEvents}}
\index{handleEvents@{handleEvents}!Beam::BeamHost@{Beam::BeamHost}}
\doxysubsubsection{\texorpdfstring{handleEvents()}{handleEvents()}}
{\footnotesize\ttfamily \label{class_beam_1_1_beam_host_ab14ad5484149d76f707b81af188eede5} 
void Beam::\+\+Beam\+Host::\+handle\+Events (\begin{DoxyParamCaption}{}{}\end{DoxyParamCaption})\hspace{0.3cm}{\ttfamily [private]}}

\Hypertarget{class_beam_1_1_beam_host_a624244ab7bcf572b1b931aa41608b168}\index{Beam::BeamHost@{Beam::BeamHost}!init@{init}}
\index{init@{init}!Beam::BeamHost@{Beam::BeamHost}}
\doxysubsubsection{\texorpdfstring{init()}{init()}}
{\footnotesize\ttfamily \label{class_beam_1_1_beam_host_a624244ab7bcf572b1b931aa41608b168} 
bool Beam::\+\+Beam\+Host::\+init (\begin{DoxyParamCaption}{}{}\end{DoxyParamCaption})}

\Hypertarget{class_beam_1_1_beam_host_a34046f705ed8e44849fa27838d64af16}\index{Beam::BeamHost@{Beam::BeamHost}!onFileSelected@{onFileSelected}}
\index{onFileSelected@{onFileSelected}!Beam::BeamHost@{Beam::BeamHost}}
\doxysubsubsection{\texorpdfstring{onFileSelected()}{onFileSelected()}}
{\footnotesize\ttfamily \label{class_beam_1_1_beam_host_a34046f705ed8e44849fa27838d64af16} 
void Beam::\+\+Beam\+Host::\+on\+File\+Selected (\begin{DoxyParamCaption}\item[{void \texorpdfstring{$\ast$}{*}}]{userdata}{, }\item[{const char \texorpdfstring{$\ast$}{*}const \texorpdfstring{$\ast$}{*}}]{filelist}{, }\item[{int}]{filter}{}\end{DoxyParamCaption})\hspace{0.3cm}{\ttfamily [static]}}

\Hypertarget{class_beam_1_1_beam_host_af7bba52e0b33db552233c3197a8b39c7}\index{Beam::BeamHost@{Beam::BeamHost}!performLayout@{performLayout}}
\index{performLayout@{performLayout}!Beam::BeamHost@{Beam::BeamHost}}
\doxysubsubsection{\texorpdfstring{performLayout()}{performLayout()}}
{\footnotesize\ttfamily \label{class_beam_1_1_beam_host_af7bba52e0b33db552233c3197a8b39c7} 
void Beam::\+\+Beam\+Host::\+perform\+Layout (\begin{DoxyParamCaption}{}{}\end{DoxyParamCaption})\hspace{0.3cm}{\ttfamily [private]}}

\Hypertarget{class_beam_1_1_beam_host_abde90804501af271a7f377bd12cfc58d}\index{Beam::BeamHost@{Beam::BeamHost}!render@{render}}
\index{render@{render}!Beam::BeamHost@{Beam::BeamHost}}
\doxysubsubsection{\texorpdfstring{render()}{render()}}
{\footnotesize\ttfamily \label{class_beam_1_1_beam_host_abde90804501af271a7f377bd12cfc58d} 
void Beam::\+\+Beam\+Host::\+render (\begin{DoxyParamCaption}\item[{float}]{dt}{}\end{DoxyParamCaption})\hspace{0.3cm}{\ttfamily [private]}}

\Hypertarget{class_beam_1_1_beam_host_a10355c107fb7450b2a0640232bbeb81e}\index{Beam::BeamHost@{Beam::BeamHost}!run@{run}}
\index{run@{run}!Beam::BeamHost@{Beam::BeamHost}}
\doxysubsubsection{\texorpdfstring{run()}{run()}}
{\footnotesize\ttfamily \label{class_beam_1_1_beam_host_a10355c107fb7450b2a0640232bbeb81e} 
void Beam::\+\+Beam\+Host::\+run (\begin{DoxyParamCaption}{}{}\end{DoxyParamCaption})}

\Hypertarget{class_beam_1_1_beam_host_aaeeaebd43fe7b6bdff057c5cf13e293c}\index{Beam::BeamHost@{Beam::BeamHost}!setMode@{setMode}}
\index{setMode@{setMode}!Beam::BeamHost@{Beam::BeamHost}}
\doxysubsubsection{\texorpdfstring{setMode()}{setMode()}}
{\footnotesize\ttfamily \label{class_beam_1_1_beam_host_aaeeaebd43fe7b6bdff057c5cf13e293c} 
void Beam::\+\+Beam\+Host::\+set\+Mode (\begin{DoxyParamCaption}\item[{\doxymbox{\hyperlink{namespace_beam_a857abf9a9358b461e5e0b3a25b0c3d88}{DAWMode}}}]{mode}{}\end{DoxyParamCaption})}

\Hypertarget{class_beam_1_1_beam_host_affc3b3fb2b16eaea708a8b07a1438c2a}\index{Beam::BeamHost@{Beam::BeamHost}!stop@{stop}}
\index{stop@{stop}!Beam::BeamHost@{Beam::BeamHost}}
\doxysubsubsection{\texorpdfstring{stop()}{stop()}}
{\footnotesize\ttfamily \label{class_beam_1_1_beam_host_affc3b3fb2b16eaea708a8b07a1438c2a} 
void Beam::\+\+Beam\+Host::\+stop (\begin{DoxyParamCaption}{}{}\end{DoxyParamCaption})}

\Hypertarget{class_beam_1_1_beam_host_a4c1ebd2058246d175889db7de8513a8b}\index{Beam::BeamHost@{Beam::BeamHost}!update@{update}}
\index{update@{update}!Beam::BeamHost@{Beam::BeamHost}}
\doxysubsubsection{\texorpdfstring{update()}{update()}}
{\footnotesize\ttfamily \label{class_beam_1_1_beam_host_a4c1ebd2058246d175889db7de8513a8b} 
void Beam::\+\+Beam\+Host::\+update (\begin{DoxyParamCaption}{}{}\end{DoxyParamCaption})\hspace{0.3cm}{\ttfamily [private]}}



\label{doc-variable-members}
\Hypertarget{class_beam_1_1_beam_host_doc-variable-members}
\doxysubsection{Member Data Documentation}
\Hypertarget{class_beam_1_1_beam_host_aac26727eb841244dc6e063eb3c39f935}\index{Beam::BeamHost@{Beam::BeamHost}!m\_audioDeviceManager@{m\_audioDeviceManager}}
\index{m\_audioDeviceManager@{m\_audioDeviceManager}!Beam::BeamHost@{Beam::BeamHost}}
\doxysubsubsection{\texorpdfstring{m\_audioDeviceManager}{m\_audioDeviceManager}}
{\footnotesize\ttfamily \label{class_beam_1_1_beam_host_aac26727eb841244dc6e063eb3c39f935} 
std::\+unique\+\_\+ptr$<$class \doxymbox{\hyperlink{class_beam_1_1_audio_device_manager}{Audio\+Device\+Manager}}$>$ Beam::\+\+Beam\+Host::\+m\+\_\+audio\+Device\+Manager\hspace{0.3cm}{\ttfamily [private]}}

\Hypertarget{class_beam_1_1_beam_host_aefb093db282894cda12f36206a7ded34}\index{Beam::BeamHost@{Beam::BeamHost}!m\_audioEngine@{m\_audioEngine}}
\index{m\_audioEngine@{m\_audioEngine}!Beam::BeamHost@{Beam::BeamHost}}
\doxysubsubsection{\texorpdfstring{m\_audioEngine}{m\_audioEngine}}
{\footnotesize\ttfamily \label{class_beam_1_1_beam_host_aefb093db282894cda12f36206a7ded34} 
std::\+unique\+\_\+ptr$<$\doxymbox{\hyperlink{class_beam_1_1_audio_engine}{Audio\+Engine}}$>$ Beam::\+\+Beam\+Host::\+m\+\_\+audio\+Engine\hspace{0.3cm}{\ttfamily [private]}}

\Hypertarget{class_beam_1_1_beam_host_a5a7952cd0cecc2865975a796ba2a5523}\index{Beam::BeamHost@{Beam::BeamHost}!m\_batcher@{m\_batcher}}
\index{m\_batcher@{m\_batcher}!Beam::BeamHost@{Beam::BeamHost}}
\doxysubsubsection{\texorpdfstring{m\_batcher}{m\_batcher}}
{\footnotesize\ttfamily \label{class_beam_1_1_beam_host_a5a7952cd0cecc2865975a796ba2a5523} 
std::\+unique\+\_\+ptr$<$\doxymbox{\hyperlink{class_beam_1_1_quad_batcher}{Quad\+Batcher}}$>$ Beam::\+\+Beam\+Host::\+m\+\_\+batcher\hspace{0.3cm}{\ttfamily [private]}}

\Hypertarget{class_beam_1_1_beam_host_a29b3d9b64a195b17b3bb4d854e2854e8}\index{Beam::BeamHost@{Beam::BeamHost}!m\_browser@{m\_browser}}
\index{m\_browser@{m\_browser}!Beam::BeamHost@{Beam::BeamHost}}
\doxysubsubsection{\texorpdfstring{m\_browser}{m\_browser}}
{\footnotesize\ttfamily \label{class_beam_1_1_beam_host_a29b3d9b64a195b17b3bb4d854e2854e8} 
std::\+shared\+\_\+ptr$<$class \doxymbox{\hyperlink{class_beam_1_1_sidebar}{Sidebar}}$>$ Beam::\+\+Beam\+Host::\+m\+\_\+browser\hspace{0.3cm}{\ttfamily [private]}}

\Hypertarget{class_beam_1_1_beam_host_a38fd61099ad675f9d816366de3ef852e}\index{Beam::BeamHost@{Beam::BeamHost}!m\_configView@{m\_configView}}
\index{m\_configView@{m\_configView}!Beam::BeamHost@{Beam::BeamHost}}
\doxysubsubsection{\texorpdfstring{m\_configView}{m\_configView}}
{\footnotesize\ttfamily \label{class_beam_1_1_beam_host_a38fd61099ad675f9d816366de3ef852e} 
std::\+shared\+\_\+ptr$<$class \doxymbox{\hyperlink{class_beam_1_1_audio_config_view}{Audio\+Config\+View}}$>$ Beam::\+\+Beam\+Host::\+m\+\_\+config\+View\hspace{0.3cm}{\ttfamily [private]}}

\Hypertarget{class_beam_1_1_beam_host_a25a31d24eb87a8254522491ad671b1d4}\index{Beam::BeamHost@{Beam::BeamHost}!m\_glContext@{m\_glContext}}
\index{m\_glContext@{m\_glContext}!Beam::BeamHost@{Beam::BeamHost}}
\doxysubsubsection{\texorpdfstring{m\_glContext}{m\_glContext}}
{\footnotesize\ttfamily \label{class_beam_1_1_beam_host_a25a31d24eb87a8254522491ad671b1d4} 
SDL\+\_\+\+GLContext Beam::\+\+Beam\+Host::\+m\+\_\+gl\+Context\hspace{0.3cm}{\ttfamily [private]}}

\Hypertarget{class_beam_1_1_beam_host_a8a402533c41936ec6b0554a03716fd1a}\index{Beam::BeamHost@{Beam::BeamHost}!m\_height@{m\_height}}
\index{m\_height@{m\_height}!Beam::BeamHost@{Beam::BeamHost}}
\doxysubsubsection{\texorpdfstring{m\_height}{m\_height}}
{\footnotesize\ttfamily \label{class_beam_1_1_beam_host_a8a402533c41936ec6b0554a03716fd1a} 
int Beam::\+\+Beam\+Host::\+m\+\_\+height\hspace{0.3cm}{\ttfamily [private]}}

\Hypertarget{class_beam_1_1_beam_host_ac7abda34255ce3126887a0fe44df7a65}\index{Beam::BeamHost@{Beam::BeamHost}!m\_isRunning@{m\_isRunning}}
\index{m\_isRunning@{m\_isRunning}!Beam::BeamHost@{Beam::BeamHost}}
\doxysubsubsection{\texorpdfstring{m\_isRunning}{m\_isRunning}}
{\footnotesize\ttfamily \label{class_beam_1_1_beam_host_ac7abda34255ce3126887a0fe44df7a65} 
bool Beam::\+\+Beam\+Host::\+m\+\_\+is\+Running\hspace{0.3cm}{\ttfamily [private]}}

\Hypertarget{class_beam_1_1_beam_host_a4c6405471172009a7ee7509e93152ff8}\index{Beam::BeamHost@{Beam::BeamHost}!m\_masterStrip@{m\_masterStrip}}
\index{m\_masterStrip@{m\_masterStrip}!Beam::BeamHost@{Beam::BeamHost}}
\doxysubsubsection{\texorpdfstring{m\_masterStrip}{m\_masterStrip}}
{\footnotesize\ttfamily \label{class_beam_1_1_beam_host_a4c6405471172009a7ee7509e93152ff8} 
std::\+shared\+\_\+ptr$<$class \doxymbox{\hyperlink{class_beam_1_1_master_strip}{Master\+Strip}}$>$ Beam::\+\+Beam\+Host::\+m\+\_\+master\+Strip\hspace{0.3cm}{\ttfamily [private]}}

\Hypertarget{class_beam_1_1_beam_host_a2aaeeb49d79e64d1b5bbe6844b67ee7a}\index{Beam::BeamHost@{Beam::BeamHost}!m\_mode@{m\_mode}}
\index{m\_mode@{m\_mode}!Beam::BeamHost@{Beam::BeamHost}}
\doxysubsubsection{\texorpdfstring{m\_mode}{m\_mode}}
{\footnotesize\ttfamily \label{class_beam_1_1_beam_host_a2aaeeb49d79e64d1b5bbe6844b67ee7a} 
\doxymbox{\hyperlink{namespace_beam_a857abf9a9358b461e5e0b3a25b0c3d88}{DAWMode}} Beam::\+\+Beam\+Host::\+m\+\_\+mode = \doxymbox{\hyperlink{namespace_beam_a857abf9a9358b461e5e0b3a25b0c3d88a1c30b35b12895df175ccd44dbb6f5ace}{DAWMode::\+\+Flux}}\hspace{0.3cm}{\ttfamily [private]}}

\Hypertarget{class_beam_1_1_beam_host_a8c40df6cc8ed1c3fa6e3751526dea8d5}\index{Beam::BeamHost@{Beam::BeamHost}!m\_project@{m\_project}}
\index{m\_project@{m\_project}!Beam::BeamHost@{Beam::BeamHost}}
\doxysubsubsection{\texorpdfstring{m\_project}{m\_project}}
{\footnotesize\ttfamily \label{class_beam_1_1_beam_host_a8c40df6cc8ed1c3fa6e3751526dea8d5} 
std::\+shared\+\_\+ptr$<$\doxymbox{\hyperlink{class_beam_1_1_flux_project}{Flux\+Project}}$>$ Beam::\+\+Beam\+Host::\+m\+\_\+project\hspace{0.3cm}{\ttfamily [private]}}

\Hypertarget{class_beam_1_1_beam_host_ad49d80c9ee8b743f20073756f1805ec3}\index{Beam::BeamHost@{Beam::BeamHost}!m\_timeline@{m\_timeline}}
\index{m\_timeline@{m\_timeline}!Beam::BeamHost@{Beam::BeamHost}}
\doxysubsubsection{\texorpdfstring{m\_timeline}{m\_timeline}}
{\footnotesize\ttfamily \label{class_beam_1_1_beam_host_ad49d80c9ee8b743f20073756f1805ec3} 
std::\+shared\+\_\+ptr$<$class \doxymbox{\hyperlink{class_beam_1_1_timeline}{Timeline}}$>$ Beam::\+\+Beam\+Host::\+m\+\_\+timeline\hspace{0.3cm}{\ttfamily [private]}}

\Hypertarget{class_beam_1_1_beam_host_a96e1bd8f9a31b93da7ad3401b1dc778f}\index{Beam::BeamHost@{Beam::BeamHost}!m\_title@{m\_title}}
\index{m\_title@{m\_title}!Beam::BeamHost@{Beam::BeamHost}}
\doxysubsubsection{\texorpdfstring{m\_title}{m\_title}}
{\footnotesize\ttfamily \label{class_beam_1_1_beam_host_a96e1bd8f9a31b93da7ad3401b1dc778f} 
std::\+string Beam::\+\+Beam\+Host::\+m\+\_\+title\hspace{0.3cm}{\ttfamily [private]}}

\Hypertarget{class_beam_1_1_beam_host_a9d7779bbadd94af671a40877e0118b8b}\index{Beam::BeamHost@{Beam::BeamHost}!m\_topBar@{m\_topBar}}
\index{m\_topBar@{m\_topBar}!Beam::BeamHost@{Beam::BeamHost}}
\doxysubsubsection{\texorpdfstring{m\_topBar}{m\_topBar}}
{\footnotesize\ttfamily \label{class_beam_1_1_beam_host_a9d7779bbadd94af671a40877e0118b8b} 
std::\+shared\+\_\+ptr$<$class \doxymbox{\hyperlink{class_beam_1_1_top_bar}{Top\+Bar}}$>$ Beam::\+\+Beam\+Host::\+m\+\_\+top\+Bar\hspace{0.3cm}{\ttfamily [private]}}

\Hypertarget{class_beam_1_1_beam_host_a5bd4a6f62b57908f5607e0fe7ef8166d}\index{Beam::BeamHost@{Beam::BeamHost}!m\_uiHandler@{m\_uiHandler}}
\index{m\_uiHandler@{m\_uiHandler}!Beam::BeamHost@{Beam::BeamHost}}
\doxysubsubsection{\texorpdfstring{m\_uiHandler}{m\_uiHandler}}
{\footnotesize\ttfamily \label{class_beam_1_1_beam_host_a5bd4a6f62b57908f5607e0fe7ef8166d} 
std::\+unique\+\_\+ptr$<$\doxymbox{\hyperlink{class_beam_1_1_input_handler}{Input\+Handler}}$>$ Beam::\+\+Beam\+Host::\+m\+\_\+ui\+Handler\hspace{0.3cm}{\ttfamily [private]}}

\Hypertarget{class_beam_1_1_beam_host_aeed02c8e1d960b2b0dbdee48c62c671a}\index{Beam::BeamHost@{Beam::BeamHost}!m\_uiShader@{m\_uiShader}}
\index{m\_uiShader@{m\_uiShader}!Beam::BeamHost@{Beam::BeamHost}}
\doxysubsubsection{\texorpdfstring{m\_uiShader}{m\_uiShader}}
{\footnotesize\ttfamily \label{class_beam_1_1_beam_host_aeed02c8e1d960b2b0dbdee48c62c671a} 
std::\+unique\+\_\+ptr$<$\doxymbox{\hyperlink{class_beam_1_1_shader}{Shader}}$>$ Beam::\+\+Beam\+Host::\+m\+\_\+ui\+Shader\hspace{0.3cm}{\ttfamily [private]}}

\Hypertarget{class_beam_1_1_beam_host_a8588100db6dadbf9f3d3bb5faff93a61}\index{Beam::BeamHost@{Beam::BeamHost}!m\_width@{m\_width}}
\index{m\_width@{m\_width}!Beam::BeamHost@{Beam::BeamHost}}
\doxysubsubsection{\texorpdfstring{m\_width}{m\_width}}
{\footnotesize\ttfamily \label{class_beam_1_1_beam_host_a8588100db6dadbf9f3d3bb5faff93a61} 
int Beam::\+\+Beam\+Host::\+m\+\_\+width\hspace{0.3cm}{\ttfamily [private]}}

\Hypertarget{class_beam_1_1_beam_host_afea4f3dffefd6d2e09e30d3e0744d82e}\index{Beam::BeamHost@{Beam::BeamHost}!m\_window@{m\_window}}
\index{m\_window@{m\_window}!Beam::BeamHost@{Beam::BeamHost}}
\doxysubsubsection{\texorpdfstring{m\_window}{m\_window}}
{\footnotesize\ttfamily \label{class_beam_1_1_beam_host_afea4f3dffefd6d2e09e30d3e0744d82e} 
SDL\+\_\+\+Window\texorpdfstring{$\ast$}{*} Beam::\+\+Beam\+Host::\+m\+\_\+window\hspace{0.3cm}{\ttfamily [private]}}

\Hypertarget{class_beam_1_1_beam_host_acb133910e5fa3a9f78e4b9f13625d0c9}\index{Beam::BeamHost@{Beam::BeamHost}!m\_workspace@{m\_workspace}}
\index{m\_workspace@{m\_workspace}!Beam::BeamHost@{Beam::BeamHost}}
\doxysubsubsection{\texorpdfstring{m\_workspace}{m\_workspace}}
{\footnotesize\ttfamily \label{class_beam_1_1_beam_host_acb133910e5fa3a9f78e4b9f13625d0c9} 
std::\+shared\+\_\+ptr$<$class \doxymbox{\hyperlink{class_beam_1_1_workspace}{Workspace}}$>$ Beam::\+\+Beam\+Host::\+m\+\_\+workspace\hspace{0.3cm}{\ttfamily [private]}}



The documentation for this class was generated from the following files:\+\begin{DoxyCompactItemize}
\item 
src/\+session/\+\doxymbox{\hyperlink{beam__host_8hpp}{beam\+\_\+host.\+hpp}}\item 
src/\+session/\+\doxymbox{\hyperlink{beam__host_8cpp}{beam\+\_\+host.\+cpp}}\end{DoxyCompactItemize}

\doxysection{Beam::\+Audio\+Engine Class Reference}
\hypertarget{class_beam_1_1_audio_engine}{}\label{class_beam_1_1_audio_engine}\index{Beam::AudioEngine@{Beam::AudioEngine}}


{\ttfamily \+\#include $<$audio\+\_\+engine.\+hpp$>$}

\doxysubsubsection*{Public Member Functions}
\begin{DoxyCompactItemize}
\item 
\doxymbox{\hyperlink{class_beam_1_1_audio_engine_a984cb2c1c5cae9f6a44be879582791e8}{Audio\+Engine}} ()
\item 
\doxymbox{\hyperlink{class_beam_1_1_audio_engine_ae6cd4742fe2edcfaea839dc12e0cd2c7}{\texorpdfstring{$\sim$}{\string~}\+Audio\+Engine}} ()
\item 
bool \doxymbox{\hyperlink{class_beam_1_1_audio_engine_a0c408623b61971329e72ad37f6bf903c}{init}} (int sample\+Rate, int channels)
\item 
void \doxymbox{\hyperlink{class_beam_1_1_audio_engine_a40648762866403dc71e99b9d63fb6947}{process}} (float \texorpdfstring{$\ast$}{*}output, int frames)
\item 
void \doxymbox{\hyperlink{class_beam_1_1_audio_engine_aedbcbbb422a975217cfafc22cfd17b0c}{set\+Graph}} (std::\+shared\+\_\+ptr$<$ \doxymbox{\hyperlink{class_beam_1_1_flux_graph}{Flux\+Graph}} $>$ graph)
\item 
std::\+shared\+\_\+ptr$<$ \doxymbox{\hyperlink{class_beam_1_1_flux_graph}{Flux\+Graph}} $>$ \doxymbox{\hyperlink{class_beam_1_1_audio_engine_a96a0cdd7ef2b6d89fbae82122caa849d}{get\+Graph}} ()
\item 
std::\+shared\+\_\+ptr$<$ \doxymbox{\hyperlink{class_beam_1_1_master_node}{Master\+Node}} $>$ \doxymbox{\hyperlink{class_beam_1_1_audio_engine_af89b35f3399cb0780e8a7d7755e13888}{get\+Master\+Node}} ()
\item 
void \doxymbox{\hyperlink{class_beam_1_1_audio_engine_a7b40e0830210096ecb44349c5e27a18c}{set\+Playing}} (bool playing)
\item 
bool \doxymbox{\hyperlink{class_beam_1_1_audio_engine_a456e847f34fbc934900413639850f924}{is\+Playing}} () const
\item 
void \doxymbox{\hyperlink{class_beam_1_1_audio_engine_a34e5112ab4743e510754758353a66a9f}{rewind}} ()
\end{DoxyCompactItemize}
\doxysubsubsection*{Private Attributes}
\begin{DoxyCompactItemize}
\item 
int \doxymbox{\hyperlink{class_beam_1_1_audio_engine_abeb2228b0d0a09fac76834dda57ad5ab}{m\+\_\+sample\+Rate}}
\item 
int \doxymbox{\hyperlink{class_beam_1_1_audio_engine_a28eb68f5efd3038d4da2a65da3bb5181}{m\+\_\+channels}}
\item 
std::\+shared\+\_\+ptr$<$ \doxymbox{\hyperlink{class_beam_1_1_flux_graph}{Flux\+Graph}} $>$ \doxymbox{\hyperlink{class_beam_1_1_audio_engine_ac35c02a04d9c946a264c735485bfa611}{m\+\_\+graph}}
\item 
size\+\_\+t \doxymbox{\hyperlink{class_beam_1_1_audio_engine_a86ed2fef95c106eaa6631713105a3e4a}{m\+\_\+master\+Node\+Id}}
\item 
std::\+shared\+\_\+ptr$<$ \doxymbox{\hyperlink{class_beam_1_1_master_node}{Master\+Node}} $>$ \doxymbox{\hyperlink{class_beam_1_1_audio_engine_a8831b1803efa6ae4ae6a060b3de0e30a}{m\+\_\+master\+Node}}
\item 
std::\+mutex \doxymbox{\hyperlink{class_beam_1_1_audio_engine_a5ea52e9dc365eb2869a1d4b417f6bc18}{m\+\_\+engine\+Mutex}}
\item 
SDL\+\_\+\+Audio\+Stream \texorpdfstring{$\ast$}{*} \doxymbox{\hyperlink{class_beam_1_1_audio_engine_a1792387f3723d0e1683aba783aef5fe4}{m\+\_\+stream}}
\item 
std::\+atomic$<$ bool $>$ \doxymbox{\hyperlink{class_beam_1_1_audio_engine_a5ace65ddbdaf0088334c2f48acd17655}{m\+\_\+is\+Playing}} \{false\}
\end{DoxyCompactItemize}


\label{doc-constructors}
\Hypertarget{class_beam_1_1_audio_engine_doc-constructors}
\doxysubsection{Constructor \& Destructor Documentation}
\Hypertarget{class_beam_1_1_audio_engine_a984cb2c1c5cae9f6a44be879582791e8}\index{Beam::AudioEngine@{Beam::AudioEngine}!AudioEngine@{AudioEngine}}
\index{AudioEngine@{AudioEngine}!Beam::AudioEngine@{Beam::AudioEngine}}
\doxysubsubsection{\texorpdfstring{AudioEngine()}{AudioEngine()}}
{\footnotesize\ttfamily \label{class_beam_1_1_audio_engine_a984cb2c1c5cae9f6a44be879582791e8} 
Beam::\+\+Audio\+Engine::\+\+Audio\+Engine (\begin{DoxyParamCaption}{}{}\end{DoxyParamCaption})}

\Hypertarget{class_beam_1_1_audio_engine_ae6cd4742fe2edcfaea839dc12e0cd2c7}\index{Beam::AudioEngine@{Beam::AudioEngine}!````~AudioEngine@{\texorpdfstring{$\sim$}{\string~}AudioEngine}}
\index{````~AudioEngine@{\texorpdfstring{$\sim$}{\string~}AudioEngine}!Beam::AudioEngine@{Beam::AudioEngine}}
\doxysubsubsection{\texorpdfstring{\texorpdfstring{$\sim$}{\string~}AudioEngine()}{\string~AudioEngine()}}
{\footnotesize\ttfamily \label{class_beam_1_1_audio_engine_ae6cd4742fe2edcfaea839dc12e0cd2c7} 
Beam::\+\+Audio\+Engine::\+\texorpdfstring{$\sim$}{\string~}\+Audio\+Engine (\begin{DoxyParamCaption}{}{}\end{DoxyParamCaption})}



\label{doc-func-members}
\Hypertarget{class_beam_1_1_audio_engine_doc-func-members}
\doxysubsection{Member Function Documentation}
\Hypertarget{class_beam_1_1_audio_engine_a96a0cdd7ef2b6d89fbae82122caa849d}\index{Beam::AudioEngine@{Beam::AudioEngine}!getGraph@{getGraph}}
\index{getGraph@{getGraph}!Beam::AudioEngine@{Beam::AudioEngine}}
\doxysubsubsection{\texorpdfstring{getGraph()}{getGraph()}}
{\footnotesize\ttfamily \label{class_beam_1_1_audio_engine_a96a0cdd7ef2b6d89fbae82122caa849d} 
std::\+shared\+\_\+ptr$<$ \doxymbox{\hyperlink{class_beam_1_1_flux_graph}{Flux\+Graph}} $>$ Beam::\+\+Audio\+Engine::\+get\+Graph (\begin{DoxyParamCaption}{}{}\end{DoxyParamCaption})\hspace{0.3cm}{\ttfamily [inline]}}

\Hypertarget{class_beam_1_1_audio_engine_af89b35f3399cb0780e8a7d7755e13888}\index{Beam::AudioEngine@{Beam::AudioEngine}!getMasterNode@{getMasterNode}}
\index{getMasterNode@{getMasterNode}!Beam::AudioEngine@{Beam::AudioEngine}}
\doxysubsubsection{\texorpdfstring{getMasterNode()}{getMasterNode()}}
{\footnotesize\ttfamily \label{class_beam_1_1_audio_engine_af89b35f3399cb0780e8a7d7755e13888} 
std::\+shared\+\_\+ptr$<$ \doxymbox{\hyperlink{class_beam_1_1_master_node}{Master\+Node}} $>$ Beam::\+\+Audio\+Engine::\+get\+Master\+Node (\begin{DoxyParamCaption}{}{}\end{DoxyParamCaption})\hspace{0.3cm}{\ttfamily [inline]}}

\Hypertarget{class_beam_1_1_audio_engine_a0c408623b61971329e72ad37f6bf903c}\index{Beam::AudioEngine@{Beam::AudioEngine}!init@{init}}
\index{init@{init}!Beam::AudioEngine@{Beam::AudioEngine}}
\doxysubsubsection{\texorpdfstring{init()}{init()}}
{\footnotesize\ttfamily \label{class_beam_1_1_audio_engine_a0c408623b61971329e72ad37f6bf903c} 
bool Beam::\+\+Audio\+Engine::\+init (\begin{DoxyParamCaption}\item[{int}]{sample\+Rate}{, }\item[{int}]{channels}{}\end{DoxyParamCaption})}

\Hypertarget{class_beam_1_1_audio_engine_a456e847f34fbc934900413639850f924}\index{Beam::AudioEngine@{Beam::AudioEngine}!isPlaying@{isPlaying}}
\index{isPlaying@{isPlaying}!Beam::AudioEngine@{Beam::AudioEngine}}
\doxysubsubsection{\texorpdfstring{isPlaying()}{isPlaying()}}
{\footnotesize\ttfamily \label{class_beam_1_1_audio_engine_a456e847f34fbc934900413639850f924} 
bool Beam::\+\+Audio\+Engine::\+is\+Playing (\begin{DoxyParamCaption}{}{}\end{DoxyParamCaption}) const\hspace{0.3cm}{\ttfamily [inline]}}

\Hypertarget{class_beam_1_1_audio_engine_a40648762866403dc71e99b9d63fb6947}\index{Beam::AudioEngine@{Beam::AudioEngine}!process@{process}}
\index{process@{process}!Beam::AudioEngine@{Beam::AudioEngine}}
\doxysubsubsection{\texorpdfstring{process()}{process()}}
{\footnotesize\ttfamily \label{class_beam_1_1_audio_engine_a40648762866403dc71e99b9d63fb6947} 
void Beam::\+\+Audio\+Engine::\+process (\begin{DoxyParamCaption}\item[{float \texorpdfstring{$\ast$}{*}}]{output}{, }\item[{int}]{frames}{}\end{DoxyParamCaption})}

\Hypertarget{class_beam_1_1_audio_engine_a34e5112ab4743e510754758353a66a9f}\index{Beam::AudioEngine@{Beam::AudioEngine}!rewind@{rewind}}
\index{rewind@{rewind}!Beam::AudioEngine@{Beam::AudioEngine}}
\doxysubsubsection{\texorpdfstring{rewind()}{rewind()}}
{\footnotesize\ttfamily \label{class_beam_1_1_audio_engine_a34e5112ab4743e510754758353a66a9f} 
void Beam::\+\+Audio\+Engine::\+rewind (\begin{DoxyParamCaption}{}{}\end{DoxyParamCaption})}

\Hypertarget{class_beam_1_1_audio_engine_aedbcbbb422a975217cfafc22cfd17b0c}\index{Beam::AudioEngine@{Beam::AudioEngine}!setGraph@{setGraph}}
\index{setGraph@{setGraph}!Beam::AudioEngine@{Beam::AudioEngine}}
\doxysubsubsection{\texorpdfstring{setGraph()}{setGraph()}}
{\footnotesize\ttfamily \label{class_beam_1_1_audio_engine_aedbcbbb422a975217cfafc22cfd17b0c} 
void Beam::\+\+Audio\+Engine::\+set\+Graph (\begin{DoxyParamCaption}\item[{std::\+shared\+\_\+ptr$<$ \doxymbox{\hyperlink{class_beam_1_1_flux_graph}{Flux\+Graph}} $>$}]{graph}{}\end{DoxyParamCaption})}

\Hypertarget{class_beam_1_1_audio_engine_a7b40e0830210096ecb44349c5e27a18c}\index{Beam::AudioEngine@{Beam::AudioEngine}!setPlaying@{setPlaying}}
\index{setPlaying@{setPlaying}!Beam::AudioEngine@{Beam::AudioEngine}}
\doxysubsubsection{\texorpdfstring{setPlaying()}{setPlaying()}}
{\footnotesize\ttfamily \label{class_beam_1_1_audio_engine_a7b40e0830210096ecb44349c5e27a18c} 
void Beam::\+\+Audio\+Engine::\+set\+Playing (\begin{DoxyParamCaption}\item[{bool}]{playing}{}\end{DoxyParamCaption})\hspace{0.3cm}{\ttfamily [inline]}}



\label{doc-variable-members}
\Hypertarget{class_beam_1_1_audio_engine_doc-variable-members}
\doxysubsection{Member Data Documentation}
\Hypertarget{class_beam_1_1_audio_engine_a28eb68f5efd3038d4da2a65da3bb5181}\index{Beam::AudioEngine@{Beam::AudioEngine}!m\_channels@{m\_channels}}
\index{m\_channels@{m\_channels}!Beam::AudioEngine@{Beam::AudioEngine}}
\doxysubsubsection{\texorpdfstring{m\_channels}{m\_channels}}
{\footnotesize\ttfamily \label{class_beam_1_1_audio_engine_a28eb68f5efd3038d4da2a65da3bb5181} 
int Beam::\+\+Audio\+Engine::\+m\+\_\+channels\hspace{0.3cm}{\ttfamily [private]}}

\Hypertarget{class_beam_1_1_audio_engine_a5ea52e9dc365eb2869a1d4b417f6bc18}\index{Beam::AudioEngine@{Beam::AudioEngine}!m\_engineMutex@{m\_engineMutex}}
\index{m\_engineMutex@{m\_engineMutex}!Beam::AudioEngine@{Beam::AudioEngine}}
\doxysubsubsection{\texorpdfstring{m\_engineMutex}{m\_engineMutex}}
{\footnotesize\ttfamily \label{class_beam_1_1_audio_engine_a5ea52e9dc365eb2869a1d4b417f6bc18} 
std::\+mutex Beam::\+\+Audio\+Engine::\+m\+\_\+engine\+Mutex\hspace{0.3cm}{\ttfamily [private]}}

\Hypertarget{class_beam_1_1_audio_engine_ac35c02a04d9c946a264c735485bfa611}\index{Beam::AudioEngine@{Beam::AudioEngine}!m\_graph@{m\_graph}}
\index{m\_graph@{m\_graph}!Beam::AudioEngine@{Beam::AudioEngine}}
\doxysubsubsection{\texorpdfstring{m\_graph}{m\_graph}}
{\footnotesize\ttfamily \label{class_beam_1_1_audio_engine_ac35c02a04d9c946a264c735485bfa611} 
std::\+shared\+\_\+ptr$<$\doxymbox{\hyperlink{class_beam_1_1_flux_graph}{Flux\+Graph}}$>$ Beam::\+\+Audio\+Engine::\+m\+\_\+graph\hspace{0.3cm}{\ttfamily [private]}}

\Hypertarget{class_beam_1_1_audio_engine_a5ace65ddbdaf0088334c2f48acd17655}\index{Beam::AudioEngine@{Beam::AudioEngine}!m\_isPlaying@{m\_isPlaying}}
\index{m\_isPlaying@{m\_isPlaying}!Beam::AudioEngine@{Beam::AudioEngine}}
\doxysubsubsection{\texorpdfstring{m\_isPlaying}{m\_isPlaying}}
{\footnotesize\ttfamily \label{class_beam_1_1_audio_engine_a5ace65ddbdaf0088334c2f48acd17655} 
std::\+atomic$<$bool$>$ Beam::\+\+Audio\+Engine::\+m\+\_\+is\+Playing \{false\}\hspace{0.3cm}{\ttfamily [private]}}

\Hypertarget{class_beam_1_1_audio_engine_a8831b1803efa6ae4ae6a060b3de0e30a}\index{Beam::AudioEngine@{Beam::AudioEngine}!m\_masterNode@{m\_masterNode}}
\index{m\_masterNode@{m\_masterNode}!Beam::AudioEngine@{Beam::AudioEngine}}
\doxysubsubsection{\texorpdfstring{m\_masterNode}{m\_masterNode}}
{\footnotesize\ttfamily \label{class_beam_1_1_audio_engine_a8831b1803efa6ae4ae6a060b3de0e30a} 
std::\+shared\+\_\+ptr$<$\doxymbox{\hyperlink{class_beam_1_1_master_node}{Master\+Node}}$>$ Beam::\+\+Audio\+Engine::\+m\+\_\+master\+Node\hspace{0.3cm}{\ttfamily [private]}}

\Hypertarget{class_beam_1_1_audio_engine_a86ed2fef95c106eaa6631713105a3e4a}\index{Beam::AudioEngine@{Beam::AudioEngine}!m\_masterNodeId@{m\_masterNodeId}}
\index{m\_masterNodeId@{m\_masterNodeId}!Beam::AudioEngine@{Beam::AudioEngine}}
\doxysubsubsection{\texorpdfstring{m\_masterNodeId}{m\_masterNodeId}}
{\footnotesize\ttfamily \label{class_beam_1_1_audio_engine_a86ed2fef95c106eaa6631713105a3e4a} 
size\+\_\+t Beam::\+\+Audio\+Engine::\+m\+\_\+master\+Node\+Id\hspace{0.3cm}{\ttfamily [private]}}

\Hypertarget{class_beam_1_1_audio_engine_abeb2228b0d0a09fac76834dda57ad5ab}\index{Beam::AudioEngine@{Beam::AudioEngine}!m\_sampleRate@{m\_sampleRate}}
\index{m\_sampleRate@{m\_sampleRate}!Beam::AudioEngine@{Beam::AudioEngine}}
\doxysubsubsection{\texorpdfstring{m\_sampleRate}{m\_sampleRate}}
{\footnotesize\ttfamily \label{class_beam_1_1_audio_engine_abeb2228b0d0a09fac76834dda57ad5ab} 
int Beam::\+\+Audio\+Engine::\+m\+\_\+sample\+Rate\hspace{0.3cm}{\ttfamily [private]}}

\Hypertarget{class_beam_1_1_audio_engine_a1792387f3723d0e1683aba783aef5fe4}\index{Beam::AudioEngine@{Beam::AudioEngine}!m\_stream@{m\_stream}}
\index{m\_stream@{m\_stream}!Beam::AudioEngine@{Beam::AudioEngine}}
\doxysubsubsection{\texorpdfstring{m\_stream}{m\_stream}}
{\footnotesize\ttfamily \label{class_beam_1_1_audio_engine_a1792387f3723d0e1683aba783aef5fe4} 
SDL\+\_\+\+Audio\+Stream\texorpdfstring{$\ast$}{*} Beam::\+\+Audio\+Engine::\+m\+\_\+stream\hspace{0.3cm}{\ttfamily [private]}}



The documentation for this class was generated from the following files:\+\begin{DoxyCompactItemize}
\item 
src/\+dsp/\+\doxymbox{\hyperlink{audio__engine_8hpp}{audio\+\_\+engine.\+hpp}}\item 
src/\+dsp/\+\doxymbox{\hyperlink{audio__engine_8cpp}{audio\+\_\+engine.\+cpp}}\end{DoxyCompactItemize}

\doxysection{Beam::\+Flux\+Graph Class Reference}
\hypertarget{class_beam_1_1_flux_graph}{}\label{class_beam_1_1_flux_graph}\index{Beam::FluxGraph@{Beam::FluxGraph}}


{\ttfamily \+\#include $<$flux\+\_\+graph.\+hpp$>$}

\doxysubsubsection*{Public Member Functions}
\begin{DoxyCompactItemize}
\item 
size\+\_\+t \doxymbox{\hyperlink{class_beam_1_1_flux_graph_a265eecf57dcd7ba6f58436a91d0624c7}{add\+Node}} (std::\+shared\+\_\+ptr$<$ \doxymbox{\hyperlink{class_beam_1_1_flux_node}{Flux\+Node}} $>$ node)
\item 
void \doxymbox{\hyperlink{class_beam_1_1_flux_graph_ab0a9ab8d7c11ea0a40585c6ffe3961ef}{remove\+Node}} (size\+\_\+t id)
\item 
void \doxymbox{\hyperlink{class_beam_1_1_flux_graph_a601ea426ce872ae69a21211ff0ea7303}{connect}} (size\+\_\+t src\+Node\+Id, int src\+Port, size\+\_\+t dst\+Node\+Id, int dst\+Port)
\item 
void \doxymbox{\hyperlink{class_beam_1_1_flux_graph_a200b5ac788ec3afa6fbce38cadf9a6d7}{disconnect}} (size\+\_\+t src\+Node\+Id, int src\+Port, size\+\_\+t dst\+Node\+Id, int dst\+Port)
\item 
std::\+shared\+\_\+ptr$<$ \doxymbox{\hyperlink{struct_beam_1_1_render_plan}{Render\+Plan}} $>$ \doxymbox{\hyperlink{class_beam_1_1_flux_graph_acbad93d7181338801a24e7093e82459e}{compile}} (int buffer\+Size\+Frames, int channels=2)
\item 
void \doxymbox{\hyperlink{class_beam_1_1_flux_graph_a4ac5823a2afb44ce639d3be71a5bfc98}{set\+Transport\+State}} (bool playing)
\item 
std::\+shared\+\_\+ptr$<$ \doxymbox{\hyperlink{class_beam_1_1_flux_node}{Flux\+Node}} $>$ \doxymbox{\hyperlink{class_beam_1_1_flux_graph_a1dae717bfb05d9167d1cfe79954807b1}{get\+Node}} (size\+\_\+t id)
\item 
const std::\+map$<$ size\+\_\+t, std::\+shared\+\_\+ptr$<$ \doxymbox{\hyperlink{class_beam_1_1_flux_node}{Flux\+Node}} $>$ $>$ \& \doxymbox{\hyperlink{class_beam_1_1_flux_graph_ab450ac7e6e4218b5f72c5ad8ad17577c}{get\+Nodes}} () const
\item 
const std::\+set$<$ \doxymbox{\hyperlink{struct_beam_1_1_flux_connection}{Flux\+Connection}} $>$ \& \doxymbox{\hyperlink{class_beam_1_1_flux_graph_abecdeddaec45c42957802cab02aa0711}{get\+Connections}} () const
\end{DoxyCompactItemize}
\doxysubsubsection*{Private Attributes}
\begin{DoxyCompactItemize}
\item 
std::\+map$<$ size\+\_\+t, std::\+shared\+\_\+ptr$<$ \doxymbox{\hyperlink{class_beam_1_1_flux_node}{Flux\+Node}} $>$ $>$ \doxymbox{\hyperlink{class_beam_1_1_flux_graph_a8bc2b841557ded5a4d18bef70addc9ac}{m\+\_\+nodes}}
\item 
std::\+set$<$ \doxymbox{\hyperlink{struct_beam_1_1_flux_connection}{Flux\+Connection}} $>$ \doxymbox{\hyperlink{class_beam_1_1_flux_graph_a8aa7b00e51cb7a4cc9123f327d2a3831}{m\+\_\+connections}}
\item 
size\+\_\+t \doxymbox{\hyperlink{class_beam_1_1_flux_graph_a02661747af83751ced185ae9b858f626}{m\+\_\+next\+Id}} = 0
\item 
bool \doxymbox{\hyperlink{class_beam_1_1_flux_graph_adc3a3d43995a4f8450e5e40afcc27905}{m\+\_\+needs\+Rebuild}} = true
\item 
std::\+mutex \doxymbox{\hyperlink{class_beam_1_1_flux_graph_a254ba77bf37f2de61b7863f42b84c93c}{m\+\_\+mutex}}
\end{DoxyCompactItemize}


\label{doc-func-members}
\Hypertarget{class_beam_1_1_flux_graph_doc-func-members}
\doxysubsection{Member Function Documentation}
\Hypertarget{class_beam_1_1_flux_graph_a265eecf57dcd7ba6f58436a91d0624c7}\index{Beam::FluxGraph@{Beam::FluxGraph}!addNode@{addNode}}
\index{addNode@{addNode}!Beam::FluxGraph@{Beam::FluxGraph}}
\doxysubsubsection{\texorpdfstring{addNode()}{addNode()}}
{\footnotesize\ttfamily \label{class_beam_1_1_flux_graph_a265eecf57dcd7ba6f58436a91d0624c7} 
size\+\_\+t Beam::\+\+Flux\+Graph::\+add\+Node (\begin{DoxyParamCaption}\item[{std::\+shared\+\_\+ptr$<$ \doxymbox{\hyperlink{class_beam_1_1_flux_node}{Flux\+Node}} $>$}]{node}{}\end{DoxyParamCaption})\hspace{0.3cm}{\ttfamily [inline]}}

\Hypertarget{class_beam_1_1_flux_graph_acbad93d7181338801a24e7093e82459e}\index{Beam::FluxGraph@{Beam::FluxGraph}!compile@{compile}}
\index{compile@{compile}!Beam::FluxGraph@{Beam::FluxGraph}}
\doxysubsubsection{\texorpdfstring{compile()}{compile()}}
{\footnotesize\ttfamily \label{class_beam_1_1_flux_graph_acbad93d7181338801a24e7093e82459e} 
std::\+shared\+\_\+ptr$<$ \doxymbox{\hyperlink{struct_beam_1_1_render_plan}{Render\+Plan}} $>$ Beam::\+\+Flux\+Graph::\+compile (\begin{DoxyParamCaption}\item[{int}]{buffer\+Size\+Frames}{, }\item[{int}]{channels}{ = {\ttfamily 2}}\end{DoxyParamCaption})\hspace{0.3cm}{\ttfamily [inline]}}

\Hypertarget{class_beam_1_1_flux_graph_a601ea426ce872ae69a21211ff0ea7303}\index{Beam::FluxGraph@{Beam::FluxGraph}!connect@{connect}}
\index{connect@{connect}!Beam::FluxGraph@{Beam::FluxGraph}}
\doxysubsubsection{\texorpdfstring{connect()}{connect()}}
{\footnotesize\ttfamily \label{class_beam_1_1_flux_graph_a601ea426ce872ae69a21211ff0ea7303} 
void Beam::\+\+Flux\+Graph::\+connect (\begin{DoxyParamCaption}\item[{size\+\_\+t}]{src\+Node\+Id}{, }\item[{int}]{src\+Port}{, }\item[{size\+\_\+t}]{dst\+Node\+Id}{, }\item[{int}]{dst\+Port}{}\end{DoxyParamCaption})\hspace{0.3cm}{\ttfamily [inline]}}

\Hypertarget{class_beam_1_1_flux_graph_a200b5ac788ec3afa6fbce38cadf9a6d7}\index{Beam::FluxGraph@{Beam::FluxGraph}!disconnect@{disconnect}}
\index{disconnect@{disconnect}!Beam::FluxGraph@{Beam::FluxGraph}}
\doxysubsubsection{\texorpdfstring{disconnect()}{disconnect()}}
{\footnotesize\ttfamily \label{class_beam_1_1_flux_graph_a200b5ac788ec3afa6fbce38cadf9a6d7} 
void Beam::\+\+Flux\+Graph::\+disconnect (\begin{DoxyParamCaption}\item[{size\+\_\+t}]{src\+Node\+Id}{, }\item[{int}]{src\+Port}{, }\item[{size\+\_\+t}]{dst\+Node\+Id}{, }\item[{int}]{dst\+Port}{}\end{DoxyParamCaption})\hspace{0.3cm}{\ttfamily [inline]}}

\Hypertarget{class_beam_1_1_flux_graph_abecdeddaec45c42957802cab02aa0711}\index{Beam::FluxGraph@{Beam::FluxGraph}!getConnections@{getConnections}}
\index{getConnections@{getConnections}!Beam::FluxGraph@{Beam::FluxGraph}}
\doxysubsubsection{\texorpdfstring{getConnections()}{getConnections()}}
{\footnotesize\ttfamily \label{class_beam_1_1_flux_graph_abecdeddaec45c42957802cab02aa0711} 
const std::\+set$<$ \doxymbox{\hyperlink{struct_beam_1_1_flux_connection}{Flux\+Connection}} $>$ \& Beam::\+\+Flux\+Graph::\+get\+Connections (\begin{DoxyParamCaption}{}{}\end{DoxyParamCaption}) const\hspace{0.3cm}{\ttfamily [inline]}}

\Hypertarget{class_beam_1_1_flux_graph_a1dae717bfb05d9167d1cfe79954807b1}\index{Beam::FluxGraph@{Beam::FluxGraph}!getNode@{getNode}}
\index{getNode@{getNode}!Beam::FluxGraph@{Beam::FluxGraph}}
\doxysubsubsection{\texorpdfstring{getNode()}{getNode()}}
{\footnotesize\ttfamily \label{class_beam_1_1_flux_graph_a1dae717bfb05d9167d1cfe79954807b1} 
std::\+shared\+\_\+ptr$<$ \doxymbox{\hyperlink{class_beam_1_1_flux_node}{Flux\+Node}} $>$ Beam::\+\+Flux\+Graph::\+get\+Node (\begin{DoxyParamCaption}\item[{size\+\_\+t}]{id}{}\end{DoxyParamCaption})\hspace{0.3cm}{\ttfamily [inline]}}

\Hypertarget{class_beam_1_1_flux_graph_ab450ac7e6e4218b5f72c5ad8ad17577c}\index{Beam::FluxGraph@{Beam::FluxGraph}!getNodes@{getNodes}}
\index{getNodes@{getNodes}!Beam::FluxGraph@{Beam::FluxGraph}}
\doxysubsubsection{\texorpdfstring{getNodes()}{getNodes()}}
{\footnotesize\ttfamily \label{class_beam_1_1_flux_graph_ab450ac7e6e4218b5f72c5ad8ad17577c} 
const std::\+map$<$ size\+\_\+t, std::\+shared\+\_\+ptr$<$ \doxymbox{\hyperlink{class_beam_1_1_flux_node}{Flux\+Node}} $>$ $>$ \& Beam::\+\+Flux\+Graph::\+get\+Nodes (\begin{DoxyParamCaption}{}{}\end{DoxyParamCaption}) const\hspace{0.3cm}{\ttfamily [inline]}}

\Hypertarget{class_beam_1_1_flux_graph_ab0a9ab8d7c11ea0a40585c6ffe3961ef}\index{Beam::FluxGraph@{Beam::FluxGraph}!removeNode@{removeNode}}
\index{removeNode@{removeNode}!Beam::FluxGraph@{Beam::FluxGraph}}
\doxysubsubsection{\texorpdfstring{removeNode()}{removeNode()}}
{\footnotesize\ttfamily \label{class_beam_1_1_flux_graph_ab0a9ab8d7c11ea0a40585c6ffe3961ef} 
void Beam::\+\+Flux\+Graph::\+remove\+Node (\begin{DoxyParamCaption}\item[{size\+\_\+t}]{id}{}\end{DoxyParamCaption})\hspace{0.3cm}{\ttfamily [inline]}}

\Hypertarget{class_beam_1_1_flux_graph_a4ac5823a2afb44ce639d3be71a5bfc98}\index{Beam::FluxGraph@{Beam::FluxGraph}!setTransportState@{setTransportState}}
\index{setTransportState@{setTransportState}!Beam::FluxGraph@{Beam::FluxGraph}}
\doxysubsubsection{\texorpdfstring{setTransportState()}{setTransportState()}}
{\footnotesize\ttfamily \label{class_beam_1_1_flux_graph_a4ac5823a2afb44ce639d3be71a5bfc98} 
void Beam::\+\+Flux\+Graph::\+set\+Transport\+State (\begin{DoxyParamCaption}\item[{bool}]{playing}{}\end{DoxyParamCaption})\hspace{0.3cm}{\ttfamily [inline]}}



\label{doc-variable-members}
\Hypertarget{class_beam_1_1_flux_graph_doc-variable-members}
\doxysubsection{Member Data Documentation}
\Hypertarget{class_beam_1_1_flux_graph_a8aa7b00e51cb7a4cc9123f327d2a3831}\index{Beam::FluxGraph@{Beam::FluxGraph}!m\_connections@{m\_connections}}
\index{m\_connections@{m\_connections}!Beam::FluxGraph@{Beam::FluxGraph}}
\doxysubsubsection{\texorpdfstring{m\_connections}{m\_connections}}
{\footnotesize\ttfamily \label{class_beam_1_1_flux_graph_a8aa7b00e51cb7a4cc9123f327d2a3831} 
std::\+set$<$\doxymbox{\hyperlink{struct_beam_1_1_flux_connection}{Flux\+Connection}}$>$ Beam::\+\+Flux\+Graph::\+m\+\_\+connections\hspace{0.3cm}{\ttfamily [private]}}

\Hypertarget{class_beam_1_1_flux_graph_a254ba77bf37f2de61b7863f42b84c93c}\index{Beam::FluxGraph@{Beam::FluxGraph}!m\_mutex@{m\_mutex}}
\index{m\_mutex@{m\_mutex}!Beam::FluxGraph@{Beam::FluxGraph}}
\doxysubsubsection{\texorpdfstring{m\_mutex}{m\_mutex}}
{\footnotesize\ttfamily \label{class_beam_1_1_flux_graph_a254ba77bf37f2de61b7863f42b84c93c} 
std::\+mutex Beam::\+\+Flux\+Graph::\+m\+\_\+mutex\hspace{0.3cm}{\ttfamily [private]}}

\Hypertarget{class_beam_1_1_flux_graph_adc3a3d43995a4f8450e5e40afcc27905}\index{Beam::FluxGraph@{Beam::FluxGraph}!m\_needsRebuild@{m\_needsRebuild}}
\index{m\_needsRebuild@{m\_needsRebuild}!Beam::FluxGraph@{Beam::FluxGraph}}
\doxysubsubsection{\texorpdfstring{m\_needsRebuild}{m\_needsRebuild}}
{\footnotesize\ttfamily \label{class_beam_1_1_flux_graph_adc3a3d43995a4f8450e5e40afcc27905} 
bool Beam::\+\+Flux\+Graph::\+m\+\_\+needs\+Rebuild = true\hspace{0.3cm}{\ttfamily [private]}}

\Hypertarget{class_beam_1_1_flux_graph_a02661747af83751ced185ae9b858f626}\index{Beam::FluxGraph@{Beam::FluxGraph}!m\_nextId@{m\_nextId}}
\index{m\_nextId@{m\_nextId}!Beam::FluxGraph@{Beam::FluxGraph}}
\doxysubsubsection{\texorpdfstring{m\_nextId}{m\_nextId}}
{\footnotesize\ttfamily \label{class_beam_1_1_flux_graph_a02661747af83751ced185ae9b858f626} 
size\+\_\+t Beam::\+\+Flux\+Graph::\+m\+\_\+next\+Id = 0\hspace{0.3cm}{\ttfamily [private]}}

\Hypertarget{class_beam_1_1_flux_graph_a8bc2b841557ded5a4d18bef70addc9ac}\index{Beam::FluxGraph@{Beam::FluxGraph}!m\_nodes@{m\_nodes}}
\index{m\_nodes@{m\_nodes}!Beam::FluxGraph@{Beam::FluxGraph}}
\doxysubsubsection{\texorpdfstring{m\_nodes}{m\_nodes}}
{\footnotesize\ttfamily \label{class_beam_1_1_flux_graph_a8bc2b841557ded5a4d18bef70addc9ac} 
std::\+map$<$size\+\_\+t, std::\+shared\+\_\+ptr$<$\doxymbox{\hyperlink{class_beam_1_1_flux_node}{Flux\+Node}}$>$ $>$ Beam::\+\+Flux\+Graph::\+m\+\_\+nodes\hspace{0.3cm}{\ttfamily [private]}}



The documentation for this class was generated from the following file:\+\begin{DoxyCompactItemize}
\item 
src/\+engine/\+\doxymbox{\hyperlink{flux__graph_8hpp}{flux\+\_\+graph.\+hpp}}\end{DoxyCompactItemize}

\doxysection{Beam::\+Flux\+Node Class Reference}
\hypertarget{class_beam_1_1_flux_node}{}\label{class_beam_1_1_flux_node}\index{Beam::FluxNode@{Beam::FluxNode}}


Base class for all audio and MIDI processing nodes in the Flux Graph.  




{\ttfamily \+\#include $<$flux\+\_\+node.\+hpp$>$}

Inheritance diagram for Beam::\+Flux\+Node:\+\begin{figure}[H]
\begin{center}
\leavevmode
\includegraphics[height=7.140255cm]{class_beam_1_1_flux_node}
\end{center}
\end{figure}
\doxysubsubsection*{Classes}
\begin{DoxyCompactItemize}
\item 
struct \doxymbox{\hyperlink{struct_beam_1_1_flux_node_1_1_port}{Port}}
\begin{DoxyCompactList}\small\item\em Describes an audio input or output port. \end{DoxyCompactList}\end{DoxyCompactItemize}
\doxysubsubsection*{Public Member Functions}
\begin{DoxyCompactItemize}
\item 
virtual \doxymbox{\hyperlink{class_beam_1_1_flux_node_a708c135cdb61e8838469998cd8a84e65}{\texorpdfstring{$\sim$}{\string~}\+Flux\+Node}} ()=default
\item 
virtual void \doxymbox{\hyperlink{class_beam_1_1_flux_node_a3c263446753fa7ae5ff6928ee57bcd4d}{process}} (int frames)=0
\begin{DoxyCompactList}\small\item\em Main audio processing method. Must be implemented by subclasses. \end{DoxyCompactList}\item 
virtual void \doxymbox{\hyperlink{class_beam_1_1_flux_node_ae9d1e151eff5166de969f45de06d5596}{process\+MIDI}} (const \doxymbox{\hyperlink{class_beam_1_1_m_i_d_i_buffer}{MIDIBuffer}} \&midi)
\begin{DoxyCompactList}\small\item\em Optional MIDI processing. Called before \doxylink{class_beam_1_1_flux_node_a3c263446753fa7ae5ff6928ee57bcd4d}{process()} in the engine loop. \end{DoxyCompactList}\item 
virtual void \doxymbox{\hyperlink{class_beam_1_1_flux_node_ace8cc49479d8924d44bca5fd4cd955e2}{on\+Transport\+State\+Changed}} (bool playing)
\begin{DoxyCompactList}\small\item\em Responds to global transport changes (Play/\+\+Pause). \end{DoxyCompactList}\item 
virtual void \doxymbox{\hyperlink{class_beam_1_1_flux_node_adc7c4e979bf27de5bfca66815ae97a67}{on\+Transport\+Seek}} (size\+\_\+t frame)
\begin{DoxyCompactList}\small\item\em Responds to timeline seeking. \end{DoxyCompactList}\item 
void \doxymbox{\hyperlink{class_beam_1_1_flux_node_aa579ec06608fd776987bbb089f27fd94}{set\+Current\+Frame}} (size\+\_\+t frame)
\begin{DoxyCompactList}\small\item\em Sets the current playhead position for this block. \end{DoxyCompactList}\item 
virtual std::\+string \doxymbox{\hyperlink{class_beam_1_1_flux_node_ac638d3d9bb1050d658294bc5470abeba}{get\+Name}} () const =0
\item 
virtual std::\+vector$<$ \doxymbox{\hyperlink{struct_beam_1_1_flux_node_1_1_port}{Port}} $>$ \doxymbox{\hyperlink{class_beam_1_1_flux_node_a17eb02187925b52bf8e53fa3ebe3da66}{get\+Input\+Ports}} () const =0
\item 
virtual std::\+vector$<$ \doxymbox{\hyperlink{struct_beam_1_1_flux_node_1_1_port}{Port}} $>$ \doxymbox{\hyperlink{class_beam_1_1_flux_node_a034f59d236afd7901ed84090422e3279}{get\+Output\+Ports}} () const =0
\item 
float \texorpdfstring{$\ast$}{*} \doxymbox{\hyperlink{class_beam_1_1_flux_node_ac90bd1a05b5bed3d68978f532386ed29}{get\+Input\+Buffer}} (int port\+Idx)
\item 
float \texorpdfstring{$\ast$}{*} \doxymbox{\hyperlink{class_beam_1_1_flux_node_abf11cfd4f2346ee0cd46d4345f1ed7d4}{get\+Output\+Buffer}} (int port\+Idx)
\item 
void \doxymbox{\hyperlink{class_beam_1_1_flux_node_af37f8c1b6b825da2ce7e35011d6f8253}{set\+Bypass}} (bool bypass)
\item 
bool \doxymbox{\hyperlink{class_beam_1_1_flux_node_a4bd30f3c8d311afdcd5c0d208e3bbf0f}{is\+Bypassed}} () const
\item 
void \doxymbox{\hyperlink{class_beam_1_1_flux_node_ad53f3fcaa5737f46d88530f40dbfbe32}{add\+Parameter}} (std::\+shared\+\_\+ptr$<$ \doxymbox{\hyperlink{class_beam_1_1_parameter}{Parameter}} $>$ \doxymbox{\hyperlink{texture_8cpp_aaded45152436a99bb4f9bda081df9f69}{param}})
\item 
std::\+shared\+\_\+ptr$<$ \doxymbox{\hyperlink{class_beam_1_1_parameter}{Parameter}} $>$ \doxymbox{\hyperlink{class_beam_1_1_flux_node_a59a32442eec144010741b9f2086c516e}{get\+Parameter}} (const std::\+string \&name)
\item 
const std::\+map$<$ std::\+string, std::\+shared\+\_\+ptr$<$ \doxymbox{\hyperlink{class_beam_1_1_parameter}{Parameter}} $>$ $>$ \& \doxymbox{\hyperlink{class_beam_1_1_flux_node_a6296c79b1ba77aa8b9526ace4a109529}{get\+Parameters}} () const
\end{DoxyCompactItemize}
\doxysubsubsection*{Protected Member Functions}
\begin{DoxyCompactItemize}
\item 
void \doxymbox{\hyperlink{class_beam_1_1_flux_node_ae3bafc1c5a1aa545167256172b3d3688}{setup\+Buffers}} (int num\+Inputs, int num\+Outputs, int buffer\+Size, int channels)
\begin{DoxyCompactList}\small\item\em Pre-\/allocates buffers for inputs and outputs. \end{DoxyCompactList}\end{DoxyCompactItemize}
\doxysubsubsection*{Protected Attributes}
\begin{DoxyCompactItemize}
\item 
std::\+vector$<$ std::\+vector$<$ float $>$ $>$ \doxymbox{\hyperlink{class_beam_1_1_flux_node_a8edab1c9ebd83e73bbfd92af29d6e92c}{m\+\_\+inputs}}
\item 
std::\+vector$<$ std::\+vector$<$ float $>$ $>$ \doxymbox{\hyperlink{class_beam_1_1_flux_node_a496905f0ff42c432eb38e19bd6135383}{m\+\_\+outputs}}
\item 
std::\+map$<$ std::\+string, std::\+shared\+\_\+ptr$<$ \doxymbox{\hyperlink{class_beam_1_1_parameter}{Parameter}} $>$ $>$ \doxymbox{\hyperlink{class_beam_1_1_flux_node_a65628a37cd2dd2832eda60e74ec1aed3}{m\+\_\+parameters}}
\item 
std::\+atomic$<$ bool $>$ \doxymbox{\hyperlink{class_beam_1_1_flux_node_a6116dcdcfa20998fe90dc75a74f25d9b}{m\+\_\+bypassed}} \{false\}
\item 
size\+\_\+t \doxymbox{\hyperlink{class_beam_1_1_flux_node_a7d8556ddb1482f997cda7749d737668b}{m\+\_\+current\+Frame}} = 0
\end{DoxyCompactItemize}


\doxysubsection{Detailed Description}
Base class for all audio and MIDI processing nodes in the Flux Graph. 

\label{doc-constructors}
\Hypertarget{class_beam_1_1_flux_node_doc-constructors}
\doxysubsection{Constructor \& Destructor Documentation}
\Hypertarget{class_beam_1_1_flux_node_a708c135cdb61e8838469998cd8a84e65}\index{Beam::FluxNode@{Beam::FluxNode}!````~FluxNode@{\texorpdfstring{$\sim$}{\string~}FluxNode}}
\index{````~FluxNode@{\texorpdfstring{$\sim$}{\string~}FluxNode}!Beam::FluxNode@{Beam::FluxNode}}
\doxysubsubsection{\texorpdfstring{\texorpdfstring{$\sim$}{\string~}FluxNode()}{\string~FluxNode()}}
{\footnotesize\ttfamily \label{class_beam_1_1_flux_node_a708c135cdb61e8838469998cd8a84e65} 
virtual Beam::\+\+Flux\+Node::\+\texorpdfstring{$\sim$}{\string~}\+Flux\+Node (\begin{DoxyParamCaption}{}{}\end{DoxyParamCaption})\hspace{0.3cm}{\ttfamily [virtual]}, {\ttfamily [default]}}



\label{doc-func-members}
\Hypertarget{class_beam_1_1_flux_node_doc-func-members}
\doxysubsection{Member Function Documentation}
\Hypertarget{class_beam_1_1_flux_node_ad53f3fcaa5737f46d88530f40dbfbe32}\index{Beam::FluxNode@{Beam::FluxNode}!addParameter@{addParameter}}
\index{addParameter@{addParameter}!Beam::FluxNode@{Beam::FluxNode}}
\doxysubsubsection{\texorpdfstring{addParameter()}{addParameter()}}
{\footnotesize\ttfamily \label{class_beam_1_1_flux_node_ad53f3fcaa5737f46d88530f40dbfbe32} 
void Beam::\+\+Flux\+Node::\+add\+Parameter (\begin{DoxyParamCaption}\item[{std::\+shared\+\_\+ptr$<$ \doxymbox{\hyperlink{class_beam_1_1_parameter}{Parameter}} $>$}]{param}{}\end{DoxyParamCaption})\hspace{0.3cm}{\ttfamily [inline]}}

\Hypertarget{class_beam_1_1_flux_node_ac90bd1a05b5bed3d68978f532386ed29}\index{Beam::FluxNode@{Beam::FluxNode}!getInputBuffer@{getInputBuffer}}
\index{getInputBuffer@{getInputBuffer}!Beam::FluxNode@{Beam::FluxNode}}
\doxysubsubsection{\texorpdfstring{getInputBuffer()}{getInputBuffer()}}
{\footnotesize\ttfamily \label{class_beam_1_1_flux_node_ac90bd1a05b5bed3d68978f532386ed29} 
float \texorpdfstring{$\ast$}{*} Beam::\+\+Flux\+Node::\+get\+Input\+Buffer (\begin{DoxyParamCaption}\item[{int}]{port\+Idx}{}\end{DoxyParamCaption})\hspace{0.3cm}{\ttfamily [inline]}}

\Hypertarget{class_beam_1_1_flux_node_a17eb02187925b52bf8e53fa3ebe3da66}\index{Beam::FluxNode@{Beam::FluxNode}!getInputPorts@{getInputPorts}}
\index{getInputPorts@{getInputPorts}!Beam::FluxNode@{Beam::FluxNode}}
\doxysubsubsection{\texorpdfstring{getInputPorts()}{getInputPorts()}}
{\footnotesize\ttfamily \label{class_beam_1_1_flux_node_a17eb02187925b52bf8e53fa3ebe3da66} 
virtual std::\+vector$<$ \doxymbox{\hyperlink{struct_beam_1_1_flux_node_1_1_port}{Port}} $>$ Beam::\+\+Flux\+Node::\+get\+Input\+Ports (\begin{DoxyParamCaption}{}{}\end{DoxyParamCaption}) const\hspace{0.3cm}{\ttfamily [pure virtual]}}



Implemented in \doxymbox{\hyperlink{class_beam_1_1_flux_plugin_ad231db67f900e8e7dd853936ad2e866a}{Beam::\+\+Flux\+Plugin}}, \doxymbox{\hyperlink{class_beam_1_1_flux_track_node_a99caf56c174e00c7814a6820879ccc19}{Beam::\+\+Flux\+Track\+Node}}, \doxymbox{\hyperlink{class_beam_1_1_input_node_a9f91b9b33f0c719360f57e012cbb1250}{Beam::\+\+Input\+Node}}, \doxymbox{\hyperlink{class_beam_1_1_master_node_adb6c1fcb26bf9b50765a3c47c9f3841a}{Beam::\+\+Master\+Node}}, \doxymbox{\hyperlink{class_beam_1_1_simple_gain_processor_a26ab1fae0c646a95dd158babd134af91}{Beam::\+\+Simple\+Gain\+Processor}}, and \doxymbox{\hyperlink{class_beam_1_1_sine_synth_node_a5bdac8c5ac96a5413cc233af8918f29b}{Beam::\+\+Sine\+Synth\+Node}}.

\Hypertarget{class_beam_1_1_flux_node_ac638d3d9bb1050d658294bc5470abeba}\index{Beam::FluxNode@{Beam::FluxNode}!getName@{getName}}
\index{getName@{getName}!Beam::FluxNode@{Beam::FluxNode}}
\doxysubsubsection{\texorpdfstring{getName()}{getName()}}
{\footnotesize\ttfamily \label{class_beam_1_1_flux_node_ac638d3d9bb1050d658294bc5470abeba} 
virtual std::\+string Beam::\+\+Flux\+Node::\+get\+Name (\begin{DoxyParamCaption}{}{}\end{DoxyParamCaption}) const\hspace{0.3cm}{\ttfamily [pure virtual]}}



Implemented in \doxymbox{\hyperlink{class_beam_1_1_flux_plugin_a450563af4d65a25b8a8e896dab77a3c6}{Beam::\+\+Flux\+Plugin}}, \doxymbox{\hyperlink{class_beam_1_1_flux_track_node_aeb9ce1b2d297d6c800bae5e0ed0fd70f}{Beam::\+\+Flux\+Track\+Node}}, \doxymbox{\hyperlink{class_beam_1_1_input_node_aeef98292e43008c4e69525146f6e4050}{Beam::\+\+Input\+Node}}, \doxymbox{\hyperlink{class_beam_1_1_master_node_ab2d3838f891f7bb46980f3eeba9ff905}{Beam::\+\+Master\+Node}}, \doxymbox{\hyperlink{class_beam_1_1_simple_gain_processor_a73bd1595872e04e5abac05d7d17ca336}{Beam::\+\+Simple\+Gain\+Processor}}, and \doxymbox{\hyperlink{class_beam_1_1_sine_synth_node_a745b626198ff61dc6196df3ff08b1de9}{Beam::\+\+Sine\+Synth\+Node}}.

\Hypertarget{class_beam_1_1_flux_node_abf11cfd4f2346ee0cd46d4345f1ed7d4}\index{Beam::FluxNode@{Beam::FluxNode}!getOutputBuffer@{getOutputBuffer}}
\index{getOutputBuffer@{getOutputBuffer}!Beam::FluxNode@{Beam::FluxNode}}
\doxysubsubsection{\texorpdfstring{getOutputBuffer()}{getOutputBuffer()}}
{\footnotesize\ttfamily \label{class_beam_1_1_flux_node_abf11cfd4f2346ee0cd46d4345f1ed7d4} 
float \texorpdfstring{$\ast$}{*} Beam::\+\+Flux\+Node::\+get\+Output\+Buffer (\begin{DoxyParamCaption}\item[{int}]{port\+Idx}{}\end{DoxyParamCaption})\hspace{0.3cm}{\ttfamily [inline]}}

\Hypertarget{class_beam_1_1_flux_node_a034f59d236afd7901ed84090422e3279}\index{Beam::FluxNode@{Beam::FluxNode}!getOutputPorts@{getOutputPorts}}
\index{getOutputPorts@{getOutputPorts}!Beam::FluxNode@{Beam::FluxNode}}
\doxysubsubsection{\texorpdfstring{getOutputPorts()}{getOutputPorts()}}
{\footnotesize\ttfamily \label{class_beam_1_1_flux_node_a034f59d236afd7901ed84090422e3279} 
virtual std::\+vector$<$ \doxymbox{\hyperlink{struct_beam_1_1_flux_node_1_1_port}{Port}} $>$ Beam::\+\+Flux\+Node::\+get\+Output\+Ports (\begin{DoxyParamCaption}{}{}\end{DoxyParamCaption}) const\hspace{0.3cm}{\ttfamily [pure virtual]}}



Implemented in \doxymbox{\hyperlink{class_beam_1_1_flux_plugin_a4ea312de74047e127e818a26e715d8bb}{Beam::\+\+Flux\+Plugin}}, \doxymbox{\hyperlink{class_beam_1_1_flux_track_node_ac77568a2dcca757d7d51486cc27b6a2f}{Beam::\+\+Flux\+Track\+Node}}, \doxymbox{\hyperlink{class_beam_1_1_input_node_a7a11ca9c74a2e08d3200feb375aa57bf}{Beam::\+\+Input\+Node}}, \doxymbox{\hyperlink{class_beam_1_1_master_node_adbe0570947ffbcfcd71e77df7e844797}{Beam::\+\+Master\+Node}}, \doxymbox{\hyperlink{class_beam_1_1_simple_gain_processor_a225eb5f513d1dd5138d020155c0978ed}{Beam::\+\+Simple\+Gain\+Processor}}, and \doxymbox{\hyperlink{class_beam_1_1_sine_synth_node_ac3bde3e31ac67687fd6470b72a0d7b6f}{Beam::\+\+Sine\+Synth\+Node}}.

\Hypertarget{class_beam_1_1_flux_node_a59a32442eec144010741b9f2086c516e}\index{Beam::FluxNode@{Beam::FluxNode}!getParameter@{getParameter}}
\index{getParameter@{getParameter}!Beam::FluxNode@{Beam::FluxNode}}
\doxysubsubsection{\texorpdfstring{getParameter()}{getParameter()}}
{\footnotesize\ttfamily \label{class_beam_1_1_flux_node_a59a32442eec144010741b9f2086c516e} 
std::\+shared\+\_\+ptr$<$ \doxymbox{\hyperlink{class_beam_1_1_parameter}{Parameter}} $>$ Beam::\+\+Flux\+Node::\+get\+Parameter (\begin{DoxyParamCaption}\item[{const std::\+string \&}]{name}{}\end{DoxyParamCaption})\hspace{0.3cm}{\ttfamily [inline]}}

\Hypertarget{class_beam_1_1_flux_node_a6296c79b1ba77aa8b9526ace4a109529}\index{Beam::FluxNode@{Beam::FluxNode}!getParameters@{getParameters}}
\index{getParameters@{getParameters}!Beam::FluxNode@{Beam::FluxNode}}
\doxysubsubsection{\texorpdfstring{getParameters()}{getParameters()}}
{\footnotesize\ttfamily \label{class_beam_1_1_flux_node_a6296c79b1ba77aa8b9526ace4a109529} 
const std::\+map$<$ std::\+string, std::\+shared\+\_\+ptr$<$ \doxymbox{\hyperlink{class_beam_1_1_parameter}{Parameter}} $>$ $>$ \& Beam::\+\+Flux\+Node::\+get\+Parameters (\begin{DoxyParamCaption}{}{}\end{DoxyParamCaption}) const\hspace{0.3cm}{\ttfamily [inline]}}

\Hypertarget{class_beam_1_1_flux_node_a4bd30f3c8d311afdcd5c0d208e3bbf0f}\index{Beam::FluxNode@{Beam::FluxNode}!isBypassed@{isBypassed}}
\index{isBypassed@{isBypassed}!Beam::FluxNode@{Beam::FluxNode}}
\doxysubsubsection{\texorpdfstring{isBypassed()}{isBypassed()}}
{\footnotesize\ttfamily \label{class_beam_1_1_flux_node_a4bd30f3c8d311afdcd5c0d208e3bbf0f} 
bool Beam::\+\+Flux\+Node::\+is\+Bypassed (\begin{DoxyParamCaption}{}{}\end{DoxyParamCaption}) const\hspace{0.3cm}{\ttfamily [inline]}}

\Hypertarget{class_beam_1_1_flux_node_adc7c4e979bf27de5bfca66815ae97a67}\index{Beam::FluxNode@{Beam::FluxNode}!onTransportSeek@{onTransportSeek}}
\index{onTransportSeek@{onTransportSeek}!Beam::FluxNode@{Beam::FluxNode}}
\doxysubsubsection{\texorpdfstring{onTransportSeek()}{onTransportSeek()}}
{\footnotesize\ttfamily \label{class_beam_1_1_flux_node_adc7c4e979bf27de5bfca66815ae97a67} 
virtual void Beam::\+\+Flux\+Node::\+on\+Transport\+Seek (\begin{DoxyParamCaption}\item[{size\+\_\+t}]{frame}{}\end{DoxyParamCaption})\hspace{0.3cm}{\ttfamily [inline]}, {\ttfamily [virtual]}}



Responds to timeline seeking. 



Reimplemented in \doxymbox{\hyperlink{class_beam_1_1_flux_track_node_a372900a79b264e873083189295b7acf3}{Beam::\+\+Flux\+Track\+Node}}.

\Hypertarget{class_beam_1_1_flux_node_ace8cc49479d8924d44bca5fd4cd955e2}\index{Beam::FluxNode@{Beam::FluxNode}!onTransportStateChanged@{onTransportStateChanged}}
\index{onTransportStateChanged@{onTransportStateChanged}!Beam::FluxNode@{Beam::FluxNode}}
\doxysubsubsection{\texorpdfstring{onTransportStateChanged()}{onTransportStateChanged()}}
{\footnotesize\ttfamily \label{class_beam_1_1_flux_node_ace8cc49479d8924d44bca5fd4cd955e2} 
virtual void Beam::\+\+Flux\+Node::\+on\+Transport\+State\+Changed (\begin{DoxyParamCaption}\item[{bool}]{playing}{}\end{DoxyParamCaption})\hspace{0.3cm}{\ttfamily [inline]}, {\ttfamily [virtual]}}



Responds to global transport changes (Play/\+\+Pause). 



Reimplemented in \doxymbox{\hyperlink{class_beam_1_1_flux_track_node_a26450c6c0de5d8f7ec68837bdde18ef3}{Beam::\+\+Flux\+Track\+Node}}.

\Hypertarget{class_beam_1_1_flux_node_a3c263446753fa7ae5ff6928ee57bcd4d}\index{Beam::FluxNode@{Beam::FluxNode}!process@{process}}
\index{process@{process}!Beam::FluxNode@{Beam::FluxNode}}
\doxysubsubsection{\texorpdfstring{process()}{process()}}
{\footnotesize\ttfamily \label{class_beam_1_1_flux_node_a3c263446753fa7ae5ff6928ee57bcd4d} 
virtual void Beam::\+\+Flux\+Node::\+process (\begin{DoxyParamCaption}\item[{int}]{frames}{}\end{DoxyParamCaption})\hspace{0.3cm}{\ttfamily [pure virtual]}}



Main audio processing method. Must be implemented by subclasses. 


\begin{DoxyParams}{Parameters}
{\em frames} & Number of frames to process in the current block. \\
\hline
\end{DoxyParams}


Implemented in \doxymbox{\hyperlink{class_beam_1_1_flux_plugin_a181430e1cbf129891fe3ed72f3905a61}{Beam::\+\+Flux\+Plugin}}, \doxymbox{\hyperlink{class_beam_1_1_flux_track_node_ab29facb24a749834aff2aa3de3be437a}{Beam::\+\+Flux\+Track\+Node}}, \doxymbox{\hyperlink{class_beam_1_1_input_node_a132e1ea5e27d21645ef371a45f377dd3}{Beam::\+\+Input\+Node}}, \doxymbox{\hyperlink{class_beam_1_1_master_node_a4d78fec9962d8b9fd8a34142d0a129f7}{Beam::\+\+Master\+Node}}, \doxymbox{\hyperlink{class_beam_1_1_simple_gain_processor_a29edd229c91eb66476785fe38c983718}{Beam::\+\+Simple\+Gain\+Processor}}, and \doxymbox{\hyperlink{class_beam_1_1_sine_synth_node_aa39837475aad06ad5ae86a8cddfd425f}{Beam::\+\+Sine\+Synth\+Node}}.

\Hypertarget{class_beam_1_1_flux_node_ae9d1e151eff5166de969f45de06d5596}\index{Beam::FluxNode@{Beam::FluxNode}!processMIDI@{processMIDI}}
\index{processMIDI@{processMIDI}!Beam::FluxNode@{Beam::FluxNode}}
\doxysubsubsection{\texorpdfstring{processMIDI()}{processMIDI()}}
{\footnotesize\ttfamily \label{class_beam_1_1_flux_node_ae9d1e151eff5166de969f45de06d5596} 
virtual void Beam::\+\+Flux\+Node::\+process\+MIDI (\begin{DoxyParamCaption}\item[{const \doxymbox{\hyperlink{class_beam_1_1_m_i_d_i_buffer}{MIDIBuffer}} \&}]{midi}{}\end{DoxyParamCaption})\hspace{0.3cm}{\ttfamily [inline]}, {\ttfamily [virtual]}}



Optional MIDI processing. Called before \doxylink{class_beam_1_1_flux_node_a3c263446753fa7ae5ff6928ee57bcd4d}{process()} in the engine loop. 



Reimplemented in \doxymbox{\hyperlink{class_beam_1_1_flux_plugin_aa1f9c569002ec23eeb5db0af686abea7}{Beam::\+\+Flux\+Plugin}}, and \doxymbox{\hyperlink{class_beam_1_1_sine_synth_node_a3440774582a70b64fa5cd67cec5ffceb}{Beam::\+\+Sine\+Synth\+Node}}.

\Hypertarget{class_beam_1_1_flux_node_af37f8c1b6b825da2ce7e35011d6f8253}\index{Beam::FluxNode@{Beam::FluxNode}!setBypass@{setBypass}}
\index{setBypass@{setBypass}!Beam::FluxNode@{Beam::FluxNode}}
\doxysubsubsection{\texorpdfstring{setBypass()}{setBypass()}}
{\footnotesize\ttfamily \label{class_beam_1_1_flux_node_af37f8c1b6b825da2ce7e35011d6f8253} 
void Beam::\+\+Flux\+Node::\+set\+Bypass (\begin{DoxyParamCaption}\item[{bool}]{bypass}{}\end{DoxyParamCaption})\hspace{0.3cm}{\ttfamily [inline]}}

\Hypertarget{class_beam_1_1_flux_node_aa579ec06608fd776987bbb089f27fd94}\index{Beam::FluxNode@{Beam::FluxNode}!setCurrentFrame@{setCurrentFrame}}
\index{setCurrentFrame@{setCurrentFrame}!Beam::FluxNode@{Beam::FluxNode}}
\doxysubsubsection{\texorpdfstring{setCurrentFrame()}{setCurrentFrame()}}
{\footnotesize\ttfamily \label{class_beam_1_1_flux_node_aa579ec06608fd776987bbb089f27fd94} 
void Beam::\+\+Flux\+Node::\+set\+Current\+Frame (\begin{DoxyParamCaption}\item[{size\+\_\+t}]{frame}{}\end{DoxyParamCaption})\hspace{0.3cm}{\ttfamily [inline]}}



Sets the current playhead position for this block. 

\Hypertarget{class_beam_1_1_flux_node_ae3bafc1c5a1aa545167256172b3d3688}\index{Beam::FluxNode@{Beam::FluxNode}!setupBuffers@{setupBuffers}}
\index{setupBuffers@{setupBuffers}!Beam::FluxNode@{Beam::FluxNode}}
\doxysubsubsection{\texorpdfstring{setupBuffers()}{setupBuffers()}}
{\footnotesize\ttfamily \label{class_beam_1_1_flux_node_ae3bafc1c5a1aa545167256172b3d3688} 
void Beam::\+\+Flux\+Node::\+setup\+Buffers (\begin{DoxyParamCaption}\item[{int}]{num\+Inputs}{, }\item[{int}]{num\+Outputs}{, }\item[{int}]{buffer\+Size}{, }\item[{int}]{channels}{}\end{DoxyParamCaption})\hspace{0.3cm}{\ttfamily [inline]}, {\ttfamily [protected]}}



Pre-\/allocates buffers for inputs and outputs. 



\label{doc-variable-members}
\Hypertarget{class_beam_1_1_flux_node_doc-variable-members}
\doxysubsection{Member Data Documentation}
\Hypertarget{class_beam_1_1_flux_node_a6116dcdcfa20998fe90dc75a74f25d9b}\index{Beam::FluxNode@{Beam::FluxNode}!m\_bypassed@{m\_bypassed}}
\index{m\_bypassed@{m\_bypassed}!Beam::FluxNode@{Beam::FluxNode}}
\doxysubsubsection{\texorpdfstring{m\_bypassed}{m\_bypassed}}
{\footnotesize\ttfamily \label{class_beam_1_1_flux_node_a6116dcdcfa20998fe90dc75a74f25d9b} 
std::\+atomic$<$bool$>$ Beam::\+\+Flux\+Node::\+m\+\_\+bypassed \{false\}\hspace{0.3cm}{\ttfamily [protected]}}

\Hypertarget{class_beam_1_1_flux_node_a7d8556ddb1482f997cda7749d737668b}\index{Beam::FluxNode@{Beam::FluxNode}!m\_currentFrame@{m\_currentFrame}}
\index{m\_currentFrame@{m\_currentFrame}!Beam::FluxNode@{Beam::FluxNode}}
\doxysubsubsection{\texorpdfstring{m\_currentFrame}{m\_currentFrame}}
{\footnotesize\ttfamily \label{class_beam_1_1_flux_node_a7d8556ddb1482f997cda7749d737668b} 
size\+\_\+t Beam::\+\+Flux\+Node::\+m\+\_\+current\+Frame = 0\hspace{0.3cm}{\ttfamily [protected]}}

\Hypertarget{class_beam_1_1_flux_node_a8edab1c9ebd83e73bbfd92af29d6e92c}\index{Beam::FluxNode@{Beam::FluxNode}!m\_inputs@{m\_inputs}}
\index{m\_inputs@{m\_inputs}!Beam::FluxNode@{Beam::FluxNode}}
\doxysubsubsection{\texorpdfstring{m\_inputs}{m\_inputs}}
{\footnotesize\ttfamily \label{class_beam_1_1_flux_node_a8edab1c9ebd83e73bbfd92af29d6e92c} 
std::\+vector$<$std::\+vector$<$float$>$ $>$ Beam::\+\+Flux\+Node::\+m\+\_\+inputs\hspace{0.3cm}{\ttfamily [protected]}}

\Hypertarget{class_beam_1_1_flux_node_a496905f0ff42c432eb38e19bd6135383}\index{Beam::FluxNode@{Beam::FluxNode}!m\_outputs@{m\_outputs}}
\index{m\_outputs@{m\_outputs}!Beam::FluxNode@{Beam::FluxNode}}
\doxysubsubsection{\texorpdfstring{m\_outputs}{m\_outputs}}
{\footnotesize\ttfamily \label{class_beam_1_1_flux_node_a496905f0ff42c432eb38e19bd6135383} 
std::\+vector$<$std::\+vector$<$float$>$ $>$ Beam::\+\+Flux\+Node::\+m\+\_\+outputs\hspace{0.3cm}{\ttfamily [protected]}}

\Hypertarget{class_beam_1_1_flux_node_a65628a37cd2dd2832eda60e74ec1aed3}\index{Beam::FluxNode@{Beam::FluxNode}!m\_parameters@{m\_parameters}}
\index{m\_parameters@{m\_parameters}!Beam::FluxNode@{Beam::FluxNode}}
\doxysubsubsection{\texorpdfstring{m\_parameters}{m\_parameters}}
{\footnotesize\ttfamily \label{class_beam_1_1_flux_node_a65628a37cd2dd2832eda60e74ec1aed3} 
std::\+map$<$std::\+string, std::\+shared\+\_\+ptr$<$\doxymbox{\hyperlink{class_beam_1_1_parameter}{Parameter}}$>$ $>$ Beam::\+\+Flux\+Node::\+m\+\_\+parameters\hspace{0.3cm}{\ttfamily [protected]}}



The documentation for this class was generated from the following file:\+\begin{DoxyCompactItemize}
\item 
src/\+engine/\+\doxymbox{\hyperlink{flux__node_8hpp}{flux\+\_\+node.\+hpp}}\end{DoxyCompactItemize}

\doxysection{Beam::\+Quad\+Batcher Class Reference}
\hypertarget{class_beam_1_1_quad_batcher}{}\label{class_beam_1_1_quad_batcher}\index{Beam::QuadBatcher@{Beam::QuadBatcher}}


{\ttfamily \+\#include $<$quad\+\_\+batcher.\+hpp$>$}

\doxysubsubsection*{Public Member Functions}
\begin{DoxyCompactItemize}
\item 
\doxymbox{\hyperlink{class_beam_1_1_quad_batcher_acb37c49599337128ac40efe6832cb714}{Quad\+Batcher}} (size\+\_\+t max\+Quads=1000)
\item 
\doxymbox{\hyperlink{class_beam_1_1_quad_batcher_ad2db51466ed3ace6c5d3876e04a35044}{\texorpdfstring{$\sim$}{\string~}\+Quad\+Batcher}} ()
\item 
void \doxymbox{\hyperlink{class_beam_1_1_quad_batcher_aab5d3d705e6dcd91f650264129380d31}{begin}} ()
\item 
void \doxymbox{\hyperlink{class_beam_1_1_quad_batcher_afeda7287047f1ee0db8203306f408503}{draw\+Quad}} (float x, float y, float w, float h, float r, float g, float b, float a)
\item 
void \doxymbox{\hyperlink{class_beam_1_1_quad_batcher_a16d867a50f08e83bc2b2242ea7c00acb}{draw\+Rounded\+Rect}} (float x, float y, float w, float h, float radius, float softness, float r, float g, float b, float a)
\item 
void \doxymbox{\hyperlink{class_beam_1_1_quad_batcher_a54b57a1fa94c48b9778c2fe1109e761a}{draw\+Text}} (const std::\+string \&text, float x, float y, float size, float r, float g, float b, float a)
\item 
void \doxymbox{\hyperlink{class_beam_1_1_quad_batcher_ae24a0b2b697e7870bebf358ac4776d81}{draw\+Line}} (float x1, float y1, float x2, float y2, float thickness, float r, float g, float b, float a)
\item 
void \doxymbox{\hyperlink{class_beam_1_1_quad_batcher_ad2f4ad2e5231b76d8b58f33f9afae7e0}{draw\+Rect}} (float x, float y, float w, float h, float thickness, float r, float g, float b, float a)
\item 
void \doxymbox{\hyperlink{class_beam_1_1_quad_batcher_a7d04ae84e0638e4531264ff5e8de383f}{flush}} ()
\item 
void \doxymbox{\hyperlink{class_beam_1_1_quad_batcher_a838b935a69a5041d714b4895c5f737dd}{set\+Shader}} (class \doxymbox{\hyperlink{class_beam_1_1_shader}{Shader}} \texorpdfstring{$\ast$}{*}shader)
\end{DoxyCompactItemize}
\doxysubsubsection*{Private Member Functions}
\begin{DoxyCompactItemize}
\item 
void \doxymbox{\hyperlink{class_beam_1_1_quad_batcher_a3f0b54c6cfd0ba9e85bd6c7c239eb928}{create\+Font\+Texture}} ()
\end{DoxyCompactItemize}
\doxysubsubsection*{Private Attributes}
\begin{DoxyCompactItemize}
\item 
size\+\_\+t \doxymbox{\hyperlink{class_beam_1_1_quad_batcher_a34e92273d14f3b91a1fc3b77ad770d97}{m\+\_\+max\+Quads}}
\item 
size\+\_\+t \doxymbox{\hyperlink{class_beam_1_1_quad_batcher_a39fbaab8381cfd108021c6dff7110370}{m\+\_\+quad\+Count}}
\item 
unsigned int \doxymbox{\hyperlink{class_beam_1_1_quad_batcher_a463cd5ae7fca431940a89ff413e522f1}{m\+\_\+vao}}
\item 
unsigned int \doxymbox{\hyperlink{class_beam_1_1_quad_batcher_a92cad15aca7b10792225fb67cbb7e322}{m\+\_\+vbo}}
\item 
unsigned int \doxymbox{\hyperlink{class_beam_1_1_quad_batcher_a6d49aeb9d17eec78b2f6acdb4615ea32}{m\+\_\+ibo}}
\item 
unsigned int \doxymbox{\hyperlink{class_beam_1_1_quad_batcher_a0b453e3a8d0b7b673a180afbbca46f29}{m\+\_\+font\+Texture}}
\item 
std::\+vector$<$ \doxymbox{\hyperlink{struct_beam_1_1_vertex}{Vertex}} $>$ \doxymbox{\hyperlink{class_beam_1_1_quad_batcher_ac47e3f9be44ce389aaa15a44c9f3bd94}{m\+\_\+vertices}}
\item 
class \doxymbox{\hyperlink{class_beam_1_1_shader}{Shader}} \texorpdfstring{$\ast$}{*} \doxymbox{\hyperlink{class_beam_1_1_quad_batcher_a0bf56fdf73e935879113849de645fb63}{m\+\_\+shader}} = nullptr
\end{DoxyCompactItemize}


\label{doc-constructors}
\Hypertarget{class_beam_1_1_quad_batcher_doc-constructors}
\doxysubsection{Constructor \& Destructor Documentation}
\Hypertarget{class_beam_1_1_quad_batcher_acb37c49599337128ac40efe6832cb714}\index{Beam::QuadBatcher@{Beam::QuadBatcher}!QuadBatcher@{QuadBatcher}}
\index{QuadBatcher@{QuadBatcher}!Beam::QuadBatcher@{Beam::QuadBatcher}}
\doxysubsubsection{\texorpdfstring{QuadBatcher()}{QuadBatcher()}}
{\footnotesize\ttfamily \label{class_beam_1_1_quad_batcher_acb37c49599337128ac40efe6832cb714} 
Beam::\+\+Quad\+Batcher::\+\+Quad\+Batcher (\begin{DoxyParamCaption}\item[{size\+\_\+t}]{max\+Quads}{ = {\ttfamily 1000}}\end{DoxyParamCaption})}

\Hypertarget{class_beam_1_1_quad_batcher_ad2db51466ed3ace6c5d3876e04a35044}\index{Beam::QuadBatcher@{Beam::QuadBatcher}!````~QuadBatcher@{\texorpdfstring{$\sim$}{\string~}QuadBatcher}}
\index{````~QuadBatcher@{\texorpdfstring{$\sim$}{\string~}QuadBatcher}!Beam::QuadBatcher@{Beam::QuadBatcher}}
\doxysubsubsection{\texorpdfstring{\texorpdfstring{$\sim$}{\string~}QuadBatcher()}{\string~QuadBatcher()}}
{\footnotesize\ttfamily \label{class_beam_1_1_quad_batcher_ad2db51466ed3ace6c5d3876e04a35044} 
Beam::\+\+Quad\+Batcher::\+\texorpdfstring{$\sim$}{\string~}\+Quad\+Batcher (\begin{DoxyParamCaption}{}{}\end{DoxyParamCaption})}



\label{doc-func-members}
\Hypertarget{class_beam_1_1_quad_batcher_doc-func-members}
\doxysubsection{Member Function Documentation}
\Hypertarget{class_beam_1_1_quad_batcher_aab5d3d705e6dcd91f650264129380d31}\index{Beam::QuadBatcher@{Beam::QuadBatcher}!begin@{begin}}
\index{begin@{begin}!Beam::QuadBatcher@{Beam::QuadBatcher}}
\doxysubsubsection{\texorpdfstring{begin()}{begin()}}
{\footnotesize\ttfamily \label{class_beam_1_1_quad_batcher_aab5d3d705e6dcd91f650264129380d31} 
void Beam::\+\+Quad\+Batcher::\+begin (\begin{DoxyParamCaption}{}{}\end{DoxyParamCaption})}

\Hypertarget{class_beam_1_1_quad_batcher_a3f0b54c6cfd0ba9e85bd6c7c239eb928}\index{Beam::QuadBatcher@{Beam::QuadBatcher}!createFontTexture@{createFontTexture}}
\index{createFontTexture@{createFontTexture}!Beam::QuadBatcher@{Beam::QuadBatcher}}
\doxysubsubsection{\texorpdfstring{createFontTexture()}{createFontTexture()}}
{\footnotesize\ttfamily \label{class_beam_1_1_quad_batcher_a3f0b54c6cfd0ba9e85bd6c7c239eb928} 
void Beam::\+\+Quad\+Batcher::\+create\+Font\+Texture (\begin{DoxyParamCaption}{}{}\end{DoxyParamCaption})\hspace{0.3cm}{\ttfamily [private]}}

\Hypertarget{class_beam_1_1_quad_batcher_ae24a0b2b697e7870bebf358ac4776d81}\index{Beam::QuadBatcher@{Beam::QuadBatcher}!drawLine@{drawLine}}
\index{drawLine@{drawLine}!Beam::QuadBatcher@{Beam::QuadBatcher}}
\doxysubsubsection{\texorpdfstring{drawLine()}{drawLine()}}
{\footnotesize\ttfamily \label{class_beam_1_1_quad_batcher_ae24a0b2b697e7870bebf358ac4776d81} 
void Beam::\+\+Quad\+Batcher::\+draw\+Line (\begin{DoxyParamCaption}\item[{float}]{x1}{, }\item[{float}]{y1}{, }\item[{float}]{x2}{, }\item[{float}]{y2}{, }\item[{float}]{thickness}{, }\item[{float}]{r}{, }\item[{float}]{g}{, }\item[{float}]{b}{, }\item[{float}]{a}{}\end{DoxyParamCaption})}

\Hypertarget{class_beam_1_1_quad_batcher_afeda7287047f1ee0db8203306f408503}\index{Beam::QuadBatcher@{Beam::QuadBatcher}!drawQuad@{drawQuad}}
\index{drawQuad@{drawQuad}!Beam::QuadBatcher@{Beam::QuadBatcher}}
\doxysubsubsection{\texorpdfstring{drawQuad()}{drawQuad()}}
{\footnotesize\ttfamily \label{class_beam_1_1_quad_batcher_afeda7287047f1ee0db8203306f408503} 
void Beam::\+\+Quad\+Batcher::\+draw\+Quad (\begin{DoxyParamCaption}\item[{float}]{x}{, }\item[{float}]{y}{, }\item[{float}]{w}{, }\item[{float}]{h}{, }\item[{float}]{r}{, }\item[{float}]{g}{, }\item[{float}]{b}{, }\item[{float}]{a}{}\end{DoxyParamCaption})}

\Hypertarget{class_beam_1_1_quad_batcher_ad2f4ad2e5231b76d8b58f33f9afae7e0}\index{Beam::QuadBatcher@{Beam::QuadBatcher}!drawRect@{drawRect}}
\index{drawRect@{drawRect}!Beam::QuadBatcher@{Beam::QuadBatcher}}
\doxysubsubsection{\texorpdfstring{drawRect()}{drawRect()}}
{\footnotesize\ttfamily \label{class_beam_1_1_quad_batcher_ad2f4ad2e5231b76d8b58f33f9afae7e0} 
void Beam::\+\+Quad\+Batcher::\+draw\+Rect (\begin{DoxyParamCaption}\item[{float}]{x}{, }\item[{float}]{y}{, }\item[{float}]{w}{, }\item[{float}]{h}{, }\item[{float}]{thickness}{, }\item[{float}]{r}{, }\item[{float}]{g}{, }\item[{float}]{b}{, }\item[{float}]{a}{}\end{DoxyParamCaption})}

\Hypertarget{class_beam_1_1_quad_batcher_a16d867a50f08e83bc2b2242ea7c00acb}\index{Beam::QuadBatcher@{Beam::QuadBatcher}!drawRoundedRect@{drawRoundedRect}}
\index{drawRoundedRect@{drawRoundedRect}!Beam::QuadBatcher@{Beam::QuadBatcher}}
\doxysubsubsection{\texorpdfstring{drawRoundedRect()}{drawRoundedRect()}}
{\footnotesize\ttfamily \label{class_beam_1_1_quad_batcher_a16d867a50f08e83bc2b2242ea7c00acb} 
void Beam::\+\+Quad\+Batcher::\+draw\+Rounded\+Rect (\begin{DoxyParamCaption}\item[{float}]{x}{, }\item[{float}]{y}{, }\item[{float}]{w}{, }\item[{float}]{h}{, }\item[{float}]{radius}{, }\item[{float}]{softness}{, }\item[{float}]{r}{, }\item[{float}]{g}{, }\item[{float}]{b}{, }\item[{float}]{a}{}\end{DoxyParamCaption})}

\Hypertarget{class_beam_1_1_quad_batcher_a54b57a1fa94c48b9778c2fe1109e761a}\index{Beam::QuadBatcher@{Beam::QuadBatcher}!drawText@{drawText}}
\index{drawText@{drawText}!Beam::QuadBatcher@{Beam::QuadBatcher}}
\doxysubsubsection{\texorpdfstring{drawText()}{drawText()}}
{\footnotesize\ttfamily \label{class_beam_1_1_quad_batcher_a54b57a1fa94c48b9778c2fe1109e761a} 
void Beam::\+\+Quad\+Batcher::\+draw\+Text (\begin{DoxyParamCaption}\item[{const std::\+string \&}]{text}{, }\item[{float}]{x}{, }\item[{float}]{y}{, }\item[{float}]{size}{, }\item[{float}]{r}{, }\item[{float}]{g}{, }\item[{float}]{b}{, }\item[{float}]{a}{}\end{DoxyParamCaption})}

\Hypertarget{class_beam_1_1_quad_batcher_a7d04ae84e0638e4531264ff5e8de383f}\index{Beam::QuadBatcher@{Beam::QuadBatcher}!flush@{flush}}
\index{flush@{flush}!Beam::QuadBatcher@{Beam::QuadBatcher}}
\doxysubsubsection{\texorpdfstring{flush()}{flush()}}
{\footnotesize\ttfamily \label{class_beam_1_1_quad_batcher_a7d04ae84e0638e4531264ff5e8de383f} 
void Beam::\+\+Quad\+Batcher::\+flush (\begin{DoxyParamCaption}{}{}\end{DoxyParamCaption})}

\Hypertarget{class_beam_1_1_quad_batcher_a838b935a69a5041d714b4895c5f737dd}\index{Beam::QuadBatcher@{Beam::QuadBatcher}!setShader@{setShader}}
\index{setShader@{setShader}!Beam::QuadBatcher@{Beam::QuadBatcher}}
\doxysubsubsection{\texorpdfstring{setShader()}{setShader()}}
{\footnotesize\ttfamily \label{class_beam_1_1_quad_batcher_a838b935a69a5041d714b4895c5f737dd} 
void Beam::\+\+Quad\+Batcher::\+set\+Shader (\begin{DoxyParamCaption}\item[{class \doxymbox{\hyperlink{class_beam_1_1_shader}{Shader}} \texorpdfstring{$\ast$}{*}}]{shader}{}\end{DoxyParamCaption})\hspace{0.3cm}{\ttfamily [inline]}}



\label{doc-variable-members}
\Hypertarget{class_beam_1_1_quad_batcher_doc-variable-members}
\doxysubsection{Member Data Documentation}
\Hypertarget{class_beam_1_1_quad_batcher_a0b453e3a8d0b7b673a180afbbca46f29}\index{Beam::QuadBatcher@{Beam::QuadBatcher}!m\_fontTexture@{m\_fontTexture}}
\index{m\_fontTexture@{m\_fontTexture}!Beam::QuadBatcher@{Beam::QuadBatcher}}
\doxysubsubsection{\texorpdfstring{m\_fontTexture}{m\_fontTexture}}
{\footnotesize\ttfamily \label{class_beam_1_1_quad_batcher_a0b453e3a8d0b7b673a180afbbca46f29} 
unsigned int Beam::\+\+Quad\+Batcher::\+m\+\_\+font\+Texture\hspace{0.3cm}{\ttfamily [private]}}

\Hypertarget{class_beam_1_1_quad_batcher_a6d49aeb9d17eec78b2f6acdb4615ea32}\index{Beam::QuadBatcher@{Beam::QuadBatcher}!m\_ibo@{m\_ibo}}
\index{m\_ibo@{m\_ibo}!Beam::QuadBatcher@{Beam::QuadBatcher}}
\doxysubsubsection{\texorpdfstring{m\_ibo}{m\_ibo}}
{\footnotesize\ttfamily \label{class_beam_1_1_quad_batcher_a6d49aeb9d17eec78b2f6acdb4615ea32} 
unsigned int Beam::\+\+Quad\+Batcher::\+m\+\_\+ibo\hspace{0.3cm}{\ttfamily [private]}}

\Hypertarget{class_beam_1_1_quad_batcher_a34e92273d14f3b91a1fc3b77ad770d97}\index{Beam::QuadBatcher@{Beam::QuadBatcher}!m\_maxQuads@{m\_maxQuads}}
\index{m\_maxQuads@{m\_maxQuads}!Beam::QuadBatcher@{Beam::QuadBatcher}}
\doxysubsubsection{\texorpdfstring{m\_maxQuads}{m\_maxQuads}}
{\footnotesize\ttfamily \label{class_beam_1_1_quad_batcher_a34e92273d14f3b91a1fc3b77ad770d97} 
size\+\_\+t Beam::\+\+Quad\+Batcher::\+m\+\_\+max\+Quads\hspace{0.3cm}{\ttfamily [private]}}

\Hypertarget{class_beam_1_1_quad_batcher_a39fbaab8381cfd108021c6dff7110370}\index{Beam::QuadBatcher@{Beam::QuadBatcher}!m\_quadCount@{m\_quadCount}}
\index{m\_quadCount@{m\_quadCount}!Beam::QuadBatcher@{Beam::QuadBatcher}}
\doxysubsubsection{\texorpdfstring{m\_quadCount}{m\_quadCount}}
{\footnotesize\ttfamily \label{class_beam_1_1_quad_batcher_a39fbaab8381cfd108021c6dff7110370} 
size\+\_\+t Beam::\+\+Quad\+Batcher::\+m\+\_\+quad\+Count\hspace{0.3cm}{\ttfamily [private]}}

\Hypertarget{class_beam_1_1_quad_batcher_a0bf56fdf73e935879113849de645fb63}\index{Beam::QuadBatcher@{Beam::QuadBatcher}!m\_shader@{m\_shader}}
\index{m\_shader@{m\_shader}!Beam::QuadBatcher@{Beam::QuadBatcher}}
\doxysubsubsection{\texorpdfstring{m\_shader}{m\_shader}}
{\footnotesize\ttfamily \label{class_beam_1_1_quad_batcher_a0bf56fdf73e935879113849de645fb63} 
class \doxymbox{\hyperlink{class_beam_1_1_shader}{Shader}}\texorpdfstring{$\ast$}{*} Beam::\+\+Quad\+Batcher::\+m\+\_\+shader = nullptr\hspace{0.3cm}{\ttfamily [private]}}

\Hypertarget{class_beam_1_1_quad_batcher_a463cd5ae7fca431940a89ff413e522f1}\index{Beam::QuadBatcher@{Beam::QuadBatcher}!m\_vao@{m\_vao}}
\index{m\_vao@{m\_vao}!Beam::QuadBatcher@{Beam::QuadBatcher}}
\doxysubsubsection{\texorpdfstring{m\_vao}{m\_vao}}
{\footnotesize\ttfamily \label{class_beam_1_1_quad_batcher_a463cd5ae7fca431940a89ff413e522f1} 
unsigned int Beam::\+\+Quad\+Batcher::\+m\+\_\+vao\hspace{0.3cm}{\ttfamily [private]}}

\Hypertarget{class_beam_1_1_quad_batcher_a92cad15aca7b10792225fb67cbb7e322}\index{Beam::QuadBatcher@{Beam::QuadBatcher}!m\_vbo@{m\_vbo}}
\index{m\_vbo@{m\_vbo}!Beam::QuadBatcher@{Beam::QuadBatcher}}
\doxysubsubsection{\texorpdfstring{m\_vbo}{m\_vbo}}
{\footnotesize\ttfamily \label{class_beam_1_1_quad_batcher_a92cad15aca7b10792225fb67cbb7e322} 
unsigned int Beam::\+\+Quad\+Batcher::\+m\+\_\+vbo\hspace{0.3cm}{\ttfamily [private]}}

\Hypertarget{class_beam_1_1_quad_batcher_ac47e3f9be44ce389aaa15a44c9f3bd94}\index{Beam::QuadBatcher@{Beam::QuadBatcher}!m\_vertices@{m\_vertices}}
\index{m\_vertices@{m\_vertices}!Beam::QuadBatcher@{Beam::QuadBatcher}}
\doxysubsubsection{\texorpdfstring{m\_vertices}{m\_vertices}}
{\footnotesize\ttfamily \label{class_beam_1_1_quad_batcher_ac47e3f9be44ce389aaa15a44c9f3bd94} 
std::\+vector$<$\doxymbox{\hyperlink{struct_beam_1_1_vertex}{Vertex}}$>$ Beam::\+\+Quad\+Batcher::\+m\+\_\+vertices\hspace{0.3cm}{\ttfamily [private]}}



The documentation for this class was generated from the following files:\+\begin{DoxyCompactItemize}
\item 
src/\+graphics/\+\doxymbox{\hyperlink{quad__batcher_8hpp}{quad\+\_\+batcher.\+hpp}}\item 
src/\+graphics/\+\doxymbox{\hyperlink{quad__batcher_8cpp}{quad\+\_\+batcher.\+cpp}}\end{DoxyCompactItemize}

\doxysection{Beam::\+Component Class Reference}
\hypertarget{class_beam_1_1_component}{}\label{class_beam_1_1_component}\index{Beam::Component@{Beam::Component}}


{\ttfamily \+\#include $<$component.\+hpp$>$}

Inheritance diagram for Beam::\+Component:\+\begin{figure}[H]
\begin{center}
\leavevmode
\includegraphics[height=10.000000cm]{class_beam_1_1_component}
\end{center}
\end{figure}
\doxysubsubsection*{Public Member Functions}
\begin{DoxyCompactItemize}
\item 
virtual \doxymbox{\hyperlink{class_beam_1_1_component_af9d734d649978e027412a87bc54362cd}{\texorpdfstring{$\sim$}{\string~}\+Component}} ()=default
\item 
virtual void \doxymbox{\hyperlink{class_beam_1_1_component_ad3d3fb19d25b4371d07620567970a158}{update}} (float dt)
\item 
virtual void \doxymbox{\hyperlink{class_beam_1_1_component_a2ed6b7841a25bd992bd46b822311ef1d}{render}} (\doxymbox{\hyperlink{class_beam_1_1_quad_batcher}{Quad\+Batcher}} \&batcher)=0
\item 
virtual bool \doxymbox{\hyperlink{class_beam_1_1_component_aec1da33d2d6e3d4e7dd6708309264e76}{on\+Mouse\+Down}} (float x, float y, int button)
\item 
virtual bool \doxymbox{\hyperlink{class_beam_1_1_component_ae36b8e9d70e8f9a1b9ba81c23c54d5c8}{on\+Mouse\+Up}} (float x, float y, int button)
\item 
virtual bool \doxymbox{\hyperlink{class_beam_1_1_component_a9d8e5970783d315044277a1228659e6c}{on\+Mouse\+Move}} (float x, float y)
\item 
virtual void \doxymbox{\hyperlink{class_beam_1_1_component_a6865b1f22388af467bf6c789120fac05}{set\+Bounds}} (float x, float y, float w, float h)
\item 
const \doxymbox{\hyperlink{struct_beam_1_1_rect}{Rect}} \& \doxymbox{\hyperlink{class_beam_1_1_component_a5746dbc69d5b0adb4cffbcf920936d00}{get\+Bounds}} () const
\item 
void \doxymbox{\hyperlink{class_beam_1_1_component_a00d4e2dfa7703e59d6486852321dbdf1}{set\+Draggable}} (bool draggable)
\item 
void \doxymbox{\hyperlink{class_beam_1_1_component_aca7b02d1dddf7cd20378db9e3242fb84}{start\+Dragging}} (float x, float y)
\end{DoxyCompactItemize}
\doxysubsubsection*{Protected Attributes}
\begin{DoxyCompactItemize}
\item 
\doxymbox{\hyperlink{struct_beam_1_1_rect}{Rect}} \doxymbox{\hyperlink{class_beam_1_1_component_a4f1ec4a5fb168c39a6c18f958b2b1495}{m\+\_\+bounds}} \{0, 0, 0, 0\}
\item 
bool \doxymbox{\hyperlink{class_beam_1_1_component_adc07913aed6ddadf1c730e7b3bb599cf}{m\+\_\+is\+Visible}} = true
\item 
bool \doxymbox{\hyperlink{class_beam_1_1_component_a0bf77b204ae374a14b5a6d7e5a3c13c6}{m\+\_\+is\+Enabled}} = true
\item 
bool \doxymbox{\hyperlink{class_beam_1_1_component_a9646efcaa9540a26a387f5da9aae4bde}{m\+\_\+is\+Draggable}} = false
\item 
bool \doxymbox{\hyperlink{class_beam_1_1_component_ab03af9a9743acf040f38e3fb11f8dc14}{m\+\_\+is\+Dragging}} = false
\item 
float \doxymbox{\hyperlink{class_beam_1_1_component_a7110b2b9dc235f724bf4689569266a63}{m\+\_\+last\+MouseX}} = 0
\item 
float \doxymbox{\hyperlink{class_beam_1_1_component_a768931a0f51394bf011f821f6ed2efe9}{m\+\_\+last\+MouseY}} = 0
\end{DoxyCompactItemize}


\label{doc-constructors}
\Hypertarget{class_beam_1_1_component_doc-constructors}
\doxysubsection{Constructor \& Destructor Documentation}
\Hypertarget{class_beam_1_1_component_af9d734d649978e027412a87bc54362cd}\index{Beam::Component@{Beam::Component}!````~Component@{\texorpdfstring{$\sim$}{\string~}Component}}
\index{````~Component@{\texorpdfstring{$\sim$}{\string~}Component}!Beam::Component@{Beam::Component}}
\doxysubsubsection{\texorpdfstring{\texorpdfstring{$\sim$}{\string~}Component()}{\string~Component()}}
{\footnotesize\ttfamily \label{class_beam_1_1_component_af9d734d649978e027412a87bc54362cd} 
virtual Beam::\+\+Component::\+\texorpdfstring{$\sim$}{\string~}\+Component (\begin{DoxyParamCaption}{}{}\end{DoxyParamCaption})\hspace{0.3cm}{\ttfamily [virtual]}, {\ttfamily [default]}}



\label{doc-func-members}
\Hypertarget{class_beam_1_1_component_doc-func-members}
\doxysubsection{Member Function Documentation}
\Hypertarget{class_beam_1_1_component_a5746dbc69d5b0adb4cffbcf920936d00}\index{Beam::Component@{Beam::Component}!getBounds@{getBounds}}
\index{getBounds@{getBounds}!Beam::Component@{Beam::Component}}
\doxysubsubsection{\texorpdfstring{getBounds()}{getBounds()}}
{\footnotesize\ttfamily \label{class_beam_1_1_component_a5746dbc69d5b0adb4cffbcf920936d00} 
const \doxymbox{\hyperlink{struct_beam_1_1_rect}{Rect}} \& Beam::\+\+Component::\+get\+Bounds (\begin{DoxyParamCaption}{}{}\end{DoxyParamCaption}) const\hspace{0.3cm}{\ttfamily [inline]}}

\Hypertarget{class_beam_1_1_component_aec1da33d2d6e3d4e7dd6708309264e76}\index{Beam::Component@{Beam::Component}!onMouseDown@{onMouseDown}}
\index{onMouseDown@{onMouseDown}!Beam::Component@{Beam::Component}}
\doxysubsubsection{\texorpdfstring{onMouseDown()}{onMouseDown()}}
{\footnotesize\ttfamily \label{class_beam_1_1_component_aec1da33d2d6e3d4e7dd6708309264e76} 
virtual bool Beam::\+\+Component::\+on\+Mouse\+Down (\begin{DoxyParamCaption}\item[{float}]{x}{, }\item[{float}]{y}{, }\item[{int}]{button}{}\end{DoxyParamCaption})\hspace{0.3cm}{\ttfamily [inline]}, {\ttfamily [virtual]}}



Reimplemented in \doxymbox{\hyperlink{class_beam_1_1_audio_module_adddaa58e40512d782c2d902917491499}{Beam::\+\+Audio\+Module}}, \doxymbox{\hyperlink{class_beam_1_1_knob_a85b8ee99461b9e8c1a86c5f2d9eba4e2}{Beam::\+\+Knob}}, \doxymbox{\hyperlink{class_beam_1_1_port_af911fc850b8cf4bbe1491f7845d16cc9}{Beam::\+\+Port}}, \doxymbox{\hyperlink{class_beam_1_1_sidebar_a7fa418614ff6c3f7ca87552af765026c}{Beam::\+\+Sidebar}}, \doxymbox{\hyperlink{class_beam_1_1_tape_reel_ad2fe7a4986075d4a3c6047f2e745460e}{Beam::\+\+Tape\+Reel}}, \doxymbox{\hyperlink{class_beam_1_1_top_bar_a00fbc22478eb3256b0e11a75cf78cec9}{Beam::\+\+Top\+Bar}}, and \doxymbox{\hyperlink{class_beam_1_1_workspace_a2d287ac679c39e3bb74a6806c627c610}{Beam::\+\+Workspace}}.

\Hypertarget{class_beam_1_1_component_a9d8e5970783d315044277a1228659e6c}\index{Beam::Component@{Beam::Component}!onMouseMove@{onMouseMove}}
\index{onMouseMove@{onMouseMove}!Beam::Component@{Beam::Component}}
\doxysubsubsection{\texorpdfstring{onMouseMove()}{onMouseMove()}}
{\footnotesize\ttfamily \label{class_beam_1_1_component_a9d8e5970783d315044277a1228659e6c} 
virtual bool Beam::\+\+Component::\+on\+Mouse\+Move (\begin{DoxyParamCaption}\item[{float}]{x}{, }\item[{float}]{y}{}\end{DoxyParamCaption})\hspace{0.3cm}{\ttfamily [inline]}, {\ttfamily [virtual]}}



Reimplemented in \doxymbox{\hyperlink{class_beam_1_1_audio_module_aff90c84092de907409b84eed2c46cbe2}{Beam::\+\+Audio\+Module}}, \doxymbox{\hyperlink{class_beam_1_1_knob_a6e918f86ad28e14eb470e00938d1b212}{Beam::\+\+Knob}}, and \doxymbox{\hyperlink{class_beam_1_1_workspace_ae030281b49f7c11a28517b4deaf69b99}{Beam::\+\+Workspace}}.

\Hypertarget{class_beam_1_1_component_ae36b8e9d70e8f9a1b9ba81c23c54d5c8}\index{Beam::Component@{Beam::Component}!onMouseUp@{onMouseUp}}
\index{onMouseUp@{onMouseUp}!Beam::Component@{Beam::Component}}
\doxysubsubsection{\texorpdfstring{onMouseUp()}{onMouseUp()}}
{\footnotesize\ttfamily \label{class_beam_1_1_component_ae36b8e9d70e8f9a1b9ba81c23c54d5c8} 
virtual bool Beam::\+\+Component::\+on\+Mouse\+Up (\begin{DoxyParamCaption}\item[{float}]{x}{, }\item[{float}]{y}{, }\item[{int}]{button}{}\end{DoxyParamCaption})\hspace{0.3cm}{\ttfamily [inline]}, {\ttfamily [virtual]}}



Reimplemented in \doxymbox{\hyperlink{class_beam_1_1_knob_a266e448c4403e85f68a581fb81922fdd}{Beam::\+\+Knob}}, and \doxymbox{\hyperlink{class_beam_1_1_workspace_aa04ebe3cc67dbd9675a6a6eca5506547}{Beam::\+\+Workspace}}.

\Hypertarget{class_beam_1_1_component_a2ed6b7841a25bd992bd46b822311ef1d}\index{Beam::Component@{Beam::Component}!render@{render}}
\index{render@{render}!Beam::Component@{Beam::Component}}
\doxysubsubsection{\texorpdfstring{render()}{render()}}
{\footnotesize\ttfamily \label{class_beam_1_1_component_a2ed6b7841a25bd992bd46b822311ef1d} 
virtual void Beam::\+\+Component::\+render (\begin{DoxyParamCaption}\item[{\doxymbox{\hyperlink{class_beam_1_1_quad_batcher}{Quad\+Batcher}} \&}]{batcher}{}\end{DoxyParamCaption})\hspace{0.3cm}{\ttfamily [pure virtual]}}



Implemented in \doxymbox{\hyperlink{class_beam_1_1_audio_module_ad135753a43c33ae78c1d60980e208868}{Beam::\+\+Audio\+Module}}, \doxymbox{\hyperlink{class_beam_1_1_knob_a67bf95ba1b02ad84e9a080530aa2e090}{Beam::\+\+Knob}}, \doxymbox{\hyperlink{class_beam_1_1_master_strip_afd7b0818c5ecb610f6fd3882b8a8ad71}{Beam::\+\+Master\+Strip}}, \doxymbox{\hyperlink{class_beam_1_1_port_ad7cba89612682a527635042473a3d4c0}{Beam::\+\+Port}}, \doxymbox{\hyperlink{class_beam_1_1_sidebar_ab53552812985613ba4cfd66428b49c04}{Beam::\+\+Sidebar}}, \doxymbox{\hyperlink{class_beam_1_1_tape_reel_aec8dbeac4cb29052b8c9306a3cc408e0}{Beam::\+\+Tape\+Reel}}, \doxymbox{\hyperlink{class_beam_1_1_timeline_a10a7b70340a61dcfb1ebead1a02d6b5a}{Beam::\+\+Timeline}}, \doxymbox{\hyperlink{class_beam_1_1_top_bar_afea870bc1388aee09948c8d5b0dd014f}{Beam::\+\+Top\+Bar}}, \doxymbox{\hyperlink{class_beam_1_1_v_u_meter_a8a49ae9ac8be0f1f148dfa6020c92fd5}{Beam::\+\+VUMeter}}, and \doxymbox{\hyperlink{class_beam_1_1_workspace_a6be84daa0a4def0d2adfc162c45d9331}{Beam::\+\+Workspace}}.

\Hypertarget{class_beam_1_1_component_a6865b1f22388af467bf6c789120fac05}\index{Beam::Component@{Beam::Component}!setBounds@{setBounds}}
\index{setBounds@{setBounds}!Beam::Component@{Beam::Component}}
\doxysubsubsection{\texorpdfstring{setBounds()}{setBounds()}}
{\footnotesize\ttfamily \label{class_beam_1_1_component_a6865b1f22388af467bf6c789120fac05} 
virtual void Beam::\+\+Component::\+set\+Bounds (\begin{DoxyParamCaption}\item[{float}]{x}{, }\item[{float}]{y}{, }\item[{float}]{w}{, }\item[{float}]{h}{}\end{DoxyParamCaption})\hspace{0.3cm}{\ttfamily [inline]}, {\ttfamily [virtual]}}



Reimplemented in \doxymbox{\hyperlink{class_beam_1_1_audio_module_a75c0758091a1cec0871134babb541135}{Beam::\+\+Audio\+Module}}, and \doxymbox{\hyperlink{class_beam_1_1_master_strip_a021372e32a0033a9ed5f3224f136c893}{Beam::\+\+Master\+Strip}}.

\Hypertarget{class_beam_1_1_component_a00d4e2dfa7703e59d6486852321dbdf1}\index{Beam::Component@{Beam::Component}!setDraggable@{setDraggable}}
\index{setDraggable@{setDraggable}!Beam::Component@{Beam::Component}}
\doxysubsubsection{\texorpdfstring{setDraggable()}{setDraggable()}}
{\footnotesize\ttfamily \label{class_beam_1_1_component_a00d4e2dfa7703e59d6486852321dbdf1} 
void Beam::\+\+Component::\+set\+Draggable (\begin{DoxyParamCaption}\item[{bool}]{draggable}{}\end{DoxyParamCaption})\hspace{0.3cm}{\ttfamily [inline]}}

\Hypertarget{class_beam_1_1_component_aca7b02d1dddf7cd20378db9e3242fb84}\index{Beam::Component@{Beam::Component}!startDragging@{startDragging}}
\index{startDragging@{startDragging}!Beam::Component@{Beam::Component}}
\doxysubsubsection{\texorpdfstring{startDragging()}{startDragging()}}
{\footnotesize\ttfamily \label{class_beam_1_1_component_aca7b02d1dddf7cd20378db9e3242fb84} 
void Beam::\+\+Component::\+start\+Dragging (\begin{DoxyParamCaption}\item[{float}]{x}{, }\item[{float}]{y}{}\end{DoxyParamCaption})\hspace{0.3cm}{\ttfamily [inline]}}

\Hypertarget{class_beam_1_1_component_ad3d3fb19d25b4371d07620567970a158}\index{Beam::Component@{Beam::Component}!update@{update}}
\index{update@{update}!Beam::Component@{Beam::Component}}
\doxysubsubsection{\texorpdfstring{update()}{update()}}
{\footnotesize\ttfamily \label{class_beam_1_1_component_ad3d3fb19d25b4371d07620567970a158} 
virtual void Beam::\+\+Component::\+update (\begin{DoxyParamCaption}\item[{float}]{dt}{}\end{DoxyParamCaption})\hspace{0.3cm}{\ttfamily [inline]}, {\ttfamily [virtual]}}



Reimplemented in \doxymbox{\hyperlink{class_beam_1_1_tape_reel_a92ce3fb77345df4a6a2a0aebf04d8bd8}{Beam::\+\+Tape\+Reel}}.



\label{doc-variable-members}
\Hypertarget{class_beam_1_1_component_doc-variable-members}
\doxysubsection{Member Data Documentation}
\Hypertarget{class_beam_1_1_component_a4f1ec4a5fb168c39a6c18f958b2b1495}\index{Beam::Component@{Beam::Component}!m\_bounds@{m\_bounds}}
\index{m\_bounds@{m\_bounds}!Beam::Component@{Beam::Component}}
\doxysubsubsection{\texorpdfstring{m\_bounds}{m\_bounds}}
{\footnotesize\ttfamily \label{class_beam_1_1_component_a4f1ec4a5fb168c39a6c18f958b2b1495} 
\doxymbox{\hyperlink{struct_beam_1_1_rect}{Rect}} Beam::\+\+Component::\+m\+\_\+bounds \{0, 0, 0, 0\}\hspace{0.3cm}{\ttfamily [protected]}}

\Hypertarget{class_beam_1_1_component_a9646efcaa9540a26a387f5da9aae4bde}\index{Beam::Component@{Beam::Component}!m\_isDraggable@{m\_isDraggable}}
\index{m\_isDraggable@{m\_isDraggable}!Beam::Component@{Beam::Component}}
\doxysubsubsection{\texorpdfstring{m\_isDraggable}{m\_isDraggable}}
{\footnotesize\ttfamily \label{class_beam_1_1_component_a9646efcaa9540a26a387f5da9aae4bde} 
bool Beam::\+\+Component::\+m\+\_\+is\+Draggable = false\hspace{0.3cm}{\ttfamily [protected]}}

\Hypertarget{class_beam_1_1_component_ab03af9a9743acf040f38e3fb11f8dc14}\index{Beam::Component@{Beam::Component}!m\_isDragging@{m\_isDragging}}
\index{m\_isDragging@{m\_isDragging}!Beam::Component@{Beam::Component}}
\doxysubsubsection{\texorpdfstring{m\_isDragging}{m\_isDragging}}
{\footnotesize\ttfamily \label{class_beam_1_1_component_ab03af9a9743acf040f38e3fb11f8dc14} 
bool Beam::\+\+Component::\+m\+\_\+is\+Dragging = false\hspace{0.3cm}{\ttfamily [protected]}}

\Hypertarget{class_beam_1_1_component_a0bf77b204ae374a14b5a6d7e5a3c13c6}\index{Beam::Component@{Beam::Component}!m\_isEnabled@{m\_isEnabled}}
\index{m\_isEnabled@{m\_isEnabled}!Beam::Component@{Beam::Component}}
\doxysubsubsection{\texorpdfstring{m\_isEnabled}{m\_isEnabled}}
{\footnotesize\ttfamily \label{class_beam_1_1_component_a0bf77b204ae374a14b5a6d7e5a3c13c6} 
bool Beam::\+\+Component::\+m\+\_\+is\+Enabled = true\hspace{0.3cm}{\ttfamily [protected]}}

\Hypertarget{class_beam_1_1_component_adc07913aed6ddadf1c730e7b3bb599cf}\index{Beam::Component@{Beam::Component}!m\_isVisible@{m\_isVisible}}
\index{m\_isVisible@{m\_isVisible}!Beam::Component@{Beam::Component}}
\doxysubsubsection{\texorpdfstring{m\_isVisible}{m\_isVisible}}
{\footnotesize\ttfamily \label{class_beam_1_1_component_adc07913aed6ddadf1c730e7b3bb599cf} 
bool Beam::\+\+Component::\+m\+\_\+is\+Visible = true\hspace{0.3cm}{\ttfamily [protected]}}

\Hypertarget{class_beam_1_1_component_a7110b2b9dc235f724bf4689569266a63}\index{Beam::Component@{Beam::Component}!m\_lastMouseX@{m\_lastMouseX}}
\index{m\_lastMouseX@{m\_lastMouseX}!Beam::Component@{Beam::Component}}
\doxysubsubsection{\texorpdfstring{m\_lastMouseX}{m\_lastMouseX}}
{\footnotesize\ttfamily \label{class_beam_1_1_component_a7110b2b9dc235f724bf4689569266a63} 
float Beam::\+\+Component::\+m\+\_\+last\+MouseX = 0\hspace{0.3cm}{\ttfamily [protected]}}

\Hypertarget{class_beam_1_1_component_a768931a0f51394bf011f821f6ed2efe9}\index{Beam::Component@{Beam::Component}!m\_lastMouseY@{m\_lastMouseY}}
\index{m\_lastMouseY@{m\_lastMouseY}!Beam::Component@{Beam::Component}}
\doxysubsubsection{\texorpdfstring{m\_lastMouseY}{m\_lastMouseY}}
{\footnotesize\ttfamily \label{class_beam_1_1_component_a768931a0f51394bf011f821f6ed2efe9} 
float Beam::\+\+Component::\+m\+\_\+last\+MouseY = 0\hspace{0.3cm}{\ttfamily [protected]}}



The documentation for this class was generated from the following file:\+\begin{DoxyCompactItemize}
\item 
src/\+ui/\+\doxymbox{\hyperlink{component_8hpp}{component.\+hpp}}\end{DoxyCompactItemize}

% Include all other classes generated by Doxygen that are relevant. 
% Since listing 100+ files manually is error prone, we usually rely on Doxygen's structure, 
% but here I will explicitly include the key ones identified in the architecture.
% Note: Ideally, we would input ALL class files.
% I will check if 'annotated.tex' or similar inputs them. No, annotated.tex just lists them.
% I will use a wildcard approach if I could, but I can't in latex.
% However, typically 'refman.tex' includes ALL of them.
% I will copy the block of inputs from the 'refman.tex' I read earlier to ensure completeness.

\doxysection{Beam::\+Analog\+Base Class Reference}
\hypertarget{class_beam_1_1_analog_base}{}\label{class_beam_1_1_analog_base}\index{Beam::AnalogBase@{Beam::AnalogBase}}


Provides reusable physics-\/based algorithms for analog emulation.  




{\ttfamily \+\#include $<$analog\+\_\+base.\+hpp$>$}

\doxysubsubsection*{Classes}
\begin{DoxyCompactItemize}
\item 
class \doxymbox{\hyperlink{class_beam_1_1_analog_base_1_1_one_pole_filter}{One\+Pole\+Filter}}
\begin{DoxyCompactList}\small\item\em Simple 6d\+B/\+oct filter for simulating cable capacitance or basic tone shaping. \end{DoxyCompactList}\item 
class \doxymbox{\hyperlink{class_beam_1_1_analog_base_1_1_wow_flutter_generator}{Wow\+Flutter\+Generator}}
\begin{DoxyCompactList}\small\item\em Generates physical tape speed fluctuations. \end{DoxyCompactList}\end{DoxyCompactItemize}
\doxysubsubsection*{Static Public Member Functions}
\begin{DoxyCompactItemize}
\item 
static float \doxymbox{\hyperlink{class_beam_1_1_analog_base_a8eed4bd2205e1ff9dd9ab036ea7cfe95}{saturate\+Langevin}} (float x, float drive)
\begin{DoxyCompactList}\small\item\em Langevin function approximation for magnetic hysteresis simulation. High-\/level saturation that mimics how iron-\/oxide particles on tape respond to flux. \end{DoxyCompactList}\item 
static float \doxymbox{\hyperlink{class_beam_1_1_analog_base_a2ee8b06f1d7ce4b8cc4e30b44685b959}{saturate\+Transformer}} (float x, float iron)
\begin{DoxyCompactList}\small\item\em Simulates odd-\/harmonic distortion typical of transformer cores. \end{DoxyCompactList}\end{DoxyCompactItemize}


\doxysubsection{Detailed Description}
Provides reusable physics-\/based algorithms for analog emulation. 

\label{doc-func-members}
\Hypertarget{class_beam_1_1_analog_base_doc-func-members}
\doxysubsection{Member Function Documentation}
\Hypertarget{class_beam_1_1_analog_base_a8eed4bd2205e1ff9dd9ab036ea7cfe95}\index{Beam::AnalogBase@{Beam::AnalogBase}!saturateLangevin@{saturateLangevin}}
\index{saturateLangevin@{saturateLangevin}!Beam::AnalogBase@{Beam::AnalogBase}}
\doxysubsubsection{\texorpdfstring{saturateLangevin()}{saturateLangevin()}}
{\footnotesize\ttfamily \label{class_beam_1_1_analog_base_a8eed4bd2205e1ff9dd9ab036ea7cfe95} 
float Beam::\+\+Analog\+Base::\+saturate\+Langevin (\begin{DoxyParamCaption}\item[{float}]{x}{, }\item[{float}]{drive}{}\end{DoxyParamCaption})\hspace{0.3cm}{\ttfamily [inline]}, {\ttfamily [static]}}



Langevin function approximation for magnetic hysteresis simulation. High-\/level saturation that mimics how iron-\/oxide particles on tape respond to flux. 

\Hypertarget{class_beam_1_1_analog_base_a2ee8b06f1d7ce4b8cc4e30b44685b959}\index{Beam::AnalogBase@{Beam::AnalogBase}!saturateTransformer@{saturateTransformer}}
\index{saturateTransformer@{saturateTransformer}!Beam::AnalogBase@{Beam::AnalogBase}}
\doxysubsubsection{\texorpdfstring{saturateTransformer()}{saturateTransformer()}}
{\footnotesize\ttfamily \label{class_beam_1_1_analog_base_a2ee8b06f1d7ce4b8cc4e30b44685b959} 
float Beam::\+\+Analog\+Base::\+saturate\+Transformer (\begin{DoxyParamCaption}\item[{float}]{x}{, }\item[{float}]{iron}{}\end{DoxyParamCaption})\hspace{0.3cm}{\ttfamily [inline]}, {\ttfamily [static]}}



Simulates odd-\/harmonic distortion typical of transformer cores. 



The documentation for this class was generated from the following file:\+\begin{DoxyCompactItemize}
\item 
src/\+engine/\+\doxymbox{\hyperlink{analog__base_8hpp}{analog\+\_\+base.\+hpp}}\end{DoxyCompactItemize}

\doxysection{Beam::\+Application\+Base Class Reference}
\hypertarget{class_beam_1_1_application_base}{}\label{class_beam_1_1_application_base}\index{Beam::ApplicationBase@{Beam::ApplicationBase}}


Base class for Flux applications, similar to JUCE\textquotesingle{}s JUCEApplication.  




{\ttfamily \+\#include $<$application\+\_\+base.\+hpp$>$}

\doxysubsubsection*{Public Member Functions}
\begin{DoxyCompactItemize}
\item 
\doxymbox{\hyperlink{class_beam_1_1_application_base_a0ddc71fdea733a70eabeb6874e349518}{Application\+Base}} ()
\item 
virtual \doxymbox{\hyperlink{class_beam_1_1_application_base_aea7046fc8c47ff5db72750a000686f66}{\texorpdfstring{$\sim$}{\string~}\+Application\+Base}} ()
\item 
virtual void \doxymbox{\hyperlink{class_beam_1_1_application_base_a31475b2c194ffdbd46799cfc28ffc3ed}{initialise}} (const std::\+string \&command\+Line)=0
\begin{DoxyCompactList}\small\item\em Called when the application starts up. \end{DoxyCompactList}\item 
virtual void \doxymbox{\hyperlink{class_beam_1_1_application_base_a64149122524540008c756daf8cfce857}{shutdown}} ()=0
\begin{DoxyCompactList}\small\item\em Called when the application is shutting down. \end{DoxyCompactList}\item 
virtual void \doxymbox{\hyperlink{class_beam_1_1_application_base_ac2fa9347ec6cd98e26b9119629a82dc1}{suspended}} ()
\begin{DoxyCompactList}\small\item\em Called periodically when there are no messages in the queue. \end{DoxyCompactList}\item 
virtual void \doxymbox{\hyperlink{class_beam_1_1_application_base_a2fb598634b8402fe7dc9e755bfd28904}{resumed}} ()
\begin{DoxyCompactList}\small\item\em Called when the application is brought back from suspended state. \end{DoxyCompactList}\item 
virtual void \doxymbox{\hyperlink{class_beam_1_1_application_base_af1cf378bee6c721e9105533ca6063707}{system\+Requested\+Quit}} ()
\begin{DoxyCompactList}\small\item\em Called when the OS wants to shut down the application. \end{DoxyCompactList}\item 
virtual std::\+string \doxymbox{\hyperlink{class_beam_1_1_application_base_a5728a4d77317835d55cd378ca2cd7be6}{get\+Application\+Name}} () const =0
\begin{DoxyCompactList}\small\item\em Returns the application name. \end{DoxyCompactList}\item 
virtual std::\+string \doxymbox{\hyperlink{class_beam_1_1_application_base_aa9edeed40d6bbcca5f1f9fb1dd50d912}{get\+Application\+Version}} () const =0
\begin{DoxyCompactList}\small\item\em Returns the application version. \end{DoxyCompactList}\item 
bool \doxymbox{\hyperlink{class_beam_1_1_application_base_a960949eecb74d22ba121398faa46e5a9}{should\+Quit}} () const
\begin{DoxyCompactList}\small\item\em Checks if the application should quit. \end{DoxyCompactList}\item 
void \doxymbox{\hyperlink{class_beam_1_1_application_base_a50c7a47a0874c1bedaaf5320f4914a33}{quit}} ()
\begin{DoxyCompactList}\small\item\em Quits the application. \end{DoxyCompactList}\end{DoxyCompactItemize}
\doxysubsubsection*{Static Public Member Functions}
\begin{DoxyCompactItemize}
\item 
static \doxymbox{\hyperlink{class_beam_1_1_application_base_a0ddc71fdea733a70eabeb6874e349518}{Application\+Base}} \texorpdfstring{$\ast$}{*} \doxymbox{\hyperlink{class_beam_1_1_application_base_a958af7667fd187ee864076a2e97401b9}{get\+Instance}} ()
\begin{DoxyCompactList}\small\item\em Gets the singleton instance of the application. \end{DoxyCompactList}\end{DoxyCompactItemize}
\doxysubsubsection*{Protected Attributes}
\begin{DoxyCompactItemize}
\item 
bool \doxymbox{\hyperlink{class_beam_1_1_application_base_a9ae9cb9b20ac4eab13025c31461b1b7b}{m\+\_\+should\+Quit}} = false
\end{DoxyCompactItemize}
\doxysubsubsection*{Static Protected Attributes}
\begin{DoxyCompactItemize}
\item 
static \doxymbox{\hyperlink{class_beam_1_1_application_base_a0ddc71fdea733a70eabeb6874e349518}{Application\+Base}} \texorpdfstring{$\ast$}{*} \doxymbox{\hyperlink{class_beam_1_1_application_base_ad1889d82fcc6d34583fb1ae2b958dc4e}{s\+\_\+instance}} = nullptr
\end{DoxyCompactItemize}


\doxysubsection{Detailed Description}
Base class for Flux applications, similar to JUCE\textquotesingle{}s JUCEApplication. 

\label{doc-constructors}
\Hypertarget{class_beam_1_1_application_base_doc-constructors}
\doxysubsection{Constructor \& Destructor Documentation}
\Hypertarget{class_beam_1_1_application_base_a0ddc71fdea733a70eabeb6874e349518}\index{Beam::ApplicationBase@{Beam::ApplicationBase}!ApplicationBase@{ApplicationBase}}
\index{ApplicationBase@{ApplicationBase}!Beam::ApplicationBase@{Beam::ApplicationBase}}
\doxysubsubsection{\texorpdfstring{ApplicationBase()}{ApplicationBase()}}
{\footnotesize\ttfamily \label{class_beam_1_1_application_base_a0ddc71fdea733a70eabeb6874e349518} 
Beam::\+\+Application\+Base::\+\+Application\+Base (\begin{DoxyParamCaption}{}{}\end{DoxyParamCaption})}

\Hypertarget{class_beam_1_1_application_base_aea7046fc8c47ff5db72750a000686f66}\index{Beam::ApplicationBase@{Beam::ApplicationBase}!````~ApplicationBase@{\texorpdfstring{$\sim$}{\string~}ApplicationBase}}
\index{````~ApplicationBase@{\texorpdfstring{$\sim$}{\string~}ApplicationBase}!Beam::ApplicationBase@{Beam::ApplicationBase}}
\doxysubsubsection{\texorpdfstring{\texorpdfstring{$\sim$}{\string~}ApplicationBase()}{\string~ApplicationBase()}}
{\footnotesize\ttfamily \label{class_beam_1_1_application_base_aea7046fc8c47ff5db72750a000686f66} 
Beam::\+\+Application\+Base::\+\texorpdfstring{$\sim$}{\string~}\+Application\+Base (\begin{DoxyParamCaption}{}{}\end{DoxyParamCaption})\hspace{0.3cm}{\ttfamily [virtual]}}



\label{doc-func-members}
\Hypertarget{class_beam_1_1_application_base_doc-func-members}
\doxysubsection{Member Function Documentation}
\Hypertarget{class_beam_1_1_application_base_a5728a4d77317835d55cd378ca2cd7be6}\index{Beam::ApplicationBase@{Beam::ApplicationBase}!getApplicationName@{getApplicationName}}
\index{getApplicationName@{getApplicationName}!Beam::ApplicationBase@{Beam::ApplicationBase}}
\doxysubsubsection{\texorpdfstring{getApplicationName()}{getApplicationName()}}
{\footnotesize\ttfamily \label{class_beam_1_1_application_base_a5728a4d77317835d55cd378ca2cd7be6} 
virtual std::\+string Beam::\+\+Application\+Base::\+get\+Application\+Name (\begin{DoxyParamCaption}{}{}\end{DoxyParamCaption}) const\hspace{0.3cm}{\ttfamily [pure virtual]}}



Returns the application name. 

\Hypertarget{class_beam_1_1_application_base_aa9edeed40d6bbcca5f1f9fb1dd50d912}\index{Beam::ApplicationBase@{Beam::ApplicationBase}!getApplicationVersion@{getApplicationVersion}}
\index{getApplicationVersion@{getApplicationVersion}!Beam::ApplicationBase@{Beam::ApplicationBase}}
\doxysubsubsection{\texorpdfstring{getApplicationVersion()}{getApplicationVersion()}}
{\footnotesize\ttfamily \label{class_beam_1_1_application_base_aa9edeed40d6bbcca5f1f9fb1dd50d912} 
virtual std::\+string Beam::\+\+Application\+Base::\+get\+Application\+Version (\begin{DoxyParamCaption}{}{}\end{DoxyParamCaption}) const\hspace{0.3cm}{\ttfamily [pure virtual]}}



Returns the application version. 

\Hypertarget{class_beam_1_1_application_base_a958af7667fd187ee864076a2e97401b9}\index{Beam::ApplicationBase@{Beam::ApplicationBase}!getInstance@{getInstance}}
\index{getInstance@{getInstance}!Beam::ApplicationBase@{Beam::ApplicationBase}}
\doxysubsubsection{\texorpdfstring{getInstance()}{getInstance()}}
{\footnotesize\ttfamily \label{class_beam_1_1_application_base_a958af7667fd187ee864076a2e97401b9} 
\doxymbox{\hyperlink{class_beam_1_1_application_base_a0ddc71fdea733a70eabeb6874e349518}{Application\+Base}} \texorpdfstring{$\ast$}{*} Beam::\+\+Application\+Base::\+get\+Instance (\begin{DoxyParamCaption}{}{}\end{DoxyParamCaption})\hspace{0.3cm}{\ttfamily [static]}}



Gets the singleton instance of the application. 

\Hypertarget{class_beam_1_1_application_base_a31475b2c194ffdbd46799cfc28ffc3ed}\index{Beam::ApplicationBase@{Beam::ApplicationBase}!initialise@{initialise}}
\index{initialise@{initialise}!Beam::ApplicationBase@{Beam::ApplicationBase}}
\doxysubsubsection{\texorpdfstring{initialise()}{initialise()}}
{\footnotesize\ttfamily \label{class_beam_1_1_application_base_a31475b2c194ffdbd46799cfc28ffc3ed} 
virtual void Beam::\+\+Application\+Base::\+initialise (\begin{DoxyParamCaption}\item[{const std::\+string \&}]{command\+Line}{}\end{DoxyParamCaption})\hspace{0.3cm}{\ttfamily [pure virtual]}}



Called when the application starts up. 

\Hypertarget{class_beam_1_1_application_base_a50c7a47a0874c1bedaaf5320f4914a33}\index{Beam::ApplicationBase@{Beam::ApplicationBase}!quit@{quit}}
\index{quit@{quit}!Beam::ApplicationBase@{Beam::ApplicationBase}}
\doxysubsubsection{\texorpdfstring{quit()}{quit()}}
{\footnotesize\ttfamily \label{class_beam_1_1_application_base_a50c7a47a0874c1bedaaf5320f4914a33} 
void Beam::\+\+Application\+Base::\+quit (\begin{DoxyParamCaption}{}{}\end{DoxyParamCaption})}



Quits the application. 

\Hypertarget{class_beam_1_1_application_base_a2fb598634b8402fe7dc9e755bfd28904}\index{Beam::ApplicationBase@{Beam::ApplicationBase}!resumed@{resumed}}
\index{resumed@{resumed}!Beam::ApplicationBase@{Beam::ApplicationBase}}
\doxysubsubsection{\texorpdfstring{resumed()}{resumed()}}
{\footnotesize\ttfamily \label{class_beam_1_1_application_base_a2fb598634b8402fe7dc9e755bfd28904} 
void Beam::\+\+Application\+Base::\+resumed (\begin{DoxyParamCaption}{}{}\end{DoxyParamCaption})\hspace{0.3cm}{\ttfamily [virtual]}}



Called when the application is brought back from suspended state. 

\Hypertarget{class_beam_1_1_application_base_a960949eecb74d22ba121398faa46e5a9}\index{Beam::ApplicationBase@{Beam::ApplicationBase}!shouldQuit@{shouldQuit}}
\index{shouldQuit@{shouldQuit}!Beam::ApplicationBase@{Beam::ApplicationBase}}
\doxysubsubsection{\texorpdfstring{shouldQuit()}{shouldQuit()}}
{\footnotesize\ttfamily \label{class_beam_1_1_application_base_a960949eecb74d22ba121398faa46e5a9} 
bool Beam::\+\+Application\+Base::\+should\+Quit (\begin{DoxyParamCaption}{}{}\end{DoxyParamCaption}) const\hspace{0.3cm}{\ttfamily [inline]}}



Checks if the application should quit. 

\Hypertarget{class_beam_1_1_application_base_a64149122524540008c756daf8cfce857}\index{Beam::ApplicationBase@{Beam::ApplicationBase}!shutdown@{shutdown}}
\index{shutdown@{shutdown}!Beam::ApplicationBase@{Beam::ApplicationBase}}
\doxysubsubsection{\texorpdfstring{shutdown()}{shutdown()}}
{\footnotesize\ttfamily \label{class_beam_1_1_application_base_a64149122524540008c756daf8cfce857} 
virtual void Beam::\+\+Application\+Base::\+shutdown (\begin{DoxyParamCaption}{}{}\end{DoxyParamCaption})\hspace{0.3cm}{\ttfamily [pure virtual]}}



Called when the application is shutting down. 

\Hypertarget{class_beam_1_1_application_base_ac2fa9347ec6cd98e26b9119629a82dc1}\index{Beam::ApplicationBase@{Beam::ApplicationBase}!suspended@{suspended}}
\index{suspended@{suspended}!Beam::ApplicationBase@{Beam::ApplicationBase}}
\doxysubsubsection{\texorpdfstring{suspended()}{suspended()}}
{\footnotesize\ttfamily \label{class_beam_1_1_application_base_ac2fa9347ec6cd98e26b9119629a82dc1} 
void Beam::\+\+Application\+Base::\+suspended (\begin{DoxyParamCaption}{}{}\end{DoxyParamCaption})\hspace{0.3cm}{\ttfamily [virtual]}}



Called periodically when there are no messages in the queue. 

\Hypertarget{class_beam_1_1_application_base_af1cf378bee6c721e9105533ca6063707}\index{Beam::ApplicationBase@{Beam::ApplicationBase}!systemRequestedQuit@{systemRequestedQuit}}
\index{systemRequestedQuit@{systemRequestedQuit}!Beam::ApplicationBase@{Beam::ApplicationBase}}
\doxysubsubsection{\texorpdfstring{systemRequestedQuit()}{systemRequestedQuit()}}
{\footnotesize\ttfamily \label{class_beam_1_1_application_base_af1cf378bee6c721e9105533ca6063707} 
void Beam::\+\+Application\+Base::\+system\+Requested\+Quit (\begin{DoxyParamCaption}{}{}\end{DoxyParamCaption})\hspace{0.3cm}{\ttfamily [virtual]}}



Called when the OS wants to shut down the application. 



\label{doc-variable-members}
\Hypertarget{class_beam_1_1_application_base_doc-variable-members}
\doxysubsection{Member Data Documentation}
\Hypertarget{class_beam_1_1_application_base_a9ae9cb9b20ac4eab13025c31461b1b7b}\index{Beam::ApplicationBase@{Beam::ApplicationBase}!m\_shouldQuit@{m\_shouldQuit}}
\index{m\_shouldQuit@{m\_shouldQuit}!Beam::ApplicationBase@{Beam::ApplicationBase}}
\doxysubsubsection{\texorpdfstring{m\_shouldQuit}{m\_shouldQuit}}
{\footnotesize\ttfamily \label{class_beam_1_1_application_base_a9ae9cb9b20ac4eab13025c31461b1b7b} 
bool Beam::\+\+Application\+Base::\+m\+\_\+should\+Quit = false\hspace{0.3cm}{\ttfamily [protected]}}

\Hypertarget{class_beam_1_1_application_base_ad1889d82fcc6d34583fb1ae2b958dc4e}\index{Beam::ApplicationBase@{Beam::ApplicationBase}!s\_instance@{s\_instance}}
\index{s\_instance@{s\_instance}!Beam::ApplicationBase@{Beam::ApplicationBase}}
\doxysubsubsection{\texorpdfstring{s\_instance}{s\_instance}}
{\footnotesize\ttfamily \label{class_beam_1_1_application_base_ad1889d82fcc6d34583fb1ae2b958dc4e} 
\doxymbox{\hyperlink{class_beam_1_1_application_base_a0ddc71fdea733a70eabeb6874e349518}{Application\+Base}} \texorpdfstring{$\ast$}{*} Beam::\+\+Application\+Base::\+s\+\_\+instance = nullptr\hspace{0.3cm}{\ttfamily [static]}, {\ttfamily [protected]}}



The documentation for this class was generated from the following files:\+\begin{DoxyCompactItemize}
\item 
src/\+application/\+\doxymbox{\hyperlink{application__base_8hpp}{application\+\_\+base.\+hpp}}\item 
src/\+application/\+\doxymbox{\hyperlink{application__base_8cpp}{application\+\_\+base.\+cpp}}\end{DoxyCompactItemize}

\doxysection{Beam::\+Asset\+Manager Class Reference}
\hypertarget{class_beam_1_1_asset_manager}{}\label{class_beam_1_1_asset_manager}\index{Beam::AssetManager@{Beam::AssetManager}}


Centralized repository for sharing heavy resources like Textures.  




{\ttfamily \+\#include $<$asset\+\_\+manager.\+hpp$>$}

\doxysubsubsection*{Public Member Functions}
\begin{DoxyCompactItemize}
\item 
std::\+shared\+\_\+ptr$<$ \doxymbox{\hyperlink{class_beam_1_1_texture}{Texture}} $>$ \doxymbox{\hyperlink{class_beam_1_1_asset_manager_aaf454d89d7fcbc66b3cc17fb88a255df}{get\+Texture}} (const std::\+string \&path)
\begin{DoxyCompactList}\small\item\em Retrieves a texture from the cache or loads it if not present. \end{DoxyCompactList}\item 
void \doxymbox{\hyperlink{class_beam_1_1_asset_manager_a58e74758db894a4e67e9dcab535141c8}{purge\+Unused}} ()
\begin{DoxyCompactList}\small\item\em Removes unused assets from the cache. \end{DoxyCompactList}\item 
void \doxymbox{\hyperlink{class_beam_1_1_asset_manager_a2617c12bc85f81dc36713b8392982042}{clear}} ()
\begin{DoxyCompactList}\small\item\em Clears the entire asset cache. \end{DoxyCompactList}\end{DoxyCompactItemize}
\doxysubsubsection*{Static Public Member Functions}
\begin{DoxyCompactItemize}
\item 
static \doxymbox{\hyperlink{class_beam_1_1_asset_manager_abb8b8d9b05b34af657919ddafc91ca87}{Asset\+Manager}} \& \doxymbox{\hyperlink{class_beam_1_1_asset_manager_a0f695c5457caab2dc9ae9e605e16cacb}{instance}} ()
\end{DoxyCompactItemize}
\doxysubsubsection*{Private Member Functions}
\begin{DoxyCompactItemize}
\item 
\doxymbox{\hyperlink{class_beam_1_1_asset_manager_abb8b8d9b05b34af657919ddafc91ca87}{Asset\+Manager}} ()=default
\end{DoxyCompactItemize}
\doxysubsubsection*{Private Attributes}
\begin{DoxyCompactItemize}
\item 
std::\+map$<$ std::\+string, std::\+shared\+\_\+ptr$<$ \doxymbox{\hyperlink{class_beam_1_1_texture}{Texture}} $>$ $>$ \doxymbox{\hyperlink{class_beam_1_1_asset_manager_af30e6eb2be3d4f729dd496f5032eed3d}{m\+\_\+textures}}
\item 
std::\+mutex \doxymbox{\hyperlink{class_beam_1_1_asset_manager_a4d8ecae42f01c83beba10a088f1faeef}{m\+\_\+mutex}}
\end{DoxyCompactItemize}


\doxysubsection{Detailed Description}
Centralized repository for sharing heavy resources like Textures. 

The \doxylink{class_beam_1_1_asset_manager}{Asset\+Manager} ensures that any given file is loaded into memory/\+\+GPU only once. It uses a reference-\/counted approach (std::\+shared\+\_\+ptr) to manage asset lifecycles. 

\label{doc-constructors}
\Hypertarget{class_beam_1_1_asset_manager_doc-constructors}
\doxysubsection{Constructor \& Destructor Documentation}
\Hypertarget{class_beam_1_1_asset_manager_abb8b8d9b05b34af657919ddafc91ca87}\index{Beam::AssetManager@{Beam::AssetManager}!AssetManager@{AssetManager}}
\index{AssetManager@{AssetManager}!Beam::AssetManager@{Beam::AssetManager}}
\doxysubsubsection{\texorpdfstring{AssetManager()}{AssetManager()}}
{\footnotesize\ttfamily \label{class_beam_1_1_asset_manager_abb8b8d9b05b34af657919ddafc91ca87} 
Beam::\+\+Asset\+Manager::\+\+Asset\+Manager (\begin{DoxyParamCaption}{}{}\end{DoxyParamCaption})\hspace{0.3cm}{\ttfamily [private]}, {\ttfamily [default]}}



\label{doc-func-members}
\Hypertarget{class_beam_1_1_asset_manager_doc-func-members}
\doxysubsection{Member Function Documentation}
\Hypertarget{class_beam_1_1_asset_manager_a2617c12bc85f81dc36713b8392982042}\index{Beam::AssetManager@{Beam::AssetManager}!clear@{clear}}
\index{clear@{clear}!Beam::AssetManager@{Beam::AssetManager}}
\doxysubsubsection{\texorpdfstring{clear()}{clear()}}
{\footnotesize\ttfamily \label{class_beam_1_1_asset_manager_a2617c12bc85f81dc36713b8392982042} 
void Beam::\+\+Asset\+Manager::\+clear (\begin{DoxyParamCaption}{}{}\end{DoxyParamCaption})}



Clears the entire asset cache. 

\Hypertarget{class_beam_1_1_asset_manager_aaf454d89d7fcbc66b3cc17fb88a255df}\index{Beam::AssetManager@{Beam::AssetManager}!getTexture@{getTexture}}
\index{getTexture@{getTexture}!Beam::AssetManager@{Beam::AssetManager}}
\doxysubsubsection{\texorpdfstring{getTexture()}{getTexture()}}
{\footnotesize\ttfamily \label{class_beam_1_1_asset_manager_aaf454d89d7fcbc66b3cc17fb88a255df} 
std::\+shared\+\_\+ptr$<$ \doxymbox{\hyperlink{class_beam_1_1_texture}{Texture}} $>$ Beam::\+\+Asset\+Manager::\+get\+Texture (\begin{DoxyParamCaption}\item[{const std::\+string \&}]{path}{}\end{DoxyParamCaption})}



Retrieves a texture from the cache or loads it if not present. 


\begin{DoxyParams}{Parameters}
{\em path} & The filesystem path to the texture file. \\
\hline
\end{DoxyParams}
\begin{DoxyReturn}{Returns}
A shared pointer to the \doxylink{class_beam_1_1_texture}{Texture} object, or nullptr if loading fails. 
\end{DoxyReturn}
\Hypertarget{class_beam_1_1_asset_manager_a0f695c5457caab2dc9ae9e605e16cacb}\index{Beam::AssetManager@{Beam::AssetManager}!instance@{instance}}
\index{instance@{instance}!Beam::AssetManager@{Beam::AssetManager}}
\doxysubsubsection{\texorpdfstring{instance()}{instance()}}
{\footnotesize\ttfamily \label{class_beam_1_1_asset_manager_a0f695c5457caab2dc9ae9e605e16cacb} 
\doxymbox{\hyperlink{class_beam_1_1_asset_manager_abb8b8d9b05b34af657919ddafc91ca87}{Asset\+Manager}} \& Beam::\+\+Asset\+Manager::\+instance (\begin{DoxyParamCaption}{}{}\end{DoxyParamCaption})\hspace{0.3cm}{\ttfamily [inline]}, {\ttfamily [static]}}

\Hypertarget{class_beam_1_1_asset_manager_a58e74758db894a4e67e9dcab535141c8}\index{Beam::AssetManager@{Beam::AssetManager}!purgeUnused@{purgeUnused}}
\index{purgeUnused@{purgeUnused}!Beam::AssetManager@{Beam::AssetManager}}
\doxysubsubsection{\texorpdfstring{purgeUnused()}{purgeUnused()}}
{\footnotesize\ttfamily \label{class_beam_1_1_asset_manager_a58e74758db894a4e67e9dcab535141c8} 
void Beam::\+\+Asset\+Manager::\+purge\+Unused (\begin{DoxyParamCaption}{}{}\end{DoxyParamCaption})}



Removes unused assets from the cache. 



\label{doc-variable-members}
\Hypertarget{class_beam_1_1_asset_manager_doc-variable-members}
\doxysubsection{Member Data Documentation}
\Hypertarget{class_beam_1_1_asset_manager_a4d8ecae42f01c83beba10a088f1faeef}\index{Beam::AssetManager@{Beam::AssetManager}!m\_mutex@{m\_mutex}}
\index{m\_mutex@{m\_mutex}!Beam::AssetManager@{Beam::AssetManager}}
\doxysubsubsection{\texorpdfstring{m\_mutex}{m\_mutex}}
{\footnotesize\ttfamily \label{class_beam_1_1_asset_manager_a4d8ecae42f01c83beba10a088f1faeef} 
std::\+mutex Beam::\+\+Asset\+Manager::\+m\+\_\+mutex\hspace{0.3cm}{\ttfamily [private]}}

\Hypertarget{class_beam_1_1_asset_manager_af30e6eb2be3d4f729dd496f5032eed3d}\index{Beam::AssetManager@{Beam::AssetManager}!m\_textures@{m\_textures}}
\index{m\_textures@{m\_textures}!Beam::AssetManager@{Beam::AssetManager}}
\doxysubsubsection{\texorpdfstring{m\_textures}{m\_textures}}
{\footnotesize\ttfamily \label{class_beam_1_1_asset_manager_af30e6eb2be3d4f729dd496f5032eed3d} 
std::\+map$<$std::\+string, std::\+shared\+\_\+ptr$<$\doxymbox{\hyperlink{class_beam_1_1_texture}{Texture}}$>$ $>$ Beam::\+\+Asset\+Manager::\+m\+\_\+textures\hspace{0.3cm}{\ttfamily [private]}}



The documentation for this class was generated from the following files:\+\begin{DoxyCompactItemize}
\item 
src/\+session/\+\doxymbox{\hyperlink{asset__manager_8hpp}{asset\+\_\+manager.\+hpp}}\item 
src/\+session/\+\doxymbox{\hyperlink{asset__manager_8cpp}{asset\+\_\+manager.\+cpp}}\end{DoxyCompactItemize}

\doxysection{Beam::\+Audio\+Buffer\texorpdfstring{$<$}{<} T \texorpdfstring{$>$}{>} Class Template Reference}
\hypertarget{class_beam_1_1_audio_buffer}{}\label{class_beam_1_1_audio_buffer}\index{Beam::AudioBuffer$<$ T $>$@{Beam::AudioBuffer$<$ T $>$}}


Container for multi-\/channel audio data, similar to JUCE\textquotesingle{}s \doxylink{class_beam_1_1_audio_buffer}{Audio\+Buffer}.  




{\ttfamily \+\#include $<$audio\+\_\+buffer.\+hpp$>$}

\doxysubsubsection*{Public Member Functions}
\begin{DoxyCompactItemize}
\item 
\doxymbox{\hyperlink{class_beam_1_1_audio_buffer_a0f248fe3e687d32f730a34f9acdb1549}{Audio\+Buffer}} ()
\begin{DoxyCompactList}\small\item\em Constructor -\/ creates an empty buffer. \end{DoxyCompactList}\item 
\doxymbox{\hyperlink{class_beam_1_1_audio_buffer_a42f6257ae229921c31561e8d546b0dfb}{Audio\+Buffer}} (int num\+Channels, int num\+Samples)
\begin{DoxyCompactList}\small\item\em Constructor -\/ creates a buffer with specified channels and samples. \end{DoxyCompactList}\item 
\doxymbox{\hyperlink{class_beam_1_1_audio_buffer_a13dae890c82b0414120d1afd298b97bf}{\texorpdfstring{$\sim$}{\string~}\+Audio\+Buffer}} ()
\begin{DoxyCompactList}\small\item\em Destructor. \end{DoxyCompactList}\item 
void \doxymbox{\hyperlink{class_beam_1_1_audio_buffer_ae63c0109aad9d7c8bb282af793eb5d6d}{set\+Size}} (int num\+Channels, int num\+Samples, bool keep\+Existing\+Content=false, bool clear\+Extra\+Space=false, bool avoid\+Reallocating=false)
\begin{DoxyCompactList}\small\item\em Sets the size of the buffer. \end{DoxyCompactList}\item 
int \doxymbox{\hyperlink{class_beam_1_1_audio_buffer_af7325c4466887942607fd6960c86cf5c}{get\+Num\+Channels}} () const
\begin{DoxyCompactList}\small\item\em Returns the number of channels. \end{DoxyCompactList}\item 
int \doxymbox{\hyperlink{class_beam_1_1_audio_buffer_ad01215e6138cba40a8e3d813481fb263}{get\+Num\+Samples}} () const
\begin{DoxyCompactList}\small\item\em Returns the number of samples per channel. \end{DoxyCompactList}\item 
T \texorpdfstring{$\ast$}{*} \doxymbox{\hyperlink{class_beam_1_1_audio_buffer_a4378ccd20308acf3bbcc4bf37aa061a3}{get\+Write\+Pointer}} (int channel)
\begin{DoxyCompactList}\small\item\em Returns a pointer to the samples in one of the channels. \end{DoxyCompactList}\item 
const T \texorpdfstring{$\ast$}{*} \doxymbox{\hyperlink{class_beam_1_1_audio_buffer_a7eda29f4b0b43bab3aeba400c163dc17}{get\+Read\+Pointer}} (int channel) const
\begin{DoxyCompactList}\small\item\em Returns a read-\/only pointer to the samples in one of the channels. \end{DoxyCompactList}\item 
void \doxymbox{\hyperlink{class_beam_1_1_audio_buffer_a20ec61008ce4c79ecb36ab05b5ba35e2}{clear}} ()
\begin{DoxyCompactList}\small\item\em Clears all the samples in all channels. \end{DoxyCompactList}\item 
void \doxymbox{\hyperlink{class_beam_1_1_audio_buffer_a06f7df144e2eeb313e9172bc724d4682}{clear}} (int start\+Sample, int num\+Samples)
\begin{DoxyCompactList}\small\item\em Clears a specified region in all channels. \end{DoxyCompactList}\item 
void \doxymbox{\hyperlink{class_beam_1_1_audio_buffer_a816402cf170a0d8dc20adc5647828caf}{add\+From}} (int dest\+Channel, int dest\+Start\+Sample, const \doxymbox{\hyperlink{class_beam_1_1_audio_buffer_a0f248fe3e687d32f730a34f9acdb1549}{Audio\+Buffer}}$<$ T $>$ \&source, int source\+Channel, int source\+Start\+Sample, int num\+Samples, T gain=static\+\_\+cast$<$ T $>$(1))
\begin{DoxyCompactList}\small\item\em Adds samples from another buffer to this one. \end{DoxyCompactList}\item 
void \doxymbox{\hyperlink{class_beam_1_1_audio_buffer_aaa4a7663a86843155ad42202561568be}{copy\+From}} (int dest\+Channel, int dest\+Start\+Sample, const \doxymbox{\hyperlink{class_beam_1_1_audio_buffer_a0f248fe3e687d32f730a34f9acdb1549}{Audio\+Buffer}}$<$ T $>$ \&source, int source\+Channel, int source\+Start\+Sample, int num\+Samples)
\begin{DoxyCompactList}\small\item\em Copies data from another buffer. \end{DoxyCompactList}\item 
void \doxymbox{\hyperlink{class_beam_1_1_audio_buffer_a3eb7f6c54881a58dcbfdc3dc73042e67}{apply\+Gain}} (int channel, int start\+Sample, int num\+Samples, T gain)
\begin{DoxyCompactList}\small\item\em Applies gain to a range of samples in a channel. \end{DoxyCompactList}\item 
void \doxymbox{\hyperlink{class_beam_1_1_audio_buffer_a4d97c0664737f80daade25dfa6f8789d}{apply\+Gain}} (int channel, T gain)
\begin{DoxyCompactList}\small\item\em Applies gain to all samples in a channel. \end{DoxyCompactList}\item 
T \doxymbox{\hyperlink{class_beam_1_1_audio_buffer_ac58b58ad1ea4a33b74a6c9b633395763}{get\+Magnitude}} (int channel, int start\+Sample, int num\+Samples) const
\begin{DoxyCompactList}\small\item\em Finds the highest absolute value in a channel. \end{DoxyCompactList}\item 
T \doxymbox{\hyperlink{class_beam_1_1_audio_buffer_aca9fcdbc6e425179714230182cfdde00}{get\+Magnitude}} () const
\begin{DoxyCompactList}\small\item\em Finds the highest absolute value in the entire buffer. \end{DoxyCompactList}\end{DoxyCompactItemize}
\doxysubsubsection*{Private Attributes}
\begin{DoxyCompactItemize}
\item 
std::\+vector$<$ std::\+vector$<$ T $>$ $>$ \doxymbox{\hyperlink{class_beam_1_1_audio_buffer_a4df4309eb88acf14bc88079c5a71f448}{m\+\_\+data}}
\item 
int \doxymbox{\hyperlink{class_beam_1_1_audio_buffer_a0d70b1b71af4dc940a64130dc82b0e7b}{m\+\_\+num\+Channels}} = 0
\item 
int \doxymbox{\hyperlink{class_beam_1_1_audio_buffer_ad50ee6e795c4721395773e9bcac934bf}{m\+\_\+num\+Samples}} = 0
\end{DoxyCompactItemize}


\doxysubsection{Detailed Description}
\subsubsection*{template$<$typename T$>$\newline
class Beam::\+\+Audio\+Buffer$<$ T $>$}
Container for multi-\/channel audio data, similar to JUCE\textquotesingle{}s \doxylink{class_beam_1_1_audio_buffer}{Audio\+Buffer}. 

\label{doc-constructors}
\Hypertarget{class_beam_1_1_audio_buffer_doc-constructors}
\doxysubsection{Constructor \& Destructor Documentation}
\Hypertarget{class_beam_1_1_audio_buffer_a0f248fe3e687d32f730a34f9acdb1549}\index{Beam::AudioBuffer$<$ T $>$@{Beam::AudioBuffer$<$ T $>$}!AudioBuffer@{AudioBuffer}}
\index{AudioBuffer@{AudioBuffer}!Beam::AudioBuffer$<$ T $>$@{Beam::AudioBuffer$<$ T $>$}}
\doxysubsubsection{\texorpdfstring{AudioBuffer()}{AudioBuffer()}\hspace{0.1cm}{\footnotesize\ttfamily [1/2]}}
{\footnotesize\ttfamily \label{class_beam_1_1_audio_buffer_a0f248fe3e687d32f730a34f9acdb1549} 
template$<$typename T$>$ \\
\doxymbox{\hyperlink{class_beam_1_1_audio_buffer}{Beam::\+\+Audio\+Buffer}}$<$ T $>$::\+\+Audio\+Buffer (\begin{DoxyParamCaption}{}{}\end{DoxyParamCaption})}



Constructor -\/ creates an empty buffer. 

\Hypertarget{class_beam_1_1_audio_buffer_a42f6257ae229921c31561e8d546b0dfb}\index{Beam::AudioBuffer$<$ T $>$@{Beam::AudioBuffer$<$ T $>$}!AudioBuffer@{AudioBuffer}}
\index{AudioBuffer@{AudioBuffer}!Beam::AudioBuffer$<$ T $>$@{Beam::AudioBuffer$<$ T $>$}}
\doxysubsubsection{\texorpdfstring{AudioBuffer()}{AudioBuffer()}\hspace{0.1cm}{\footnotesize\ttfamily [2/2]}}
{\footnotesize\ttfamily \label{class_beam_1_1_audio_buffer_a42f6257ae229921c31561e8d546b0dfb} 
template$<$typename T$>$ \\
\doxymbox{\hyperlink{class_beam_1_1_audio_buffer}{Beam::\+\+Audio\+Buffer}}$<$ T $>$::\+\+Audio\+Buffer (\begin{DoxyParamCaption}\item[{int}]{num\+Channels}{, }\item[{int}]{num\+Samples}{}\end{DoxyParamCaption})}



Constructor -\/ creates a buffer with specified channels and samples. 

\Hypertarget{class_beam_1_1_audio_buffer_a13dae890c82b0414120d1afd298b97bf}\index{Beam::AudioBuffer$<$ T $>$@{Beam::AudioBuffer$<$ T $>$}!````~AudioBuffer@{\texorpdfstring{$\sim$}{\string~}AudioBuffer}}
\index{````~AudioBuffer@{\texorpdfstring{$\sim$}{\string~}AudioBuffer}!Beam::AudioBuffer$<$ T $>$@{Beam::AudioBuffer$<$ T $>$}}
\doxysubsubsection{\texorpdfstring{\texorpdfstring{$\sim$}{\string~}AudioBuffer()}{\string~AudioBuffer()}}
{\footnotesize\ttfamily \label{class_beam_1_1_audio_buffer_a13dae890c82b0414120d1afd298b97bf} 
template$<$typename T$>$ \\
\doxymbox{\hyperlink{class_beam_1_1_audio_buffer}{Beam::\+\+Audio\+Buffer}}$<$ T $>$::\+\texorpdfstring{$\sim$}{\string~}\doxymbox{\hyperlink{class_beam_1_1_audio_buffer_a0f248fe3e687d32f730a34f9acdb1549}{Audio\+Buffer}} (\begin{DoxyParamCaption}{}{}\end{DoxyParamCaption})}



Destructor. 



\label{doc-func-members}
\Hypertarget{class_beam_1_1_audio_buffer_doc-func-members}
\doxysubsection{Member Function Documentation}
\Hypertarget{class_beam_1_1_audio_buffer_a816402cf170a0d8dc20adc5647828caf}\index{Beam::AudioBuffer$<$ T $>$@{Beam::AudioBuffer$<$ T $>$}!addFrom@{addFrom}}
\index{addFrom@{addFrom}!Beam::AudioBuffer$<$ T $>$@{Beam::AudioBuffer$<$ T $>$}}
\doxysubsubsection{\texorpdfstring{addFrom()}{addFrom()}}
{\footnotesize\ttfamily \label{class_beam_1_1_audio_buffer_a816402cf170a0d8dc20adc5647828caf} 
template$<$typename T$>$ \\
void \doxymbox{\hyperlink{class_beam_1_1_audio_buffer}{Beam::\+\+Audio\+Buffer}}$<$ T $>$::\+add\+From (\begin{DoxyParamCaption}\item[{int}]{dest\+Channel}{, }\item[{int}]{dest\+Start\+Sample}{, }\item[{const \doxymbox{\hyperlink{class_beam_1_1_audio_buffer_a0f248fe3e687d32f730a34f9acdb1549}{Audio\+Buffer}}$<$ T $>$ \&}]{source}{, }\item[{int}]{source\+Channel}{, }\item[{int}]{source\+Start\+Sample}{, }\item[{int}]{num\+Samples}{, }\item[{T}]{gain}{ = {\ttfamily static\+\_\+cast$<$T$>$(1)}}\end{DoxyParamCaption})}



Adds samples from another buffer to this one. 

\Hypertarget{class_beam_1_1_audio_buffer_a3eb7f6c54881a58dcbfdc3dc73042e67}\index{Beam::AudioBuffer$<$ T $>$@{Beam::AudioBuffer$<$ T $>$}!applyGain@{applyGain}}
\index{applyGain@{applyGain}!Beam::AudioBuffer$<$ T $>$@{Beam::AudioBuffer$<$ T $>$}}
\doxysubsubsection{\texorpdfstring{applyGain()}{applyGain()}\hspace{0.1cm}{\footnotesize\ttfamily [1/2]}}
{\footnotesize\ttfamily \label{class_beam_1_1_audio_buffer_a3eb7f6c54881a58dcbfdc3dc73042e67} 
template$<$typename T$>$ \\
void \doxymbox{\hyperlink{class_beam_1_1_audio_buffer}{Beam::\+\+Audio\+Buffer}}$<$ T $>$::\+apply\+Gain (\begin{DoxyParamCaption}\item[{int}]{channel}{, }\item[{int}]{start\+Sample}{, }\item[{int}]{num\+Samples}{, }\item[{T}]{gain}{}\end{DoxyParamCaption})}



Applies gain to a range of samples in a channel. 

\Hypertarget{class_beam_1_1_audio_buffer_a4d97c0664737f80daade25dfa6f8789d}\index{Beam::AudioBuffer$<$ T $>$@{Beam::AudioBuffer$<$ T $>$}!applyGain@{applyGain}}
\index{applyGain@{applyGain}!Beam::AudioBuffer$<$ T $>$@{Beam::AudioBuffer$<$ T $>$}}
\doxysubsubsection{\texorpdfstring{applyGain()}{applyGain()}\hspace{0.1cm}{\footnotesize\ttfamily [2/2]}}
{\footnotesize\ttfamily \label{class_beam_1_1_audio_buffer_a4d97c0664737f80daade25dfa6f8789d} 
template$<$typename T$>$ \\
void \doxymbox{\hyperlink{class_beam_1_1_audio_buffer}{Beam::\+\+Audio\+Buffer}}$<$ T $>$::\+apply\+Gain (\begin{DoxyParamCaption}\item[{int}]{channel}{, }\item[{T}]{gain}{}\end{DoxyParamCaption})}



Applies gain to all samples in a channel. 

\Hypertarget{class_beam_1_1_audio_buffer_a20ec61008ce4c79ecb36ab05b5ba35e2}\index{Beam::AudioBuffer$<$ T $>$@{Beam::AudioBuffer$<$ T $>$}!clear@{clear}}
\index{clear@{clear}!Beam::AudioBuffer$<$ T $>$@{Beam::AudioBuffer$<$ T $>$}}
\doxysubsubsection{\texorpdfstring{clear()}{clear()}\hspace{0.1cm}{\footnotesize\ttfamily [1/2]}}
{\footnotesize\ttfamily \label{class_beam_1_1_audio_buffer_a20ec61008ce4c79ecb36ab05b5ba35e2} 
template$<$typename T$>$ \\
void \doxymbox{\hyperlink{class_beam_1_1_audio_buffer}{Beam::\+\+Audio\+Buffer}}$<$ T $>$::\+clear (\begin{DoxyParamCaption}{}{}\end{DoxyParamCaption})}



Clears all the samples in all channels. 

\Hypertarget{class_beam_1_1_audio_buffer_a06f7df144e2eeb313e9172bc724d4682}\index{Beam::AudioBuffer$<$ T $>$@{Beam::AudioBuffer$<$ T $>$}!clear@{clear}}
\index{clear@{clear}!Beam::AudioBuffer$<$ T $>$@{Beam::AudioBuffer$<$ T $>$}}
\doxysubsubsection{\texorpdfstring{clear()}{clear()}\hspace{0.1cm}{\footnotesize\ttfamily [2/2]}}
{\footnotesize\ttfamily \label{class_beam_1_1_audio_buffer_a06f7df144e2eeb313e9172bc724d4682} 
template$<$typename T$>$ \\
void \doxymbox{\hyperlink{class_beam_1_1_audio_buffer}{Beam::\+\+Audio\+Buffer}}$<$ T $>$::\+clear (\begin{DoxyParamCaption}\item[{int}]{start\+Sample}{, }\item[{int}]{num\+Samples}{}\end{DoxyParamCaption})}



Clears a specified region in all channels. 

\Hypertarget{class_beam_1_1_audio_buffer_aaa4a7663a86843155ad42202561568be}\index{Beam::AudioBuffer$<$ T $>$@{Beam::AudioBuffer$<$ T $>$}!copyFrom@{copyFrom}}
\index{copyFrom@{copyFrom}!Beam::AudioBuffer$<$ T $>$@{Beam::AudioBuffer$<$ T $>$}}
\doxysubsubsection{\texorpdfstring{copyFrom()}{copyFrom()}}
{\footnotesize\ttfamily \label{class_beam_1_1_audio_buffer_aaa4a7663a86843155ad42202561568be} 
template$<$typename T$>$ \\
void \doxymbox{\hyperlink{class_beam_1_1_audio_buffer}{Beam::\+\+Audio\+Buffer}}$<$ T $>$::\+copy\+From (\begin{DoxyParamCaption}\item[{int}]{dest\+Channel}{, }\item[{int}]{dest\+Start\+Sample}{, }\item[{const \doxymbox{\hyperlink{class_beam_1_1_audio_buffer_a0f248fe3e687d32f730a34f9acdb1549}{Audio\+Buffer}}$<$ T $>$ \&}]{source}{, }\item[{int}]{source\+Channel}{, }\item[{int}]{source\+Start\+Sample}{, }\item[{int}]{num\+Samples}{}\end{DoxyParamCaption})}



Copies data from another buffer. 

\Hypertarget{class_beam_1_1_audio_buffer_aca9fcdbc6e425179714230182cfdde00}\index{Beam::AudioBuffer$<$ T $>$@{Beam::AudioBuffer$<$ T $>$}!getMagnitude@{getMagnitude}}
\index{getMagnitude@{getMagnitude}!Beam::AudioBuffer$<$ T $>$@{Beam::AudioBuffer$<$ T $>$}}
\doxysubsubsection{\texorpdfstring{getMagnitude()}{getMagnitude()}\hspace{0.1cm}{\footnotesize\ttfamily [1/2]}}
{\footnotesize\ttfamily \label{class_beam_1_1_audio_buffer_aca9fcdbc6e425179714230182cfdde00} 
template$<$typename T$>$ \\
T \doxymbox{\hyperlink{class_beam_1_1_audio_buffer}{Beam::\+\+Audio\+Buffer}}$<$ T $>$::\+get\+Magnitude (\begin{DoxyParamCaption}{}{}\end{DoxyParamCaption}) const}



Finds the highest absolute value in the entire buffer. 

\Hypertarget{class_beam_1_1_audio_buffer_ac58b58ad1ea4a33b74a6c9b633395763}\index{Beam::AudioBuffer$<$ T $>$@{Beam::AudioBuffer$<$ T $>$}!getMagnitude@{getMagnitude}}
\index{getMagnitude@{getMagnitude}!Beam::AudioBuffer$<$ T $>$@{Beam::AudioBuffer$<$ T $>$}}
\doxysubsubsection{\texorpdfstring{getMagnitude()}{getMagnitude()}\hspace{0.1cm}{\footnotesize\ttfamily [2/2]}}
{\footnotesize\ttfamily \label{class_beam_1_1_audio_buffer_ac58b58ad1ea4a33b74a6c9b633395763} 
template$<$typename T$>$ \\
T \doxymbox{\hyperlink{class_beam_1_1_audio_buffer}{Beam::\+\+Audio\+Buffer}}$<$ T $>$::\+get\+Magnitude (\begin{DoxyParamCaption}\item[{int}]{channel}{, }\item[{int}]{start\+Sample}{, }\item[{int}]{num\+Samples}{}\end{DoxyParamCaption}) const}



Finds the highest absolute value in a channel. 

\Hypertarget{class_beam_1_1_audio_buffer_af7325c4466887942607fd6960c86cf5c}\index{Beam::AudioBuffer$<$ T $>$@{Beam::AudioBuffer$<$ T $>$}!getNumChannels@{getNumChannels}}
\index{getNumChannels@{getNumChannels}!Beam::AudioBuffer$<$ T $>$@{Beam::AudioBuffer$<$ T $>$}}
\doxysubsubsection{\texorpdfstring{getNumChannels()}{getNumChannels()}}
{\footnotesize\ttfamily \label{class_beam_1_1_audio_buffer_af7325c4466887942607fd6960c86cf5c} 
template$<$typename T$>$ \\
int \doxymbox{\hyperlink{class_beam_1_1_audio_buffer}{Beam::\+\+Audio\+Buffer}}$<$ T $>$::\+get\+Num\+Channels (\begin{DoxyParamCaption}{}{}\end{DoxyParamCaption}) const\hspace{0.3cm}{\ttfamily [inline]}}



Returns the number of channels. 

\Hypertarget{class_beam_1_1_audio_buffer_ad01215e6138cba40a8e3d813481fb263}\index{Beam::AudioBuffer$<$ T $>$@{Beam::AudioBuffer$<$ T $>$}!getNumSamples@{getNumSamples}}
\index{getNumSamples@{getNumSamples}!Beam::AudioBuffer$<$ T $>$@{Beam::AudioBuffer$<$ T $>$}}
\doxysubsubsection{\texorpdfstring{getNumSamples()}{getNumSamples()}}
{\footnotesize\ttfamily \label{class_beam_1_1_audio_buffer_ad01215e6138cba40a8e3d813481fb263} 
template$<$typename T$>$ \\
int \doxymbox{\hyperlink{class_beam_1_1_audio_buffer}{Beam::\+\+Audio\+Buffer}}$<$ T $>$::\+get\+Num\+Samples (\begin{DoxyParamCaption}{}{}\end{DoxyParamCaption}) const\hspace{0.3cm}{\ttfamily [inline]}}



Returns the number of samples per channel. 

\Hypertarget{class_beam_1_1_audio_buffer_a7eda29f4b0b43bab3aeba400c163dc17}\index{Beam::AudioBuffer$<$ T $>$@{Beam::AudioBuffer$<$ T $>$}!getReadPointer@{getReadPointer}}
\index{getReadPointer@{getReadPointer}!Beam::AudioBuffer$<$ T $>$@{Beam::AudioBuffer$<$ T $>$}}
\doxysubsubsection{\texorpdfstring{getReadPointer()}{getReadPointer()}}
{\footnotesize\ttfamily \label{class_beam_1_1_audio_buffer_a7eda29f4b0b43bab3aeba400c163dc17} 
template$<$typename T$>$ \\
const T \texorpdfstring{$\ast$}{*} \doxymbox{\hyperlink{class_beam_1_1_audio_buffer}{Beam::\+\+Audio\+Buffer}}$<$ T $>$::\+get\+Read\+Pointer (\begin{DoxyParamCaption}\item[{int}]{channel}{}\end{DoxyParamCaption}) const\hspace{0.3cm}{\ttfamily [inline]}}



Returns a read-\/only pointer to the samples in one of the channels. 

\Hypertarget{class_beam_1_1_audio_buffer_a4378ccd20308acf3bbcc4bf37aa061a3}\index{Beam::AudioBuffer$<$ T $>$@{Beam::AudioBuffer$<$ T $>$}!getWritePointer@{getWritePointer}}
\index{getWritePointer@{getWritePointer}!Beam::AudioBuffer$<$ T $>$@{Beam::AudioBuffer$<$ T $>$}}
\doxysubsubsection{\texorpdfstring{getWritePointer()}{getWritePointer()}}
{\footnotesize\ttfamily \label{class_beam_1_1_audio_buffer_a4378ccd20308acf3bbcc4bf37aa061a3} 
template$<$typename T$>$ \\
T \texorpdfstring{$\ast$}{*} \doxymbox{\hyperlink{class_beam_1_1_audio_buffer}{Beam::\+\+Audio\+Buffer}}$<$ T $>$::\+get\+Write\+Pointer (\begin{DoxyParamCaption}\item[{int}]{channel}{}\end{DoxyParamCaption})\hspace{0.3cm}{\ttfamily [inline]}}



Returns a pointer to the samples in one of the channels. 

\Hypertarget{class_beam_1_1_audio_buffer_ae63c0109aad9d7c8bb282af793eb5d6d}\index{Beam::AudioBuffer$<$ T $>$@{Beam::AudioBuffer$<$ T $>$}!setSize@{setSize}}
\index{setSize@{setSize}!Beam::AudioBuffer$<$ T $>$@{Beam::AudioBuffer$<$ T $>$}}
\doxysubsubsection{\texorpdfstring{setSize()}{setSize()}}
{\footnotesize\ttfamily \label{class_beam_1_1_audio_buffer_ae63c0109aad9d7c8bb282af793eb5d6d} 
template$<$typename T$>$ \\
void \doxymbox{\hyperlink{class_beam_1_1_audio_buffer}{Beam::\+\+Audio\+Buffer}}$<$ T $>$::\+set\+Size (\begin{DoxyParamCaption}\item[{int}]{num\+Channels}{, }\item[{int}]{num\+Samples}{, }\item[{bool}]{keep\+Existing\+Content}{ = {\ttfamily false}, }\item[{bool}]{clear\+Extra\+Space}{ = {\ttfamily false}, }\item[{bool}]{avoid\+Reallocating}{ = {\ttfamily false}}\end{DoxyParamCaption})}



Sets the size of the buffer. 



\label{doc-variable-members}
\Hypertarget{class_beam_1_1_audio_buffer_doc-variable-members}
\doxysubsection{Member Data Documentation}
\Hypertarget{class_beam_1_1_audio_buffer_a4df4309eb88acf14bc88079c5a71f448}\index{Beam::AudioBuffer$<$ T $>$@{Beam::AudioBuffer$<$ T $>$}!m\_data@{m\_data}}
\index{m\_data@{m\_data}!Beam::AudioBuffer$<$ T $>$@{Beam::AudioBuffer$<$ T $>$}}
\doxysubsubsection{\texorpdfstring{m\_data}{m\_data}}
{\footnotesize\ttfamily \label{class_beam_1_1_audio_buffer_a4df4309eb88acf14bc88079c5a71f448} 
template$<$typename T$>$ \\
std::\+vector$<$std::\+vector$<$T$>$ $>$ \doxymbox{\hyperlink{class_beam_1_1_audio_buffer}{Beam::\+\+Audio\+Buffer}}$<$ T $>$::\+m\+\_\+data\hspace{0.3cm}{\ttfamily [private]}}

\Hypertarget{class_beam_1_1_audio_buffer_a0d70b1b71af4dc940a64130dc82b0e7b}\index{Beam::AudioBuffer$<$ T $>$@{Beam::AudioBuffer$<$ T $>$}!m\_numChannels@{m\_numChannels}}
\index{m\_numChannels@{m\_numChannels}!Beam::AudioBuffer$<$ T $>$@{Beam::AudioBuffer$<$ T $>$}}
\doxysubsubsection{\texorpdfstring{m\_numChannels}{m\_numChannels}}
{\footnotesize\ttfamily \label{class_beam_1_1_audio_buffer_a0d70b1b71af4dc940a64130dc82b0e7b} 
template$<$typename T$>$ \\
int \doxymbox{\hyperlink{class_beam_1_1_audio_buffer}{Beam::\+\+Audio\+Buffer}}$<$ T $>$::\+m\+\_\+num\+Channels = 0\hspace{0.3cm}{\ttfamily [private]}}

\Hypertarget{class_beam_1_1_audio_buffer_ad50ee6e795c4721395773e9bcac934bf}\index{Beam::AudioBuffer$<$ T $>$@{Beam::AudioBuffer$<$ T $>$}!m\_numSamples@{m\_numSamples}}
\index{m\_numSamples@{m\_numSamples}!Beam::AudioBuffer$<$ T $>$@{Beam::AudioBuffer$<$ T $>$}}
\doxysubsubsection{\texorpdfstring{m\_numSamples}{m\_numSamples}}
{\footnotesize\ttfamily \label{class_beam_1_1_audio_buffer_ad50ee6e795c4721395773e9bcac934bf} 
template$<$typename T$>$ \\
int \doxymbox{\hyperlink{class_beam_1_1_audio_buffer}{Beam::\+\+Audio\+Buffer}}$<$ T $>$::\+m\+\_\+num\+Samples = 0\hspace{0.3cm}{\ttfamily [private]}}



The documentation for this class was generated from the following file:\+\begin{DoxyCompactItemize}
\item 
src/\+engine/\+\doxymbox{\hyperlink{audio__buffer_8hpp}{audio\+\_\+buffer.\+hpp}}\end{DoxyCompactItemize}

\doxysection{Beam::\+Audio\+Config\+View Class Reference}
\hypertarget{class_beam_1_1_audio_config_view}{}\label{class_beam_1_1_audio_config_view}\index{Beam::AudioConfigView@{Beam::AudioConfigView}}


{\ttfamily \+\#include $<$audio\+\_\+config\+\_\+view.\+hpp$>$}

Inheritance diagram for Beam::\+Audio\+Config\+View:\+\begin{figure}[H]
\begin{center}
\leavevmode
\includegraphics[height=2.000000cm]{class_beam_1_1_audio_config_view}
\end{center}
\end{figure}
\doxysubsubsection*{Public Member Functions}
\begin{DoxyCompactItemize}
\item 
\doxymbox{\hyperlink{class_beam_1_1_audio_config_view_afce70f191ccfe5ea9fdc5c6532e495a1}{Audio\+Config\+View}} (\doxymbox{\hyperlink{class_beam_1_1_audio_device_manager}{Audio\+Device\+Manager}} \texorpdfstring{$\ast$}{*}manager, \doxymbox{\hyperlink{class_beam_1_1_audio_engine}{Audio\+Engine}} \texorpdfstring{$\ast$}{*}engine)
\item 
void \doxymbox{\hyperlink{class_beam_1_1_audio_config_view_a675df5ea2944369623cce601bcbbe2f0}{refresh\+Devices}} ()
\item 
bool \doxymbox{\hyperlink{class_beam_1_1_audio_config_view_a7f30e01005fc003399b5c065a20ad972}{is\+Visible}} () const
\item 
void \doxymbox{\hyperlink{class_beam_1_1_audio_config_view_a84a4602b4abe0940696ea961dc4dbc9f}{render}} (\doxymbox{\hyperlink{class_beam_1_1_quad_batcher}{Quad\+Batcher}} \&batcher, float dt, float screenW, float screenH) override
\item 
bool \doxymbox{\hyperlink{class_beam_1_1_audio_config_view_a72c69a9574555c43904eca4d0e693c8b}{on\+Mouse\+Down}} (float x, float y, int button) override
\item 
void \doxymbox{\hyperlink{class_beam_1_1_audio_config_view_ab8f60e4f1d311d6041d1a7bb1fac3410}{set\+Visible}} (bool visible)
\item 
void \doxymbox{\hyperlink{class_beam_1_1_audio_config_view_a1655b8c737e88850051161417777c8d4}{apply}} ()
\end{DoxyCompactItemize}
\doxysubsection*{Public Member Functions inherited from \doxymbox{\hyperlink{class_beam_1_1_component}{Beam::\+\+Component}}}
\begin{DoxyCompactItemize}
\item 
virtual \doxymbox{\hyperlink{class_beam_1_1_component_af9d734d649978e027412a87bc54362cd}{\texorpdfstring{$\sim$}{\string~}\+Component}} ()=default
\item 
virtual void \doxymbox{\hyperlink{class_beam_1_1_component_ad3d3fb19d25b4371d07620567970a158}{update}} (float dt)
\item 
virtual bool \doxymbox{\hyperlink{class_beam_1_1_component_ae36b8e9d70e8f9a1b9ba81c23c54d5c8}{on\+Mouse\+Up}} (float x, float y, int button)
\item 
virtual bool \doxymbox{\hyperlink{class_beam_1_1_component_a9d8e5970783d315044277a1228659e6c}{on\+Mouse\+Move}} (float x, float y)
\item 
virtual bool \doxymbox{\hyperlink{class_beam_1_1_component_ab92e884903f8a621fcd57bc00a24b041}{on\+Mouse\+Wheel}} (float x, float y, float delta)
\item 
virtual void \doxymbox{\hyperlink{class_beam_1_1_component_a6865b1f22388af467bf6c789120fac05}{set\+Bounds}} (float x, float y, float w, float h)
\item 
const \doxymbox{\hyperlink{struct_beam_1_1_rect}{Rect}} \& \doxymbox{\hyperlink{class_beam_1_1_component_a5746dbc69d5b0adb4cffbcf920936d00}{get\+Bounds}} () const
\item 
void \doxymbox{\hyperlink{class_beam_1_1_component_a00d4e2dfa7703e59d6486852321dbdf1}{set\+Draggable}} (bool draggable)
\item 
void \doxymbox{\hyperlink{class_beam_1_1_component_aca7b02d1dddf7cd20378db9e3242fb84}{start\+Dragging}} (float x, float y)
\end{DoxyCompactItemize}
\doxysubsubsection*{Public Attributes}
\begin{DoxyCompactItemize}
\item 
std::\+function$<$ void()$>$ \doxymbox{\hyperlink{class_beam_1_1_audio_config_view_a0911e1d5228c4e9378dbedd6776c0a13}{on\+Config\+Changed}}
\end{DoxyCompactItemize}
\doxysubsubsection*{Private Attributes}
\begin{DoxyCompactItemize}
\item 
\doxymbox{\hyperlink{class_beam_1_1_audio_device_manager}{Audio\+Device\+Manager}} \texorpdfstring{$\ast$}{*} \doxymbox{\hyperlink{class_beam_1_1_audio_config_view_a3bb4e5289b74ba2d3fe607425a88aacd}{m\+\_\+manager}}
\item 
\doxymbox{\hyperlink{class_beam_1_1_audio_engine}{Audio\+Engine}} \texorpdfstring{$\ast$}{*} \doxymbox{\hyperlink{class_beam_1_1_audio_config_view_aa419698acbbc3f5aa203eebaff4d76dc}{m\+\_\+engine}}
\item 
std::\+vector$<$ \doxymbox{\hyperlink{struct_beam_1_1_audio_device_info}{Audio\+Device\+Info}} $>$ \doxymbox{\hyperlink{class_beam_1_1_audio_config_view_ac11caf4e0798119957ec5e61b92cc82b}{m\+\_\+output\+Devices}}
\item 
std::\+vector$<$ \doxymbox{\hyperlink{struct_beam_1_1_audio_device_info}{Audio\+Device\+Info}} $>$ \doxymbox{\hyperlink{class_beam_1_1_audio_config_view_a1189cc259b0409afebc874b157eb6a7b}{m\+\_\+input\+Devices}}
\item 
\doxymbox{\hyperlink{struct_beam_1_1_audio_device_setup}{Audio\+Device\+Setup}} \doxymbox{\hyperlink{class_beam_1_1_audio_config_view_adb09efd9b894a29cfce0e3dd8f7501a7}{m\+\_\+current\+Setup}}
\item 
bool \doxymbox{\hyperlink{class_beam_1_1_audio_config_view_aa395b9a67b489f91fb568cc52a7363bd}{m\+\_\+is\+Visible}} = false
\end{DoxyCompactItemize}
\doxysubsubsection*{Additional Inherited Members}
\doxysubsection*{Protected Attributes inherited from \doxymbox{\hyperlink{class_beam_1_1_component}{Beam::\+\+Component}}}
\begin{DoxyCompactItemize}
\item 
\doxymbox{\hyperlink{struct_beam_1_1_rect}{Rect}} \doxymbox{\hyperlink{class_beam_1_1_component_a4f1ec4a5fb168c39a6c18f958b2b1495}{m\+\_\+bounds}} \{0, 0, 0, 0\}
\item 
bool \doxymbox{\hyperlink{class_beam_1_1_component_adc07913aed6ddadf1c730e7b3bb599cf}{m\+\_\+is\+Visible}} = true
\item 
bool \doxymbox{\hyperlink{class_beam_1_1_component_a0bf77b204ae374a14b5a6d7e5a3c13c6}{m\+\_\+is\+Enabled}} = true
\item 
bool \doxymbox{\hyperlink{class_beam_1_1_component_a9646efcaa9540a26a387f5da9aae4bde}{m\+\_\+is\+Draggable}} = false
\item 
bool \doxymbox{\hyperlink{class_beam_1_1_component_ab03af9a9743acf040f38e3fb11f8dc14}{m\+\_\+is\+Dragging}} = false
\item 
float \doxymbox{\hyperlink{class_beam_1_1_component_a7110b2b9dc235f724bf4689569266a63}{m\+\_\+last\+MouseX}} = 0
\item 
float \doxymbox{\hyperlink{class_beam_1_1_component_a768931a0f51394bf011f821f6ed2efe9}{m\+\_\+last\+MouseY}} = 0
\end{DoxyCompactItemize}


\label{doc-constructors}
\Hypertarget{class_beam_1_1_audio_config_view_doc-constructors}
\doxysubsection{Constructor \& Destructor Documentation}
\Hypertarget{class_beam_1_1_audio_config_view_afce70f191ccfe5ea9fdc5c6532e495a1}\index{Beam::AudioConfigView@{Beam::AudioConfigView}!AudioConfigView@{AudioConfigView}}
\index{AudioConfigView@{AudioConfigView}!Beam::AudioConfigView@{Beam::AudioConfigView}}
\doxysubsubsection{\texorpdfstring{AudioConfigView()}{AudioConfigView()}}
{\footnotesize\ttfamily \label{class_beam_1_1_audio_config_view_afce70f191ccfe5ea9fdc5c6532e495a1} 
Beam::\+\+Audio\+Config\+View::\+\+Audio\+Config\+View (\begin{DoxyParamCaption}\item[{\doxymbox{\hyperlink{class_beam_1_1_audio_device_manager}{Audio\+Device\+Manager}} \texorpdfstring{$\ast$}{*}}]{manager}{, }\item[{\doxymbox{\hyperlink{class_beam_1_1_audio_engine}{Audio\+Engine}} \texorpdfstring{$\ast$}{*}}]{engine}{}\end{DoxyParamCaption})\hspace{0.3cm}{\ttfamily [inline]}}



\label{doc-func-members}
\Hypertarget{class_beam_1_1_audio_config_view_doc-func-members}
\doxysubsection{Member Function Documentation}
\Hypertarget{class_beam_1_1_audio_config_view_a1655b8c737e88850051161417777c8d4}\index{Beam::AudioConfigView@{Beam::AudioConfigView}!apply@{apply}}
\index{apply@{apply}!Beam::AudioConfigView@{Beam::AudioConfigView}}
\doxysubsubsection{\texorpdfstring{apply()}{apply()}}
{\footnotesize\ttfamily \label{class_beam_1_1_audio_config_view_a1655b8c737e88850051161417777c8d4} 
void Beam::\+\+Audio\+Config\+View::\+apply (\begin{DoxyParamCaption}{}{}\end{DoxyParamCaption})\hspace{0.3cm}{\ttfamily [inline]}}

\Hypertarget{class_beam_1_1_audio_config_view_a7f30e01005fc003399b5c065a20ad972}\index{Beam::AudioConfigView@{Beam::AudioConfigView}!isVisible@{isVisible}}
\index{isVisible@{isVisible}!Beam::AudioConfigView@{Beam::AudioConfigView}}
\doxysubsubsection{\texorpdfstring{isVisible()}{isVisible()}}
{\footnotesize\ttfamily \label{class_beam_1_1_audio_config_view_a7f30e01005fc003399b5c065a20ad972} 
bool Beam::\+\+Audio\+Config\+View::\+is\+Visible (\begin{DoxyParamCaption}{}{}\end{DoxyParamCaption}) const\hspace{0.3cm}{\ttfamily [inline]}}

\Hypertarget{class_beam_1_1_audio_config_view_a72c69a9574555c43904eca4d0e693c8b}\index{Beam::AudioConfigView@{Beam::AudioConfigView}!onMouseDown@{onMouseDown}}
\index{onMouseDown@{onMouseDown}!Beam::AudioConfigView@{Beam::AudioConfigView}}
\doxysubsubsection{\texorpdfstring{onMouseDown()}{onMouseDown()}}
{\footnotesize\ttfamily \label{class_beam_1_1_audio_config_view_a72c69a9574555c43904eca4d0e693c8b} 
bool Beam::\+\+Audio\+Config\+View::\+on\+Mouse\+Down (\begin{DoxyParamCaption}\item[{float}]{x}{, }\item[{float}]{y}{, }\item[{int}]{button}{}\end{DoxyParamCaption})\hspace{0.3cm}{\ttfamily [inline]}, {\ttfamily [override]}, {\ttfamily [virtual]}}



Reimplemented from \doxymbox{\hyperlink{class_beam_1_1_component_aec1da33d2d6e3d4e7dd6708309264e76}{Beam::\+\+Component}}.

\Hypertarget{class_beam_1_1_audio_config_view_a675df5ea2944369623cce601bcbbe2f0}\index{Beam::AudioConfigView@{Beam::AudioConfigView}!refreshDevices@{refreshDevices}}
\index{refreshDevices@{refreshDevices}!Beam::AudioConfigView@{Beam::AudioConfigView}}
\doxysubsubsection{\texorpdfstring{refreshDevices()}{refreshDevices()}}
{\footnotesize\ttfamily \label{class_beam_1_1_audio_config_view_a675df5ea2944369623cce601bcbbe2f0} 
void Beam::\+\+Audio\+Config\+View::\+refresh\+Devices (\begin{DoxyParamCaption}{}{}\end{DoxyParamCaption})\hspace{0.3cm}{\ttfamily [inline]}}

\Hypertarget{class_beam_1_1_audio_config_view_a84a4602b4abe0940696ea961dc4dbc9f}\index{Beam::AudioConfigView@{Beam::AudioConfigView}!render@{render}}
\index{render@{render}!Beam::AudioConfigView@{Beam::AudioConfigView}}
\doxysubsubsection{\texorpdfstring{render()}{render()}}
{\footnotesize\ttfamily \label{class_beam_1_1_audio_config_view_a84a4602b4abe0940696ea961dc4dbc9f} 
void Beam::\+\+Audio\+Config\+View::\+render (\begin{DoxyParamCaption}\item[{\doxymbox{\hyperlink{class_beam_1_1_quad_batcher}{Quad\+Batcher}} \&}]{batcher}{, }\item[{float}]{dt}{, }\item[{float}]{screenW}{, }\item[{float}]{screenH}{}\end{DoxyParamCaption})\hspace{0.3cm}{\ttfamily [inline]}, {\ttfamily [override]}, {\ttfamily [virtual]}}



Implements \doxymbox{\hyperlink{class_beam_1_1_component_acef3496a55f0d94c8678f6049dbaa7cd}{Beam::\+\+Component}}.

\Hypertarget{class_beam_1_1_audio_config_view_ab8f60e4f1d311d6041d1a7bb1fac3410}\index{Beam::AudioConfigView@{Beam::AudioConfigView}!setVisible@{setVisible}}
\index{setVisible@{setVisible}!Beam::AudioConfigView@{Beam::AudioConfigView}}
\doxysubsubsection{\texorpdfstring{setVisible()}{setVisible()}}
{\footnotesize\ttfamily \label{class_beam_1_1_audio_config_view_ab8f60e4f1d311d6041d1a7bb1fac3410} 
void Beam::\+\+Audio\+Config\+View::\+set\+Visible (\begin{DoxyParamCaption}\item[{bool}]{visible}{}\end{DoxyParamCaption})\hspace{0.3cm}{\ttfamily [inline]}}



\label{doc-variable-members}
\Hypertarget{class_beam_1_1_audio_config_view_doc-variable-members}
\doxysubsection{Member Data Documentation}
\Hypertarget{class_beam_1_1_audio_config_view_adb09efd9b894a29cfce0e3dd8f7501a7}\index{Beam::AudioConfigView@{Beam::AudioConfigView}!m\_currentSetup@{m\_currentSetup}}
\index{m\_currentSetup@{m\_currentSetup}!Beam::AudioConfigView@{Beam::AudioConfigView}}
\doxysubsubsection{\texorpdfstring{m\_currentSetup}{m\_currentSetup}}
{\footnotesize\ttfamily \label{class_beam_1_1_audio_config_view_adb09efd9b894a29cfce0e3dd8f7501a7} 
\doxymbox{\hyperlink{struct_beam_1_1_audio_device_setup}{Audio\+Device\+Setup}} Beam::\+\+Audio\+Config\+View::\+m\+\_\+current\+Setup\hspace{0.3cm}{\ttfamily [private]}}

\Hypertarget{class_beam_1_1_audio_config_view_aa419698acbbc3f5aa203eebaff4d76dc}\index{Beam::AudioConfigView@{Beam::AudioConfigView}!m\_engine@{m\_engine}}
\index{m\_engine@{m\_engine}!Beam::AudioConfigView@{Beam::AudioConfigView}}
\doxysubsubsection{\texorpdfstring{m\_engine}{m\_engine}}
{\footnotesize\ttfamily \label{class_beam_1_1_audio_config_view_aa419698acbbc3f5aa203eebaff4d76dc} 
\doxymbox{\hyperlink{class_beam_1_1_audio_engine}{Audio\+Engine}}\texorpdfstring{$\ast$}{*} Beam::\+\+Audio\+Config\+View::\+m\+\_\+engine\hspace{0.3cm}{\ttfamily [private]}}

\Hypertarget{class_beam_1_1_audio_config_view_a1189cc259b0409afebc874b157eb6a7b}\index{Beam::AudioConfigView@{Beam::AudioConfigView}!m\_inputDevices@{m\_inputDevices}}
\index{m\_inputDevices@{m\_inputDevices}!Beam::AudioConfigView@{Beam::AudioConfigView}}
\doxysubsubsection{\texorpdfstring{m\_inputDevices}{m\_inputDevices}}
{\footnotesize\ttfamily \label{class_beam_1_1_audio_config_view_a1189cc259b0409afebc874b157eb6a7b} 
std::\+vector$<$\doxymbox{\hyperlink{struct_beam_1_1_audio_device_info}{Audio\+Device\+Info}}$>$ Beam::\+\+Audio\+Config\+View::\+m\+\_\+input\+Devices\hspace{0.3cm}{\ttfamily [private]}}

\Hypertarget{class_beam_1_1_audio_config_view_aa395b9a67b489f91fb568cc52a7363bd}\index{Beam::AudioConfigView@{Beam::AudioConfigView}!m\_isVisible@{m\_isVisible}}
\index{m\_isVisible@{m\_isVisible}!Beam::AudioConfigView@{Beam::AudioConfigView}}
\doxysubsubsection{\texorpdfstring{m\_isVisible}{m\_isVisible}}
{\footnotesize\ttfamily \label{class_beam_1_1_audio_config_view_aa395b9a67b489f91fb568cc52a7363bd} 
bool Beam::\+\+Audio\+Config\+View::\+m\+\_\+is\+Visible = false\hspace{0.3cm}{\ttfamily [private]}}

\Hypertarget{class_beam_1_1_audio_config_view_a3bb4e5289b74ba2d3fe607425a88aacd}\index{Beam::AudioConfigView@{Beam::AudioConfigView}!m\_manager@{m\_manager}}
\index{m\_manager@{m\_manager}!Beam::AudioConfigView@{Beam::AudioConfigView}}
\doxysubsubsection{\texorpdfstring{m\_manager}{m\_manager}}
{\footnotesize\ttfamily \label{class_beam_1_1_audio_config_view_a3bb4e5289b74ba2d3fe607425a88aacd} 
\doxymbox{\hyperlink{class_beam_1_1_audio_device_manager}{Audio\+Device\+Manager}}\texorpdfstring{$\ast$}{*} Beam::\+\+Audio\+Config\+View::\+m\+\_\+manager\hspace{0.3cm}{\ttfamily [private]}}

\Hypertarget{class_beam_1_1_audio_config_view_ac11caf4e0798119957ec5e61b92cc82b}\index{Beam::AudioConfigView@{Beam::AudioConfigView}!m\_outputDevices@{m\_outputDevices}}
\index{m\_outputDevices@{m\_outputDevices}!Beam::AudioConfigView@{Beam::AudioConfigView}}
\doxysubsubsection{\texorpdfstring{m\_outputDevices}{m\_outputDevices}}
{\footnotesize\ttfamily \label{class_beam_1_1_audio_config_view_ac11caf4e0798119957ec5e61b92cc82b} 
std::\+vector$<$\doxymbox{\hyperlink{struct_beam_1_1_audio_device_info}{Audio\+Device\+Info}}$>$ Beam::\+\+Audio\+Config\+View::\+m\+\_\+output\+Devices\hspace{0.3cm}{\ttfamily [private]}}

\Hypertarget{class_beam_1_1_audio_config_view_a0911e1d5228c4e9378dbedd6776c0a13}\index{Beam::AudioConfigView@{Beam::AudioConfigView}!onConfigChanged@{onConfigChanged}}
\index{onConfigChanged@{onConfigChanged}!Beam::AudioConfigView@{Beam::AudioConfigView}}
\doxysubsubsection{\texorpdfstring{onConfigChanged}{onConfigChanged}}
{\footnotesize\ttfamily \label{class_beam_1_1_audio_config_view_a0911e1d5228c4e9378dbedd6776c0a13} 
std::\+function$<$void()$>$ Beam::\+\+Audio\+Config\+View::\+on\+Config\+Changed}



The documentation for this class was generated from the following file:\+\begin{DoxyCompactItemize}
\item 
src/\+interface/\+\doxymbox{\hyperlink{audio__config__view_8hpp}{audio\+\_\+config\+\_\+view.\+hpp}}\end{DoxyCompactItemize}

\doxysection{Beam::\+Audio\+Device\+Info Struct Reference}
\hypertarget{struct_beam_1_1_audio_device_info}{}\label{struct_beam_1_1_audio_device_info}\index{Beam::AudioDeviceInfo@{Beam::AudioDeviceInfo}}


{\ttfamily \+\#include $<$audio\+\_\+device\+\_\+manager.\+hpp$>$}

\doxysubsubsection*{Public Attributes}
\begin{DoxyCompactItemize}
\item 
std::\+string \doxymbox{\hyperlink{struct_beam_1_1_audio_device_info_aad650253fa12979446b87320f5c96ccc}{name}}
\item 
std::\+string \doxymbox{\hyperlink{struct_beam_1_1_audio_device_info_ac9c055b47d9dc9276745f3a0e3651c56}{device\+Id}}
\item 
int \doxymbox{\hyperlink{struct_beam_1_1_audio_device_info_a5c41d38a90ece0e9ad24005b74a9a7d3}{max\+Input\+Channels}} = 0
\item 
int \doxymbox{\hyperlink{struct_beam_1_1_audio_device_info_a11cf70a35302921ab90eefdc005a46c7}{max\+Output\+Channels}} = 0
\item 
std::\+vector$<$ double $>$ \doxymbox{\hyperlink{struct_beam_1_1_audio_device_info_a47e48b5e0836e3d00d3341eedaabd487}{sample\+Rates}}
\item 
std::\+vector$<$ int $>$ \doxymbox{\hyperlink{struct_beam_1_1_audio_device_info_a2bba9e6fc7df2635fb42ce12f479853e}{buffer\+Sizes}}
\end{DoxyCompactItemize}


\label{doc-variable-members}
\Hypertarget{struct_beam_1_1_audio_device_info_doc-variable-members}
\doxysubsection{Member Data Documentation}
\Hypertarget{struct_beam_1_1_audio_device_info_a2bba9e6fc7df2635fb42ce12f479853e}\index{Beam::AudioDeviceInfo@{Beam::AudioDeviceInfo}!bufferSizes@{bufferSizes}}
\index{bufferSizes@{bufferSizes}!Beam::AudioDeviceInfo@{Beam::AudioDeviceInfo}}
\doxysubsubsection{\texorpdfstring{bufferSizes}{bufferSizes}}
{\footnotesize\ttfamily \label{struct_beam_1_1_audio_device_info_a2bba9e6fc7df2635fb42ce12f479853e} 
std::\+vector$<$int$>$ Beam::\+\+Audio\+Device\+Info::\+buffer\+Sizes}

\Hypertarget{struct_beam_1_1_audio_device_info_ac9c055b47d9dc9276745f3a0e3651c56}\index{Beam::AudioDeviceInfo@{Beam::AudioDeviceInfo}!deviceId@{deviceId}}
\index{deviceId@{deviceId}!Beam::AudioDeviceInfo@{Beam::AudioDeviceInfo}}
\doxysubsubsection{\texorpdfstring{deviceId}{deviceId}}
{\footnotesize\ttfamily \label{struct_beam_1_1_audio_device_info_ac9c055b47d9dc9276745f3a0e3651c56} 
std::\+string Beam::\+\+Audio\+Device\+Info::\+device\+Id}

\Hypertarget{struct_beam_1_1_audio_device_info_a5c41d38a90ece0e9ad24005b74a9a7d3}\index{Beam::AudioDeviceInfo@{Beam::AudioDeviceInfo}!maxInputChannels@{maxInputChannels}}
\index{maxInputChannels@{maxInputChannels}!Beam::AudioDeviceInfo@{Beam::AudioDeviceInfo}}
\doxysubsubsection{\texorpdfstring{maxInputChannels}{maxInputChannels}}
{\footnotesize\ttfamily \label{struct_beam_1_1_audio_device_info_a5c41d38a90ece0e9ad24005b74a9a7d3} 
int Beam::\+\+Audio\+Device\+Info::\+max\+Input\+Channels = 0}

\Hypertarget{struct_beam_1_1_audio_device_info_a11cf70a35302921ab90eefdc005a46c7}\index{Beam::AudioDeviceInfo@{Beam::AudioDeviceInfo}!maxOutputChannels@{maxOutputChannels}}
\index{maxOutputChannels@{maxOutputChannels}!Beam::AudioDeviceInfo@{Beam::AudioDeviceInfo}}
\doxysubsubsection{\texorpdfstring{maxOutputChannels}{maxOutputChannels}}
{\footnotesize\ttfamily \label{struct_beam_1_1_audio_device_info_a11cf70a35302921ab90eefdc005a46c7} 
int Beam::\+\+Audio\+Device\+Info::\+max\+Output\+Channels = 0}

\Hypertarget{struct_beam_1_1_audio_device_info_aad650253fa12979446b87320f5c96ccc}\index{Beam::AudioDeviceInfo@{Beam::AudioDeviceInfo}!name@{name}}
\index{name@{name}!Beam::AudioDeviceInfo@{Beam::AudioDeviceInfo}}
\doxysubsubsection{\texorpdfstring{name}{name}}
{\footnotesize\ttfamily \label{struct_beam_1_1_audio_device_info_aad650253fa12979446b87320f5c96ccc} 
std::\+string Beam::\+\+Audio\+Device\+Info::\+name}

\Hypertarget{struct_beam_1_1_audio_device_info_a47e48b5e0836e3d00d3341eedaabd487}\index{Beam::AudioDeviceInfo@{Beam::AudioDeviceInfo}!sampleRates@{sampleRates}}
\index{sampleRates@{sampleRates}!Beam::AudioDeviceInfo@{Beam::AudioDeviceInfo}}
\doxysubsubsection{\texorpdfstring{sampleRates}{sampleRates}}
{\footnotesize\ttfamily \label{struct_beam_1_1_audio_device_info_a47e48b5e0836e3d00d3341eedaabd487} 
std::\+vector$<$double$>$ Beam::\+\+Audio\+Device\+Info::\+sample\+Rates}



The documentation for this struct was generated from the following file:\+\begin{DoxyCompactItemize}
\item 
src/\+engine/\+\doxymbox{\hyperlink{audio__device__manager_8hpp}{audio\+\_\+device\+\_\+manager.\+hpp}}\end{DoxyCompactItemize}

\doxysection{Beam::\+Audio\+Device\+Manager Class Reference}
\hypertarget{class_beam_1_1_audio_device_manager}{}\label{class_beam_1_1_audio_device_manager}\index{Beam::AudioDeviceManager@{Beam::AudioDeviceManager}}


Manages audio devices, similar to JUCE\textquotesingle{}s \doxylink{class_beam_1_1_audio_device_manager}{Audio\+Device\+Manager}.  




{\ttfamily \+\#include $<$audio\+\_\+device\+\_\+manager.\+hpp$>$}

\doxysubsubsection*{Public Member Functions}
\begin{DoxyCompactItemize}
\item 
\doxymbox{\hyperlink{class_beam_1_1_audio_device_manager_a9a04a7a0ffc1320123152d9b62c31a66}{Audio\+Device\+Manager}} ()
\item 
\doxymbox{\hyperlink{class_beam_1_1_audio_device_manager_a9b897b43315c9639a308175a2a031d61}{\texorpdfstring{$\sim$}{\string~}\+Audio\+Device\+Manager}} ()
\item 
int \doxymbox{\hyperlink{class_beam_1_1_audio_device_manager_ac2f1820ef1c285f957a22c01c309ce47}{initialise}} (int num\+Input\+Channels\+Needed, int num\+Output\+Channels\+Needed, const \doxymbox{\hyperlink{struct_beam_1_1_audio_device_setup}{Audio\+Device\+Setup}} \texorpdfstring{$\ast$}{*}preferred\+Setup=nullptr, bool select\+Default\+Device\+On\+Failure=true)
\begin{DoxyCompactList}\small\item\em Initializes the audio device manager. \end{DoxyCompactList}\item 
void \doxymbox{\hyperlink{class_beam_1_1_audio_device_manager_abbb9b08f8cf14312ae4c97805037fa26}{set\+Audio\+Callback}} (std::\+function$<$ void(float \texorpdfstring{$\ast$}{*}\texorpdfstring{$\ast$}{*}, float \texorpdfstring{$\ast$}{*}\texorpdfstring{$\ast$}{*}, int, int, int, double)$>$ callback)
\begin{DoxyCompactList}\small\item\em Sets the audio callback. \end{DoxyCompactList}\item 
std::\+vector$<$ \doxymbox{\hyperlink{struct_beam_1_1_audio_device_info}{Audio\+Device\+Info}} $>$ \doxymbox{\hyperlink{class_beam_1_1_audio_device_manager_a5c9a75f931952a669030f2f857ec7ad4}{get\+Available\+Output\+Devices}} () const
\begin{DoxyCompactList}\small\item\em Gets a list of available output devices. \end{DoxyCompactList}\item 
std::\+vector$<$ \doxymbox{\hyperlink{struct_beam_1_1_audio_device_info}{Audio\+Device\+Info}} $>$ \doxymbox{\hyperlink{class_beam_1_1_audio_device_manager_adb352085db8d068ce1f9377b5b45cf41}{get\+Available\+Input\+Devices}} () const
\begin{DoxyCompactList}\small\item\em Gets a list of available input devices. \end{DoxyCompactList}\item 
int \doxymbox{\hyperlink{class_beam_1_1_audio_device_manager_ae48b8b030b046121a00f66d451b53c2c}{set\+Current\+Audio\+Device}} (const std::\+string \&output\+Device\+Name, const std::\+string \&output\+Device\+Id, const std::\+string \&input\+Device\+Name, const std::\+string \&input\+Device\+Id, double sample\+Rate, int buffer\+Size)
\begin{DoxyCompactList}\small\item\em Sets the current audio device. \end{DoxyCompactList}\item 
\doxymbox{\hyperlink{struct_beam_1_1_audio_device_setup}{Audio\+Device\+Setup}} \doxymbox{\hyperlink{class_beam_1_1_audio_device_manager_a63164a22008735b5b36a2dc9c6c5d521}{get\+Current\+Device\+Setup}} () const
\begin{DoxyCompactList}\small\item\em Gets the current device setup. \end{DoxyCompactList}\item 
int \doxymbox{\hyperlink{class_beam_1_1_audio_device_manager_a0bb774a182b11b77fee7252be4070eae}{start\+Audio}} ()
\begin{DoxyCompactList}\small\item\em Starts the audio stream. \end{DoxyCompactList}\item 
void \doxymbox{\hyperlink{class_beam_1_1_audio_device_manager_a7acb3781fec8511e5aab3dc042c78c64}{stop\+Audio}} ()
\begin{DoxyCompactList}\small\item\em Stops the audio stream. \end{DoxyCompactList}\item 
bool \doxymbox{\hyperlink{class_beam_1_1_audio_device_manager_ad6dd75b4831e8122a4dea3973d870ec3}{is\+Audio\+Running}} () const
\begin{DoxyCompactList}\small\item\em Checks if audio is currently running. \end{DoxyCompactList}\item 
double \doxymbox{\hyperlink{class_beam_1_1_audio_device_manager_a73e3a8cea55888afb28ae26bee4bb097}{get\+Current\+Sample\+Rate}} () const
\begin{DoxyCompactList}\small\item\em Gets the current sample rate. \end{DoxyCompactList}\item 
int \doxymbox{\hyperlink{class_beam_1_1_audio_device_manager_ac0fe3ed1cb2a41923d9a425a698e5cf6}{get\+Current\+Buffer\+Size\+Samples}} () const
\begin{DoxyCompactList}\small\item\em Gets the current block size. \end{DoxyCompactList}\end{DoxyCompactItemize}
\doxysubsubsection*{Private Attributes}
\begin{DoxyCompactItemize}
\item 
\doxymbox{\hyperlink{struct_beam_1_1_audio_device_setup}{Audio\+Device\+Setup}} \doxymbox{\hyperlink{class_beam_1_1_audio_device_manager_ac6fd1663b68d22105f5a18ac01b11c7d}{m\+\_\+device\+Setup}}
\item 
bool \doxymbox{\hyperlink{class_beam_1_1_audio_device_manager_ae560d602be900a9725278496efb94453}{m\+\_\+is\+Running}} = false
\item 
std::\+function$<$ void(float \texorpdfstring{$\ast$}{*}\texorpdfstring{$\ast$}{*}, float \texorpdfstring{$\ast$}{*}\texorpdfstring{$\ast$}{*}, int, int, int, double)$>$ \doxymbox{\hyperlink{class_beam_1_1_audio_device_manager_a51b4a88bb21309c8e830310064672cfd}{m\+\_\+audio\+Callback}}
\end{DoxyCompactItemize}


\doxysubsection{Detailed Description}
Manages audio devices, similar to JUCE\textquotesingle{}s \doxylink{class_beam_1_1_audio_device_manager}{Audio\+Device\+Manager}. 

\label{doc-constructors}
\Hypertarget{class_beam_1_1_audio_device_manager_doc-constructors}
\doxysubsection{Constructor \& Destructor Documentation}
\Hypertarget{class_beam_1_1_audio_device_manager_a9a04a7a0ffc1320123152d9b62c31a66}\index{Beam::AudioDeviceManager@{Beam::AudioDeviceManager}!AudioDeviceManager@{AudioDeviceManager}}
\index{AudioDeviceManager@{AudioDeviceManager}!Beam::AudioDeviceManager@{Beam::AudioDeviceManager}}
\doxysubsubsection{\texorpdfstring{AudioDeviceManager()}{AudioDeviceManager()}}
{\footnotesize\ttfamily \label{class_beam_1_1_audio_device_manager_a9a04a7a0ffc1320123152d9b62c31a66} 
Beam::\+\+Audio\+Device\+Manager::\+\+Audio\+Device\+Manager (\begin{DoxyParamCaption}{}{}\end{DoxyParamCaption})}

\Hypertarget{class_beam_1_1_audio_device_manager_a9b897b43315c9639a308175a2a031d61}\index{Beam::AudioDeviceManager@{Beam::AudioDeviceManager}!````~AudioDeviceManager@{\texorpdfstring{$\sim$}{\string~}AudioDeviceManager}}
\index{````~AudioDeviceManager@{\texorpdfstring{$\sim$}{\string~}AudioDeviceManager}!Beam::AudioDeviceManager@{Beam::AudioDeviceManager}}
\doxysubsubsection{\texorpdfstring{\texorpdfstring{$\sim$}{\string~}AudioDeviceManager()}{\string~AudioDeviceManager()}}
{\footnotesize\ttfamily \label{class_beam_1_1_audio_device_manager_a9b897b43315c9639a308175a2a031d61} 
Beam::\+\+Audio\+Device\+Manager::\+\texorpdfstring{$\sim$}{\string~}\+Audio\+Device\+Manager (\begin{DoxyParamCaption}{}{}\end{DoxyParamCaption})}



\label{doc-func-members}
\Hypertarget{class_beam_1_1_audio_device_manager_doc-func-members}
\doxysubsection{Member Function Documentation}
\Hypertarget{class_beam_1_1_audio_device_manager_adb352085db8d068ce1f9377b5b45cf41}\index{Beam::AudioDeviceManager@{Beam::AudioDeviceManager}!getAvailableInputDevices@{getAvailableInputDevices}}
\index{getAvailableInputDevices@{getAvailableInputDevices}!Beam::AudioDeviceManager@{Beam::AudioDeviceManager}}
\doxysubsubsection{\texorpdfstring{getAvailableInputDevices()}{getAvailableInputDevices()}}
{\footnotesize\ttfamily \label{class_beam_1_1_audio_device_manager_adb352085db8d068ce1f9377b5b45cf41} 
std::\+vector$<$ \doxymbox{\hyperlink{struct_beam_1_1_audio_device_info}{Audio\+Device\+Info}} $>$ Beam::\+\+Audio\+Device\+Manager::\+get\+Available\+Input\+Devices (\begin{DoxyParamCaption}{}{}\end{DoxyParamCaption}) const}



Gets a list of available input devices. 

\Hypertarget{class_beam_1_1_audio_device_manager_a5c9a75f931952a669030f2f857ec7ad4}\index{Beam::AudioDeviceManager@{Beam::AudioDeviceManager}!getAvailableOutputDevices@{getAvailableOutputDevices}}
\index{getAvailableOutputDevices@{getAvailableOutputDevices}!Beam::AudioDeviceManager@{Beam::AudioDeviceManager}}
\doxysubsubsection{\texorpdfstring{getAvailableOutputDevices()}{getAvailableOutputDevices()}}
{\footnotesize\ttfamily \label{class_beam_1_1_audio_device_manager_a5c9a75f931952a669030f2f857ec7ad4} 
std::\+vector$<$ \doxymbox{\hyperlink{struct_beam_1_1_audio_device_info}{Audio\+Device\+Info}} $>$ Beam::\+\+Audio\+Device\+Manager::\+get\+Available\+Output\+Devices (\begin{DoxyParamCaption}{}{}\end{DoxyParamCaption}) const}



Gets a list of available output devices. 

\Hypertarget{class_beam_1_1_audio_device_manager_ac0fe3ed1cb2a41923d9a425a698e5cf6}\index{Beam::AudioDeviceManager@{Beam::AudioDeviceManager}!getCurrentBufferSizeSamples@{getCurrentBufferSizeSamples}}
\index{getCurrentBufferSizeSamples@{getCurrentBufferSizeSamples}!Beam::AudioDeviceManager@{Beam::AudioDeviceManager}}
\doxysubsubsection{\texorpdfstring{getCurrentBufferSizeSamples()}{getCurrentBufferSizeSamples()}}
{\footnotesize\ttfamily \label{class_beam_1_1_audio_device_manager_ac0fe3ed1cb2a41923d9a425a698e5cf6} 
int Beam::\+\+Audio\+Device\+Manager::\+get\+Current\+Buffer\+Size\+Samples (\begin{DoxyParamCaption}{}{}\end{DoxyParamCaption}) const}



Gets the current block size. 

\Hypertarget{class_beam_1_1_audio_device_manager_a63164a22008735b5b36a2dc9c6c5d521}\index{Beam::AudioDeviceManager@{Beam::AudioDeviceManager}!getCurrentDeviceSetup@{getCurrentDeviceSetup}}
\index{getCurrentDeviceSetup@{getCurrentDeviceSetup}!Beam::AudioDeviceManager@{Beam::AudioDeviceManager}}
\doxysubsubsection{\texorpdfstring{getCurrentDeviceSetup()}{getCurrentDeviceSetup()}}
{\footnotesize\ttfamily \label{class_beam_1_1_audio_device_manager_a63164a22008735b5b36a2dc9c6c5d521} 
\doxymbox{\hyperlink{struct_beam_1_1_audio_device_setup}{Audio\+Device\+Setup}} Beam::\+\+Audio\+Device\+Manager::\+get\+Current\+Device\+Setup (\begin{DoxyParamCaption}{}{}\end{DoxyParamCaption}) const}



Gets the current device setup. 

\Hypertarget{class_beam_1_1_audio_device_manager_a73e3a8cea55888afb28ae26bee4bb097}\index{Beam::AudioDeviceManager@{Beam::AudioDeviceManager}!getCurrentSampleRate@{getCurrentSampleRate}}
\index{getCurrentSampleRate@{getCurrentSampleRate}!Beam::AudioDeviceManager@{Beam::AudioDeviceManager}}
\doxysubsubsection{\texorpdfstring{getCurrentSampleRate()}{getCurrentSampleRate()}}
{\footnotesize\ttfamily \label{class_beam_1_1_audio_device_manager_a73e3a8cea55888afb28ae26bee4bb097} 
double Beam::\+\+Audio\+Device\+Manager::\+get\+Current\+Sample\+Rate (\begin{DoxyParamCaption}{}{}\end{DoxyParamCaption}) const}



Gets the current sample rate. 

\Hypertarget{class_beam_1_1_audio_device_manager_ac2f1820ef1c285f957a22c01c309ce47}\index{Beam::AudioDeviceManager@{Beam::AudioDeviceManager}!initialise@{initialise}}
\index{initialise@{initialise}!Beam::AudioDeviceManager@{Beam::AudioDeviceManager}}
\doxysubsubsection{\texorpdfstring{initialise()}{initialise()}}
{\footnotesize\ttfamily \label{class_beam_1_1_audio_device_manager_ac2f1820ef1c285f957a22c01c309ce47} 
int Beam::\+\+Audio\+Device\+Manager::\+initialise (\begin{DoxyParamCaption}\item[{int}]{num\+Input\+Channels\+Needed}{, }\item[{int}]{num\+Output\+Channels\+Needed}{, }\item[{const \doxymbox{\hyperlink{struct_beam_1_1_audio_device_setup}{Audio\+Device\+Setup}} \texorpdfstring{$\ast$}{*}}]{preferred\+Setup}{ = {\ttfamily nullptr}, }\item[{bool}]{select\+Default\+Device\+On\+Failure}{ = {\ttfamily true}}\end{DoxyParamCaption})}



Initializes the audio device manager. 

\Hypertarget{class_beam_1_1_audio_device_manager_ad6dd75b4831e8122a4dea3973d870ec3}\index{Beam::AudioDeviceManager@{Beam::AudioDeviceManager}!isAudioRunning@{isAudioRunning}}
\index{isAudioRunning@{isAudioRunning}!Beam::AudioDeviceManager@{Beam::AudioDeviceManager}}
\doxysubsubsection{\texorpdfstring{isAudioRunning()}{isAudioRunning()}}
{\footnotesize\ttfamily \label{class_beam_1_1_audio_device_manager_ad6dd75b4831e8122a4dea3973d870ec3} 
bool Beam::\+\+Audio\+Device\+Manager::\+is\+Audio\+Running (\begin{DoxyParamCaption}{}{}\end{DoxyParamCaption}) const}



Checks if audio is currently running. 

\Hypertarget{class_beam_1_1_audio_device_manager_abbb9b08f8cf14312ae4c97805037fa26}\index{Beam::AudioDeviceManager@{Beam::AudioDeviceManager}!setAudioCallback@{setAudioCallback}}
\index{setAudioCallback@{setAudioCallback}!Beam::AudioDeviceManager@{Beam::AudioDeviceManager}}
\doxysubsubsection{\texorpdfstring{setAudioCallback()}{setAudioCallback()}}
{\footnotesize\ttfamily \label{class_beam_1_1_audio_device_manager_abbb9b08f8cf14312ae4c97805037fa26} 
void Beam::\+\+Audio\+Device\+Manager::\+set\+Audio\+Callback (\begin{DoxyParamCaption}\item[{std::\+function$<$ void(float \texorpdfstring{$\ast$}{*}\texorpdfstring{$\ast$}{*}, float \texorpdfstring{$\ast$}{*}\texorpdfstring{$\ast$}{*}, int, int, int, double)$>$}]{callback}{}\end{DoxyParamCaption})}



Sets the audio callback. 

\Hypertarget{class_beam_1_1_audio_device_manager_ae48b8b030b046121a00f66d451b53c2c}\index{Beam::AudioDeviceManager@{Beam::AudioDeviceManager}!setCurrentAudioDevice@{setCurrentAudioDevice}}
\index{setCurrentAudioDevice@{setCurrentAudioDevice}!Beam::AudioDeviceManager@{Beam::AudioDeviceManager}}
\doxysubsubsection{\texorpdfstring{setCurrentAudioDevice()}{setCurrentAudioDevice()}}
{\footnotesize\ttfamily \label{class_beam_1_1_audio_device_manager_ae48b8b030b046121a00f66d451b53c2c} 
int Beam::\+\+Audio\+Device\+Manager::\+set\+Current\+Audio\+Device (\begin{DoxyParamCaption}\item[{const std::\+string \&}]{output\+Device\+Name}{, }\item[{const std::\+string \&}]{output\+Device\+Id}{, }\item[{const std::\+string \&}]{input\+Device\+Name}{, }\item[{const std::\+string \&}]{input\+Device\+Id}{, }\item[{double}]{sample\+Rate}{, }\item[{int}]{buffer\+Size}{}\end{DoxyParamCaption})}



Sets the current audio device. 

\Hypertarget{class_beam_1_1_audio_device_manager_a0bb774a182b11b77fee7252be4070eae}\index{Beam::AudioDeviceManager@{Beam::AudioDeviceManager}!startAudio@{startAudio}}
\index{startAudio@{startAudio}!Beam::AudioDeviceManager@{Beam::AudioDeviceManager}}
\doxysubsubsection{\texorpdfstring{startAudio()}{startAudio()}}
{\footnotesize\ttfamily \label{class_beam_1_1_audio_device_manager_a0bb774a182b11b77fee7252be4070eae} 
int Beam::\+\+Audio\+Device\+Manager::\+start\+Audio (\begin{DoxyParamCaption}{}{}\end{DoxyParamCaption})}



Starts the audio stream. 

\Hypertarget{class_beam_1_1_audio_device_manager_a7acb3781fec8511e5aab3dc042c78c64}\index{Beam::AudioDeviceManager@{Beam::AudioDeviceManager}!stopAudio@{stopAudio}}
\index{stopAudio@{stopAudio}!Beam::AudioDeviceManager@{Beam::AudioDeviceManager}}
\doxysubsubsection{\texorpdfstring{stopAudio()}{stopAudio()}}
{\footnotesize\ttfamily \label{class_beam_1_1_audio_device_manager_a7acb3781fec8511e5aab3dc042c78c64} 
void Beam::\+\+Audio\+Device\+Manager::\+stop\+Audio (\begin{DoxyParamCaption}{}{}\end{DoxyParamCaption})}



Stops the audio stream. 



\label{doc-variable-members}
\Hypertarget{class_beam_1_1_audio_device_manager_doc-variable-members}
\doxysubsection{Member Data Documentation}
\Hypertarget{class_beam_1_1_audio_device_manager_a51b4a88bb21309c8e830310064672cfd}\index{Beam::AudioDeviceManager@{Beam::AudioDeviceManager}!m\_audioCallback@{m\_audioCallback}}
\index{m\_audioCallback@{m\_audioCallback}!Beam::AudioDeviceManager@{Beam::AudioDeviceManager}}
\doxysubsubsection{\texorpdfstring{m\_audioCallback}{m\_audioCallback}}
{\footnotesize\ttfamily \label{class_beam_1_1_audio_device_manager_a51b4a88bb21309c8e830310064672cfd} 
std::\+function$<$void(float\texorpdfstring{$\ast$}{*}\texorpdfstring{$\ast$}{*}, float\texorpdfstring{$\ast$}{*}\texorpdfstring{$\ast$}{*}, int, int, int, double)$>$ Beam::\+\+Audio\+Device\+Manager::\+m\+\_\+audio\+Callback\hspace{0.3cm}{\ttfamily [private]}}

\Hypertarget{class_beam_1_1_audio_device_manager_ac6fd1663b68d22105f5a18ac01b11c7d}\index{Beam::AudioDeviceManager@{Beam::AudioDeviceManager}!m\_deviceSetup@{m\_deviceSetup}}
\index{m\_deviceSetup@{m\_deviceSetup}!Beam::AudioDeviceManager@{Beam::AudioDeviceManager}}
\doxysubsubsection{\texorpdfstring{m\_deviceSetup}{m\_deviceSetup}}
{\footnotesize\ttfamily \label{class_beam_1_1_audio_device_manager_ac6fd1663b68d22105f5a18ac01b11c7d} 
\doxymbox{\hyperlink{struct_beam_1_1_audio_device_setup}{Audio\+Device\+Setup}} Beam::\+\+Audio\+Device\+Manager::\+m\+\_\+device\+Setup\hspace{0.3cm}{\ttfamily [private]}}

\Hypertarget{class_beam_1_1_audio_device_manager_ae560d602be900a9725278496efb94453}\index{Beam::AudioDeviceManager@{Beam::AudioDeviceManager}!m\_isRunning@{m\_isRunning}}
\index{m\_isRunning@{m\_isRunning}!Beam::AudioDeviceManager@{Beam::AudioDeviceManager}}
\doxysubsubsection{\texorpdfstring{m\_isRunning}{m\_isRunning}}
{\footnotesize\ttfamily \label{class_beam_1_1_audio_device_manager_ae560d602be900a9725278496efb94453} 
bool Beam::\+\+Audio\+Device\+Manager::\+m\+\_\+is\+Running = false\hspace{0.3cm}{\ttfamily [private]}}



The documentation for this class was generated from the following files:\+\begin{DoxyCompactItemize}
\item 
src/\+engine/\+\doxymbox{\hyperlink{audio__device__manager_8hpp}{audio\+\_\+device\+\_\+manager.\+hpp}}\item 
src/\+engine/\+\doxymbox{\hyperlink{audio__device__manager_8cpp}{audio\+\_\+device\+\_\+manager.\+cpp}}\end{DoxyCompactItemize}

\doxysection{Beam::\+Audio\+Device\+Setup Struct Reference}
\hypertarget{struct_beam_1_1_audio_device_setup}{}\label{struct_beam_1_1_audio_device_setup}\index{Beam::AudioDeviceSetup@{Beam::AudioDeviceSetup}}


{\ttfamily \+\#include $<$audio\+\_\+device\+\_\+manager.\+hpp$>$}

\doxysubsubsection*{Public Attributes}
\begin{DoxyCompactItemize}
\item 
std::\+string \doxymbox{\hyperlink{struct_beam_1_1_audio_device_setup_a910039315978037aabf9d2b5719e1b16}{output\+Device\+Name}}
\item 
std::\+string \doxymbox{\hyperlink{struct_beam_1_1_audio_device_setup_a385ddcb28e546f205f5c1d16881c8982}{output\+Device\+Id}}
\item 
std::\+string \doxymbox{\hyperlink{struct_beam_1_1_audio_device_setup_ad13f73799686138a9b239aa0dc699d20}{input\+Device\+Name}}
\item 
std::\+string \doxymbox{\hyperlink{struct_beam_1_1_audio_device_setup_a7aa39a5536e8ff3165e6a676e8b77b66}{input\+Device\+Id}}
\item 
double \doxymbox{\hyperlink{struct_beam_1_1_audio_device_setup_a774d7a2a8c5ac18b4d5a5f69acb737ba}{sample\+Rate}} = 44100.\+0
\item 
int \doxymbox{\hyperlink{struct_beam_1_1_audio_device_setup_af93cb616e91a168437c7029dde2cca48}{block\+Size}} = 512
\item 
int \doxymbox{\hyperlink{struct_beam_1_1_audio_device_setup_ad242eac81acfd9ed549fef5346efd5b0}{input\+Channels}} = 2
\item 
int \doxymbox{\hyperlink{struct_beam_1_1_audio_device_setup_ac452465d89be8dea339f95edee0f9a00}{output\+Channels}} = 2
\end{DoxyCompactItemize}


\label{doc-variable-members}
\Hypertarget{struct_beam_1_1_audio_device_setup_doc-variable-members}
\doxysubsection{Member Data Documentation}
\Hypertarget{struct_beam_1_1_audio_device_setup_af93cb616e91a168437c7029dde2cca48}\index{Beam::AudioDeviceSetup@{Beam::AudioDeviceSetup}!blockSize@{blockSize}}
\index{blockSize@{blockSize}!Beam::AudioDeviceSetup@{Beam::AudioDeviceSetup}}
\doxysubsubsection{\texorpdfstring{blockSize}{blockSize}}
{\footnotesize\ttfamily \label{struct_beam_1_1_audio_device_setup_af93cb616e91a168437c7029dde2cca48} 
int Beam::\+\+Audio\+Device\+Setup::\+block\+Size = 512}

\Hypertarget{struct_beam_1_1_audio_device_setup_ad242eac81acfd9ed549fef5346efd5b0}\index{Beam::AudioDeviceSetup@{Beam::AudioDeviceSetup}!inputChannels@{inputChannels}}
\index{inputChannels@{inputChannels}!Beam::AudioDeviceSetup@{Beam::AudioDeviceSetup}}
\doxysubsubsection{\texorpdfstring{inputChannels}{inputChannels}}
{\footnotesize\ttfamily \label{struct_beam_1_1_audio_device_setup_ad242eac81acfd9ed549fef5346efd5b0} 
int Beam::\+\+Audio\+Device\+Setup::\+input\+Channels = 2}

\Hypertarget{struct_beam_1_1_audio_device_setup_a7aa39a5536e8ff3165e6a676e8b77b66}\index{Beam::AudioDeviceSetup@{Beam::AudioDeviceSetup}!inputDeviceId@{inputDeviceId}}
\index{inputDeviceId@{inputDeviceId}!Beam::AudioDeviceSetup@{Beam::AudioDeviceSetup}}
\doxysubsubsection{\texorpdfstring{inputDeviceId}{inputDeviceId}}
{\footnotesize\ttfamily \label{struct_beam_1_1_audio_device_setup_a7aa39a5536e8ff3165e6a676e8b77b66} 
std::\+string Beam::\+\+Audio\+Device\+Setup::\+input\+Device\+Id}

\Hypertarget{struct_beam_1_1_audio_device_setup_ad13f73799686138a9b239aa0dc699d20}\index{Beam::AudioDeviceSetup@{Beam::AudioDeviceSetup}!inputDeviceName@{inputDeviceName}}
\index{inputDeviceName@{inputDeviceName}!Beam::AudioDeviceSetup@{Beam::AudioDeviceSetup}}
\doxysubsubsection{\texorpdfstring{inputDeviceName}{inputDeviceName}}
{\footnotesize\ttfamily \label{struct_beam_1_1_audio_device_setup_ad13f73799686138a9b239aa0dc699d20} 
std::\+string Beam::\+\+Audio\+Device\+Setup::\+input\+Device\+Name}

\Hypertarget{struct_beam_1_1_audio_device_setup_ac452465d89be8dea339f95edee0f9a00}\index{Beam::AudioDeviceSetup@{Beam::AudioDeviceSetup}!outputChannels@{outputChannels}}
\index{outputChannels@{outputChannels}!Beam::AudioDeviceSetup@{Beam::AudioDeviceSetup}}
\doxysubsubsection{\texorpdfstring{outputChannels}{outputChannels}}
{\footnotesize\ttfamily \label{struct_beam_1_1_audio_device_setup_ac452465d89be8dea339f95edee0f9a00} 
int Beam::\+\+Audio\+Device\+Setup::\+output\+Channels = 2}

\Hypertarget{struct_beam_1_1_audio_device_setup_a385ddcb28e546f205f5c1d16881c8982}\index{Beam::AudioDeviceSetup@{Beam::AudioDeviceSetup}!outputDeviceId@{outputDeviceId}}
\index{outputDeviceId@{outputDeviceId}!Beam::AudioDeviceSetup@{Beam::AudioDeviceSetup}}
\doxysubsubsection{\texorpdfstring{outputDeviceId}{outputDeviceId}}
{\footnotesize\ttfamily \label{struct_beam_1_1_audio_device_setup_a385ddcb28e546f205f5c1d16881c8982} 
std::\+string Beam::\+\+Audio\+Device\+Setup::\+output\+Device\+Id}

\Hypertarget{struct_beam_1_1_audio_device_setup_a910039315978037aabf9d2b5719e1b16}\index{Beam::AudioDeviceSetup@{Beam::AudioDeviceSetup}!outputDeviceName@{outputDeviceName}}
\index{outputDeviceName@{outputDeviceName}!Beam::AudioDeviceSetup@{Beam::AudioDeviceSetup}}
\doxysubsubsection{\texorpdfstring{outputDeviceName}{outputDeviceName}}
{\footnotesize\ttfamily \label{struct_beam_1_1_audio_device_setup_a910039315978037aabf9d2b5719e1b16} 
std::\+string Beam::\+\+Audio\+Device\+Setup::\+output\+Device\+Name}

\Hypertarget{struct_beam_1_1_audio_device_setup_a774d7a2a8c5ac18b4d5a5f69acb737ba}\index{Beam::AudioDeviceSetup@{Beam::AudioDeviceSetup}!sampleRate@{sampleRate}}
\index{sampleRate@{sampleRate}!Beam::AudioDeviceSetup@{Beam::AudioDeviceSetup}}
\doxysubsubsection{\texorpdfstring{sampleRate}{sampleRate}}
{\footnotesize\ttfamily \label{struct_beam_1_1_audio_device_setup_a774d7a2a8c5ac18b4d5a5f69acb737ba} 
double Beam::\+\+Audio\+Device\+Setup::\+sample\+Rate = 44100.\+0}



The documentation for this struct was generated from the following file:\+\begin{DoxyCompactItemize}
\item 
src/\+engine/\+\doxymbox{\hyperlink{audio__device__manager_8hpp}{audio\+\_\+device\+\_\+manager.\+hpp}}\end{DoxyCompactItemize}

\doxysection{Beam::\+Audio\+Module Class Reference}
\hypertarget{class_beam_1_1_audio_module}{}\label{class_beam_1_1_audio_module}\index{Beam::AudioModule@{Beam::AudioModule}}


{\ttfamily \+\#include $<$audio\+\_\+module.\+hpp$>$}

Inheritance diagram for Beam::\+Audio\+Module:\+\begin{figure}[H]
\begin{center}
\leavevmode
\includegraphics[height=2.270270cm]{class_beam_1_1_audio_module}
\end{center}
\end{figure}
\doxysubsubsection*{Public Member Functions}
\begin{DoxyCompactItemize}
\item 
\doxymbox{\hyperlink{class_beam_1_1_audio_module_a409c75189798146d7f556b1d50f4ba98}{Audio\+Module}} (std::\+shared\+\_\+ptr$<$ \doxymbox{\hyperlink{class_beam_1_1_flux_node}{Flux\+Node}} $>$ node, size\+\_\+t node\+Id, float x, float y)
\item 
size\+\_\+t \doxymbox{\hyperlink{class_beam_1_1_audio_module_a7fc32b3bfadec2badbfd067d2f37da96}{get\+Node\+Id}} () const
\item 
void \doxymbox{\hyperlink{class_beam_1_1_audio_module_a655aa14548b3d4e977294a9bdaafa879}{auto\+Generate\+UI}} ()
\item 
void \doxymbox{\hyperlink{class_beam_1_1_audio_module_a75c0758091a1cec0871134babb541135}{set\+Bounds}} (float x, float y, float w, float h) override
\item 
void \doxymbox{\hyperlink{class_beam_1_1_audio_module_a0e43bace4dcd9eb159ff2237b0c7c3fd}{render}} (\doxymbox{\hyperlink{class_beam_1_1_quad_batcher}{Quad\+Batcher}} \&batcher, float dt, float screenW, float screenH) override
\item 
bool \doxymbox{\hyperlink{class_beam_1_1_audio_module_adddaa58e40512d782c2d902917491499}{on\+Mouse\+Down}} (float x, float y, int button) override
\item 
bool \doxymbox{\hyperlink{class_beam_1_1_audio_module_a4ac204a52c61e603ae6a51daa610bc19}{on\+Mouse\+Up}} (float x, float y, int button) override
\item 
bool \doxymbox{\hyperlink{class_beam_1_1_audio_module_aff90c84092de907409b84eed2c46cbe2}{on\+Mouse\+Move}} (float x, float y) override
\item 
void \doxymbox{\hyperlink{class_beam_1_1_audio_module_a68a3d4bef3787a290f9a313b975f7925}{add\+Child}} (std::\+shared\+\_\+ptr$<$ \doxymbox{\hyperlink{class_beam_1_1_component}{Component}} $>$ child)
\item 
std::\+shared\+\_\+ptr$<$ \doxymbox{\hyperlink{class_beam_1_1_port}{Port}} $>$ \doxymbox{\hyperlink{class_beam_1_1_audio_module_a0d0da8bdbfb3a2355994bba086e0e721}{get\+Input\+Port}} ()
\item 
std::\+shared\+\_\+ptr$<$ \doxymbox{\hyperlink{class_beam_1_1_port}{Port}} $>$ \doxymbox{\hyperlink{class_beam_1_1_audio_module_a75e91aa2e7da1c9a6d31f4c90915403a}{get\+Output\+Port}} ()
\end{DoxyCompactItemize}
\doxysubsection*{Public Member Functions inherited from \doxymbox{\hyperlink{class_beam_1_1_component}{Beam::\+\+Component}}}
\begin{DoxyCompactItemize}
\item 
virtual \doxymbox{\hyperlink{class_beam_1_1_component_af9d734d649978e027412a87bc54362cd}{\texorpdfstring{$\sim$}{\string~}\+Component}} ()=default
\item 
virtual void \doxymbox{\hyperlink{class_beam_1_1_component_ad3d3fb19d25b4371d07620567970a158}{update}} (float dt)
\item 
virtual bool \doxymbox{\hyperlink{class_beam_1_1_component_ab92e884903f8a621fcd57bc00a24b041}{on\+Mouse\+Wheel}} (float x, float y, float delta)
\item 
const \doxymbox{\hyperlink{struct_beam_1_1_rect}{Rect}} \& \doxymbox{\hyperlink{class_beam_1_1_component_a5746dbc69d5b0adb4cffbcf920936d00}{get\+Bounds}} () const
\item 
void \doxymbox{\hyperlink{class_beam_1_1_component_a00d4e2dfa7703e59d6486852321dbdf1}{set\+Draggable}} (bool draggable)
\item 
void \doxymbox{\hyperlink{class_beam_1_1_component_aca7b02d1dddf7cd20378db9e3242fb84}{start\+Dragging}} (float x, float y)
\end{DoxyCompactItemize}
\doxysubsubsection*{Public Attributes}
\begin{DoxyCompactItemize}
\item 
std::\+function$<$ void(\doxymbox{\hyperlink{class_beam_1_1_audio_module_a409c75189798146d7f556b1d50f4ba98}{Audio\+Module}} \texorpdfstring{$\ast$}{*})$>$ \doxymbox{\hyperlink{class_beam_1_1_audio_module_af95229e9824037c0c9916cc04bf67e90}{on\+Delete\+Requested}}
\end{DoxyCompactItemize}
\doxysubsubsection*{Protected Attributes}
\begin{DoxyCompactItemize}
\item 
std::\+vector$<$ std::\+shared\+\_\+ptr$<$ \doxymbox{\hyperlink{class_beam_1_1_component}{Component}} $>$ $>$ \doxymbox{\hyperlink{class_beam_1_1_audio_module_a1af200892e351f910e6447a50913a560}{m\+\_\+children}}
\end{DoxyCompactItemize}
\doxysubsection*{Protected Attributes inherited from \doxymbox{\hyperlink{class_beam_1_1_component}{Beam::\+\+Component}}}
\begin{DoxyCompactItemize}
\item 
\doxymbox{\hyperlink{struct_beam_1_1_rect}{Rect}} \doxymbox{\hyperlink{class_beam_1_1_component_a4f1ec4a5fb168c39a6c18f958b2b1495}{m\+\_\+bounds}} \{0, 0, 0, 0\}
\item 
bool \doxymbox{\hyperlink{class_beam_1_1_component_adc07913aed6ddadf1c730e7b3bb599cf}{m\+\_\+is\+Visible}} = true
\item 
bool \doxymbox{\hyperlink{class_beam_1_1_component_a0bf77b204ae374a14b5a6d7e5a3c13c6}{m\+\_\+is\+Enabled}} = true
\item 
bool \doxymbox{\hyperlink{class_beam_1_1_component_a9646efcaa9540a26a387f5da9aae4bde}{m\+\_\+is\+Draggable}} = false
\item 
bool \doxymbox{\hyperlink{class_beam_1_1_component_ab03af9a9743acf040f38e3fb11f8dc14}{m\+\_\+is\+Dragging}} = false
\item 
float \doxymbox{\hyperlink{class_beam_1_1_component_a7110b2b9dc235f724bf4689569266a63}{m\+\_\+last\+MouseX}} = 0
\item 
float \doxymbox{\hyperlink{class_beam_1_1_component_a768931a0f51394bf011f821f6ed2efe9}{m\+\_\+last\+MouseY}} = 0
\end{DoxyCompactItemize}
\doxysubsubsection*{Private Attributes}
\begin{DoxyCompactItemize}
\item 
std::\+shared\+\_\+ptr$<$ \doxymbox{\hyperlink{class_beam_1_1_flux_node}{Flux\+Node}} $>$ \doxymbox{\hyperlink{class_beam_1_1_audio_module_aafbc58a945b32ba82e3635a6d915f498}{m\+\_\+node}}
\item 
size\+\_\+t \doxymbox{\hyperlink{class_beam_1_1_audio_module_a9b8bb7bd8aaa48881475de44550013ad}{m\+\_\+node\+Id}}
\item 
std::\+string \doxymbox{\hyperlink{class_beam_1_1_audio_module_a1b3da6406c733f1a4b0e19b6d1cbcfb6}{m\+\_\+name}}
\item 
std::\+shared\+\_\+ptr$<$ \doxymbox{\hyperlink{class_beam_1_1_port}{Port}} $>$ \doxymbox{\hyperlink{class_beam_1_1_audio_module_a2e0f72cba2f16ff63cedee7d2417c0bf}{m\+\_\+input\+Port}}
\item 
std::\+shared\+\_\+ptr$<$ \doxymbox{\hyperlink{class_beam_1_1_port}{Port}} $>$ \doxymbox{\hyperlink{class_beam_1_1_audio_module_ab2959617ad001fc0bb8bc9ed9828a558}{m\+\_\+output\+Port}}
\item 
\doxymbox{\hyperlink{struct_beam_1_1_rect}{Rect}} \doxymbox{\hyperlink{class_beam_1_1_audio_module_acec6c6786886fc1b3c1ff8d01428d325}{m\+\_\+delete\+Btn\+Bounds}}
\item 
float \doxymbox{\hyperlink{class_beam_1_1_audio_module_ab93c8ba97f1b2f29113c31a52c0c36eb}{m\+\_\+scroll\+Timer}} = 0.\+0f
\end{DoxyCompactItemize}


\label{doc-constructors}
\Hypertarget{class_beam_1_1_audio_module_doc-constructors}
\doxysubsection{Constructor \& Destructor Documentation}
\Hypertarget{class_beam_1_1_audio_module_a409c75189798146d7f556b1d50f4ba98}\index{Beam::AudioModule@{Beam::AudioModule}!AudioModule@{AudioModule}}
\index{AudioModule@{AudioModule}!Beam::AudioModule@{Beam::AudioModule}}
\doxysubsubsection{\texorpdfstring{AudioModule()}{AudioModule()}}
{\footnotesize\ttfamily \label{class_beam_1_1_audio_module_a409c75189798146d7f556b1d50f4ba98} 
Beam::\+\+Audio\+Module::\+\+Audio\+Module (\begin{DoxyParamCaption}\item[{std::\+shared\+\_\+ptr$<$ \doxymbox{\hyperlink{class_beam_1_1_flux_node}{Flux\+Node}} $>$}]{node}{, }\item[{size\+\_\+t}]{node\+Id}{, }\item[{float}]{x}{, }\item[{float}]{y}{}\end{DoxyParamCaption})\hspace{0.3cm}{\ttfamily [inline]}}



\label{doc-func-members}
\Hypertarget{class_beam_1_1_audio_module_doc-func-members}
\doxysubsection{Member Function Documentation}
\Hypertarget{class_beam_1_1_audio_module_a68a3d4bef3787a290f9a313b975f7925}\index{Beam::AudioModule@{Beam::AudioModule}!addChild@{addChild}}
\index{addChild@{addChild}!Beam::AudioModule@{Beam::AudioModule}}
\doxysubsubsection{\texorpdfstring{addChild()}{addChild()}}
{\footnotesize\ttfamily \label{class_beam_1_1_audio_module_a68a3d4bef3787a290f9a313b975f7925} 
void Beam::\+\+Audio\+Module::\+add\+Child (\begin{DoxyParamCaption}\item[{std::\+shared\+\_\+ptr$<$ \doxymbox{\hyperlink{class_beam_1_1_component}{Component}} $>$}]{child}{}\end{DoxyParamCaption})\hspace{0.3cm}{\ttfamily [inline]}}

\Hypertarget{class_beam_1_1_audio_module_a655aa14548b3d4e977294a9bdaafa879}\index{Beam::AudioModule@{Beam::AudioModule}!autoGenerateUI@{autoGenerateUI}}
\index{autoGenerateUI@{autoGenerateUI}!Beam::AudioModule@{Beam::AudioModule}}
\doxysubsubsection{\texorpdfstring{autoGenerateUI()}{autoGenerateUI()}}
{\footnotesize\ttfamily \label{class_beam_1_1_audio_module_a655aa14548b3d4e977294a9bdaafa879} 
void Beam::\+\+Audio\+Module::\+auto\+Generate\+UI (\begin{DoxyParamCaption}{}{}\end{DoxyParamCaption})\hspace{0.3cm}{\ttfamily [inline]}}

\Hypertarget{class_beam_1_1_audio_module_a0d0da8bdbfb3a2355994bba086e0e721}\index{Beam::AudioModule@{Beam::AudioModule}!getInputPort@{getInputPort}}
\index{getInputPort@{getInputPort}!Beam::AudioModule@{Beam::AudioModule}}
\doxysubsubsection{\texorpdfstring{getInputPort()}{getInputPort()}}
{\footnotesize\ttfamily \label{class_beam_1_1_audio_module_a0d0da8bdbfb3a2355994bba086e0e721} 
std::\+shared\+\_\+ptr$<$ \doxymbox{\hyperlink{class_beam_1_1_port}{Port}} $>$ Beam::\+\+Audio\+Module::\+get\+Input\+Port (\begin{DoxyParamCaption}{}{}\end{DoxyParamCaption})\hspace{0.3cm}{\ttfamily [inline]}}

\Hypertarget{class_beam_1_1_audio_module_a7fc32b3bfadec2badbfd067d2f37da96}\index{Beam::AudioModule@{Beam::AudioModule}!getNodeId@{getNodeId}}
\index{getNodeId@{getNodeId}!Beam::AudioModule@{Beam::AudioModule}}
\doxysubsubsection{\texorpdfstring{getNodeId()}{getNodeId()}}
{\footnotesize\ttfamily \label{class_beam_1_1_audio_module_a7fc32b3bfadec2badbfd067d2f37da96} 
size\+\_\+t Beam::\+\+Audio\+Module::\+get\+Node\+Id (\begin{DoxyParamCaption}{}{}\end{DoxyParamCaption}) const\hspace{0.3cm}{\ttfamily [inline]}}

\Hypertarget{class_beam_1_1_audio_module_a75e91aa2e7da1c9a6d31f4c90915403a}\index{Beam::AudioModule@{Beam::AudioModule}!getOutputPort@{getOutputPort}}
\index{getOutputPort@{getOutputPort}!Beam::AudioModule@{Beam::AudioModule}}
\doxysubsubsection{\texorpdfstring{getOutputPort()}{getOutputPort()}}
{\footnotesize\ttfamily \label{class_beam_1_1_audio_module_a75e91aa2e7da1c9a6d31f4c90915403a} 
std::\+shared\+\_\+ptr$<$ \doxymbox{\hyperlink{class_beam_1_1_port}{Port}} $>$ Beam::\+\+Audio\+Module::\+get\+Output\+Port (\begin{DoxyParamCaption}{}{}\end{DoxyParamCaption})\hspace{0.3cm}{\ttfamily [inline]}}

\Hypertarget{class_beam_1_1_audio_module_adddaa58e40512d782c2d902917491499}\index{Beam::AudioModule@{Beam::AudioModule}!onMouseDown@{onMouseDown}}
\index{onMouseDown@{onMouseDown}!Beam::AudioModule@{Beam::AudioModule}}
\doxysubsubsection{\texorpdfstring{onMouseDown()}{onMouseDown()}}
{\footnotesize\ttfamily \label{class_beam_1_1_audio_module_adddaa58e40512d782c2d902917491499} 
bool Beam::\+\+Audio\+Module::\+on\+Mouse\+Down (\begin{DoxyParamCaption}\item[{float}]{x}{, }\item[{float}]{y}{, }\item[{int}]{button}{}\end{DoxyParamCaption})\hspace{0.3cm}{\ttfamily [inline]}, {\ttfamily [override]}, {\ttfamily [virtual]}}



Reimplemented from \doxymbox{\hyperlink{class_beam_1_1_component_aec1da33d2d6e3d4e7dd6708309264e76}{Beam::\+\+Component}}.



Reimplemented in \doxymbox{\hyperlink{class_beam_1_1_tape_reel_ad2fe7a4986075d4a3c6047f2e745460e}{Beam::\+\+Tape\+Reel}}.

\Hypertarget{class_beam_1_1_audio_module_aff90c84092de907409b84eed2c46cbe2}\index{Beam::AudioModule@{Beam::AudioModule}!onMouseMove@{onMouseMove}}
\index{onMouseMove@{onMouseMove}!Beam::AudioModule@{Beam::AudioModule}}
\doxysubsubsection{\texorpdfstring{onMouseMove()}{onMouseMove()}}
{\footnotesize\ttfamily \label{class_beam_1_1_audio_module_aff90c84092de907409b84eed2c46cbe2} 
bool Beam::\+\+Audio\+Module::\+on\+Mouse\+Move (\begin{DoxyParamCaption}\item[{float}]{x}{, }\item[{float}]{y}{}\end{DoxyParamCaption})\hspace{0.3cm}{\ttfamily [inline]}, {\ttfamily [override]}, {\ttfamily [virtual]}}



Reimplemented from \doxymbox{\hyperlink{class_beam_1_1_component_a9d8e5970783d315044277a1228659e6c}{Beam::\+\+Component}}.

\Hypertarget{class_beam_1_1_audio_module_a4ac204a52c61e603ae6a51daa610bc19}\index{Beam::AudioModule@{Beam::AudioModule}!onMouseUp@{onMouseUp}}
\index{onMouseUp@{onMouseUp}!Beam::AudioModule@{Beam::AudioModule}}
\doxysubsubsection{\texorpdfstring{onMouseUp()}{onMouseUp()}}
{\footnotesize\ttfamily \label{class_beam_1_1_audio_module_a4ac204a52c61e603ae6a51daa610bc19} 
bool Beam::\+\+Audio\+Module::\+on\+Mouse\+Up (\begin{DoxyParamCaption}\item[{float}]{x}{, }\item[{float}]{y}{, }\item[{int}]{button}{}\end{DoxyParamCaption})\hspace{0.3cm}{\ttfamily [inline]}, {\ttfamily [override]}, {\ttfamily [virtual]}}



Reimplemented from \doxymbox{\hyperlink{class_beam_1_1_component_ae36b8e9d70e8f9a1b9ba81c23c54d5c8}{Beam::\+\+Component}}.

\Hypertarget{class_beam_1_1_audio_module_a0e43bace4dcd9eb159ff2237b0c7c3fd}\index{Beam::AudioModule@{Beam::AudioModule}!render@{render}}
\index{render@{render}!Beam::AudioModule@{Beam::AudioModule}}
\doxysubsubsection{\texorpdfstring{render()}{render()}}
{\footnotesize\ttfamily \label{class_beam_1_1_audio_module_a0e43bace4dcd9eb159ff2237b0c7c3fd} 
void Beam::\+\+Audio\+Module::\+render (\begin{DoxyParamCaption}\item[{\doxymbox{\hyperlink{class_beam_1_1_quad_batcher}{Quad\+Batcher}} \&}]{batcher}{, }\item[{float}]{dt}{, }\item[{float}]{screenW}{, }\item[{float}]{screenH}{}\end{DoxyParamCaption})\hspace{0.3cm}{\ttfamily [inline]}, {\ttfamily [override]}, {\ttfamily [virtual]}}



Implements \doxymbox{\hyperlink{class_beam_1_1_component_acef3496a55f0d94c8678f6049dbaa7cd}{Beam::\+\+Component}}.



Reimplemented in \doxymbox{\hyperlink{class_beam_1_1_dynamics_module_a89e85fb3938bb6e5f1c764c52e5e6033}{Beam::\+\+Dynamics\+Module$<$ T $>$}}, \doxymbox{\hyperlink{class_beam_1_1_tape_reel_a3c7a82b43aed2a8f48528bf64c2cde8e}{Beam::\+\+Tape\+Reel}}, and \doxymbox{\hyperlink{class_beam_1_1_tube_compressor_u_i_a03af59494f87e40910fb9a38380def0a}{Beam::\+\+Tube\+Compressor\+UI}}.

\Hypertarget{class_beam_1_1_audio_module_a75c0758091a1cec0871134babb541135}\index{Beam::AudioModule@{Beam::AudioModule}!setBounds@{setBounds}}
\index{setBounds@{setBounds}!Beam::AudioModule@{Beam::AudioModule}}
\doxysubsubsection{\texorpdfstring{setBounds()}{setBounds()}}
{\footnotesize\ttfamily \label{class_beam_1_1_audio_module_a75c0758091a1cec0871134babb541135} 
void Beam::\+\+Audio\+Module::\+set\+Bounds (\begin{DoxyParamCaption}\item[{float}]{x}{, }\item[{float}]{y}{, }\item[{float}]{w}{, }\item[{float}]{h}{}\end{DoxyParamCaption})\hspace{0.3cm}{\ttfamily [inline]}, {\ttfamily [override]}, {\ttfamily [virtual]}}



Reimplemented from \doxymbox{\hyperlink{class_beam_1_1_component_a6865b1f22388af467bf6c789120fac05}{Beam::\+\+Component}}.



Reimplemented in \doxymbox{\hyperlink{class_beam_1_1_filter_module_a05cb71fbd7f6cd82181d4a2c81abc44a}{Beam::\+\+Filter\+Module}}.



\label{doc-variable-members}
\Hypertarget{class_beam_1_1_audio_module_doc-variable-members}
\doxysubsection{Member Data Documentation}
\Hypertarget{class_beam_1_1_audio_module_a1af200892e351f910e6447a50913a560}\index{Beam::AudioModule@{Beam::AudioModule}!m\_children@{m\_children}}
\index{m\_children@{m\_children}!Beam::AudioModule@{Beam::AudioModule}}
\doxysubsubsection{\texorpdfstring{m\_children}{m\_children}}
{\footnotesize\ttfamily \label{class_beam_1_1_audio_module_a1af200892e351f910e6447a50913a560} 
std::\+vector$<$std::\+shared\+\_\+ptr$<$\doxymbox{\hyperlink{class_beam_1_1_component}{Component}}$>$ $>$ Beam::\+\+Audio\+Module::\+m\+\_\+children\hspace{0.3cm}{\ttfamily [protected]}}

\Hypertarget{class_beam_1_1_audio_module_acec6c6786886fc1b3c1ff8d01428d325}\index{Beam::AudioModule@{Beam::AudioModule}!m\_deleteBtnBounds@{m\_deleteBtnBounds}}
\index{m\_deleteBtnBounds@{m\_deleteBtnBounds}!Beam::AudioModule@{Beam::AudioModule}}
\doxysubsubsection{\texorpdfstring{m\_deleteBtnBounds}{m\_deleteBtnBounds}}
{\footnotesize\ttfamily \label{class_beam_1_1_audio_module_acec6c6786886fc1b3c1ff8d01428d325} 
\doxymbox{\hyperlink{struct_beam_1_1_rect}{Rect}} Beam::\+\+Audio\+Module::\+m\+\_\+delete\+Btn\+Bounds\hspace{0.3cm}{\ttfamily [private]}}

\Hypertarget{class_beam_1_1_audio_module_a2e0f72cba2f16ff63cedee7d2417c0bf}\index{Beam::AudioModule@{Beam::AudioModule}!m\_inputPort@{m\_inputPort}}
\index{m\_inputPort@{m\_inputPort}!Beam::AudioModule@{Beam::AudioModule}}
\doxysubsubsection{\texorpdfstring{m\_inputPort}{m\_inputPort}}
{\footnotesize\ttfamily \label{class_beam_1_1_audio_module_a2e0f72cba2f16ff63cedee7d2417c0bf} 
std::\+shared\+\_\+ptr$<$\doxymbox{\hyperlink{class_beam_1_1_port}{Port}}$>$ Beam::\+\+Audio\+Module::\+m\+\_\+input\+Port\hspace{0.3cm}{\ttfamily [private]}}

\Hypertarget{class_beam_1_1_audio_module_a1b3da6406c733f1a4b0e19b6d1cbcfb6}\index{Beam::AudioModule@{Beam::AudioModule}!m\_name@{m\_name}}
\index{m\_name@{m\_name}!Beam::AudioModule@{Beam::AudioModule}}
\doxysubsubsection{\texorpdfstring{m\_name}{m\_name}}
{\footnotesize\ttfamily \label{class_beam_1_1_audio_module_a1b3da6406c733f1a4b0e19b6d1cbcfb6} 
std::\+string Beam::\+\+Audio\+Module::\+m\+\_\+name\hspace{0.3cm}{\ttfamily [private]}}

\Hypertarget{class_beam_1_1_audio_module_aafbc58a945b32ba82e3635a6d915f498}\index{Beam::AudioModule@{Beam::AudioModule}!m\_node@{m\_node}}
\index{m\_node@{m\_node}!Beam::AudioModule@{Beam::AudioModule}}
\doxysubsubsection{\texorpdfstring{m\_node}{m\_node}}
{\footnotesize\ttfamily \label{class_beam_1_1_audio_module_aafbc58a945b32ba82e3635a6d915f498} 
std::\+shared\+\_\+ptr$<$\doxymbox{\hyperlink{class_beam_1_1_flux_node}{Flux\+Node}}$>$ Beam::\+\+Audio\+Module::\+m\+\_\+node\hspace{0.3cm}{\ttfamily [private]}}

\Hypertarget{class_beam_1_1_audio_module_a9b8bb7bd8aaa48881475de44550013ad}\index{Beam::AudioModule@{Beam::AudioModule}!m\_nodeId@{m\_nodeId}}
\index{m\_nodeId@{m\_nodeId}!Beam::AudioModule@{Beam::AudioModule}}
\doxysubsubsection{\texorpdfstring{m\_nodeId}{m\_nodeId}}
{\footnotesize\ttfamily \label{class_beam_1_1_audio_module_a9b8bb7bd8aaa48881475de44550013ad} 
size\+\_\+t Beam::\+\+Audio\+Module::\+m\+\_\+node\+Id\hspace{0.3cm}{\ttfamily [private]}}

\Hypertarget{class_beam_1_1_audio_module_ab2959617ad001fc0bb8bc9ed9828a558}\index{Beam::AudioModule@{Beam::AudioModule}!m\_outputPort@{m\_outputPort}}
\index{m\_outputPort@{m\_outputPort}!Beam::AudioModule@{Beam::AudioModule}}
\doxysubsubsection{\texorpdfstring{m\_outputPort}{m\_outputPort}}
{\footnotesize\ttfamily \label{class_beam_1_1_audio_module_ab2959617ad001fc0bb8bc9ed9828a558} 
std::\+shared\+\_\+ptr$<$\doxymbox{\hyperlink{class_beam_1_1_port}{Port}}$>$ Beam::\+\+Audio\+Module::\+m\+\_\+output\+Port\hspace{0.3cm}{\ttfamily [private]}}

\Hypertarget{class_beam_1_1_audio_module_ab93c8ba97f1b2f29113c31a52c0c36eb}\index{Beam::AudioModule@{Beam::AudioModule}!m\_scrollTimer@{m\_scrollTimer}}
\index{m\_scrollTimer@{m\_scrollTimer}!Beam::AudioModule@{Beam::AudioModule}}
\doxysubsubsection{\texorpdfstring{m\_scrollTimer}{m\_scrollTimer}}
{\footnotesize\ttfamily \label{class_beam_1_1_audio_module_ab93c8ba97f1b2f29113c31a52c0c36eb} 
float Beam::\+\+Audio\+Module::\+m\+\_\+scroll\+Timer = 0.\+0f\hspace{0.3cm}{\ttfamily [private]}}

\Hypertarget{class_beam_1_1_audio_module_af95229e9824037c0c9916cc04bf67e90}\index{Beam::AudioModule@{Beam::AudioModule}!onDeleteRequested@{onDeleteRequested}}
\index{onDeleteRequested@{onDeleteRequested}!Beam::AudioModule@{Beam::AudioModule}}
\doxysubsubsection{\texorpdfstring{onDeleteRequested}{onDeleteRequested}}
{\footnotesize\ttfamily \label{class_beam_1_1_audio_module_af95229e9824037c0c9916cc04bf67e90} 
std::\+function$<$void(\doxymbox{\hyperlink{class_beam_1_1_audio_module_a409c75189798146d7f556b1d50f4ba98}{Audio\+Module}}\texorpdfstring{$\ast$}{*})$>$ Beam::\+\+Audio\+Module::\+on\+Delete\+Requested}



The documentation for this class was generated from the following file:\+\begin{DoxyCompactItemize}
\item 
src/\+interface/\+\doxymbox{\hyperlink{audio__module_8hpp}{audio\+\_\+module.\+hpp}}\end{DoxyCompactItemize}

\doxysection{Beam::\+Audio\+Node Class Reference}
\hypertarget{class_beam_1_1_audio_node}{}\label{class_beam_1_1_audio_node}\index{Beam::AudioNode@{Beam::AudioNode}}


{\ttfamily \+\#include $<$audio\+\_\+node.\+hpp$>$}

Inheritance diagram for Beam::\+Audio\+Node:\+\begin{figure}[H]
\begin{center}
\leavevmode
\includegraphics[height=2.000000cm]{class_beam_1_1_audio_node}
\end{center}
\end{figure}
\doxysubsubsection*{Public Member Functions}
\begin{DoxyCompactItemize}
\item 
virtual \doxymbox{\hyperlink{class_beam_1_1_audio_node_afbea31954b50918131d31fc0d1f6de8c}{\texorpdfstring{$\sim$}{\string~}\+Audio\+Node}} ()=default
\item 
virtual void \doxymbox{\hyperlink{class_beam_1_1_audio_node_ab0cb6fa1aba031e703be16be26e0d6b7}{process}} (float \texorpdfstring{$\ast$}{*}buffer, int frames, int channels, size\+\_\+t start\+Frame=0)=0
\item 
virtual std::\+string \doxymbox{\hyperlink{class_beam_1_1_audio_node_a864b3bf9095638e43ad334b7b3706bec}{get\+Name}} () const =0
\item 
void \doxymbox{\hyperlink{class_beam_1_1_audio_node_a3fcc68eab5b1adf547a4205f258b212c}{set\+Bypass}} (bool bypass)
\item 
bool \doxymbox{\hyperlink{class_beam_1_1_audio_node_a6ba1724cff34b5bc0f811ee2537caae5}{is\+Bypassed}} () const
\end{DoxyCompactItemize}
\doxysubsubsection*{Protected Attributes}
\begin{DoxyCompactItemize}
\item 
bool \doxymbox{\hyperlink{class_beam_1_1_audio_node_ac5ad81de4a5d0abe555fe9f06219b09f}{m\+\_\+is\+Bypassed}} = false
\end{DoxyCompactItemize}


\label{doc-constructors}
\Hypertarget{class_beam_1_1_audio_node_doc-constructors}
\doxysubsection{Constructor \& Destructor Documentation}
\Hypertarget{class_beam_1_1_audio_node_afbea31954b50918131d31fc0d1f6de8c}\index{Beam::AudioNode@{Beam::AudioNode}!````~AudioNode@{\texorpdfstring{$\sim$}{\string~}AudioNode}}
\index{````~AudioNode@{\texorpdfstring{$\sim$}{\string~}AudioNode}!Beam::AudioNode@{Beam::AudioNode}}
\doxysubsubsection{\texorpdfstring{\texorpdfstring{$\sim$}{\string~}AudioNode()}{\string~AudioNode()}}
{\footnotesize\ttfamily \label{class_beam_1_1_audio_node_afbea31954b50918131d31fc0d1f6de8c} 
virtual Beam::\+\+Audio\+Node::\+\texorpdfstring{$\sim$}{\string~}\+Audio\+Node (\begin{DoxyParamCaption}{}{}\end{DoxyParamCaption})\hspace{0.3cm}{\ttfamily [virtual]}, {\ttfamily [default]}}



\label{doc-func-members}
\Hypertarget{class_beam_1_1_audio_node_doc-func-members}
\doxysubsection{Member Function Documentation}
\Hypertarget{class_beam_1_1_audio_node_a864b3bf9095638e43ad334b7b3706bec}\index{Beam::AudioNode@{Beam::AudioNode}!getName@{getName}}
\index{getName@{getName}!Beam::AudioNode@{Beam::AudioNode}}
\doxysubsubsection{\texorpdfstring{getName()}{getName()}}
{\footnotesize\ttfamily \label{class_beam_1_1_audio_node_a864b3bf9095638e43ad334b7b3706bec} 
virtual std::\+string Beam::\+\+Audio\+Node::\+get\+Name (\begin{DoxyParamCaption}{}{}\end{DoxyParamCaption}) const\hspace{0.3cm}{\ttfamily [pure virtual]}}



Implemented in \doxymbox{\hyperlink{class_beam_1_1_biquad_filter_node_a928e2017ed0dacfc547360c6fb982181}{Beam::\+\+Biquad\+Filter\+Node}}, \doxymbox{\hyperlink{class_beam_1_1_delay_node_aac469a6ff6f2ea95f5ffbb06f60dbf2e}{Beam::\+\+Delay\+Node}}, and \doxymbox{\hyperlink{class_beam_1_1_track_node_a8c233bb481da69fd335c86c286ae9606}{Beam::\+\+Track\+Node}}.

\Hypertarget{class_beam_1_1_audio_node_a6ba1724cff34b5bc0f811ee2537caae5}\index{Beam::AudioNode@{Beam::AudioNode}!isBypassed@{isBypassed}}
\index{isBypassed@{isBypassed}!Beam::AudioNode@{Beam::AudioNode}}
\doxysubsubsection{\texorpdfstring{isBypassed()}{isBypassed()}}
{\footnotesize\ttfamily \label{class_beam_1_1_audio_node_a6ba1724cff34b5bc0f811ee2537caae5} 
bool Beam::\+\+Audio\+Node::\+is\+Bypassed (\begin{DoxyParamCaption}{}{}\end{DoxyParamCaption}) const\hspace{0.3cm}{\ttfamily [inline]}}

\Hypertarget{class_beam_1_1_audio_node_ab0cb6fa1aba031e703be16be26e0d6b7}\index{Beam::AudioNode@{Beam::AudioNode}!process@{process}}
\index{process@{process}!Beam::AudioNode@{Beam::AudioNode}}
\doxysubsubsection{\texorpdfstring{process()}{process()}}
{\footnotesize\ttfamily \label{class_beam_1_1_audio_node_ab0cb6fa1aba031e703be16be26e0d6b7} 
virtual void Beam::\+\+Audio\+Node::\+process (\begin{DoxyParamCaption}\item[{float \texorpdfstring{$\ast$}{*}}]{buffer}{, }\item[{int}]{frames}{, }\item[{int}]{channels}{, }\item[{size\+\_\+t}]{start\+Frame}{ = {\ttfamily 0}}\end{DoxyParamCaption})\hspace{0.3cm}{\ttfamily [pure virtual]}}



Implemented in \doxymbox{\hyperlink{class_beam_1_1_biquad_filter_node_a9ca4ea9dffee4a78f9c25457c2303142}{Beam::\+\+Biquad\+Filter\+Node}}, \doxymbox{\hyperlink{class_beam_1_1_delay_node_a44fdc2abfe4112bf5e89f97d1c9ebe20}{Beam::\+\+Delay\+Node}}, and \doxymbox{\hyperlink{class_beam_1_1_track_node_a41898cb4d3b6b3f09c5c144a66dc4a5d}{Beam::\+\+Track\+Node}}.

\Hypertarget{class_beam_1_1_audio_node_a3fcc68eab5b1adf547a4205f258b212c}\index{Beam::AudioNode@{Beam::AudioNode}!setBypass@{setBypass}}
\index{setBypass@{setBypass}!Beam::AudioNode@{Beam::AudioNode}}
\doxysubsubsection{\texorpdfstring{setBypass()}{setBypass()}}
{\footnotesize\ttfamily \label{class_beam_1_1_audio_node_a3fcc68eab5b1adf547a4205f258b212c} 
void Beam::\+\+Audio\+Node::\+set\+Bypass (\begin{DoxyParamCaption}\item[{bool}]{bypass}{}\end{DoxyParamCaption})\hspace{0.3cm}{\ttfamily [inline]}}



\label{doc-variable-members}
\Hypertarget{class_beam_1_1_audio_node_doc-variable-members}
\doxysubsection{Member Data Documentation}
\Hypertarget{class_beam_1_1_audio_node_ac5ad81de4a5d0abe555fe9f06219b09f}\index{Beam::AudioNode@{Beam::AudioNode}!m\_isBypassed@{m\_isBypassed}}
\index{m\_isBypassed@{m\_isBypassed}!Beam::AudioNode@{Beam::AudioNode}}
\doxysubsubsection{\texorpdfstring{m\_isBypassed}{m\_isBypassed}}
{\footnotesize\ttfamily \label{class_beam_1_1_audio_node_ac5ad81de4a5d0abe555fe9f06219b09f} 
bool Beam::\+\+Audio\+Node::\+m\+\_\+is\+Bypassed = false\hspace{0.3cm}{\ttfamily [protected]}}



The documentation for this class was generated from the following file:\+\begin{DoxyCompactItemize}
\item 
src/\+engine/\+\doxymbox{\hyperlink{audio__node_8hpp}{audio\+\_\+node.\+hpp}}\end{DoxyCompactItemize}

\doxysection{Beam::\+Audio\+Processor Class Reference}
\hypertarget{class_beam_1_1_audio_processor}{}\label{class_beam_1_1_audio_processor}\index{Beam::AudioProcessor@{Beam::AudioProcessor}}


Base class for audio processing units, similar to JUCE\textquotesingle{}s \doxylink{class_beam_1_1_audio_processor}{Audio\+Processor}.  




{\ttfamily \+\#include $<$audio\+\_\+processor.\+hpp$>$}

Inheritance diagram for Beam::\+Audio\+Processor:\+\begin{figure}[H]
\begin{center}
\leavevmode
\includegraphics[height=2.000000cm]{class_beam_1_1_audio_processor}
\end{center}
\end{figure}
\doxysubsubsection*{Public Member Functions}
\begin{DoxyCompactItemize}
\item 
\doxymbox{\hyperlink{class_beam_1_1_audio_processor_a8f0f6af900e612f1e95407eb6c8878cd}{Audio\+Processor}} ()
\item 
virtual \doxymbox{\hyperlink{class_beam_1_1_audio_processor_aa760c6d60a8045dc32aa1578b9c74afe}{\texorpdfstring{$\sim$}{\string~}\+Audio\+Processor}} ()
\item 
virtual void \doxymbox{\hyperlink{class_beam_1_1_audio_processor_a0ef7279db2dc6cef108efcad0c6edb8a}{prepare\+To\+Play}} (double sample\+Rate, int samples\+Per\+Block)=0
\begin{DoxyCompactList}\small\item\em Called before audio processing begins. \end{DoxyCompactList}\item 
virtual void \doxymbox{\hyperlink{class_beam_1_1_audio_processor_adf8005d9b36d70188aa1379a61f7f5a9}{release\+Resources}} ()=0
\begin{DoxyCompactList}\small\item\em Called after audio stops. \end{DoxyCompactList}\item 
virtual void \doxymbox{\hyperlink{class_beam_1_1_audio_processor_a831818fed7ed10115cd3983addf21936}{process\+Block}} (float \texorpdfstring{$\ast$}{*}\texorpdfstring{$\ast$}{*}audio\+Input\+Output, int num\+Input\+Channels, int num\+Output\+Channels, int num\+Samples, const \doxymbox{\hyperlink{class_beam_1_1_m_i_d_i_buffer}{MIDIBuffer}} \&midi\+Messages)=0
\begin{DoxyCompactList}\small\item\em Main processing function. \end{DoxyCompactList}\item 
virtual std::\+string \doxymbox{\hyperlink{class_beam_1_1_audio_processor_af4f88f01a436dba1fd3db520503b68c7}{get\+Name}} () const =0
\begin{DoxyCompactList}\small\item\em Returns the name of this processor. \end{DoxyCompactList}\item 
virtual int \doxymbox{\hyperlink{class_beam_1_1_audio_processor_a13c1f31b340d69fce391411402082211}{get\+Num\+Input\+Channels}} () const =0
\begin{DoxyCompactList}\small\item\em Returns the number of input channels. \end{DoxyCompactList}\item 
virtual int \doxymbox{\hyperlink{class_beam_1_1_audio_processor_a403c0cf559133347ee2753ef11766f21}{get\+Num\+Output\+Channels}} () const =0
\begin{DoxyCompactList}\small\item\em Returns the number of output channels. \end{DoxyCompactList}\item 
int \doxymbox{\hyperlink{class_beam_1_1_audio_processor_a0e0668105ca3844b5d0947a3fcf25b7b}{get\+Num\+Parameters}} () const
\begin{DoxyCompactList}\small\item\em Returns the total number of parameters. \end{DoxyCompactList}\item 
virtual std::\+string \doxymbox{\hyperlink{class_beam_1_1_audio_processor_a8dd4277153a28700b0b043ef218a4d15}{get\+Parameter\+Name}} (int parameter\+Index) const
\begin{DoxyCompactList}\small\item\em Returns the name of a parameter. \end{DoxyCompactList}\item 
virtual float \doxymbox{\hyperlink{class_beam_1_1_audio_processor_a456045a6f38f955db222bfdbb1e54be0}{get\+Parameter}} (int parameter\+Index) const
\begin{DoxyCompactList}\small\item\em Returns the value of a parameter. \end{DoxyCompactList}\item 
virtual void \doxymbox{\hyperlink{class_beam_1_1_audio_processor_a3e9a8c5b7470f9a957d85b84efff64b1}{set\+Parameter}} (int parameter\+Index, float new\+Value)
\begin{DoxyCompactList}\small\item\em Sets the value of a parameter. \end{DoxyCompactList}\item 
virtual std::\+string \doxymbox{\hyperlink{class_beam_1_1_audio_processor_ad0c8e5cece50693d131dc9476f5a7835}{get\+Parameter\+Text}} (int parameter\+Index, float value) const
\begin{DoxyCompactList}\small\item\em Returns the text representation of a parameter\textquotesingle{}s value. \end{DoxyCompactList}\item 
void \doxymbox{\hyperlink{class_beam_1_1_audio_processor_a7e9f840858284252d4a288f4e89e0c8e}{add\+Parameter}} (std::\+shared\+\_\+ptr$<$ \doxymbox{\hyperlink{class_beam_1_1_parameter}{Parameter}} $>$ parameter)
\begin{DoxyCompactList}\small\item\em Adds a parameter to this processor. \end{DoxyCompactList}\item 
std::\+shared\+\_\+ptr$<$ \doxymbox{\hyperlink{class_beam_1_1_parameter}{Parameter}} $>$ \doxymbox{\hyperlink{class_beam_1_1_audio_processor_a7298dde8e5844676fea306d579ae1b5c}{get\+Parameter\+By\+Index}} (int index) const
\begin{DoxyCompactList}\small\item\em Gets a parameter by index. \end{DoxyCompactList}\item 
std::\+shared\+\_\+ptr$<$ \doxymbox{\hyperlink{class_beam_1_1_parameter}{Parameter}} $>$ \doxymbox{\hyperlink{class_beam_1_1_audio_processor_a3b570dc9bcbeb3208e026e436c1fe9e7}{get\+Parameter\+By\+Name}} (const std::\+string \&name) const
\begin{DoxyCompactList}\small\item\em Gets a parameter by name. \end{DoxyCompactList}\item 
virtual void \doxymbox{\hyperlink{class_beam_1_1_audio_processor_a7da84e7885fe1660401485dd1f46c8f9}{set\+Current\+Playback\+State}} (bool is\+Playing, double current\+Time\+Seconds, double tempo)
\begin{DoxyCompactList}\small\item\em Called when the play head state changes. \end{DoxyCompactList}\item 
double \doxymbox{\hyperlink{class_beam_1_1_audio_processor_a480e589f13d2a2046d7d46896c17daaa}{get\+Sample\+Rate}} () const
\begin{DoxyCompactList}\small\item\em Returns the current sample rate. \end{DoxyCompactList}\item 
int \doxymbox{\hyperlink{class_beam_1_1_audio_processor_a49a5218a84064deb3c05aad0087be7a3}{get\+Block\+Size}} () const
\begin{DoxyCompactList}\small\item\em Returns the current block size. \end{DoxyCompactList}\end{DoxyCompactItemize}
\doxysubsubsection*{Protected Attributes}
\begin{DoxyCompactItemize}
\item 
double \doxymbox{\hyperlink{class_beam_1_1_audio_processor_add21a9644aaa71b9457b4f508e1e0840}{m\+\_\+sample\+Rate}} = 44100.\+0
\item 
int \doxymbox{\hyperlink{class_beam_1_1_audio_processor_a3d2ab6b1f0c86689d55c8e72ce428466}{m\+\_\+block\+Size}} = 512
\item 
std::\+vector$<$ std::\+shared\+\_\+ptr$<$ \doxymbox{\hyperlink{class_beam_1_1_parameter}{Parameter}} $>$ $>$ \doxymbox{\hyperlink{class_beam_1_1_audio_processor_a89bb21df73830e6b14b8bf69d3ba68a1}{m\+\_\+parameters}}
\item 
std::\+map$<$ std::\+string, std::\+shared\+\_\+ptr$<$ \doxymbox{\hyperlink{class_beam_1_1_parameter}{Parameter}} $>$ $>$ \doxymbox{\hyperlink{class_beam_1_1_audio_processor_a0430f0f07132e944c8af000cae93f321}{m\+\_\+named\+Parameters}}
\end{DoxyCompactItemize}


\doxysubsection{Detailed Description}
Base class for audio processing units, similar to JUCE\textquotesingle{}s \doxylink{class_beam_1_1_audio_processor}{Audio\+Processor}. 

\label{doc-constructors}
\Hypertarget{class_beam_1_1_audio_processor_doc-constructors}
\doxysubsection{Constructor \& Destructor Documentation}
\Hypertarget{class_beam_1_1_audio_processor_a8f0f6af900e612f1e95407eb6c8878cd}\index{Beam::AudioProcessor@{Beam::AudioProcessor}!AudioProcessor@{AudioProcessor}}
\index{AudioProcessor@{AudioProcessor}!Beam::AudioProcessor@{Beam::AudioProcessor}}
\doxysubsubsection{\texorpdfstring{AudioProcessor()}{AudioProcessor()}}
{\footnotesize\ttfamily \label{class_beam_1_1_audio_processor_a8f0f6af900e612f1e95407eb6c8878cd} 
Beam::\+\+Audio\+Processor::\+\+Audio\+Processor (\begin{DoxyParamCaption}{}{}\end{DoxyParamCaption})}

\Hypertarget{class_beam_1_1_audio_processor_aa760c6d60a8045dc32aa1578b9c74afe}\index{Beam::AudioProcessor@{Beam::AudioProcessor}!````~AudioProcessor@{\texorpdfstring{$\sim$}{\string~}AudioProcessor}}
\index{````~AudioProcessor@{\texorpdfstring{$\sim$}{\string~}AudioProcessor}!Beam::AudioProcessor@{Beam::AudioProcessor}}
\doxysubsubsection{\texorpdfstring{\texorpdfstring{$\sim$}{\string~}AudioProcessor()}{\string~AudioProcessor()}}
{\footnotesize\ttfamily \label{class_beam_1_1_audio_processor_aa760c6d60a8045dc32aa1578b9c74afe} 
Beam::\+\+Audio\+Processor::\+\texorpdfstring{$\sim$}{\string~}\+Audio\+Processor (\begin{DoxyParamCaption}{}{}\end{DoxyParamCaption})\hspace{0.3cm}{\ttfamily [virtual]}}



\label{doc-func-members}
\Hypertarget{class_beam_1_1_audio_processor_doc-func-members}
\doxysubsection{Member Function Documentation}
\Hypertarget{class_beam_1_1_audio_processor_a7e9f840858284252d4a288f4e89e0c8e}\index{Beam::AudioProcessor@{Beam::AudioProcessor}!addParameter@{addParameter}}
\index{addParameter@{addParameter}!Beam::AudioProcessor@{Beam::AudioProcessor}}
\doxysubsubsection{\texorpdfstring{addParameter()}{addParameter()}}
{\footnotesize\ttfamily \label{class_beam_1_1_audio_processor_a7e9f840858284252d4a288f4e89e0c8e} 
void Beam::\+\+Audio\+Processor::\+add\+Parameter (\begin{DoxyParamCaption}\item[{std::\+shared\+\_\+ptr$<$ \doxymbox{\hyperlink{class_beam_1_1_parameter}{Parameter}} $>$}]{parameter}{}\end{DoxyParamCaption})}



Adds a parameter to this processor. 

\Hypertarget{class_beam_1_1_audio_processor_a49a5218a84064deb3c05aad0087be7a3}\index{Beam::AudioProcessor@{Beam::AudioProcessor}!getBlockSize@{getBlockSize}}
\index{getBlockSize@{getBlockSize}!Beam::AudioProcessor@{Beam::AudioProcessor}}
\doxysubsubsection{\texorpdfstring{getBlockSize()}{getBlockSize()}}
{\footnotesize\ttfamily \label{class_beam_1_1_audio_processor_a49a5218a84064deb3c05aad0087be7a3} 
int Beam::\+\+Audio\+Processor::\+get\+Block\+Size (\begin{DoxyParamCaption}{}{}\end{DoxyParamCaption}) const\hspace{0.3cm}{\ttfamily [inline]}}



Returns the current block size. 

\Hypertarget{class_beam_1_1_audio_processor_af4f88f01a436dba1fd3db520503b68c7}\index{Beam::AudioProcessor@{Beam::AudioProcessor}!getName@{getName}}
\index{getName@{getName}!Beam::AudioProcessor@{Beam::AudioProcessor}}
\doxysubsubsection{\texorpdfstring{getName()}{getName()}}
{\footnotesize\ttfamily \label{class_beam_1_1_audio_processor_af4f88f01a436dba1fd3db520503b68c7} 
virtual std::\+string Beam::\+\+Audio\+Processor::\+get\+Name (\begin{DoxyParamCaption}{}{}\end{DoxyParamCaption}) const\hspace{0.3cm}{\ttfamily [pure virtual]}}



Returns the name of this processor. 



Implemented in \doxymbox{\hyperlink{class_beam_1_1_flux_node_audio_processor_wrapper_aa2b20d72dcf9a64018cfa617ee29df8c}{Beam::\+\+Flux\+Node\+Audio\+Processor\+Wrapper}}.

\Hypertarget{class_beam_1_1_audio_processor_a13c1f31b340d69fce391411402082211}\index{Beam::AudioProcessor@{Beam::AudioProcessor}!getNumInputChannels@{getNumInputChannels}}
\index{getNumInputChannels@{getNumInputChannels}!Beam::AudioProcessor@{Beam::AudioProcessor}}
\doxysubsubsection{\texorpdfstring{getNumInputChannels()}{getNumInputChannels()}}
{\footnotesize\ttfamily \label{class_beam_1_1_audio_processor_a13c1f31b340d69fce391411402082211} 
virtual int Beam::\+\+Audio\+Processor::\+get\+Num\+Input\+Channels (\begin{DoxyParamCaption}{}{}\end{DoxyParamCaption}) const\hspace{0.3cm}{\ttfamily [pure virtual]}}



Returns the number of input channels. 



Implemented in \doxymbox{\hyperlink{class_beam_1_1_flux_node_audio_processor_wrapper_af5f7eb577be8e860cd7e15126a05d285}{Beam::\+\+Flux\+Node\+Audio\+Processor\+Wrapper}}.

\Hypertarget{class_beam_1_1_audio_processor_a403c0cf559133347ee2753ef11766f21}\index{Beam::AudioProcessor@{Beam::AudioProcessor}!getNumOutputChannels@{getNumOutputChannels}}
\index{getNumOutputChannels@{getNumOutputChannels}!Beam::AudioProcessor@{Beam::AudioProcessor}}
\doxysubsubsection{\texorpdfstring{getNumOutputChannels()}{getNumOutputChannels()}}
{\footnotesize\ttfamily \label{class_beam_1_1_audio_processor_a403c0cf559133347ee2753ef11766f21} 
virtual int Beam::\+\+Audio\+Processor::\+get\+Num\+Output\+Channels (\begin{DoxyParamCaption}{}{}\end{DoxyParamCaption}) const\hspace{0.3cm}{\ttfamily [pure virtual]}}



Returns the number of output channels. 



Implemented in \doxymbox{\hyperlink{class_beam_1_1_flux_node_audio_processor_wrapper_a2e3a9d06433fc35e14071301544125df}{Beam::\+\+Flux\+Node\+Audio\+Processor\+Wrapper}}.

\Hypertarget{class_beam_1_1_audio_processor_a0e0668105ca3844b5d0947a3fcf25b7b}\index{Beam::AudioProcessor@{Beam::AudioProcessor}!getNumParameters@{getNumParameters}}
\index{getNumParameters@{getNumParameters}!Beam::AudioProcessor@{Beam::AudioProcessor}}
\doxysubsubsection{\texorpdfstring{getNumParameters()}{getNumParameters()}}
{\footnotesize\ttfamily \label{class_beam_1_1_audio_processor_a0e0668105ca3844b5d0947a3fcf25b7b} 
int Beam::\+\+Audio\+Processor::\+get\+Num\+Parameters (\begin{DoxyParamCaption}{}{}\end{DoxyParamCaption}) const}



Returns the total number of parameters. 

\Hypertarget{class_beam_1_1_audio_processor_a456045a6f38f955db222bfdbb1e54be0}\index{Beam::AudioProcessor@{Beam::AudioProcessor}!getParameter@{getParameter}}
\index{getParameter@{getParameter}!Beam::AudioProcessor@{Beam::AudioProcessor}}
\doxysubsubsection{\texorpdfstring{getParameter()}{getParameter()}}
{\footnotesize\ttfamily \label{class_beam_1_1_audio_processor_a456045a6f38f955db222bfdbb1e54be0} 
float Beam::\+\+Audio\+Processor::\+get\+Parameter (\begin{DoxyParamCaption}\item[{int}]{parameter\+Index}{}\end{DoxyParamCaption}) const\hspace{0.3cm}{\ttfamily [virtual]}}



Returns the value of a parameter. 

\Hypertarget{class_beam_1_1_audio_processor_a7298dde8e5844676fea306d579ae1b5c}\index{Beam::AudioProcessor@{Beam::AudioProcessor}!getParameterByIndex@{getParameterByIndex}}
\index{getParameterByIndex@{getParameterByIndex}!Beam::AudioProcessor@{Beam::AudioProcessor}}
\doxysubsubsection{\texorpdfstring{getParameterByIndex()}{getParameterByIndex()}}
{\footnotesize\ttfamily \label{class_beam_1_1_audio_processor_a7298dde8e5844676fea306d579ae1b5c} 
std::\+shared\+\_\+ptr$<$ \doxymbox{\hyperlink{class_beam_1_1_parameter}{Parameter}} $>$ Beam::\+\+Audio\+Processor::\+get\+Parameter\+By\+Index (\begin{DoxyParamCaption}\item[{int}]{index}{}\end{DoxyParamCaption}) const}



Gets a parameter by index. 

\Hypertarget{class_beam_1_1_audio_processor_a3b570dc9bcbeb3208e026e436c1fe9e7}\index{Beam::AudioProcessor@{Beam::AudioProcessor}!getParameterByName@{getParameterByName}}
\index{getParameterByName@{getParameterByName}!Beam::AudioProcessor@{Beam::AudioProcessor}}
\doxysubsubsection{\texorpdfstring{getParameterByName()}{getParameterByName()}}
{\footnotesize\ttfamily \label{class_beam_1_1_audio_processor_a3b570dc9bcbeb3208e026e436c1fe9e7} 
std::\+shared\+\_\+ptr$<$ \doxymbox{\hyperlink{class_beam_1_1_parameter}{Parameter}} $>$ Beam::\+\+Audio\+Processor::\+get\+Parameter\+By\+Name (\begin{DoxyParamCaption}\item[{const std::\+string \&}]{name}{}\end{DoxyParamCaption}) const}



Gets a parameter by name. 

\Hypertarget{class_beam_1_1_audio_processor_a8dd4277153a28700b0b043ef218a4d15}\index{Beam::AudioProcessor@{Beam::AudioProcessor}!getParameterName@{getParameterName}}
\index{getParameterName@{getParameterName}!Beam::AudioProcessor@{Beam::AudioProcessor}}
\doxysubsubsection{\texorpdfstring{getParameterName()}{getParameterName()}}
{\footnotesize\ttfamily \label{class_beam_1_1_audio_processor_a8dd4277153a28700b0b043ef218a4d15} 
std::\+string Beam::\+\+Audio\+Processor::\+get\+Parameter\+Name (\begin{DoxyParamCaption}\item[{int}]{parameter\+Index}{}\end{DoxyParamCaption}) const\hspace{0.3cm}{\ttfamily [virtual]}}



Returns the name of a parameter. 

\Hypertarget{class_beam_1_1_audio_processor_ad0c8e5cece50693d131dc9476f5a7835}\index{Beam::AudioProcessor@{Beam::AudioProcessor}!getParameterText@{getParameterText}}
\index{getParameterText@{getParameterText}!Beam::AudioProcessor@{Beam::AudioProcessor}}
\doxysubsubsection{\texorpdfstring{getParameterText()}{getParameterText()}}
{\footnotesize\ttfamily \label{class_beam_1_1_audio_processor_ad0c8e5cece50693d131dc9476f5a7835} 
std::\+string Beam::\+\+Audio\+Processor::\+get\+Parameter\+Text (\begin{DoxyParamCaption}\item[{int}]{parameter\+Index}{, }\item[{float}]{value}{}\end{DoxyParamCaption}) const\hspace{0.3cm}{\ttfamily [virtual]}}



Returns the text representation of a parameter\textquotesingle{}s value. 

\Hypertarget{class_beam_1_1_audio_processor_a480e589f13d2a2046d7d46896c17daaa}\index{Beam::AudioProcessor@{Beam::AudioProcessor}!getSampleRate@{getSampleRate}}
\index{getSampleRate@{getSampleRate}!Beam::AudioProcessor@{Beam::AudioProcessor}}
\doxysubsubsection{\texorpdfstring{getSampleRate()}{getSampleRate()}}
{\footnotesize\ttfamily \label{class_beam_1_1_audio_processor_a480e589f13d2a2046d7d46896c17daaa} 
double Beam::\+\+Audio\+Processor::\+get\+Sample\+Rate (\begin{DoxyParamCaption}{}{}\end{DoxyParamCaption}) const\hspace{0.3cm}{\ttfamily [inline]}}



Returns the current sample rate. 

\Hypertarget{class_beam_1_1_audio_processor_a0ef7279db2dc6cef108efcad0c6edb8a}\index{Beam::AudioProcessor@{Beam::AudioProcessor}!prepareToPlay@{prepareToPlay}}
\index{prepareToPlay@{prepareToPlay}!Beam::AudioProcessor@{Beam::AudioProcessor}}
\doxysubsubsection{\texorpdfstring{prepareToPlay()}{prepareToPlay()}}
{\footnotesize\ttfamily \label{class_beam_1_1_audio_processor_a0ef7279db2dc6cef108efcad0c6edb8a} 
virtual void Beam::\+\+Audio\+Processor::\+prepare\+To\+Play (\begin{DoxyParamCaption}\item[{double}]{sample\+Rate}{, }\item[{int}]{samples\+Per\+Block}{}\end{DoxyParamCaption})\hspace{0.3cm}{\ttfamily [pure virtual]}}



Called before audio processing begins. 



Implemented in \doxymbox{\hyperlink{class_beam_1_1_flux_node_audio_processor_wrapper_a26b885175501fb2cad51c69ef85489c1}{Beam::\+\+Flux\+Node\+Audio\+Processor\+Wrapper}}.

\Hypertarget{class_beam_1_1_audio_processor_a831818fed7ed10115cd3983addf21936}\index{Beam::AudioProcessor@{Beam::AudioProcessor}!processBlock@{processBlock}}
\index{processBlock@{processBlock}!Beam::AudioProcessor@{Beam::AudioProcessor}}
\doxysubsubsection{\texorpdfstring{processBlock()}{processBlock()}}
{\footnotesize\ttfamily \label{class_beam_1_1_audio_processor_a831818fed7ed10115cd3983addf21936} 
virtual void Beam::\+\+Audio\+Processor::\+process\+Block (\begin{DoxyParamCaption}\item[{float \texorpdfstring{$\ast$}{*}\texorpdfstring{$\ast$}{*}}]{audio\+Input\+Output}{, }\item[{int}]{num\+Input\+Channels}{, }\item[{int}]{num\+Output\+Channels}{, }\item[{int}]{num\+Samples}{, }\item[{const \doxymbox{\hyperlink{class_beam_1_1_m_i_d_i_buffer}{MIDIBuffer}} \&}]{midi\+Messages}{}\end{DoxyParamCaption})\hspace{0.3cm}{\ttfamily [pure virtual]}}



Main processing function. 



Implemented in \doxymbox{\hyperlink{class_beam_1_1_flux_node_audio_processor_wrapper_ad4916adf7b3423d1407a4058e43d3a8f}{Beam::\+\+Flux\+Node\+Audio\+Processor\+Wrapper}}.

\Hypertarget{class_beam_1_1_audio_processor_adf8005d9b36d70188aa1379a61f7f5a9}\index{Beam::AudioProcessor@{Beam::AudioProcessor}!releaseResources@{releaseResources}}
\index{releaseResources@{releaseResources}!Beam::AudioProcessor@{Beam::AudioProcessor}}
\doxysubsubsection{\texorpdfstring{releaseResources()}{releaseResources()}}
{\footnotesize\ttfamily \label{class_beam_1_1_audio_processor_adf8005d9b36d70188aa1379a61f7f5a9} 
virtual void Beam::\+\+Audio\+Processor::\+release\+Resources (\begin{DoxyParamCaption}{}{}\end{DoxyParamCaption})\hspace{0.3cm}{\ttfamily [pure virtual]}}



Called after audio stops. 



Implemented in \doxymbox{\hyperlink{class_beam_1_1_flux_node_audio_processor_wrapper_a124968314929cbbf897db8a23ef5ed3b}{Beam::\+\+Flux\+Node\+Audio\+Processor\+Wrapper}}.

\Hypertarget{class_beam_1_1_audio_processor_a7da84e7885fe1660401485dd1f46c8f9}\index{Beam::AudioProcessor@{Beam::AudioProcessor}!setCurrentPlaybackState@{setCurrentPlaybackState}}
\index{setCurrentPlaybackState@{setCurrentPlaybackState}!Beam::AudioProcessor@{Beam::AudioProcessor}}
\doxysubsubsection{\texorpdfstring{setCurrentPlaybackState()}{setCurrentPlaybackState()}}
{\footnotesize\ttfamily \label{class_beam_1_1_audio_processor_a7da84e7885fe1660401485dd1f46c8f9} 
void Beam::\+\+Audio\+Processor::\+set\+Current\+Playback\+State (\begin{DoxyParamCaption}\item[{bool}]{is\+Playing}{, }\item[{double}]{current\+Time\+Seconds}{, }\item[{double}]{tempo}{}\end{DoxyParamCaption})\hspace{0.3cm}{\ttfamily [virtual]}}



Called when the play head state changes. 

\Hypertarget{class_beam_1_1_audio_processor_a3e9a8c5b7470f9a957d85b84efff64b1}\index{Beam::AudioProcessor@{Beam::AudioProcessor}!setParameter@{setParameter}}
\index{setParameter@{setParameter}!Beam::AudioProcessor@{Beam::AudioProcessor}}
\doxysubsubsection{\texorpdfstring{setParameter()}{setParameter()}}
{\footnotesize\ttfamily \label{class_beam_1_1_audio_processor_a3e9a8c5b7470f9a957d85b84efff64b1} 
void Beam::\+\+Audio\+Processor::\+set\+Parameter (\begin{DoxyParamCaption}\item[{int}]{parameter\+Index}{, }\item[{float}]{new\+Value}{}\end{DoxyParamCaption})\hspace{0.3cm}{\ttfamily [virtual]}}



Sets the value of a parameter. 



\label{doc-variable-members}
\Hypertarget{class_beam_1_1_audio_processor_doc-variable-members}
\doxysubsection{Member Data Documentation}
\Hypertarget{class_beam_1_1_audio_processor_a3d2ab6b1f0c86689d55c8e72ce428466}\index{Beam::AudioProcessor@{Beam::AudioProcessor}!m\_blockSize@{m\_blockSize}}
\index{m\_blockSize@{m\_blockSize}!Beam::AudioProcessor@{Beam::AudioProcessor}}
\doxysubsubsection{\texorpdfstring{m\_blockSize}{m\_blockSize}}
{\footnotesize\ttfamily \label{class_beam_1_1_audio_processor_a3d2ab6b1f0c86689d55c8e72ce428466} 
int Beam::\+\+Audio\+Processor::\+m\+\_\+block\+Size = 512\hspace{0.3cm}{\ttfamily [protected]}}

\Hypertarget{class_beam_1_1_audio_processor_a0430f0f07132e944c8af000cae93f321}\index{Beam::AudioProcessor@{Beam::AudioProcessor}!m\_namedParameters@{m\_namedParameters}}
\index{m\_namedParameters@{m\_namedParameters}!Beam::AudioProcessor@{Beam::AudioProcessor}}
\doxysubsubsection{\texorpdfstring{m\_namedParameters}{m\_namedParameters}}
{\footnotesize\ttfamily \label{class_beam_1_1_audio_processor_a0430f0f07132e944c8af000cae93f321} 
std::\+map$<$std::\+string, std::\+shared\+\_\+ptr$<$\doxymbox{\hyperlink{class_beam_1_1_parameter}{Parameter}}$>$ $>$ Beam::\+\+Audio\+Processor::\+m\+\_\+named\+Parameters\hspace{0.3cm}{\ttfamily [protected]}}

\Hypertarget{class_beam_1_1_audio_processor_a89bb21df73830e6b14b8bf69d3ba68a1}\index{Beam::AudioProcessor@{Beam::AudioProcessor}!m\_parameters@{m\_parameters}}
\index{m\_parameters@{m\_parameters}!Beam::AudioProcessor@{Beam::AudioProcessor}}
\doxysubsubsection{\texorpdfstring{m\_parameters}{m\_parameters}}
{\footnotesize\ttfamily \label{class_beam_1_1_audio_processor_a89bb21df73830e6b14b8bf69d3ba68a1} 
std::\+vector$<$std::\+shared\+\_\+ptr$<$\doxymbox{\hyperlink{class_beam_1_1_parameter}{Parameter}}$>$ $>$ Beam::\+\+Audio\+Processor::\+m\+\_\+parameters\hspace{0.3cm}{\ttfamily [protected]}}

\Hypertarget{class_beam_1_1_audio_processor_add21a9644aaa71b9457b4f508e1e0840}\index{Beam::AudioProcessor@{Beam::AudioProcessor}!m\_sampleRate@{m\_sampleRate}}
\index{m\_sampleRate@{m\_sampleRate}!Beam::AudioProcessor@{Beam::AudioProcessor}}
\doxysubsubsection{\texorpdfstring{m\_sampleRate}{m\_sampleRate}}
{\footnotesize\ttfamily \label{class_beam_1_1_audio_processor_add21a9644aaa71b9457b4f508e1e0840} 
double Beam::\+\+Audio\+Processor::\+m\+\_\+sample\+Rate = 44100.\+0\hspace{0.3cm}{\ttfamily [protected]}}



The documentation for this class was generated from the following files:\+\begin{DoxyCompactItemize}
\item 
src/\+engine/\+\doxymbox{\hyperlink{audio__processor_8hpp}{audio\+\_\+processor.\+hpp}}\item 
src/\+engine/\+\doxymbox{\hyperlink{audio__processor_8cpp}{audio\+\_\+processor.\+cpp}}\end{DoxyCompactItemize}

\doxysection{Beam::\+Audio\+Processor\+Value\+Tree\+State Class Reference}
\hypertarget{class_beam_1_1_audio_processor_value_tree_state}{}\label{class_beam_1_1_audio_processor_value_tree_state}\index{Beam::AudioProcessorValueTreeState@{Beam::AudioProcessorValueTreeState}}


Manages parameters for an audio processor, similar to JUCE\textquotesingle{}s \doxylink{class_beam_1_1_audio_processor_value_tree_state}{Audio\+Processor\+Value\+Tree\+State}.  




{\ttfamily \+\#include $<$audio\+\_\+processor\+\_\+value\+\_\+tree\+\_\+state.\+hpp$>$}

\doxysubsubsection*{Public Member Functions}
\begin{DoxyCompactItemize}
\item 
\doxymbox{\hyperlink{class_beam_1_1_audio_processor_value_tree_state_a89af59fc9dce810673b152f01f20d46d}{Audio\+Processor\+Value\+Tree\+State}} ()
\begin{DoxyCompactList}\small\item\em Constructor. \end{DoxyCompactList}\item 
\doxymbox{\hyperlink{class_beam_1_1_audio_processor_value_tree_state_a488f6f94f2e597cd12e28755ffca7610}{\texorpdfstring{$\sim$}{\string~}\+Audio\+Processor\+Value\+Tree\+State}} ()
\begin{DoxyCompactList}\small\item\em Destructor. \end{DoxyCompactList}\item 
std::\+shared\+\_\+ptr$<$ \doxymbox{\hyperlink{class_beam_1_1_parameter}{Parameter}} $>$ \doxymbox{\hyperlink{class_beam_1_1_audio_processor_value_tree_state_a31631bce42f2e551e522a4ad2a157b6f}{create\+And\+Add\+Parameter}} (const std::\+string \&parameter\+ID, const std::\+string \&parameter\+Name, const std::\+string \&parameter\+Label, float min\+Value, float max\+Value, float default\+Value)
\begin{DoxyCompactList}\small\item\em Creates a parameter and adds it to the state. \end{DoxyCompactList}\item 
std::\+shared\+\_\+ptr$<$ \doxymbox{\hyperlink{class_beam_1_1_parameter}{Parameter}} $>$ \doxymbox{\hyperlink{class_beam_1_1_audio_processor_value_tree_state_a8a5de1df5fa2357ad7a3a91935bbb544}{get\+Parameter}} (const std::\+string \&parameter\+ID) const
\begin{DoxyCompactList}\small\item\em Gets a parameter by ID. \end{DoxyCompactList}\item 
float \doxymbox{\hyperlink{class_beam_1_1_audio_processor_value_tree_state_a879cc7a0234ecb1bef47f35d647318e2}{get\+Parameter\+As\+Float}} (const std::\+string \&parameter\+ID) const
\begin{DoxyCompactList}\small\item\em Gets the value of a parameter as a float. \end{DoxyCompactList}\item 
void \doxymbox{\hyperlink{class_beam_1_1_audio_processor_value_tree_state_ab7baf7ad6dff25ac215bae830afcadee}{set\+Parameter\+As\+Float}} (const std::\+string \&parameter\+ID, float new\+Value)
\begin{DoxyCompactList}\small\item\em Sets the value of a parameter. \end{DoxyCompactList}\item 
void \doxymbox{\hyperlink{class_beam_1_1_audio_processor_value_tree_state_a2c650ebb150473f29e716f086ced90d1}{add\+Parameter\+Listener}} (const std::\+string \&parameter\+ID, std::\+function$<$ void(float)$>$ listener)
\begin{DoxyCompactList}\small\item\em Adds a parameter listener. \end{DoxyCompactList}\item 
void \doxymbox{\hyperlink{class_beam_1_1_audio_processor_value_tree_state_a4245a0a626551851646a1680f6ab4437}{remove\+Parameter\+Listener}} (const std::\+string \&parameter\+ID, std::\+function$<$ void(float)$>$ listener)
\begin{DoxyCompactList}\small\item\em Removes a parameter listener. \end{DoxyCompactList}\item 
const std::\+map$<$ std::\+string, std::\+shared\+\_\+ptr$<$ \doxymbox{\hyperlink{class_beam_1_1_parameter}{Parameter}} $>$ $>$ \& \doxymbox{\hyperlink{class_beam_1_1_audio_processor_value_tree_state_af22fa436bc83c00011ab6f0757ed9aac}{get\+Parameters}} () const
\begin{DoxyCompactList}\small\item\em Gets all parameters. \end{DoxyCompactList}\item 
void \doxymbox{\hyperlink{class_beam_1_1_audio_processor_value_tree_state_ab2b9936e83868de2541ea9c092e3aa19}{reset\+To\+Default}} ()
\begin{DoxyCompactList}\small\item\em Resets all parameters to their default values. \end{DoxyCompactList}\end{DoxyCompactItemize}
\doxysubsubsection*{Private Attributes}
\begin{DoxyCompactItemize}
\item 
std::\+map$<$ std::\+string, std::\+shared\+\_\+ptr$<$ \doxymbox{\hyperlink{class_beam_1_1_parameter}{Parameter}} $>$ $>$ \doxymbox{\hyperlink{class_beam_1_1_audio_processor_value_tree_state_a9058146e725431e074007e89ed76ded5}{m\+\_\+parameters}}
\item 
std::\+map$<$ std::\+string, std::\+function$<$ void(float)$>$ $>$ \doxymbox{\hyperlink{class_beam_1_1_audio_processor_value_tree_state_a0cee656b3be3f606cc2f530327b7432f}{m\+\_\+listeners}}
\end{DoxyCompactItemize}


\doxysubsection{Detailed Description}
Manages parameters for an audio processor, similar to JUCE\textquotesingle{}s \doxylink{class_beam_1_1_audio_processor_value_tree_state}{Audio\+Processor\+Value\+Tree\+State}. 

\label{doc-constructors}
\Hypertarget{class_beam_1_1_audio_processor_value_tree_state_doc-constructors}
\doxysubsection{Constructor \& Destructor Documentation}
\Hypertarget{class_beam_1_1_audio_processor_value_tree_state_a89af59fc9dce810673b152f01f20d46d}\index{Beam::AudioProcessorValueTreeState@{Beam::AudioProcessorValueTreeState}!AudioProcessorValueTreeState@{AudioProcessorValueTreeState}}
\index{AudioProcessorValueTreeState@{AudioProcessorValueTreeState}!Beam::AudioProcessorValueTreeState@{Beam::AudioProcessorValueTreeState}}
\doxysubsubsection{\texorpdfstring{AudioProcessorValueTreeState()}{AudioProcessorValueTreeState()}}
{\footnotesize\ttfamily \label{class_beam_1_1_audio_processor_value_tree_state_a89af59fc9dce810673b152f01f20d46d} 
Beam::\+\+Audio\+Processor\+Value\+Tree\+State::\+\+Audio\+Processor\+Value\+Tree\+State (\begin{DoxyParamCaption}{}{}\end{DoxyParamCaption})}



Constructor. 

\Hypertarget{class_beam_1_1_audio_processor_value_tree_state_a488f6f94f2e597cd12e28755ffca7610}\index{Beam::AudioProcessorValueTreeState@{Beam::AudioProcessorValueTreeState}!````~AudioProcessorValueTreeState@{\texorpdfstring{$\sim$}{\string~}AudioProcessorValueTreeState}}
\index{````~AudioProcessorValueTreeState@{\texorpdfstring{$\sim$}{\string~}AudioProcessorValueTreeState}!Beam::AudioProcessorValueTreeState@{Beam::AudioProcessorValueTreeState}}
\doxysubsubsection{\texorpdfstring{\texorpdfstring{$\sim$}{\string~}AudioProcessorValueTreeState()}{\string~AudioProcessorValueTreeState()}}
{\footnotesize\ttfamily \label{class_beam_1_1_audio_processor_value_tree_state_a488f6f94f2e597cd12e28755ffca7610} 
Beam::\+\+Audio\+Processor\+Value\+Tree\+State::\+\texorpdfstring{$\sim$}{\string~}\+Audio\+Processor\+Value\+Tree\+State (\begin{DoxyParamCaption}{}{}\end{DoxyParamCaption})}



Destructor. 



\label{doc-func-members}
\Hypertarget{class_beam_1_1_audio_processor_value_tree_state_doc-func-members}
\doxysubsection{Member Function Documentation}
\Hypertarget{class_beam_1_1_audio_processor_value_tree_state_a2c650ebb150473f29e716f086ced90d1}\index{Beam::AudioProcessorValueTreeState@{Beam::AudioProcessorValueTreeState}!addParameterListener@{addParameterListener}}
\index{addParameterListener@{addParameterListener}!Beam::AudioProcessorValueTreeState@{Beam::AudioProcessorValueTreeState}}
\doxysubsubsection{\texorpdfstring{addParameterListener()}{addParameterListener()}}
{\footnotesize\ttfamily \label{class_beam_1_1_audio_processor_value_tree_state_a2c650ebb150473f29e716f086ced90d1} 
void Beam::\+\+Audio\+Processor\+Value\+Tree\+State::\+add\+Parameter\+Listener (\begin{DoxyParamCaption}\item[{const std::\+string \&}]{parameter\+ID}{, }\item[{std::\+function$<$ void(float)$>$}]{listener}{}\end{DoxyParamCaption})}



Adds a parameter listener. 

\Hypertarget{class_beam_1_1_audio_processor_value_tree_state_a31631bce42f2e551e522a4ad2a157b6f}\index{Beam::AudioProcessorValueTreeState@{Beam::AudioProcessorValueTreeState}!createAndAddParameter@{createAndAddParameter}}
\index{createAndAddParameter@{createAndAddParameter}!Beam::AudioProcessorValueTreeState@{Beam::AudioProcessorValueTreeState}}
\doxysubsubsection{\texorpdfstring{createAndAddParameter()}{createAndAddParameter()}}
{\footnotesize\ttfamily \label{class_beam_1_1_audio_processor_value_tree_state_a31631bce42f2e551e522a4ad2a157b6f} 
std::\+shared\+\_\+ptr$<$ \doxymbox{\hyperlink{class_beam_1_1_parameter}{Parameter}} $>$ Beam::\+\+Audio\+Processor\+Value\+Tree\+State::\+create\+And\+Add\+Parameter (\begin{DoxyParamCaption}\item[{const std::\+string \&}]{parameter\+ID}{, }\item[{const std::\+string \&}]{parameter\+Name}{, }\item[{const std::\+string \&}]{parameter\+Label}{, }\item[{float}]{min\+Value}{, }\item[{float}]{max\+Value}{, }\item[{float}]{default\+Value}{}\end{DoxyParamCaption})}



Creates a parameter and adds it to the state. 

\Hypertarget{class_beam_1_1_audio_processor_value_tree_state_a8a5de1df5fa2357ad7a3a91935bbb544}\index{Beam::AudioProcessorValueTreeState@{Beam::AudioProcessorValueTreeState}!getParameter@{getParameter}}
\index{getParameter@{getParameter}!Beam::AudioProcessorValueTreeState@{Beam::AudioProcessorValueTreeState}}
\doxysubsubsection{\texorpdfstring{getParameter()}{getParameter()}}
{\footnotesize\ttfamily \label{class_beam_1_1_audio_processor_value_tree_state_a8a5de1df5fa2357ad7a3a91935bbb544} 
std::\+shared\+\_\+ptr$<$ \doxymbox{\hyperlink{class_beam_1_1_parameter}{Parameter}} $>$ Beam::\+\+Audio\+Processor\+Value\+Tree\+State::\+get\+Parameter (\begin{DoxyParamCaption}\item[{const std::\+string \&}]{parameter\+ID}{}\end{DoxyParamCaption}) const}



Gets a parameter by ID. 

\Hypertarget{class_beam_1_1_audio_processor_value_tree_state_a879cc7a0234ecb1bef47f35d647318e2}\index{Beam::AudioProcessorValueTreeState@{Beam::AudioProcessorValueTreeState}!getParameterAsFloat@{getParameterAsFloat}}
\index{getParameterAsFloat@{getParameterAsFloat}!Beam::AudioProcessorValueTreeState@{Beam::AudioProcessorValueTreeState}}
\doxysubsubsection{\texorpdfstring{getParameterAsFloat()}{getParameterAsFloat()}}
{\footnotesize\ttfamily \label{class_beam_1_1_audio_processor_value_tree_state_a879cc7a0234ecb1bef47f35d647318e2} 
float Beam::\+\+Audio\+Processor\+Value\+Tree\+State::\+get\+Parameter\+As\+Float (\begin{DoxyParamCaption}\item[{const std::\+string \&}]{parameter\+ID}{}\end{DoxyParamCaption}) const}



Gets the value of a parameter as a float. 

\Hypertarget{class_beam_1_1_audio_processor_value_tree_state_af22fa436bc83c00011ab6f0757ed9aac}\index{Beam::AudioProcessorValueTreeState@{Beam::AudioProcessorValueTreeState}!getParameters@{getParameters}}
\index{getParameters@{getParameters}!Beam::AudioProcessorValueTreeState@{Beam::AudioProcessorValueTreeState}}
\doxysubsubsection{\texorpdfstring{getParameters()}{getParameters()}}
{\footnotesize\ttfamily \label{class_beam_1_1_audio_processor_value_tree_state_af22fa436bc83c00011ab6f0757ed9aac} 
const std::\+map$<$ std::\+string, std::\+shared\+\_\+ptr$<$ \doxymbox{\hyperlink{class_beam_1_1_parameter}{Parameter}} $>$ $>$ \& Beam::\+\+Audio\+Processor\+Value\+Tree\+State::\+get\+Parameters (\begin{DoxyParamCaption}{}{}\end{DoxyParamCaption}) const}



Gets all parameters. 

\Hypertarget{class_beam_1_1_audio_processor_value_tree_state_a4245a0a626551851646a1680f6ab4437}\index{Beam::AudioProcessorValueTreeState@{Beam::AudioProcessorValueTreeState}!removeParameterListener@{removeParameterListener}}
\index{removeParameterListener@{removeParameterListener}!Beam::AudioProcessorValueTreeState@{Beam::AudioProcessorValueTreeState}}
\doxysubsubsection{\texorpdfstring{removeParameterListener()}{removeParameterListener()}}
{\footnotesize\ttfamily \label{class_beam_1_1_audio_processor_value_tree_state_a4245a0a626551851646a1680f6ab4437} 
void Beam::\+\+Audio\+Processor\+Value\+Tree\+State::\+remove\+Parameter\+Listener (\begin{DoxyParamCaption}\item[{const std::\+string \&}]{parameter\+ID}{, }\item[{std::\+function$<$ void(float)$>$}]{listener}{}\end{DoxyParamCaption})}



Removes a parameter listener. 

\Hypertarget{class_beam_1_1_audio_processor_value_tree_state_ab2b9936e83868de2541ea9c092e3aa19}\index{Beam::AudioProcessorValueTreeState@{Beam::AudioProcessorValueTreeState}!resetToDefault@{resetToDefault}}
\index{resetToDefault@{resetToDefault}!Beam::AudioProcessorValueTreeState@{Beam::AudioProcessorValueTreeState}}
\doxysubsubsection{\texorpdfstring{resetToDefault()}{resetToDefault()}}
{\footnotesize\ttfamily \label{class_beam_1_1_audio_processor_value_tree_state_ab2b9936e83868de2541ea9c092e3aa19} 
void Beam::\+\+Audio\+Processor\+Value\+Tree\+State::\+reset\+To\+Default (\begin{DoxyParamCaption}{}{}\end{DoxyParamCaption})}



Resets all parameters to their default values. 

\Hypertarget{class_beam_1_1_audio_processor_value_tree_state_ab7baf7ad6dff25ac215bae830afcadee}\index{Beam::AudioProcessorValueTreeState@{Beam::AudioProcessorValueTreeState}!setParameterAsFloat@{setParameterAsFloat}}
\index{setParameterAsFloat@{setParameterAsFloat}!Beam::AudioProcessorValueTreeState@{Beam::AudioProcessorValueTreeState}}
\doxysubsubsection{\texorpdfstring{setParameterAsFloat()}{setParameterAsFloat()}}
{\footnotesize\ttfamily \label{class_beam_1_1_audio_processor_value_tree_state_ab7baf7ad6dff25ac215bae830afcadee} 
void Beam::\+\+Audio\+Processor\+Value\+Tree\+State::\+set\+Parameter\+As\+Float (\begin{DoxyParamCaption}\item[{const std::\+string \&}]{parameter\+ID}{, }\item[{float}]{new\+Value}{}\end{DoxyParamCaption})}



Sets the value of a parameter. 



\label{doc-variable-members}
\Hypertarget{class_beam_1_1_audio_processor_value_tree_state_doc-variable-members}
\doxysubsection{Member Data Documentation}
\Hypertarget{class_beam_1_1_audio_processor_value_tree_state_a0cee656b3be3f606cc2f530327b7432f}\index{Beam::AudioProcessorValueTreeState@{Beam::AudioProcessorValueTreeState}!m\_listeners@{m\_listeners}}
\index{m\_listeners@{m\_listeners}!Beam::AudioProcessorValueTreeState@{Beam::AudioProcessorValueTreeState}}
\doxysubsubsection{\texorpdfstring{m\_listeners}{m\_listeners}}
{\footnotesize\ttfamily \label{class_beam_1_1_audio_processor_value_tree_state_a0cee656b3be3f606cc2f530327b7432f} 
std::\+map$<$std::\+string, std::\+function$<$void(float)$>$ $>$ Beam::\+\+Audio\+Processor\+Value\+Tree\+State::\+m\+\_\+listeners\hspace{0.3cm}{\ttfamily [private]}}

\Hypertarget{class_beam_1_1_audio_processor_value_tree_state_a9058146e725431e074007e89ed76ded5}\index{Beam::AudioProcessorValueTreeState@{Beam::AudioProcessorValueTreeState}!m\_parameters@{m\_parameters}}
\index{m\_parameters@{m\_parameters}!Beam::AudioProcessorValueTreeState@{Beam::AudioProcessorValueTreeState}}
\doxysubsubsection{\texorpdfstring{m\_parameters}{m\_parameters}}
{\footnotesize\ttfamily \label{class_beam_1_1_audio_processor_value_tree_state_a9058146e725431e074007e89ed76ded5} 
std::\+map$<$std::\+string, std::\+shared\+\_\+ptr$<$\doxymbox{\hyperlink{class_beam_1_1_parameter}{Parameter}}$>$ $>$ Beam::\+\+Audio\+Processor\+Value\+Tree\+State::\+m\+\_\+parameters\hspace{0.3cm}{\ttfamily [private]}}



The documentation for this class was generated from the following files:\+\begin{DoxyCompactItemize}
\item 
src/\+engine/\+\doxymbox{\hyperlink{audio__processor__value__tree__state_8hpp}{audio\+\_\+processor\+\_\+value\+\_\+tree\+\_\+state.\+hpp}}\item 
src/\+engine/\+\doxymbox{\hyperlink{audio__processor__value__tree__state_8cpp}{audio\+\_\+processor\+\_\+value\+\_\+tree\+\_\+state.\+cpp}}\end{DoxyCompactItemize}

\doxysection{Beam::\+Audio\+Reader Class Reference}
\hypertarget{class_beam_1_1_audio_reader}{}\label{class_beam_1_1_audio_reader}\index{Beam::AudioReader@{Beam::AudioReader}}


Universal audio reader using miniaudio\textquotesingle{}s decoder. Supports WAV, MP3, FLAC, etc.  




{\ttfamily \+\#include $<$audio\+\_\+reader.\+hpp$>$}

\doxysubsubsection*{Public Member Functions}
\begin{DoxyCompactItemize}
\item 
\doxymbox{\hyperlink{class_beam_1_1_audio_reader_ae4a56c5f148770c773af2a9fb420310a}{Audio\+Reader}} ()
\item 
\doxymbox{\hyperlink{class_beam_1_1_audio_reader_a21636c4b3318702dfd8ee9a44673ab7a}{\texorpdfstring{$\sim$}{\string~}\+Audio\+Reader}} ()
\item 
bool \doxymbox{\hyperlink{class_beam_1_1_audio_reader_a9b7d56e818f7713f66dc7034ef4ec2f6}{open}} (const std::\+string \&file\+Path, int target\+Channels=2)
\item 
void \doxymbox{\hyperlink{class_beam_1_1_audio_reader_a77131e7bd2db6d44ae899de90f4fb9d4}{close}} ()
\item 
size\+\_\+t \doxymbox{\hyperlink{class_beam_1_1_audio_reader_adc6266eb0274d4e1449f1c70391a3b7e}{read\+Frames}} (float \texorpdfstring{$\ast$}{*}buffer, size\+\_\+t frames, int dest\+Channels)
\item 
void \doxymbox{\hyperlink{class_beam_1_1_audio_reader_a1fa33f3e75524650944a9570b08ca533}{seek}} (size\+\_\+t frame)
\item 
uint32\+\_\+t \doxymbox{\hyperlink{class_beam_1_1_audio_reader_afead01b7db78e8e597b2d527f5375680}{get\+Sample\+Rate}} () const
\item 
uint32\+\_\+t \doxymbox{\hyperlink{class_beam_1_1_audio_reader_a7916ff1975fb3596ffd6418716c18563}{get\+Channels}} () const
\item 
uint64\+\_\+t \doxymbox{\hyperlink{class_beam_1_1_audio_reader_aadf285f15de29c86b809762174cffc5a}{get\+Total\+Frames}} () const
\item 
std::\+vector$<$ std::\+vector$<$ float $>$ $>$ \doxymbox{\hyperlink{class_beam_1_1_audio_reader_ade48fa4295efbeda7fb10cd8f4e06551}{get\+Peak\+Data}} (int num\+Points)
\end{DoxyCompactItemize}
\doxysubsubsection*{Private Attributes}
\begin{DoxyCompactItemize}
\item 
ma\+\_\+decoder \doxymbox{\hyperlink{class_beam_1_1_audio_reader_a1ed857bef5848ff75653fea75d1e7c51}{m\+\_\+decoder}}
\item 
bool \doxymbox{\hyperlink{class_beam_1_1_audio_reader_a6d9fca276cd0aef64371cee9d9bac9fb}{m\+\_\+is\+Initialized}}
\item 
std::\+mutex \doxymbox{\hyperlink{class_beam_1_1_audio_reader_a3536c1167f89c47f55d5f438395c02a2}{m\+\_\+mutex}}
\end{DoxyCompactItemize}


\doxysubsection{Detailed Description}
Universal audio reader using miniaudio\textquotesingle{}s decoder. Supports WAV, MP3, FLAC, etc. 

\label{doc-constructors}
\Hypertarget{class_beam_1_1_audio_reader_doc-constructors}
\doxysubsection{Constructor \& Destructor Documentation}
\Hypertarget{class_beam_1_1_audio_reader_ae4a56c5f148770c773af2a9fb420310a}\index{Beam::AudioReader@{Beam::AudioReader}!AudioReader@{AudioReader}}
\index{AudioReader@{AudioReader}!Beam::AudioReader@{Beam::AudioReader}}
\doxysubsubsection{\texorpdfstring{AudioReader()}{AudioReader()}}
{\footnotesize\ttfamily \label{class_beam_1_1_audio_reader_ae4a56c5f148770c773af2a9fb420310a} 
Beam::\+\+Audio\+Reader::\+\+Audio\+Reader (\begin{DoxyParamCaption}{}{}\end{DoxyParamCaption})\hspace{0.3cm}{\ttfamily [inline]}}

\Hypertarget{class_beam_1_1_audio_reader_a21636c4b3318702dfd8ee9a44673ab7a}\index{Beam::AudioReader@{Beam::AudioReader}!````~AudioReader@{\texorpdfstring{$\sim$}{\string~}AudioReader}}
\index{````~AudioReader@{\texorpdfstring{$\sim$}{\string~}AudioReader}!Beam::AudioReader@{Beam::AudioReader}}
\doxysubsubsection{\texorpdfstring{\texorpdfstring{$\sim$}{\string~}AudioReader()}{\string~AudioReader()}}
{\footnotesize\ttfamily \label{class_beam_1_1_audio_reader_a21636c4b3318702dfd8ee9a44673ab7a} 
Beam::\+\+Audio\+Reader::\+\texorpdfstring{$\sim$}{\string~}\+Audio\+Reader (\begin{DoxyParamCaption}{}{}\end{DoxyParamCaption})\hspace{0.3cm}{\ttfamily [inline]}}



\label{doc-func-members}
\Hypertarget{class_beam_1_1_audio_reader_doc-func-members}
\doxysubsection{Member Function Documentation}
\Hypertarget{class_beam_1_1_audio_reader_a77131e7bd2db6d44ae899de90f4fb9d4}\index{Beam::AudioReader@{Beam::AudioReader}!close@{close}}
\index{close@{close}!Beam::AudioReader@{Beam::AudioReader}}
\doxysubsubsection{\texorpdfstring{close()}{close()}}
{\footnotesize\ttfamily \label{class_beam_1_1_audio_reader_a77131e7bd2db6d44ae899de90f4fb9d4} 
void Beam::\+\+Audio\+Reader::\+close (\begin{DoxyParamCaption}{}{}\end{DoxyParamCaption})\hspace{0.3cm}{\ttfamily [inline]}}

\Hypertarget{class_beam_1_1_audio_reader_a7916ff1975fb3596ffd6418716c18563}\index{Beam::AudioReader@{Beam::AudioReader}!getChannels@{getChannels}}
\index{getChannels@{getChannels}!Beam::AudioReader@{Beam::AudioReader}}
\doxysubsubsection{\texorpdfstring{getChannels()}{getChannels()}}
{\footnotesize\ttfamily \label{class_beam_1_1_audio_reader_a7916ff1975fb3596ffd6418716c18563} 
uint32\+\_\+t Beam::\+\+Audio\+Reader::\+get\+Channels (\begin{DoxyParamCaption}{}{}\end{DoxyParamCaption}) const\hspace{0.3cm}{\ttfamily [inline]}}

\Hypertarget{class_beam_1_1_audio_reader_ade48fa4295efbeda7fb10cd8f4e06551}\index{Beam::AudioReader@{Beam::AudioReader}!getPeakData@{getPeakData}}
\index{getPeakData@{getPeakData}!Beam::AudioReader@{Beam::AudioReader}}
\doxysubsubsection{\texorpdfstring{getPeakData()}{getPeakData()}}
{\footnotesize\ttfamily \label{class_beam_1_1_audio_reader_ade48fa4295efbeda7fb10cd8f4e06551} 
std::\+vector$<$ std::\+vector$<$ float $>$ $>$ Beam::\+\+Audio\+Reader::\+get\+Peak\+Data (\begin{DoxyParamCaption}\item[{int}]{num\+Points}{}\end{DoxyParamCaption})\hspace{0.3cm}{\ttfamily [inline]}}

\Hypertarget{class_beam_1_1_audio_reader_afead01b7db78e8e597b2d527f5375680}\index{Beam::AudioReader@{Beam::AudioReader}!getSampleRate@{getSampleRate}}
\index{getSampleRate@{getSampleRate}!Beam::AudioReader@{Beam::AudioReader}}
\doxysubsubsection{\texorpdfstring{getSampleRate()}{getSampleRate()}}
{\footnotesize\ttfamily \label{class_beam_1_1_audio_reader_afead01b7db78e8e597b2d527f5375680} 
uint32\+\_\+t Beam::\+\+Audio\+Reader::\+get\+Sample\+Rate (\begin{DoxyParamCaption}{}{}\end{DoxyParamCaption}) const\hspace{0.3cm}{\ttfamily [inline]}}

\Hypertarget{class_beam_1_1_audio_reader_aadf285f15de29c86b809762174cffc5a}\index{Beam::AudioReader@{Beam::AudioReader}!getTotalFrames@{getTotalFrames}}
\index{getTotalFrames@{getTotalFrames}!Beam::AudioReader@{Beam::AudioReader}}
\doxysubsubsection{\texorpdfstring{getTotalFrames()}{getTotalFrames()}}
{\footnotesize\ttfamily \label{class_beam_1_1_audio_reader_aadf285f15de29c86b809762174cffc5a} 
uint64\+\_\+t Beam::\+\+Audio\+Reader::\+get\+Total\+Frames (\begin{DoxyParamCaption}{}{}\end{DoxyParamCaption}) const\hspace{0.3cm}{\ttfamily [inline]}}

\Hypertarget{class_beam_1_1_audio_reader_a9b7d56e818f7713f66dc7034ef4ec2f6}\index{Beam::AudioReader@{Beam::AudioReader}!open@{open}}
\index{open@{open}!Beam::AudioReader@{Beam::AudioReader}}
\doxysubsubsection{\texorpdfstring{open()}{open()}}
{\footnotesize\ttfamily \label{class_beam_1_1_audio_reader_a9b7d56e818f7713f66dc7034ef4ec2f6} 
bool Beam::\+\+Audio\+Reader::\+open (\begin{DoxyParamCaption}\item[{const std::\+string \&}]{file\+Path}{, }\item[{int}]{target\+Channels}{ = {\ttfamily 2}}\end{DoxyParamCaption})\hspace{0.3cm}{\ttfamily [inline]}}

\Hypertarget{class_beam_1_1_audio_reader_adc6266eb0274d4e1449f1c70391a3b7e}\index{Beam::AudioReader@{Beam::AudioReader}!readFrames@{readFrames}}
\index{readFrames@{readFrames}!Beam::AudioReader@{Beam::AudioReader}}
\doxysubsubsection{\texorpdfstring{readFrames()}{readFrames()}}
{\footnotesize\ttfamily \label{class_beam_1_1_audio_reader_adc6266eb0274d4e1449f1c70391a3b7e} 
size\+\_\+t Beam::\+\+Audio\+Reader::\+read\+Frames (\begin{DoxyParamCaption}\item[{float \texorpdfstring{$\ast$}{*}}]{buffer}{, }\item[{size\+\_\+t}]{frames}{, }\item[{int}]{dest\+Channels}{}\end{DoxyParamCaption})\hspace{0.3cm}{\ttfamily [inline]}}

\Hypertarget{class_beam_1_1_audio_reader_a1fa33f3e75524650944a9570b08ca533}\index{Beam::AudioReader@{Beam::AudioReader}!seek@{seek}}
\index{seek@{seek}!Beam::AudioReader@{Beam::AudioReader}}
\doxysubsubsection{\texorpdfstring{seek()}{seek()}}
{\footnotesize\ttfamily \label{class_beam_1_1_audio_reader_a1fa33f3e75524650944a9570b08ca533} 
void Beam::\+\+Audio\+Reader::\+seek (\begin{DoxyParamCaption}\item[{size\+\_\+t}]{frame}{}\end{DoxyParamCaption})\hspace{0.3cm}{\ttfamily [inline]}}



\label{doc-variable-members}
\Hypertarget{class_beam_1_1_audio_reader_doc-variable-members}
\doxysubsection{Member Data Documentation}
\Hypertarget{class_beam_1_1_audio_reader_a1ed857bef5848ff75653fea75d1e7c51}\index{Beam::AudioReader@{Beam::AudioReader}!m\_decoder@{m\_decoder}}
\index{m\_decoder@{m\_decoder}!Beam::AudioReader@{Beam::AudioReader}}
\doxysubsubsection{\texorpdfstring{m\_decoder}{m\_decoder}}
{\footnotesize\ttfamily \label{class_beam_1_1_audio_reader_a1ed857bef5848ff75653fea75d1e7c51} 
ma\+\_\+decoder Beam::\+\+Audio\+Reader::\+m\+\_\+decoder\hspace{0.3cm}{\ttfamily [private]}}

\Hypertarget{class_beam_1_1_audio_reader_a6d9fca276cd0aef64371cee9d9bac9fb}\index{Beam::AudioReader@{Beam::AudioReader}!m\_isInitialized@{m\_isInitialized}}
\index{m\_isInitialized@{m\_isInitialized}!Beam::AudioReader@{Beam::AudioReader}}
\doxysubsubsection{\texorpdfstring{m\_isInitialized}{m\_isInitialized}}
{\footnotesize\ttfamily \label{class_beam_1_1_audio_reader_a6d9fca276cd0aef64371cee9d9bac9fb} 
bool Beam::\+\+Audio\+Reader::\+m\+\_\+is\+Initialized\hspace{0.3cm}{\ttfamily [private]}}

\Hypertarget{class_beam_1_1_audio_reader_a3536c1167f89c47f55d5f438395c02a2}\index{Beam::AudioReader@{Beam::AudioReader}!m\_mutex@{m\_mutex}}
\index{m\_mutex@{m\_mutex}!Beam::AudioReader@{Beam::AudioReader}}
\doxysubsubsection{\texorpdfstring{m\_mutex}{m\_mutex}}
{\footnotesize\ttfamily \label{class_beam_1_1_audio_reader_a3536c1167f89c47f55d5f438395c02a2} 
std::\+mutex Beam::\+\+Audio\+Reader::\+m\+\_\+mutex\hspace{0.3cm}{\ttfamily [private]}}



The documentation for this class was generated from the following file:\+\begin{DoxyCompactItemize}
\item 
src/\+engine/\+\doxymbox{\hyperlink{audio__reader_8hpp}{audio\+\_\+reader.\+hpp}}\end{DoxyCompactItemize}

\doxysection{Beam::\+Audio\+Utils Class Reference}
\hypertarget{class_beam_1_1_audio_utils}{}\label{class_beam_1_1_audio_utils}\index{Beam::AudioUtils@{Beam::AudioUtils}}


Utility functions for audio processing, similar to JUCE\textquotesingle{}s dsp module.  




{\ttfamily \+\#include $<$flux\+\_\+audio\+\_\+utils.\+hpp$>$}

\doxysubsubsection*{Static Public Member Functions}
\begin{DoxyCompactItemize}
\item 
{\footnotesize template$<$typename T$>$ }\\static void \doxymbox{\hyperlink{class_beam_1_1_audio_utils_ad739e5556afef878bc1279e69a19ad4e}{apply\+Gain}} (\doxymbox{\hyperlink{class_beam_1_1_audio_buffer}{Audio\+Buffer}}$<$ T $>$ \&buffer, T gain)
\begin{DoxyCompactList}\small\item\em Applies a simple linear gain to a buffer. \end{DoxyCompactList}\item 
{\footnotesize template$<$typename T$>$ }\\static void \doxymbox{\hyperlink{class_beam_1_1_audio_utils_a3630d5e4134ec025ceb4e62c67191f08}{apply\+Gain\+Ramp}} (\doxymbox{\hyperlink{class_beam_1_1_audio_buffer}{Audio\+Buffer}}$<$ T $>$ \&buffer, int channel, int start\+Sample, int num\+Samples, T start\+Gain, T end\+Gain)
\begin{DoxyCompactList}\small\item\em Applies a gain ramp over a range of samples. \end{DoxyCompactList}\item 
{\footnotesize template$<$typename T$>$ }\\static void \doxymbox{\hyperlink{class_beam_1_1_audio_utils_ad70a2259392bee7768cab6e7cb39db75}{copy\+Buffer}} (\doxymbox{\hyperlink{class_beam_1_1_audio_buffer}{Audio\+Buffer}}$<$ T $>$ \&dest, const \doxymbox{\hyperlink{class_beam_1_1_audio_buffer}{Audio\+Buffer}}$<$ T $>$ \&src, int dest\+Start\+Sample=0, int src\+Start\+Sample=0, int num\+Samples=-\/1)
\begin{DoxyCompactList}\small\item\em Copies audio data from source to destination. \end{DoxyCompactList}\item 
{\footnotesize template$<$typename T$>$ }\\static void \doxymbox{\hyperlink{class_beam_1_1_audio_utils_a5e5e34b0ed28b3f849757c6848fa1c6c}{add\+Buffer}} (\doxymbox{\hyperlink{class_beam_1_1_audio_buffer}{Audio\+Buffer}}$<$ T $>$ \&dest, const \doxymbox{\hyperlink{class_beam_1_1_audio_buffer}{Audio\+Buffer}}$<$ T $>$ \&src, int dest\+Start\+Sample=0, int src\+Start\+Sample=0, int num\+Samples=-\/1, T gain=static\+\_\+cast$<$ T $>$(1))
\begin{DoxyCompactList}\small\item\em Adds audio data from source to destination. \end{DoxyCompactList}\item 
{\footnotesize template$<$typename T$>$ }\\static void \doxymbox{\hyperlink{class_beam_1_1_audio_utils_af1720f182e432d86f3f5f12b5d42c254}{generate\+Sine\+Wave}} (\doxymbox{\hyperlink{class_beam_1_1_audio_buffer}{Audio\+Buffer}}$<$ T $>$ \&buffer, T frequency, T sample\+Rate, T amplitude=static\+\_\+cast$<$ T $>$(0.\+5))
\begin{DoxyCompactList}\small\item\em Generates a sine wave into the buffer. \end{DoxyCompactList}\item 
{\footnotesize template$<$typename T$>$ }\\static T \doxymbox{\hyperlink{class_beam_1_1_audio_utils_a289e69e3e48a00249414ac8af58e6b66}{clamp}} (T value, T min, T max)
\begin{DoxyCompactList}\small\item\em Clamps a value to a specified range. \end{DoxyCompactList}\item 
{\footnotesize template$<$typename T$>$ }\\static T \doxymbox{\hyperlink{class_beam_1_1_audio_utils_ae385f12bc0a6bb9f50341c59b39a3504}{linear\+To\+Decibels}} (T linear)
\begin{DoxyCompactList}\small\item\em Converts from linear amplitude to decibels. \end{DoxyCompactList}\item 
{\footnotesize template$<$typename T$>$ }\\static T \doxymbox{\hyperlink{class_beam_1_1_audio_utils_ace53ae1468004ff96950d1d2d20e2612}{decibels\+To\+Linear}} (T db)
\begin{DoxyCompactList}\small\item\em Converts from decibels to linear amplitude. \end{DoxyCompactList}\item 
static void \doxymbox{\hyperlink{class_beam_1_1_audio_utils_a47dc61c21cf5d067b8c584a35d360aa9}{draw\+Scrolling\+Text}} (\doxymbox{\hyperlink{class_beam_1_1_quad_batcher}{Quad\+Batcher}} \&batcher, const std::\+string \&text, float x, float y, float w, float h, float size, float dt, float \&timer, float screen\+Height)
\begin{DoxyCompactList}\small\item\em Draws text that scrolls horizontally if it exceeds the available width. \end{DoxyCompactList}\end{DoxyCompactItemize}


\doxysubsection{Detailed Description}
Utility functions for audio processing, similar to JUCE\textquotesingle{}s dsp module. 

\label{doc-func-members}
\Hypertarget{class_beam_1_1_audio_utils_doc-func-members}
\doxysubsection{Member Function Documentation}
\Hypertarget{class_beam_1_1_audio_utils_a5e5e34b0ed28b3f849757c6848fa1c6c}\index{Beam::AudioUtils@{Beam::AudioUtils}!addBuffer@{addBuffer}}
\index{addBuffer@{addBuffer}!Beam::AudioUtils@{Beam::AudioUtils}}
\doxysubsubsection{\texorpdfstring{addBuffer()}{addBuffer()}}
{\footnotesize\ttfamily \label{class_beam_1_1_audio_utils_a5e5e34b0ed28b3f849757c6848fa1c6c} 
template$<$typename T$>$ \\
void Beam::\+\+Audio\+Utils::\+add\+Buffer (\begin{DoxyParamCaption}\item[{\doxymbox{\hyperlink{class_beam_1_1_audio_buffer}{Audio\+Buffer}}$<$ T $>$ \&}]{dest}{, }\item[{const \doxymbox{\hyperlink{class_beam_1_1_audio_buffer}{Audio\+Buffer}}$<$ T $>$ \&}]{src}{, }\item[{int}]{dest\+Start\+Sample}{ = {\ttfamily 0}, }\item[{int}]{src\+Start\+Sample}{ = {\ttfamily 0}, }\item[{int}]{num\+Samples}{ = {\ttfamily -\/1}, }\item[{T}]{gain}{ = {\ttfamily static\+\_\+cast$<$T$>$(1)}}\end{DoxyParamCaption})\hspace{0.3cm}{\ttfamily [inline]}, {\ttfamily [static]}}



Adds audio data from source to destination. 

\Hypertarget{class_beam_1_1_audio_utils_ad739e5556afef878bc1279e69a19ad4e}\index{Beam::AudioUtils@{Beam::AudioUtils}!applyGain@{applyGain}}
\index{applyGain@{applyGain}!Beam::AudioUtils@{Beam::AudioUtils}}
\doxysubsubsection{\texorpdfstring{applyGain()}{applyGain()}}
{\footnotesize\ttfamily \label{class_beam_1_1_audio_utils_ad739e5556afef878bc1279e69a19ad4e} 
template$<$typename T$>$ \\
void Beam::\+\+Audio\+Utils::\+apply\+Gain (\begin{DoxyParamCaption}\item[{\doxymbox{\hyperlink{class_beam_1_1_audio_buffer}{Audio\+Buffer}}$<$ T $>$ \&}]{buffer}{, }\item[{T}]{gain}{}\end{DoxyParamCaption})\hspace{0.3cm}{\ttfamily [inline]}, {\ttfamily [static]}}



Applies a simple linear gain to a buffer. 

\Hypertarget{class_beam_1_1_audio_utils_a3630d5e4134ec025ceb4e62c67191f08}\index{Beam::AudioUtils@{Beam::AudioUtils}!applyGainRamp@{applyGainRamp}}
\index{applyGainRamp@{applyGainRamp}!Beam::AudioUtils@{Beam::AudioUtils}}
\doxysubsubsection{\texorpdfstring{applyGainRamp()}{applyGainRamp()}}
{\footnotesize\ttfamily \label{class_beam_1_1_audio_utils_a3630d5e4134ec025ceb4e62c67191f08} 
template$<$typename T$>$ \\
void Beam::\+\+Audio\+Utils::\+apply\+Gain\+Ramp (\begin{DoxyParamCaption}\item[{\doxymbox{\hyperlink{class_beam_1_1_audio_buffer}{Audio\+Buffer}}$<$ T $>$ \&}]{buffer}{, }\item[{int}]{channel}{, }\item[{int}]{start\+Sample}{, }\item[{int}]{num\+Samples}{, }\item[{T}]{start\+Gain}{, }\item[{T}]{end\+Gain}{}\end{DoxyParamCaption})\hspace{0.3cm}{\ttfamily [inline]}, {\ttfamily [static]}}



Applies a gain ramp over a range of samples. 

\Hypertarget{class_beam_1_1_audio_utils_a289e69e3e48a00249414ac8af58e6b66}\index{Beam::AudioUtils@{Beam::AudioUtils}!clamp@{clamp}}
\index{clamp@{clamp}!Beam::AudioUtils@{Beam::AudioUtils}}
\doxysubsubsection{\texorpdfstring{clamp()}{clamp()}}
{\footnotesize\ttfamily \label{class_beam_1_1_audio_utils_a289e69e3e48a00249414ac8af58e6b66} 
template$<$typename T$>$ \\
T Beam::\+\+Audio\+Utils::\+clamp (\begin{DoxyParamCaption}\item[{T}]{value}{, }\item[{T}]{min}{, }\item[{T}]{max}{}\end{DoxyParamCaption})\hspace{0.3cm}{\ttfamily [inline]}, {\ttfamily [static]}}



Clamps a value to a specified range. 

\Hypertarget{class_beam_1_1_audio_utils_ad70a2259392bee7768cab6e7cb39db75}\index{Beam::AudioUtils@{Beam::AudioUtils}!copyBuffer@{copyBuffer}}
\index{copyBuffer@{copyBuffer}!Beam::AudioUtils@{Beam::AudioUtils}}
\doxysubsubsection{\texorpdfstring{copyBuffer()}{copyBuffer()}}
{\footnotesize\ttfamily \label{class_beam_1_1_audio_utils_ad70a2259392bee7768cab6e7cb39db75} 
template$<$typename T$>$ \\
void Beam::\+\+Audio\+Utils::\+copy\+Buffer (\begin{DoxyParamCaption}\item[{\doxymbox{\hyperlink{class_beam_1_1_audio_buffer}{Audio\+Buffer}}$<$ T $>$ \&}]{dest}{, }\item[{const \doxymbox{\hyperlink{class_beam_1_1_audio_buffer}{Audio\+Buffer}}$<$ T $>$ \&}]{src}{, }\item[{int}]{dest\+Start\+Sample}{ = {\ttfamily 0}, }\item[{int}]{src\+Start\+Sample}{ = {\ttfamily 0}, }\item[{int}]{num\+Samples}{ = {\ttfamily -\/1}}\end{DoxyParamCaption})\hspace{0.3cm}{\ttfamily [inline]}, {\ttfamily [static]}}



Copies audio data from source to destination. 

\Hypertarget{class_beam_1_1_audio_utils_ace53ae1468004ff96950d1d2d20e2612}\index{Beam::AudioUtils@{Beam::AudioUtils}!decibelsToLinear@{decibelsToLinear}}
\index{decibelsToLinear@{decibelsToLinear}!Beam::AudioUtils@{Beam::AudioUtils}}
\doxysubsubsection{\texorpdfstring{decibelsToLinear()}{decibelsToLinear()}}
{\footnotesize\ttfamily \label{class_beam_1_1_audio_utils_ace53ae1468004ff96950d1d2d20e2612} 
template$<$typename T$>$ \\
T Beam::\+\+Audio\+Utils::\+decibels\+To\+Linear (\begin{DoxyParamCaption}\item[{T}]{db}{}\end{DoxyParamCaption})\hspace{0.3cm}{\ttfamily [inline]}, {\ttfamily [static]}}



Converts from decibels to linear amplitude. 

\Hypertarget{class_beam_1_1_audio_utils_a47dc61c21cf5d067b8c584a35d360aa9}\index{Beam::AudioUtils@{Beam::AudioUtils}!drawScrollingText@{drawScrollingText}}
\index{drawScrollingText@{drawScrollingText}!Beam::AudioUtils@{Beam::AudioUtils}}
\doxysubsubsection{\texorpdfstring{drawScrollingText()}{drawScrollingText()}}
{\footnotesize\ttfamily \label{class_beam_1_1_audio_utils_a47dc61c21cf5d067b8c584a35d360aa9} 
void Beam::\+\+Audio\+Utils::\+draw\+Scrolling\+Text (\begin{DoxyParamCaption}\item[{\doxymbox{\hyperlink{class_beam_1_1_quad_batcher}{Quad\+Batcher}} \&}]{batcher}{, }\item[{const std::\+string \&}]{text}{, }\item[{float}]{x}{, }\item[{float}]{y}{, }\item[{float}]{w}{, }\item[{float}]{h}{, }\item[{float}]{size}{, }\item[{float}]{dt}{, }\item[{float \&}]{timer}{, }\item[{float}]{screen\+Height}{}\end{DoxyParamCaption})\hspace{0.3cm}{\ttfamily [inline]}, {\ttfamily [static]}}



Draws text that scrolls horizontally if it exceeds the available width. 

\Hypertarget{class_beam_1_1_audio_utils_af1720f182e432d86f3f5f12b5d42c254}\index{Beam::AudioUtils@{Beam::AudioUtils}!generateSineWave@{generateSineWave}}
\index{generateSineWave@{generateSineWave}!Beam::AudioUtils@{Beam::AudioUtils}}
\doxysubsubsection{\texorpdfstring{generateSineWave()}{generateSineWave()}}
{\footnotesize\ttfamily \label{class_beam_1_1_audio_utils_af1720f182e432d86f3f5f12b5d42c254} 
template$<$typename T$>$ \\
void Beam::\+\+Audio\+Utils::\+generate\+Sine\+Wave (\begin{DoxyParamCaption}\item[{\doxymbox{\hyperlink{class_beam_1_1_audio_buffer}{Audio\+Buffer}}$<$ T $>$ \&}]{buffer}{, }\item[{T}]{frequency}{, }\item[{T}]{sample\+Rate}{, }\item[{T}]{amplitude}{ = {\ttfamily static\+\_\+cast$<$T$>$(0.5)}}\end{DoxyParamCaption})\hspace{0.3cm}{\ttfamily [inline]}, {\ttfamily [static]}}



Generates a sine wave into the buffer. 

\Hypertarget{class_beam_1_1_audio_utils_ae385f12bc0a6bb9f50341c59b39a3504}\index{Beam::AudioUtils@{Beam::AudioUtils}!linearToDecibels@{linearToDecibels}}
\index{linearToDecibels@{linearToDecibels}!Beam::AudioUtils@{Beam::AudioUtils}}
\doxysubsubsection{\texorpdfstring{linearToDecibels()}{linearToDecibels()}}
{\footnotesize\ttfamily \label{class_beam_1_1_audio_utils_ae385f12bc0a6bb9f50341c59b39a3504} 
template$<$typename T$>$ \\
T Beam::\+\+Audio\+Utils::\+linear\+To\+Decibels (\begin{DoxyParamCaption}\item[{T}]{linear}{}\end{DoxyParamCaption})\hspace{0.3cm}{\ttfamily [inline]}, {\ttfamily [static]}}



Converts from linear amplitude to decibels. 



The documentation for this class was generated from the following file:\+\begin{DoxyCompactItemize}
\item 
src/\+utilities/\+\doxymbox{\hyperlink{flux__audio__utils_8hpp}{flux\+\_\+audio\+\_\+utils.\+hpp}}\end{DoxyCompactItemize}

\doxysection{Beam::\+Automation\+Lane Class Reference}
\hypertarget{class_beam_1_1_automation_lane}{}\label{class_beam_1_1_automation_lane}\index{Beam::AutomationLane@{Beam::AutomationLane}}


Manages a sequence of automation points for a single parameter.  




{\ttfamily \+\#include $<$automation.\+hpp$>$}

\doxysubsubsection*{Public Member Functions}
\begin{DoxyCompactItemize}
\item 
\doxymbox{\hyperlink{class_beam_1_1_automation_lane_ae8cbacaae7fc64fa0e13780176573350}{Automation\+Lane}} (std::\+shared\+\_\+ptr$<$ \doxymbox{\hyperlink{class_beam_1_1_parameter}{Parameter}} $>$ \doxymbox{\hyperlink{texture_8cpp_aaded45152436a99bb4f9bda081df9f69}{param}})
\item 
void \doxymbox{\hyperlink{class_beam_1_1_automation_lane_a67f68a5b1e55c0ab0afeac41576fc57c}{add\+Point}} (size\+\_\+t frame, float value)
\begin{DoxyCompactList}\small\item\em Adds or updates a point at a specific frame. \end{DoxyCompactList}\item 
float \doxymbox{\hyperlink{class_beam_1_1_automation_lane_a5d6f16ed5be148993ac6d6f3bd4f8261}{get\+Value\+At}} (size\+\_\+t frame) const
\begin{DoxyCompactList}\small\item\em Interpolates the value for a specific frame. \end{DoxyCompactList}\item 
void \doxymbox{\hyperlink{class_beam_1_1_automation_lane_a1a775ca71ceb821f40e04796bf23b652}{apply\+At}} (size\+\_\+t frame)
\begin{DoxyCompactList}\small\item\em Applies the interpolated value to the linked parameter. \end{DoxyCompactList}\item 
std::\+shared\+\_\+ptr$<$ \doxymbox{\hyperlink{class_beam_1_1_parameter}{Parameter}} $>$ \doxymbox{\hyperlink{class_beam_1_1_automation_lane_afe88351f80f91b4a13febf7ab3eb43f1}{get\+Parameter}} ()
\end{DoxyCompactItemize}
\doxysubsubsection*{Private Attributes}
\begin{DoxyCompactItemize}
\item 
std::\+shared\+\_\+ptr$<$ \doxymbox{\hyperlink{class_beam_1_1_parameter}{Parameter}} $>$ \doxymbox{\hyperlink{class_beam_1_1_automation_lane_a0e828bb1c71c02818525fd17fcf24913}{m\+\_\+parameter}}
\item 
std::\+vector$<$ \doxymbox{\hyperlink{struct_beam_1_1_automation_point}{Automation\+Point}} $>$ \doxymbox{\hyperlink{class_beam_1_1_automation_lane_a028d89dc1c675f6b58bcb72c484c3180}{m\+\_\+points}}
\end{DoxyCompactItemize}


\doxysubsection{Detailed Description}
Manages a sequence of automation points for a single parameter. 

\label{doc-constructors}
\Hypertarget{class_beam_1_1_automation_lane_doc-constructors}
\doxysubsection{Constructor \& Destructor Documentation}
\Hypertarget{class_beam_1_1_automation_lane_ae8cbacaae7fc64fa0e13780176573350}\index{Beam::AutomationLane@{Beam::AutomationLane}!AutomationLane@{AutomationLane}}
\index{AutomationLane@{AutomationLane}!Beam::AutomationLane@{Beam::AutomationLane}}
\doxysubsubsection{\texorpdfstring{AutomationLane()}{AutomationLane()}}
{\footnotesize\ttfamily \label{class_beam_1_1_automation_lane_ae8cbacaae7fc64fa0e13780176573350} 
Beam::\+\+Automation\+Lane::\+\+Automation\+Lane (\begin{DoxyParamCaption}\item[{std::\+shared\+\_\+ptr$<$ \doxymbox{\hyperlink{class_beam_1_1_parameter}{Parameter}} $>$}]{param}{}\end{DoxyParamCaption})\hspace{0.3cm}{\ttfamily [inline]}, {\ttfamily [explicit]}}



\label{doc-func-members}
\Hypertarget{class_beam_1_1_automation_lane_doc-func-members}
\doxysubsection{Member Function Documentation}
\Hypertarget{class_beam_1_1_automation_lane_a67f68a5b1e55c0ab0afeac41576fc57c}\index{Beam::AutomationLane@{Beam::AutomationLane}!addPoint@{addPoint}}
\index{addPoint@{addPoint}!Beam::AutomationLane@{Beam::AutomationLane}}
\doxysubsubsection{\texorpdfstring{addPoint()}{addPoint()}}
{\footnotesize\ttfamily \label{class_beam_1_1_automation_lane_a67f68a5b1e55c0ab0afeac41576fc57c} 
void Beam::\+\+Automation\+Lane::\+add\+Point (\begin{DoxyParamCaption}\item[{size\+\_\+t}]{frame}{, }\item[{float}]{value}{}\end{DoxyParamCaption})\hspace{0.3cm}{\ttfamily [inline]}}



Adds or updates a point at a specific frame. 

\Hypertarget{class_beam_1_1_automation_lane_a1a775ca71ceb821f40e04796bf23b652}\index{Beam::AutomationLane@{Beam::AutomationLane}!applyAt@{applyAt}}
\index{applyAt@{applyAt}!Beam::AutomationLane@{Beam::AutomationLane}}
\doxysubsubsection{\texorpdfstring{applyAt()}{applyAt()}}
{\footnotesize\ttfamily \label{class_beam_1_1_automation_lane_a1a775ca71ceb821f40e04796bf23b652} 
void Beam::\+\+Automation\+Lane::\+apply\+At (\begin{DoxyParamCaption}\item[{size\+\_\+t}]{frame}{}\end{DoxyParamCaption})\hspace{0.3cm}{\ttfamily [inline]}}



Applies the interpolated value to the linked parameter. 

\Hypertarget{class_beam_1_1_automation_lane_afe88351f80f91b4a13febf7ab3eb43f1}\index{Beam::AutomationLane@{Beam::AutomationLane}!getParameter@{getParameter}}
\index{getParameter@{getParameter}!Beam::AutomationLane@{Beam::AutomationLane}}
\doxysubsubsection{\texorpdfstring{getParameter()}{getParameter()}}
{\footnotesize\ttfamily \label{class_beam_1_1_automation_lane_afe88351f80f91b4a13febf7ab3eb43f1} 
std::\+shared\+\_\+ptr$<$ \doxymbox{\hyperlink{class_beam_1_1_parameter}{Parameter}} $>$ Beam::\+\+Automation\+Lane::\+get\+Parameter (\begin{DoxyParamCaption}{}{}\end{DoxyParamCaption})\hspace{0.3cm}{\ttfamily [inline]}}

\Hypertarget{class_beam_1_1_automation_lane_a5d6f16ed5be148993ac6d6f3bd4f8261}\index{Beam::AutomationLane@{Beam::AutomationLane}!getValueAt@{getValueAt}}
\index{getValueAt@{getValueAt}!Beam::AutomationLane@{Beam::AutomationLane}}
\doxysubsubsection{\texorpdfstring{getValueAt()}{getValueAt()}}
{\footnotesize\ttfamily \label{class_beam_1_1_automation_lane_a5d6f16ed5be148993ac6d6f3bd4f8261} 
float Beam::\+\+Automation\+Lane::\+get\+Value\+At (\begin{DoxyParamCaption}\item[{size\+\_\+t}]{frame}{}\end{DoxyParamCaption}) const\hspace{0.3cm}{\ttfamily [inline]}}



Interpolates the value for a specific frame. 



\label{doc-variable-members}
\Hypertarget{class_beam_1_1_automation_lane_doc-variable-members}
\doxysubsection{Member Data Documentation}
\Hypertarget{class_beam_1_1_automation_lane_a0e828bb1c71c02818525fd17fcf24913}\index{Beam::AutomationLane@{Beam::AutomationLane}!m\_parameter@{m\_parameter}}
\index{m\_parameter@{m\_parameter}!Beam::AutomationLane@{Beam::AutomationLane}}
\doxysubsubsection{\texorpdfstring{m\_parameter}{m\_parameter}}
{\footnotesize\ttfamily \label{class_beam_1_1_automation_lane_a0e828bb1c71c02818525fd17fcf24913} 
std::\+shared\+\_\+ptr$<$\doxymbox{\hyperlink{class_beam_1_1_parameter}{Parameter}}$>$ Beam::\+\+Automation\+Lane::\+m\+\_\+parameter\hspace{0.3cm}{\ttfamily [private]}}

\Hypertarget{class_beam_1_1_automation_lane_a028d89dc1c675f6b58bcb72c484c3180}\index{Beam::AutomationLane@{Beam::AutomationLane}!m\_points@{m\_points}}
\index{m\_points@{m\_points}!Beam::AutomationLane@{Beam::AutomationLane}}
\doxysubsubsection{\texorpdfstring{m\_points}{m\_points}}
{\footnotesize\ttfamily \label{class_beam_1_1_automation_lane_a028d89dc1c675f6b58bcb72c484c3180} 
std::\+vector$<$\doxymbox{\hyperlink{struct_beam_1_1_automation_point}{Automation\+Point}}$>$ Beam::\+\+Automation\+Lane::\+m\+\_\+points\hspace{0.3cm}{\ttfamily [private]}}



The documentation for this class was generated from the following file:\+\begin{DoxyCompactItemize}
\item 
src/\+session/\+\doxymbox{\hyperlink{automation_8hpp}{automation.\+hpp}}\end{DoxyCompactItemize}

\doxysection{Beam::\+Automation\+Point Struct Reference}
\hypertarget{struct_beam_1_1_automation_point}{}\label{struct_beam_1_1_automation_point}\index{Beam::AutomationPoint@{Beam::AutomationPoint}}


A single point in an automation lane.  




{\ttfamily \+\#include $<$automation.\+hpp$>$}

\doxysubsubsection*{Public Attributes}
\begin{DoxyCompactItemize}
\item 
size\+\_\+t \doxymbox{\hyperlink{struct_beam_1_1_automation_point_a36f0e31abcdfd0df016bd2cdb8036cbc}{frame}}
\begin{DoxyCompactList}\small\item\em \doxylink{class_beam_1_1_timeline}{Timeline} position in frames. \end{DoxyCompactList}\item 
float \doxymbox{\hyperlink{struct_beam_1_1_automation_point_ad351d506767e558f336485f17c4e7106}{value}}
\begin{DoxyCompactList}\small\item\em \doxylink{class_beam_1_1_parameter}{Parameter} value at this position. \end{DoxyCompactList}\end{DoxyCompactItemize}


\doxysubsection{Detailed Description}
A single point in an automation lane. 

\label{doc-variable-members}
\Hypertarget{struct_beam_1_1_automation_point_doc-variable-members}
\doxysubsection{Member Data Documentation}
\Hypertarget{struct_beam_1_1_automation_point_a36f0e31abcdfd0df016bd2cdb8036cbc}\index{Beam::AutomationPoint@{Beam::AutomationPoint}!frame@{frame}}
\index{frame@{frame}!Beam::AutomationPoint@{Beam::AutomationPoint}}
\doxysubsubsection{\texorpdfstring{frame}{frame}}
{\footnotesize\ttfamily \label{struct_beam_1_1_automation_point_a36f0e31abcdfd0df016bd2cdb8036cbc} 
size\+\_\+t Beam::\+\+Automation\+Point::\+frame}



\doxylink{class_beam_1_1_timeline}{Timeline} position in frames. 

\Hypertarget{struct_beam_1_1_automation_point_ad351d506767e558f336485f17c4e7106}\index{Beam::AutomationPoint@{Beam::AutomationPoint}!value@{value}}
\index{value@{value}!Beam::AutomationPoint@{Beam::AutomationPoint}}
\doxysubsubsection{\texorpdfstring{value}{value}}
{\footnotesize\ttfamily \label{struct_beam_1_1_automation_point_ad351d506767e558f336485f17c4e7106} 
float Beam::\+\+Automation\+Point::\+value}



\doxylink{class_beam_1_1_parameter}{Parameter} value at this position. 



The documentation for this struct was generated from the following file:\+\begin{DoxyCompactItemize}
\item 
src/\+session/\+\doxymbox{\hyperlink{automation_8hpp}{automation.\+hpp}}\end{DoxyCompactItemize}

\doxysection{Beam::\+Biquad\+Filter\+Node Class Reference}
\hypertarget{class_beam_1_1_biquad_filter_node}{}\label{class_beam_1_1_biquad_filter_node}\index{Beam::BiquadFilterNode@{Beam::BiquadFilterNode}}


{\ttfamily \+\#include $<$biquad\+\_\+filter\+\_\+node.\+hpp$>$}

Inheritance diagram for Beam::\+Biquad\+Filter\+Node:\+\begin{figure}[H]
\begin{center}
\leavevmode
\includegraphics[height=2.000000cm]{class_beam_1_1_biquad_filter_node}
\end{center}
\end{figure}
\doxysubsubsection*{Public Member Functions}
\begin{DoxyCompactItemize}
\item 
\doxymbox{\hyperlink{class_beam_1_1_biquad_filter_node_af9d8020ffe968d1c127ea2105209bb26}{Biquad\+Filter\+Node}} (\doxymbox{\hyperlink{namespace_beam_aa79c53bfd2295b09750b964fa2a326d6}{Filter\+Type}} \doxymbox{\hyperlink{texture_8cpp_ad9d356d2dd3763f53f907d0f196e48fa}{type}}, float frequency, float q, float sample\+Rate)
\item 
void \doxymbox{\hyperlink{class_beam_1_1_biquad_filter_node_a9ca4ea9dffee4a78f9c25457c2303142}{process}} (float \texorpdfstring{$\ast$}{*}buffer, int frames, int channels, size\+\_\+t start\+Frame=0) override
\item 
std::\+string \doxymbox{\hyperlink{class_beam_1_1_biquad_filter_node_a928e2017ed0dacfc547360c6fb982181}{get\+Name}} () const override
\item 
void \doxymbox{\hyperlink{class_beam_1_1_biquad_filter_node_ad291d833705c4629e1b1016426f85bf8}{set\+Cutoff}} (float freq)
\item 
void \doxymbox{\hyperlink{class_beam_1_1_biquad_filter_node_a239d00981422f9f84dd8c6694bf35f94}{setQ}} (float q)
\item 
float \doxymbox{\hyperlink{class_beam_1_1_biquad_filter_node_a0a857df3e78f9f3bb258d0661fbad824}{get\+Magnitude\+Response}} (float normalized\+Freq)
\begin{DoxyCompactList}\small\item\em Calculates the magnitude response at a given normalized frequency (0..1, where 1 is Nyquist). \end{DoxyCompactList}\end{DoxyCompactItemize}
\doxysubsection*{Public Member Functions inherited from \doxymbox{\hyperlink{class_beam_1_1_audio_node}{Beam::\+\+Audio\+Node}}}
\begin{DoxyCompactItemize}
\item 
virtual \doxymbox{\hyperlink{class_beam_1_1_audio_node_afbea31954b50918131d31fc0d1f6de8c}{\texorpdfstring{$\sim$}{\string~}\+Audio\+Node}} ()=default
\item 
void \doxymbox{\hyperlink{class_beam_1_1_audio_node_a3fcc68eab5b1adf547a4205f258b212c}{set\+Bypass}} (bool bypass)
\item 
bool \doxymbox{\hyperlink{class_beam_1_1_audio_node_a6ba1724cff34b5bc0f811ee2537caae5}{is\+Bypassed}} () const
\end{DoxyCompactItemize}
\doxysubsubsection*{Private Member Functions}
\begin{DoxyCompactItemize}
\item 
void \doxymbox{\hyperlink{class_beam_1_1_biquad_filter_node_a9c0e833373e0a400667ab1e9f28fca99}{calculate\+Coefficients}} ()
\end{DoxyCompactItemize}
\doxysubsubsection*{Private Attributes}
\begin{DoxyCompactItemize}
\item 
\doxymbox{\hyperlink{namespace_beam_aa79c53bfd2295b09750b964fa2a326d6}{Filter\+Type}} \doxymbox{\hyperlink{class_beam_1_1_biquad_filter_node_af192ad68fc1a2e246e0a262f143829bd}{m\+\_\+type}}
\item 
float \doxymbox{\hyperlink{class_beam_1_1_biquad_filter_node_a6e60d0bd884240f8dd68bf329c7f0b02}{m\+\_\+frequency}}
\item 
float \doxymbox{\hyperlink{class_beam_1_1_biquad_filter_node_a6fd09afdc0987fd14b036640808ebe32}{m\+\_\+q}}
\item 
float \doxymbox{\hyperlink{class_beam_1_1_biquad_filter_node_a286bb954a110d6aca89e5538efcd048f}{m\+\_\+sample\+Rate}}
\item 
float \doxymbox{\hyperlink{class_beam_1_1_biquad_filter_node_a4587479c8ba49ab9a4b2de94a6f0762d}{m\+\_\+b0}}
\item 
float \doxymbox{\hyperlink{class_beam_1_1_biquad_filter_node_af8fa1d7d77e7c505b91f193b6abbc430}{m\+\_\+b1}}
\item 
float \doxymbox{\hyperlink{class_beam_1_1_biquad_filter_node_ad38af20f8589c09d399dda3d67974388}{m\+\_\+b2}}
\item 
float \doxymbox{\hyperlink{class_beam_1_1_biquad_filter_node_a540ecacb972cb9ffaf89b7a1b590db23}{m\+\_\+a0}}
\item 
float \doxymbox{\hyperlink{class_beam_1_1_biquad_filter_node_a591ffbe6ab67f70256077313a73e115b}{m\+\_\+a1}}
\item 
float \doxymbox{\hyperlink{class_beam_1_1_biquad_filter_node_ae3562ee8ee5b0aad417518d88ad4ac9b}{m\+\_\+a2}}
\item 
std::\+vector$<$ float $>$ \doxymbox{\hyperlink{class_beam_1_1_biquad_filter_node_a03c950f82a534c8a03288b26c36d46a0}{m\+\_\+x1}}
\item 
std::\+vector$<$ float $>$ \doxymbox{\hyperlink{class_beam_1_1_biquad_filter_node_ab51d7636914b505a233bfe06610c9cba}{m\+\_\+x2}}
\item 
std::\+vector$<$ float $>$ \doxymbox{\hyperlink{class_beam_1_1_biquad_filter_node_a628729d1e9a421749eb4d92601ff9bb0}{m\+\_\+y1}}
\item 
std::\+vector$<$ float $>$ \doxymbox{\hyperlink{class_beam_1_1_biquad_filter_node_a1608507a8153863f66811e35d0a640ad}{m\+\_\+y2}}
\end{DoxyCompactItemize}
\doxysubsubsection*{Additional Inherited Members}
\doxysubsection*{Protected Attributes inherited from \doxymbox{\hyperlink{class_beam_1_1_audio_node}{Beam::\+\+Audio\+Node}}}
\begin{DoxyCompactItemize}
\item 
bool \doxymbox{\hyperlink{class_beam_1_1_audio_node_ac5ad81de4a5d0abe555fe9f06219b09f}{m\+\_\+is\+Bypassed}} = false
\end{DoxyCompactItemize}


\label{doc-constructors}
\Hypertarget{class_beam_1_1_biquad_filter_node_doc-constructors}
\doxysubsection{Constructor \& Destructor Documentation}
\Hypertarget{class_beam_1_1_biquad_filter_node_af9d8020ffe968d1c127ea2105209bb26}\index{Beam::BiquadFilterNode@{Beam::BiquadFilterNode}!BiquadFilterNode@{BiquadFilterNode}}
\index{BiquadFilterNode@{BiquadFilterNode}!Beam::BiquadFilterNode@{Beam::BiquadFilterNode}}
\doxysubsubsection{\texorpdfstring{BiquadFilterNode()}{BiquadFilterNode()}}
{\footnotesize\ttfamily \label{class_beam_1_1_biquad_filter_node_af9d8020ffe968d1c127ea2105209bb26} 
Beam::\+\+Biquad\+Filter\+Node::\+\+Biquad\+Filter\+Node (\begin{DoxyParamCaption}\item[{\doxymbox{\hyperlink{namespace_beam_aa79c53bfd2295b09750b964fa2a326d6}{Filter\+Type}}}]{type}{, }\item[{float}]{frequency}{, }\item[{float}]{q}{, }\item[{float}]{sample\+Rate}{}\end{DoxyParamCaption})\hspace{0.3cm}{\ttfamily [inline]}}



\label{doc-func-members}
\Hypertarget{class_beam_1_1_biquad_filter_node_doc-func-members}
\doxysubsection{Member Function Documentation}
\Hypertarget{class_beam_1_1_biquad_filter_node_a9c0e833373e0a400667ab1e9f28fca99}\index{Beam::BiquadFilterNode@{Beam::BiquadFilterNode}!calculateCoefficients@{calculateCoefficients}}
\index{calculateCoefficients@{calculateCoefficients}!Beam::BiquadFilterNode@{Beam::BiquadFilterNode}}
\doxysubsubsection{\texorpdfstring{calculateCoefficients()}{calculateCoefficients()}}
{\footnotesize\ttfamily \label{class_beam_1_1_biquad_filter_node_a9c0e833373e0a400667ab1e9f28fca99} 
void Beam::\+\+Biquad\+Filter\+Node::\+calculate\+Coefficients (\begin{DoxyParamCaption}{}{}\end{DoxyParamCaption})\hspace{0.3cm}{\ttfamily [inline]}, {\ttfamily [private]}}

\Hypertarget{class_beam_1_1_biquad_filter_node_a0a857df3e78f9f3bb258d0661fbad824}\index{Beam::BiquadFilterNode@{Beam::BiquadFilterNode}!getMagnitudeResponse@{getMagnitudeResponse}}
\index{getMagnitudeResponse@{getMagnitudeResponse}!Beam::BiquadFilterNode@{Beam::BiquadFilterNode}}
\doxysubsubsection{\texorpdfstring{getMagnitudeResponse()}{getMagnitudeResponse()}}
{\footnotesize\ttfamily \label{class_beam_1_1_biquad_filter_node_a0a857df3e78f9f3bb258d0661fbad824} 
float Beam::\+\+Biquad\+Filter\+Node::\+get\+Magnitude\+Response (\begin{DoxyParamCaption}\item[{float}]{normalized\+Freq}{}\end{DoxyParamCaption})\hspace{0.3cm}{\ttfamily [inline]}}



Calculates the magnitude response at a given normalized frequency (0..1, where 1 is Nyquist). 

\Hypertarget{class_beam_1_1_biquad_filter_node_a928e2017ed0dacfc547360c6fb982181}\index{Beam::BiquadFilterNode@{Beam::BiquadFilterNode}!getName@{getName}}
\index{getName@{getName}!Beam::BiquadFilterNode@{Beam::BiquadFilterNode}}
\doxysubsubsection{\texorpdfstring{getName()}{getName()}}
{\footnotesize\ttfamily \label{class_beam_1_1_biquad_filter_node_a928e2017ed0dacfc547360c6fb982181} 
std::\+string Beam::\+\+Biquad\+Filter\+Node::\+get\+Name (\begin{DoxyParamCaption}{}{}\end{DoxyParamCaption}) const\hspace{0.3cm}{\ttfamily [inline]}, {\ttfamily [override]}, {\ttfamily [virtual]}}



Implements \doxymbox{\hyperlink{class_beam_1_1_audio_node_a864b3bf9095638e43ad334b7b3706bec}{Beam::\+\+Audio\+Node}}.

\Hypertarget{class_beam_1_1_biquad_filter_node_a9ca4ea9dffee4a78f9c25457c2303142}\index{Beam::BiquadFilterNode@{Beam::BiquadFilterNode}!process@{process}}
\index{process@{process}!Beam::BiquadFilterNode@{Beam::BiquadFilterNode}}
\doxysubsubsection{\texorpdfstring{process()}{process()}}
{\footnotesize\ttfamily \label{class_beam_1_1_biquad_filter_node_a9ca4ea9dffee4a78f9c25457c2303142} 
void Beam::\+\+Biquad\+Filter\+Node::\+process (\begin{DoxyParamCaption}\item[{float \texorpdfstring{$\ast$}{*}}]{buffer}{, }\item[{int}]{frames}{, }\item[{int}]{channels}{, }\item[{size\+\_\+t}]{start\+Frame}{ = {\ttfamily 0}}\end{DoxyParamCaption})\hspace{0.3cm}{\ttfamily [inline]}, {\ttfamily [override]}, {\ttfamily [virtual]}}



Implements \doxymbox{\hyperlink{class_beam_1_1_audio_node_ab0cb6fa1aba031e703be16be26e0d6b7}{Beam::\+\+Audio\+Node}}.

\Hypertarget{class_beam_1_1_biquad_filter_node_ad291d833705c4629e1b1016426f85bf8}\index{Beam::BiquadFilterNode@{Beam::BiquadFilterNode}!setCutoff@{setCutoff}}
\index{setCutoff@{setCutoff}!Beam::BiquadFilterNode@{Beam::BiquadFilterNode}}
\doxysubsubsection{\texorpdfstring{setCutoff()}{setCutoff()}}
{\footnotesize\ttfamily \label{class_beam_1_1_biquad_filter_node_ad291d833705c4629e1b1016426f85bf8} 
void Beam::\+\+Biquad\+Filter\+Node::\+set\+Cutoff (\begin{DoxyParamCaption}\item[{float}]{freq}{}\end{DoxyParamCaption})\hspace{0.3cm}{\ttfamily [inline]}}

\Hypertarget{class_beam_1_1_biquad_filter_node_a239d00981422f9f84dd8c6694bf35f94}\index{Beam::BiquadFilterNode@{Beam::BiquadFilterNode}!setQ@{setQ}}
\index{setQ@{setQ}!Beam::BiquadFilterNode@{Beam::BiquadFilterNode}}
\doxysubsubsection{\texorpdfstring{setQ()}{setQ()}}
{\footnotesize\ttfamily \label{class_beam_1_1_biquad_filter_node_a239d00981422f9f84dd8c6694bf35f94} 
void Beam::\+\+Biquad\+Filter\+Node::\+setQ (\begin{DoxyParamCaption}\item[{float}]{q}{}\end{DoxyParamCaption})\hspace{0.3cm}{\ttfamily [inline]}}



\label{doc-variable-members}
\Hypertarget{class_beam_1_1_biquad_filter_node_doc-variable-members}
\doxysubsection{Member Data Documentation}
\Hypertarget{class_beam_1_1_biquad_filter_node_a540ecacb972cb9ffaf89b7a1b590db23}\index{Beam::BiquadFilterNode@{Beam::BiquadFilterNode}!m\_a0@{m\_a0}}
\index{m\_a0@{m\_a0}!Beam::BiquadFilterNode@{Beam::BiquadFilterNode}}
\doxysubsubsection{\texorpdfstring{m\_a0}{m\_a0}}
{\footnotesize\ttfamily \label{class_beam_1_1_biquad_filter_node_a540ecacb972cb9ffaf89b7a1b590db23} 
float Beam::\+\+Biquad\+Filter\+Node::\+m\+\_\+a0\hspace{0.3cm}{\ttfamily [private]}}

\Hypertarget{class_beam_1_1_biquad_filter_node_a591ffbe6ab67f70256077313a73e115b}\index{Beam::BiquadFilterNode@{Beam::BiquadFilterNode}!m\_a1@{m\_a1}}
\index{m\_a1@{m\_a1}!Beam::BiquadFilterNode@{Beam::BiquadFilterNode}}
\doxysubsubsection{\texorpdfstring{m\_a1}{m\_a1}}
{\footnotesize\ttfamily \label{class_beam_1_1_biquad_filter_node_a591ffbe6ab67f70256077313a73e115b} 
float Beam::\+\+Biquad\+Filter\+Node::\+m\+\_\+a1\hspace{0.3cm}{\ttfamily [private]}}

\Hypertarget{class_beam_1_1_biquad_filter_node_ae3562ee8ee5b0aad417518d88ad4ac9b}\index{Beam::BiquadFilterNode@{Beam::BiquadFilterNode}!m\_a2@{m\_a2}}
\index{m\_a2@{m\_a2}!Beam::BiquadFilterNode@{Beam::BiquadFilterNode}}
\doxysubsubsection{\texorpdfstring{m\_a2}{m\_a2}}
{\footnotesize\ttfamily \label{class_beam_1_1_biquad_filter_node_ae3562ee8ee5b0aad417518d88ad4ac9b} 
float Beam::\+\+Biquad\+Filter\+Node::\+m\+\_\+a2\hspace{0.3cm}{\ttfamily [private]}}

\Hypertarget{class_beam_1_1_biquad_filter_node_a4587479c8ba49ab9a4b2de94a6f0762d}\index{Beam::BiquadFilterNode@{Beam::BiquadFilterNode}!m\_b0@{m\_b0}}
\index{m\_b0@{m\_b0}!Beam::BiquadFilterNode@{Beam::BiquadFilterNode}}
\doxysubsubsection{\texorpdfstring{m\_b0}{m\_b0}}
{\footnotesize\ttfamily \label{class_beam_1_1_biquad_filter_node_a4587479c8ba49ab9a4b2de94a6f0762d} 
float Beam::\+\+Biquad\+Filter\+Node::\+m\+\_\+b0\hspace{0.3cm}{\ttfamily [private]}}

\Hypertarget{class_beam_1_1_biquad_filter_node_af8fa1d7d77e7c505b91f193b6abbc430}\index{Beam::BiquadFilterNode@{Beam::BiquadFilterNode}!m\_b1@{m\_b1}}
\index{m\_b1@{m\_b1}!Beam::BiquadFilterNode@{Beam::BiquadFilterNode}}
\doxysubsubsection{\texorpdfstring{m\_b1}{m\_b1}}
{\footnotesize\ttfamily \label{class_beam_1_1_biquad_filter_node_af8fa1d7d77e7c505b91f193b6abbc430} 
float Beam::\+\+Biquad\+Filter\+Node::\+m\+\_\+b1\hspace{0.3cm}{\ttfamily [private]}}

\Hypertarget{class_beam_1_1_biquad_filter_node_ad38af20f8589c09d399dda3d67974388}\index{Beam::BiquadFilterNode@{Beam::BiquadFilterNode}!m\_b2@{m\_b2}}
\index{m\_b2@{m\_b2}!Beam::BiquadFilterNode@{Beam::BiquadFilterNode}}
\doxysubsubsection{\texorpdfstring{m\_b2}{m\_b2}}
{\footnotesize\ttfamily \label{class_beam_1_1_biquad_filter_node_ad38af20f8589c09d399dda3d67974388} 
float Beam::\+\+Biquad\+Filter\+Node::\+m\+\_\+b2\hspace{0.3cm}{\ttfamily [private]}}

\Hypertarget{class_beam_1_1_biquad_filter_node_a6e60d0bd884240f8dd68bf329c7f0b02}\index{Beam::BiquadFilterNode@{Beam::BiquadFilterNode}!m\_frequency@{m\_frequency}}
\index{m\_frequency@{m\_frequency}!Beam::BiquadFilterNode@{Beam::BiquadFilterNode}}
\doxysubsubsection{\texorpdfstring{m\_frequency}{m\_frequency}}
{\footnotesize\ttfamily \label{class_beam_1_1_biquad_filter_node_a6e60d0bd884240f8dd68bf329c7f0b02} 
float Beam::\+\+Biquad\+Filter\+Node::\+m\+\_\+frequency\hspace{0.3cm}{\ttfamily [private]}}

\Hypertarget{class_beam_1_1_biquad_filter_node_a6fd09afdc0987fd14b036640808ebe32}\index{Beam::BiquadFilterNode@{Beam::BiquadFilterNode}!m\_q@{m\_q}}
\index{m\_q@{m\_q}!Beam::BiquadFilterNode@{Beam::BiquadFilterNode}}
\doxysubsubsection{\texorpdfstring{m\_q}{m\_q}}
{\footnotesize\ttfamily \label{class_beam_1_1_biquad_filter_node_a6fd09afdc0987fd14b036640808ebe32} 
float Beam::\+\+Biquad\+Filter\+Node::\+m\+\_\+q\hspace{0.3cm}{\ttfamily [private]}}

\Hypertarget{class_beam_1_1_biquad_filter_node_a286bb954a110d6aca89e5538efcd048f}\index{Beam::BiquadFilterNode@{Beam::BiquadFilterNode}!m\_sampleRate@{m\_sampleRate}}
\index{m\_sampleRate@{m\_sampleRate}!Beam::BiquadFilterNode@{Beam::BiquadFilterNode}}
\doxysubsubsection{\texorpdfstring{m\_sampleRate}{m\_sampleRate}}
{\footnotesize\ttfamily \label{class_beam_1_1_biquad_filter_node_a286bb954a110d6aca89e5538efcd048f} 
float Beam::\+\+Biquad\+Filter\+Node::\+m\+\_\+sample\+Rate\hspace{0.3cm}{\ttfamily [private]}}

\Hypertarget{class_beam_1_1_biquad_filter_node_af192ad68fc1a2e246e0a262f143829bd}\index{Beam::BiquadFilterNode@{Beam::BiquadFilterNode}!m\_type@{m\_type}}
\index{m\_type@{m\_type}!Beam::BiquadFilterNode@{Beam::BiquadFilterNode}}
\doxysubsubsection{\texorpdfstring{m\_type}{m\_type}}
{\footnotesize\ttfamily \label{class_beam_1_1_biquad_filter_node_af192ad68fc1a2e246e0a262f143829bd} 
\doxymbox{\hyperlink{namespace_beam_aa79c53bfd2295b09750b964fa2a326d6}{Filter\+Type}} Beam::\+\+Biquad\+Filter\+Node::\+m\+\_\+type\hspace{0.3cm}{\ttfamily [private]}}

\Hypertarget{class_beam_1_1_biquad_filter_node_a03c950f82a534c8a03288b26c36d46a0}\index{Beam::BiquadFilterNode@{Beam::BiquadFilterNode}!m\_x1@{m\_x1}}
\index{m\_x1@{m\_x1}!Beam::BiquadFilterNode@{Beam::BiquadFilterNode}}
\doxysubsubsection{\texorpdfstring{m\_x1}{m\_x1}}
{\footnotesize\ttfamily \label{class_beam_1_1_biquad_filter_node_a03c950f82a534c8a03288b26c36d46a0} 
std::\+vector$<$float$>$ Beam::\+\+Biquad\+Filter\+Node::\+m\+\_\+x1\hspace{0.3cm}{\ttfamily [private]}}

\Hypertarget{class_beam_1_1_biquad_filter_node_ab51d7636914b505a233bfe06610c9cba}\index{Beam::BiquadFilterNode@{Beam::BiquadFilterNode}!m\_x2@{m\_x2}}
\index{m\_x2@{m\_x2}!Beam::BiquadFilterNode@{Beam::BiquadFilterNode}}
\doxysubsubsection{\texorpdfstring{m\_x2}{m\_x2}}
{\footnotesize\ttfamily \label{class_beam_1_1_biquad_filter_node_ab51d7636914b505a233bfe06610c9cba} 
std::\+vector$<$float$>$ Beam::\+\+Biquad\+Filter\+Node::\+m\+\_\+x2\hspace{0.3cm}{\ttfamily [private]}}

\Hypertarget{class_beam_1_1_biquad_filter_node_a628729d1e9a421749eb4d92601ff9bb0}\index{Beam::BiquadFilterNode@{Beam::BiquadFilterNode}!m\_y1@{m\_y1}}
\index{m\_y1@{m\_y1}!Beam::BiquadFilterNode@{Beam::BiquadFilterNode}}
\doxysubsubsection{\texorpdfstring{m\_y1}{m\_y1}}
{\footnotesize\ttfamily \label{class_beam_1_1_biquad_filter_node_a628729d1e9a421749eb4d92601ff9bb0} 
std::\+vector$<$float$>$ Beam::\+\+Biquad\+Filter\+Node::\+m\+\_\+y1\hspace{0.3cm}{\ttfamily [private]}}

\Hypertarget{class_beam_1_1_biquad_filter_node_a1608507a8153863f66811e35d0a640ad}\index{Beam::BiquadFilterNode@{Beam::BiquadFilterNode}!m\_y2@{m\_y2}}
\index{m\_y2@{m\_y2}!Beam::BiquadFilterNode@{Beam::BiquadFilterNode}}
\doxysubsubsection{\texorpdfstring{m\_y2}{m\_y2}}
{\footnotesize\ttfamily \label{class_beam_1_1_biquad_filter_node_a1608507a8153863f66811e35d0a640ad} 
std::\+vector$<$float$>$ Beam::\+\+Biquad\+Filter\+Node::\+m\+\_\+y2\hspace{0.3cm}{\ttfamily [private]}}



The documentation for this class was generated from the following file:\+\begin{DoxyCompactItemize}
\item 
src/\+engine/\+\doxymbox{\hyperlink{biquad__filter__node_8hpp}{biquad\+\_\+filter\+\_\+node.\+hpp}}\end{DoxyCompactItemize}

\doxysection{Beam::\+Render\+Plan::\+Buffer\+Clear\+Op Struct Reference}
\hypertarget{struct_beam_1_1_render_plan_1_1_buffer_clear_op}{}\label{struct_beam_1_1_render_plan_1_1_buffer_clear_op}\index{Beam::RenderPlan::BufferClearOp@{Beam::RenderPlan::BufferClearOp}}


{\ttfamily \+\#include $<$render\+\_\+plan.\+hpp$>$}

\doxysubsubsection*{Public Attributes}
\begin{DoxyCompactItemize}
\item 
std::\+shared\+\_\+ptr$<$ \doxymbox{\hyperlink{class_beam_1_1_flux_node}{Flux\+Node}} $>$ \doxymbox{\hyperlink{struct_beam_1_1_render_plan_1_1_buffer_clear_op_a86757a99d31ed231e1011800922b49f6}{node}}
\item 
int \doxymbox{\hyperlink{struct_beam_1_1_render_plan_1_1_buffer_clear_op_aee4cd65a8532f653c984b8c5c711bb92}{port\+Idx}}
\end{DoxyCompactItemize}


\label{doc-variable-members}
\Hypertarget{struct_beam_1_1_render_plan_1_1_buffer_clear_op_doc-variable-members}
\doxysubsection{Member Data Documentation}
\Hypertarget{struct_beam_1_1_render_plan_1_1_buffer_clear_op_a86757a99d31ed231e1011800922b49f6}\index{Beam::RenderPlan::BufferClearOp@{Beam::RenderPlan::BufferClearOp}!node@{node}}
\index{node@{node}!Beam::RenderPlan::BufferClearOp@{Beam::RenderPlan::BufferClearOp}}
\doxysubsubsection{\texorpdfstring{node}{node}}
{\footnotesize\ttfamily \label{struct_beam_1_1_render_plan_1_1_buffer_clear_op_a86757a99d31ed231e1011800922b49f6} 
std::\+shared\+\_\+ptr$<$\doxymbox{\hyperlink{class_beam_1_1_flux_node}{Flux\+Node}}$>$ Beam::\+\+Render\+Plan::\+\+Buffer\+Clear\+Op::\+node}

\Hypertarget{struct_beam_1_1_render_plan_1_1_buffer_clear_op_aee4cd65a8532f653c984b8c5c711bb92}\index{Beam::RenderPlan::BufferClearOp@{Beam::RenderPlan::BufferClearOp}!portIdx@{portIdx}}
\index{portIdx@{portIdx}!Beam::RenderPlan::BufferClearOp@{Beam::RenderPlan::BufferClearOp}}
\doxysubsubsection{\texorpdfstring{portIdx}{portIdx}}
{\footnotesize\ttfamily \label{struct_beam_1_1_render_plan_1_1_buffer_clear_op_aee4cd65a8532f653c984b8c5c711bb92} 
int Beam::\+\+Render\+Plan::\+\+Buffer\+Clear\+Op::\+port\+Idx}



The documentation for this struct was generated from the following file:\+\begin{DoxyCompactItemize}
\item 
src/\+engine/\+\doxymbox{\hyperlink{render__plan_8hpp}{render\+\_\+plan.\+hpp}}\end{DoxyCompactItemize}

\doxysection{Beam::\+Button Class Reference}
\hypertarget{class_beam_1_1_button}{}\label{class_beam_1_1_button}\index{Beam::Button@{Beam::Button}}


A button component, similar to JUCE\textquotesingle{}s \doxylink{class_beam_1_1_button}{Button}.  




{\ttfamily \+\#include $<$button.\+hpp$>$}

Inheritance diagram for Beam::\+Button:\+\begin{figure}[H]
\begin{center}
\leavevmode
\includegraphics[height=2.000000cm]{class_beam_1_1_button}
\end{center}
\end{figure}
\doxysubsubsection*{Public Member Functions}
\begin{DoxyCompactItemize}
\item 
\doxymbox{\hyperlink{class_beam_1_1_button_aa7013762bb47b10d9748aa3743536c69}{Button}} ()
\item 
\doxymbox{\hyperlink{class_beam_1_1_button_a3b86c2730e80c497b63aa89afd84caa8}{Button}} (const std::\+string \&button\+Text)
\item 
\doxymbox{\hyperlink{class_beam_1_1_button_a2a82ff4be16b8663c3fc3764e7d89e60}{\texorpdfstring{$\sim$}{\string~}\+Button}} () override
\item 
void \doxymbox{\hyperlink{class_beam_1_1_button_a6888e552294af9fd29776d60a39fd2b8}{set\+Button\+Text}} (const std::\+string \&new\+Text)
\begin{DoxyCompactList}\small\item\em Sets the button text. \end{DoxyCompactList}\item 
const std::\+string \& \doxymbox{\hyperlink{class_beam_1_1_button_af0a4f0dfc387b238855bf85618a87878}{get\+Button\+Text}} () const
\begin{DoxyCompactList}\small\item\em Gets the button text. \end{DoxyCompactList}\item 
void \doxymbox{\hyperlink{class_beam_1_1_button_a1f0a56958e4b38a933b37efe1c07da2c}{set\+Enabled}} (bool should\+Be\+Enabled)
\begin{DoxyCompactList}\small\item\em Sets whether the button is clickable. \end{DoxyCompactList}\item 
bool \doxymbox{\hyperlink{class_beam_1_1_button_a2c5dc63117eaf80e7eb7b2cf1ee4d33c}{is\+Enabled}} () const
\begin{DoxyCompactList}\small\item\em Checks if the button is enabled. \end{DoxyCompactList}\item 
void \doxymbox{\hyperlink{class_beam_1_1_button_a0c1a6bc52f4d5e3112b6e7629fd836a7}{on\+Click}} (std::\+function$<$ void()$>$ callback)
\begin{DoxyCompactList}\small\item\em Sets the click listener. \end{DoxyCompactList}\item 
void \doxymbox{\hyperlink{class_beam_1_1_button_a5a5ec975b5bac74a85126cfe41563a26}{paint}} (\doxymbox{\hyperlink{class_beam_1_1_quad_batcher}{Quad\+Batcher}} \&g) override
\begin{DoxyCompactList}\small\item\em Paints the button. \end{DoxyCompactList}\item 
void \doxymbox{\hyperlink{class_beam_1_1_button_a82487a148cdbd4f66bd401320b5032a2}{mouse\+Down}} (const \doxymbox{\hyperlink{class_beam_1_1_mouse_event}{Mouse\+Event}} \&event) override
\begin{DoxyCompactList}\small\item\em Called when the mouse is pressed. \end{DoxyCompactList}\item 
void \doxymbox{\hyperlink{class_beam_1_1_button_a19bc963ab2ef912af08bcf75701b25b8}{mouse\+Up}} (const \doxymbox{\hyperlink{class_beam_1_1_mouse_event}{Mouse\+Event}} \&event) override
\begin{DoxyCompactList}\small\item\em Called when the mouse is released. \end{DoxyCompactList}\item 
void \doxymbox{\hyperlink{class_beam_1_1_button_acd3724e0cdc77bcf9c25d1958dd9c5c3}{mouse\+Enter}} (const \doxymbox{\hyperlink{class_beam_1_1_mouse_event}{Mouse\+Event}} \&event) override
\begin{DoxyCompactList}\small\item\em Called when the mouse enters the component. \end{DoxyCompactList}\item 
void \doxymbox{\hyperlink{class_beam_1_1_button_a8199f40b2ff5391df56fbe339649c4cb}{mouse\+Exit}} (const \doxymbox{\hyperlink{class_beam_1_1_mouse_event}{Mouse\+Event}} \&event) override
\begin{DoxyCompactList}\small\item\em Called when the mouse exits the component. \end{DoxyCompactList}\end{DoxyCompactItemize}
\doxysubsection*{Public Member Functions inherited from \doxymbox{\hyperlink{class_beam_1_1_gui_component}{Beam::\+\+Gui\+Component}}}
\begin{DoxyCompactItemize}
\item 
\doxymbox{\hyperlink{class_beam_1_1_gui_component_a3a2d448a1f30616384f16a41102c613a}{Gui\+Component}} ()
\item 
virtual \doxymbox{\hyperlink{class_beam_1_1_gui_component_a255a25fc9e1a2abdfdfcb7c11da1e66e}{\texorpdfstring{$\sim$}{\string~}\+Gui\+Component}} ()
\item 
virtual void \doxymbox{\hyperlink{class_beam_1_1_gui_component_a6ecaca5ca83fc45c232093a1012191d2}{set\+Bounds}} (float x, float y, float \doxymbox{\hyperlink{texture_8cpp_a8710f3c5c66c09e158c8619b3fca614a}{width}}, float \doxymbox{\hyperlink{texture_8cpp_a1055637f17e35a0ca82b396bb94914e5}{height}})
\begin{DoxyCompactList}\small\item\em Sets the bounds of this component. \end{DoxyCompactList}\item 
const \doxymbox{\hyperlink{struct_beam_1_1_rect}{Rect}} \& \doxymbox{\hyperlink{class_beam_1_1_gui_component_a2de58576d7e55fd82acbdb2f18207982}{get\+Bounds}} () const
\begin{DoxyCompactList}\small\item\em Gets the bounds of this component. \end{DoxyCompactList}\item 
void \doxymbox{\hyperlink{class_beam_1_1_gui_component_abc37535b6bef8b996c41849df8c95e96}{set\+Visible}} (bool should\+Be\+Visible)
\begin{DoxyCompactList}\small\item\em Sets the component\textquotesingle{}s visibility. \end{DoxyCompactList}\item 
bool \doxymbox{\hyperlink{class_beam_1_1_gui_component_ae4752b8375293063779e88a83c1a799a}{is\+Visible}} () const
\begin{DoxyCompactList}\small\item\em Checks if the component is visible. \end{DoxyCompactList}\item 
virtual void \doxymbox{\hyperlink{class_beam_1_1_gui_component_a683989837c3ab83ffb1148e0b473e573}{resized}} ()
\begin{DoxyCompactList}\small\item\em Called when the component\textquotesingle{}s size changes. \end{DoxyCompactList}\item 
virtual void \doxymbox{\hyperlink{class_beam_1_1_gui_component_abb1547968352dd7ed45ec761e4ceac07}{mouse\+Move}} (const \doxymbox{\hyperlink{class_beam_1_1_mouse_event}{Mouse\+Event}} \&event)
\begin{DoxyCompactList}\small\item\em Called when the mouse is moved. \end{DoxyCompactList}\item 
virtual void \doxymbox{\hyperlink{class_beam_1_1_gui_component_a5f9e28ec7aea30422982fcb748bd54c2}{mouse\+Drag}} (const \doxymbox{\hyperlink{class_beam_1_1_mouse_event}{Mouse\+Event}} \&event)
\begin{DoxyCompactList}\small\item\em Called when the mouse is dragged. \end{DoxyCompactList}\item 
virtual bool \doxymbox{\hyperlink{class_beam_1_1_gui_component_a2459c6228fcbc914614e6252a43b016f}{key\+Pressed}} (const \doxymbox{\hyperlink{class_beam_1_1_key_press}{Key\+Press}} \&key)
\begin{DoxyCompactList}\small\item\em Called when a key is pressed. \end{DoxyCompactList}\item 
void \doxymbox{\hyperlink{class_beam_1_1_gui_component_af85677b7e220a2b1bf69659d4167bce5}{add\+Child\+Component}} (std::\+shared\+\_\+ptr$<$ \doxymbox{\hyperlink{class_beam_1_1_gui_component_a3a2d448a1f30616384f16a41102c613a}{Gui\+Component}} $>$ child)
\begin{DoxyCompactList}\small\item\em Adds a child component. \end{DoxyCompactList}\item 
void \doxymbox{\hyperlink{class_beam_1_1_gui_component_a2b68395b5a1ebdfe673053b2aa83122b}{remove\+Child\+Component}} (std::\+shared\+\_\+ptr$<$ \doxymbox{\hyperlink{class_beam_1_1_gui_component_a3a2d448a1f30616384f16a41102c613a}{Gui\+Component}} $>$ child)
\begin{DoxyCompactList}\small\item\em Removes a child component. \end{DoxyCompactList}\item 
void \doxymbox{\hyperlink{class_beam_1_1_gui_component_a7dbd4e5e0b8955748fe0962bcb476089}{paint\+Entire\+Component}} (\doxymbox{\hyperlink{class_beam_1_1_quad_batcher}{Quad\+Batcher}} \&g)
\begin{DoxyCompactList}\small\item\em Paints this component and all its children. \end{DoxyCompactList}\item 
bool \doxymbox{\hyperlink{class_beam_1_1_gui_component_abad300b88267731a68afd9a94e5225c4}{contains}} (float x, float y) const
\begin{DoxyCompactList}\small\item\em Checks if a point is inside this component. \end{DoxyCompactList}\item 
void \doxymbox{\hyperlink{class_beam_1_1_gui_component_ae7e5d9dde8c3bae347fdaaac29b5d96a}{set\+Name}} (const std::\+string \&name)
\begin{DoxyCompactList}\small\item\em Sets the component\textquotesingle{}s name. \end{DoxyCompactList}\item 
const std::\+string \& \doxymbox{\hyperlink{class_beam_1_1_gui_component_ac21fae6abb5616da7b0882c924961ae9}{get\+Name}} () const
\begin{DoxyCompactList}\small\item\em Gets the component\textquotesingle{}s name. \end{DoxyCompactList}\end{DoxyCompactItemize}
\doxysubsubsection*{Private Attributes}
\begin{DoxyCompactItemize}
\item 
std::\+string \doxymbox{\hyperlink{class_beam_1_1_button_aecf93c6e7336e54b56eaf8c4e6f5a44b}{m\+\_\+text}}
\item 
bool \doxymbox{\hyperlink{class_beam_1_1_button_af48b8160b17dd70d521ad1b7493bd02c}{m\+\_\+enabled}} = true
\item 
bool \doxymbox{\hyperlink{class_beam_1_1_button_a364565a2e4ac0c02808d6ead456941dd}{m\+\_\+is\+Over}} = false
\item 
bool \doxymbox{\hyperlink{class_beam_1_1_button_aa0f3556827981e4495ba692ee3f5b17c}{m\+\_\+is\+Down}} = false
\item 
std::\+function$<$ void()$>$ \doxymbox{\hyperlink{class_beam_1_1_button_afd1e47a77922d765b1387bdf075eaa0e}{m\+\_\+click\+Callback}}
\end{DoxyCompactItemize}
\doxysubsubsection*{Additional Inherited Members}
\doxysubsection*{Protected Attributes inherited from \doxymbox{\hyperlink{class_beam_1_1_gui_component}{Beam::\+\+Gui\+Component}}}
\begin{DoxyCompactItemize}
\item 
\doxymbox{\hyperlink{struct_beam_1_1_rect}{Rect}} \doxymbox{\hyperlink{class_beam_1_1_gui_component_ad3c42f55d6a7e47f65bd1d0f3ffb5291}{m\+\_\+bounds}} \{0, 0, 0, 0\}
\item 
bool \doxymbox{\hyperlink{class_beam_1_1_gui_component_afbfe066e00bfffc064d80082c839ebe6}{m\+\_\+visible}} = true
\item 
std::\+vector$<$ std::\+shared\+\_\+ptr$<$ \doxymbox{\hyperlink{class_beam_1_1_gui_component_a3a2d448a1f30616384f16a41102c613a}{Gui\+Component}} $>$ $>$ \doxymbox{\hyperlink{class_beam_1_1_gui_component_ada191fb579394c85ec4fa977b1b108b0}{m\+\_\+children}}
\item 
std::\+string \doxymbox{\hyperlink{class_beam_1_1_gui_component_a5e462cb9a299d364cb53b1a768369246}{m\+\_\+name}}
\item 
std::\+function$<$ void()$>$ \doxymbox{\hyperlink{class_beam_1_1_gui_component_a46e1692a3cd91b4464cce1680c759a1e}{m\+\_\+paint\+Callback}}
\item 
std::\+function$<$ void()$>$ \doxymbox{\hyperlink{class_beam_1_1_gui_component_a5ea91ef234ce12ce8dc7fcc143178072}{m\+\_\+resized\+Callback}}
\end{DoxyCompactItemize}


\doxysubsection{Detailed Description}
A button component, similar to JUCE\textquotesingle{}s \doxylink{class_beam_1_1_button}{Button}. 

\label{doc-constructors}
\Hypertarget{class_beam_1_1_button_doc-constructors}
\doxysubsection{Constructor \& Destructor Documentation}
\Hypertarget{class_beam_1_1_button_aa7013762bb47b10d9748aa3743536c69}\index{Beam::Button@{Beam::Button}!Button@{Button}}
\index{Button@{Button}!Beam::Button@{Beam::Button}}
\doxysubsubsection{\texorpdfstring{Button()}{Button()}\hspace{0.1cm}{\footnotesize\ttfamily [1/2]}}
{\footnotesize\ttfamily \label{class_beam_1_1_button_aa7013762bb47b10d9748aa3743536c69} 
Beam::\+\+Button::\+\+Button (\begin{DoxyParamCaption}{}{}\end{DoxyParamCaption})}

\Hypertarget{class_beam_1_1_button_a3b86c2730e80c497b63aa89afd84caa8}\index{Beam::Button@{Beam::Button}!Button@{Button}}
\index{Button@{Button}!Beam::Button@{Beam::Button}}
\doxysubsubsection{\texorpdfstring{Button()}{Button()}\hspace{0.1cm}{\footnotesize\ttfamily [2/2]}}
{\footnotesize\ttfamily \label{class_beam_1_1_button_a3b86c2730e80c497b63aa89afd84caa8} 
Beam::\+\+Button::\+\+Button (\begin{DoxyParamCaption}\item[{const std::\+string \&}]{button\+Text}{}\end{DoxyParamCaption})\hspace{0.3cm}{\ttfamily [explicit]}}

\Hypertarget{class_beam_1_1_button_a2a82ff4be16b8663c3fc3764e7d89e60}\index{Beam::Button@{Beam::Button}!````~Button@{\texorpdfstring{$\sim$}{\string~}Button}}
\index{````~Button@{\texorpdfstring{$\sim$}{\string~}Button}!Beam::Button@{Beam::Button}}
\doxysubsubsection{\texorpdfstring{\texorpdfstring{$\sim$}{\string~}Button()}{\string~Button()}}
{\footnotesize\ttfamily \label{class_beam_1_1_button_a2a82ff4be16b8663c3fc3764e7d89e60} 
Beam::\+\+Button::\+\texorpdfstring{$\sim$}{\string~}\+Button (\begin{DoxyParamCaption}{}{}\end{DoxyParamCaption})\hspace{0.3cm}{\ttfamily [override]}}



\label{doc-func-members}
\Hypertarget{class_beam_1_1_button_doc-func-members}
\doxysubsection{Member Function Documentation}
\Hypertarget{class_beam_1_1_button_af0a4f0dfc387b238855bf85618a87878}\index{Beam::Button@{Beam::Button}!getButtonText@{getButtonText}}
\index{getButtonText@{getButtonText}!Beam::Button@{Beam::Button}}
\doxysubsubsection{\texorpdfstring{getButtonText()}{getButtonText()}}
{\footnotesize\ttfamily \label{class_beam_1_1_button_af0a4f0dfc387b238855bf85618a87878} 
const std::\+string \& Beam::\+\+Button::\+get\+Button\+Text (\begin{DoxyParamCaption}{}{}\end{DoxyParamCaption}) const\hspace{0.3cm}{\ttfamily [inline]}}



Gets the button text. 

\Hypertarget{class_beam_1_1_button_a2c5dc63117eaf80e7eb7b2cf1ee4d33c}\index{Beam::Button@{Beam::Button}!isEnabled@{isEnabled}}
\index{isEnabled@{isEnabled}!Beam::Button@{Beam::Button}}
\doxysubsubsection{\texorpdfstring{isEnabled()}{isEnabled()}}
{\footnotesize\ttfamily \label{class_beam_1_1_button_a2c5dc63117eaf80e7eb7b2cf1ee4d33c} 
bool Beam::\+\+Button::\+is\+Enabled (\begin{DoxyParamCaption}{}{}\end{DoxyParamCaption}) const\hspace{0.3cm}{\ttfamily [inline]}}



Checks if the button is enabled. 

\Hypertarget{class_beam_1_1_button_a82487a148cdbd4f66bd401320b5032a2}\index{Beam::Button@{Beam::Button}!mouseDown@{mouseDown}}
\index{mouseDown@{mouseDown}!Beam::Button@{Beam::Button}}
\doxysubsubsection{\texorpdfstring{mouseDown()}{mouseDown()}}
{\footnotesize\ttfamily \label{class_beam_1_1_button_a82487a148cdbd4f66bd401320b5032a2} 
void Beam::\+\+Button::\+mouse\+Down (\begin{DoxyParamCaption}\item[{const \doxymbox{\hyperlink{class_beam_1_1_mouse_event}{Mouse\+Event}} \&}]{event}{}\end{DoxyParamCaption})\hspace{0.3cm}{\ttfamily [override]}, {\ttfamily [virtual]}}



Called when the mouse is pressed. 



Reimplemented from \doxymbox{\hyperlink{class_beam_1_1_gui_component_a3f1bac930389048f3ab5217ece24e032}{Beam::\+\+Gui\+Component}}.

\Hypertarget{class_beam_1_1_button_acd3724e0cdc77bcf9c25d1958dd9c5c3}\index{Beam::Button@{Beam::Button}!mouseEnter@{mouseEnter}}
\index{mouseEnter@{mouseEnter}!Beam::Button@{Beam::Button}}
\doxysubsubsection{\texorpdfstring{mouseEnter()}{mouseEnter()}}
{\footnotesize\ttfamily \label{class_beam_1_1_button_acd3724e0cdc77bcf9c25d1958dd9c5c3} 
void Beam::\+\+Button::\+mouse\+Enter (\begin{DoxyParamCaption}\item[{const \doxymbox{\hyperlink{class_beam_1_1_mouse_event}{Mouse\+Event}} \&}]{event}{}\end{DoxyParamCaption})\hspace{0.3cm}{\ttfamily [override]}, {\ttfamily [virtual]}}



Called when the mouse enters the component. 



Reimplemented from \doxymbox{\hyperlink{class_beam_1_1_gui_component_a357d8829f546299e6fda6ff119c582c1}{Beam::\+\+Gui\+Component}}.

\Hypertarget{class_beam_1_1_button_a8199f40b2ff5391df56fbe339649c4cb}\index{Beam::Button@{Beam::Button}!mouseExit@{mouseExit}}
\index{mouseExit@{mouseExit}!Beam::Button@{Beam::Button}}
\doxysubsubsection{\texorpdfstring{mouseExit()}{mouseExit()}}
{\footnotesize\ttfamily \label{class_beam_1_1_button_a8199f40b2ff5391df56fbe339649c4cb} 
void Beam::\+\+Button::\+mouse\+Exit (\begin{DoxyParamCaption}\item[{const \doxymbox{\hyperlink{class_beam_1_1_mouse_event}{Mouse\+Event}} \&}]{event}{}\end{DoxyParamCaption})\hspace{0.3cm}{\ttfamily [override]}, {\ttfamily [virtual]}}



Called when the mouse exits the component. 



Reimplemented from \doxymbox{\hyperlink{class_beam_1_1_gui_component_a9c580600fa3396a3085b2c7ca9bd6869}{Beam::\+\+Gui\+Component}}.

\Hypertarget{class_beam_1_1_button_a19bc963ab2ef912af08bcf75701b25b8}\index{Beam::Button@{Beam::Button}!mouseUp@{mouseUp}}
\index{mouseUp@{mouseUp}!Beam::Button@{Beam::Button}}
\doxysubsubsection{\texorpdfstring{mouseUp()}{mouseUp()}}
{\footnotesize\ttfamily \label{class_beam_1_1_button_a19bc963ab2ef912af08bcf75701b25b8} 
void Beam::\+\+Button::\+mouse\+Up (\begin{DoxyParamCaption}\item[{const \doxymbox{\hyperlink{class_beam_1_1_mouse_event}{Mouse\+Event}} \&}]{event}{}\end{DoxyParamCaption})\hspace{0.3cm}{\ttfamily [override]}, {\ttfamily [virtual]}}



Called when the mouse is released. 



Reimplemented from \doxymbox{\hyperlink{class_beam_1_1_gui_component_aeaa0f5b76f80ee669de3a9bf06158a04}{Beam::\+\+Gui\+Component}}.

\Hypertarget{class_beam_1_1_button_a0c1a6bc52f4d5e3112b6e7629fd836a7}\index{Beam::Button@{Beam::Button}!onClick@{onClick}}
\index{onClick@{onClick}!Beam::Button@{Beam::Button}}
\doxysubsubsection{\texorpdfstring{onClick()}{onClick()}}
{\footnotesize\ttfamily \label{class_beam_1_1_button_a0c1a6bc52f4d5e3112b6e7629fd836a7} 
void Beam::\+\+Button::\+on\+Click (\begin{DoxyParamCaption}\item[{std::\+function$<$ void()$>$}]{callback}{}\end{DoxyParamCaption})}



Sets the click listener. 

\Hypertarget{class_beam_1_1_button_a5a5ec975b5bac74a85126cfe41563a26}\index{Beam::Button@{Beam::Button}!paint@{paint}}
\index{paint@{paint}!Beam::Button@{Beam::Button}}
\doxysubsubsection{\texorpdfstring{paint()}{paint()}}
{\footnotesize\ttfamily \label{class_beam_1_1_button_a5a5ec975b5bac74a85126cfe41563a26} 
void Beam::\+\+Button::\+paint (\begin{DoxyParamCaption}\item[{\doxymbox{\hyperlink{class_beam_1_1_quad_batcher}{Quad\+Batcher}} \&}]{g}{}\end{DoxyParamCaption})\hspace{0.3cm}{\ttfamily [override]}, {\ttfamily [virtual]}}



Paints the button. 



Reimplemented from \doxymbox{\hyperlink{class_beam_1_1_gui_component_afd0b01a0cf776f3e5746622e7a4e7c5c}{Beam::\+\+Gui\+Component}}.

\Hypertarget{class_beam_1_1_button_a6888e552294af9fd29776d60a39fd2b8}\index{Beam::Button@{Beam::Button}!setButtonText@{setButtonText}}
\index{setButtonText@{setButtonText}!Beam::Button@{Beam::Button}}
\doxysubsubsection{\texorpdfstring{setButtonText()}{setButtonText()}}
{\footnotesize\ttfamily \label{class_beam_1_1_button_a6888e552294af9fd29776d60a39fd2b8} 
void Beam::\+\+Button::\+set\+Button\+Text (\begin{DoxyParamCaption}\item[{const std::\+string \&}]{new\+Text}{}\end{DoxyParamCaption})}



Sets the button text. 

\Hypertarget{class_beam_1_1_button_a1f0a56958e4b38a933b37efe1c07da2c}\index{Beam::Button@{Beam::Button}!setEnabled@{setEnabled}}
\index{setEnabled@{setEnabled}!Beam::Button@{Beam::Button}}
\doxysubsubsection{\texorpdfstring{setEnabled()}{setEnabled()}}
{\footnotesize\ttfamily \label{class_beam_1_1_button_a1f0a56958e4b38a933b37efe1c07da2c} 
void Beam::\+\+Button::\+set\+Enabled (\begin{DoxyParamCaption}\item[{bool}]{should\+Be\+Enabled}{}\end{DoxyParamCaption})}



Sets whether the button is clickable. 



\label{doc-variable-members}
\Hypertarget{class_beam_1_1_button_doc-variable-members}
\doxysubsection{Member Data Documentation}
\Hypertarget{class_beam_1_1_button_afd1e47a77922d765b1387bdf075eaa0e}\index{Beam::Button@{Beam::Button}!m\_clickCallback@{m\_clickCallback}}
\index{m\_clickCallback@{m\_clickCallback}!Beam::Button@{Beam::Button}}
\doxysubsubsection{\texorpdfstring{m\_clickCallback}{m\_clickCallback}}
{\footnotesize\ttfamily \label{class_beam_1_1_button_afd1e47a77922d765b1387bdf075eaa0e} 
std::\+function$<$void()$>$ Beam::\+\+Button::\+m\+\_\+click\+Callback\hspace{0.3cm}{\ttfamily [private]}}

\Hypertarget{class_beam_1_1_button_af48b8160b17dd70d521ad1b7493bd02c}\index{Beam::Button@{Beam::Button}!m\_enabled@{m\_enabled}}
\index{m\_enabled@{m\_enabled}!Beam::Button@{Beam::Button}}
\doxysubsubsection{\texorpdfstring{m\_enabled}{m\_enabled}}
{\footnotesize\ttfamily \label{class_beam_1_1_button_af48b8160b17dd70d521ad1b7493bd02c} 
bool Beam::\+\+Button::\+m\+\_\+enabled = true\hspace{0.3cm}{\ttfamily [private]}}

\Hypertarget{class_beam_1_1_button_aa0f3556827981e4495ba692ee3f5b17c}\index{Beam::Button@{Beam::Button}!m\_isDown@{m\_isDown}}
\index{m\_isDown@{m\_isDown}!Beam::Button@{Beam::Button}}
\doxysubsubsection{\texorpdfstring{m\_isDown}{m\_isDown}}
{\footnotesize\ttfamily \label{class_beam_1_1_button_aa0f3556827981e4495ba692ee3f5b17c} 
bool Beam::\+\+Button::\+m\+\_\+is\+Down = false\hspace{0.3cm}{\ttfamily [private]}}

\Hypertarget{class_beam_1_1_button_a364565a2e4ac0c02808d6ead456941dd}\index{Beam::Button@{Beam::Button}!m\_isOver@{m\_isOver}}
\index{m\_isOver@{m\_isOver}!Beam::Button@{Beam::Button}}
\doxysubsubsection{\texorpdfstring{m\_isOver}{m\_isOver}}
{\footnotesize\ttfamily \label{class_beam_1_1_button_a364565a2e4ac0c02808d6ead456941dd} 
bool Beam::\+\+Button::\+m\+\_\+is\+Over = false\hspace{0.3cm}{\ttfamily [private]}}

\Hypertarget{class_beam_1_1_button_aecf93c6e7336e54b56eaf8c4e6f5a44b}\index{Beam::Button@{Beam::Button}!m\_text@{m\_text}}
\index{m\_text@{m\_text}!Beam::Button@{Beam::Button}}
\doxysubsubsection{\texorpdfstring{m\_text}{m\_text}}
{\footnotesize\ttfamily \label{class_beam_1_1_button_aecf93c6e7336e54b56eaf8c4e6f5a44b} 
std::\+string Beam::\+\+Button::\+m\+\_\+text\hspace{0.3cm}{\ttfamily [private]}}



The documentation for this class was generated from the following files:\+\begin{DoxyCompactItemize}
\item 
src/\+interface/\+\doxymbox{\hyperlink{button_8hpp}{button.\+hpp}}\item 
src/\+interface/\+\doxymbox{\hyperlink{button_8cpp}{button.\+cpp}}\end{DoxyCompactItemize}

\doxysection{Beam::\+Cable Struct Reference}
\hypertarget{struct_beam_1_1_cable}{}\label{struct_beam_1_1_cable}\index{Beam::Cable@{Beam::Cable}}


{\ttfamily \+\#include $<$cable.\+hpp$>$}

\doxysubsubsection*{Public Member Functions}
\begin{DoxyCompactItemize}
\item 
void \doxymbox{\hyperlink{struct_beam_1_1_cable_a84adeb1426f567be118fbcfec2cf73f5}{render}} (\doxymbox{\hyperlink{class_beam_1_1_quad_batcher}{Quad\+Batcher}} \&batcher)
\end{DoxyCompactItemize}
\doxysubsubsection*{Public Attributes}
\begin{DoxyCompactItemize}
\item 
\doxymbox{\hyperlink{class_beam_1_1_port}{Port}} \texorpdfstring{$\ast$}{*} \doxymbox{\hyperlink{struct_beam_1_1_cable_a053fa9bcaca869bbe55a6e969a5d191f}{output}}
\item 
\doxymbox{\hyperlink{class_beam_1_1_port}{Port}} \texorpdfstring{$\ast$}{*} \doxymbox{\hyperlink{struct_beam_1_1_cable_a1d97b8c95a75addc1ec772f6328fdc43}{input}}
\end{DoxyCompactItemize}


\label{doc-func-members}
\Hypertarget{struct_beam_1_1_cable_doc-func-members}
\doxysubsection{Member Function Documentation}
\Hypertarget{struct_beam_1_1_cable_a84adeb1426f567be118fbcfec2cf73f5}\index{Beam::Cable@{Beam::Cable}!render@{render}}
\index{render@{render}!Beam::Cable@{Beam::Cable}}
\doxysubsubsection{\texorpdfstring{render()}{render()}}
{\footnotesize\ttfamily \label{struct_beam_1_1_cable_a84adeb1426f567be118fbcfec2cf73f5} 
void Beam::\+\+Cable::\+render (\begin{DoxyParamCaption}\item[{\doxymbox{\hyperlink{class_beam_1_1_quad_batcher}{Quad\+Batcher}} \&}]{batcher}{}\end{DoxyParamCaption})\hspace{0.3cm}{\ttfamily [inline]}}



\label{doc-variable-members}
\Hypertarget{struct_beam_1_1_cable_doc-variable-members}
\doxysubsection{Member Data Documentation}
\Hypertarget{struct_beam_1_1_cable_a1d97b8c95a75addc1ec772f6328fdc43}\index{Beam::Cable@{Beam::Cable}!input@{input}}
\index{input@{input}!Beam::Cable@{Beam::Cable}}
\doxysubsubsection{\texorpdfstring{input}{input}}
{\footnotesize\ttfamily \label{struct_beam_1_1_cable_a1d97b8c95a75addc1ec772f6328fdc43} 
\doxymbox{\hyperlink{class_beam_1_1_port}{Port}}\texorpdfstring{$\ast$}{*} Beam::\+\+Cable::\+input}

\Hypertarget{struct_beam_1_1_cable_a053fa9bcaca869bbe55a6e969a5d191f}\index{Beam::Cable@{Beam::Cable}!output@{output}}
\index{output@{output}!Beam::Cable@{Beam::Cable}}
\doxysubsubsection{\texorpdfstring{output}{output}}
{\footnotesize\ttfamily \label{struct_beam_1_1_cable_a053fa9bcaca869bbe55a6e969a5d191f} 
\doxymbox{\hyperlink{class_beam_1_1_port}{Port}}\texorpdfstring{$\ast$}{*} Beam::\+\+Cable::\+output}



The documentation for this struct was generated from the following file:\+\begin{DoxyCompactItemize}
\item 
src/\+ui/\+\doxymbox{\hyperlink{cable_8hpp}{cable.\+hpp}}\end{DoxyCompactItemize}

\doxysection{Beam::\+Console\+E\+\_\+\+EQ Class Reference}
\hypertarget{class_beam_1_1_console_e___e_q}{}\label{class_beam_1_1_console_e___e_q}\index{Beam::ConsoleE\_EQ@{Beam::ConsoleE\_EQ}}


{\ttfamily \+\#include $<$analog\+\_\+suite.\+hpp$>$}

Inheritance diagram for Beam::\+Console\+E\+\_\+\+EQ:\+\begin{figure}[H]
\begin{center}
\leavevmode
\includegraphics[height=3.000000cm]{class_beam_1_1_console_e___e_q}
\end{center}
\end{figure}
\doxysubsubsection*{Public Member Functions}
\begin{DoxyCompactItemize}
\item 
\doxymbox{\hyperlink{class_beam_1_1_console_e___e_q_a5d5eba473f0987818112706308ecec41}{Console\+E\+\_\+\+EQ}} (int buf, float sr)
\item 
void \doxymbox{\hyperlink{class_beam_1_1_console_e___e_q_a957a65f7ecdbe08cd5f6ae9c5c8796da}{process\+Block}} (const float \texorpdfstring{$\ast$}{*}in, float \texorpdfstring{$\ast$}{*}out, int total) override
\end{DoxyCompactItemize}
\doxysubsection*{Public Member Functions inherited from \doxymbox{\hyperlink{class_beam_1_1_flux_plugin}{Beam::\+\+Flux\+Plugin}}}
\begin{DoxyCompactItemize}
\item 
\doxymbox{\hyperlink{class_beam_1_1_flux_plugin_a9d91b56960799fdbfc4575e2fcfa6689}{Flux\+Plugin}} (const std::\+string \&name, int buffer\+Size, float sample\+Rate)
\item 
virtual void \doxymbox{\hyperlink{class_beam_1_1_flux_plugin_a87fb076475f20b062493efa7ca00e045}{process\+Events}} (const \doxymbox{\hyperlink{class_beam_1_1_m_i_d_i_buffer}{MIDIBuffer}} \&midi)
\begin{DoxyCompactList}\small\item\em Handle MIDI events in your plugin. \end{DoxyCompactList}\item 
void \doxymbox{\hyperlink{class_beam_1_1_flux_plugin_aa1f9c569002ec23eeb5db0af686abea7}{process\+MIDI}} (const \doxymbox{\hyperlink{class_beam_1_1_m_i_d_i_buffer}{MIDIBuffer}} \&midi) override
\begin{DoxyCompactList}\small\item\em Optional MIDI processing. Called before \doxylink{class_beam_1_1_flux_plugin_a181430e1cbf129891fe3ed72f3905a61}{process()} in the engine loop. \end{DoxyCompactList}\item 
void \doxymbox{\hyperlink{class_beam_1_1_flux_plugin_a181430e1cbf129891fe3ed72f3905a61}{process}} (int frames) override
\begin{DoxyCompactList}\small\item\em Main audio processing method. Must be implemented by subclasses. \end{DoxyCompactList}\item 
std::\+string \doxymbox{\hyperlink{class_beam_1_1_flux_plugin_a450563af4d65a25b8a8e896dab77a3c6}{get\+Name}} () const override
\item 
std::\+vector$<$ \doxymbox{\hyperlink{struct_beam_1_1_flux_node_1_1_port}{Port}} $>$ \doxymbox{\hyperlink{class_beam_1_1_flux_plugin_ad231db67f900e8e7dd853936ad2e866a}{get\+Input\+Ports}} () const override
\item 
std::\+vector$<$ \doxymbox{\hyperlink{struct_beam_1_1_flux_node_1_1_port}{Port}} $>$ \doxymbox{\hyperlink{class_beam_1_1_flux_plugin_a4ea312de74047e127e818a26e715d8bb}{get\+Output\+Ports}} () const override
\end{DoxyCompactItemize}
\doxysubsection*{Public Member Functions inherited from \doxymbox{\hyperlink{class_beam_1_1_flux_node}{Beam::\+\+Flux\+Node}}}
\begin{DoxyCompactItemize}
\item 
virtual \doxymbox{\hyperlink{class_beam_1_1_flux_node_a708c135cdb61e8838469998cd8a84e65}{\texorpdfstring{$\sim$}{\string~}\+Flux\+Node}} ()=default
\item 
virtual void \doxymbox{\hyperlink{class_beam_1_1_flux_node_ace8cc49479d8924d44bca5fd4cd955e2}{on\+Transport\+State\+Changed}} (bool playing)
\begin{DoxyCompactList}\small\item\em Responds to global transport changes (Play/\+\+Pause). \end{DoxyCompactList}\item 
virtual void \doxymbox{\hyperlink{class_beam_1_1_flux_node_adc7c4e979bf27de5bfca66815ae97a67}{on\+Transport\+Seek}} (size\+\_\+t frame)
\begin{DoxyCompactList}\small\item\em Responds to timeline seeking. \end{DoxyCompactList}\item 
void \doxymbox{\hyperlink{class_beam_1_1_flux_node_aa579ec06608fd776987bbb089f27fd94}{set\+Current\+Frame}} (size\+\_\+t frame)
\begin{DoxyCompactList}\small\item\em Sets the current playhead position for this block. \end{DoxyCompactList}\item 
float \texorpdfstring{$\ast$}{*} \doxymbox{\hyperlink{class_beam_1_1_flux_node_ac90bd1a05b5bed3d68978f532386ed29}{get\+Input\+Buffer}} (int port\+Idx)
\item 
float \texorpdfstring{$\ast$}{*} \doxymbox{\hyperlink{class_beam_1_1_flux_node_abf11cfd4f2346ee0cd46d4345f1ed7d4}{get\+Output\+Buffer}} (int port\+Idx)
\item 
void \doxymbox{\hyperlink{class_beam_1_1_flux_node_af37f8c1b6b825da2ce7e35011d6f8253}{set\+Bypass}} (bool bypass)
\item 
bool \doxymbox{\hyperlink{class_beam_1_1_flux_node_a4bd30f3c8d311afdcd5c0d208e3bbf0f}{is\+Bypassed}} () const
\item 
void \doxymbox{\hyperlink{class_beam_1_1_flux_node_ad53f3fcaa5737f46d88530f40dbfbe32}{add\+Parameter}} (std::\+shared\+\_\+ptr$<$ \doxymbox{\hyperlink{class_beam_1_1_parameter}{Parameter}} $>$ \doxymbox{\hyperlink{texture_8cpp_aaded45152436a99bb4f9bda081df9f69}{param}})
\item 
std::\+shared\+\_\+ptr$<$ \doxymbox{\hyperlink{class_beam_1_1_parameter}{Parameter}} $>$ \doxymbox{\hyperlink{class_beam_1_1_flux_node_a59a32442eec144010741b9f2086c516e}{get\+Parameter}} (const std::\+string \&name)
\item 
const std::\+map$<$ std::\+string, std::\+shared\+\_\+ptr$<$ \doxymbox{\hyperlink{class_beam_1_1_parameter}{Parameter}} $>$ $>$ \& \doxymbox{\hyperlink{class_beam_1_1_flux_node_a6296c79b1ba77aa8b9526ace4a109529}{get\+Parameters}} () const
\end{DoxyCompactItemize}
\doxysubsubsection*{Additional Inherited Members}
\doxysubsection*{Protected Member Functions inherited from \doxymbox{\hyperlink{class_beam_1_1_flux_plugin}{Beam::\+\+Flux\+Plugin}}}
\begin{DoxyCompactItemize}
\item 
void \doxymbox{\hyperlink{class_beam_1_1_flux_plugin_a1768cc84018f8de19bcbf781e9b7ac3f}{add\+Param}} (const std::\+string \&name, float min, float max, float initial)
\item 
float \doxymbox{\hyperlink{class_beam_1_1_flux_plugin_a1b292b033caaaf9ec67154dfee2e577b}{get\+Param}} (const std::\+string \&name)
\item 
float \doxymbox{\hyperlink{class_beam_1_1_flux_plugin_a3e65f35944360e4e6ac370a967cf5eb3}{get\+Sample\+Rate}} () const
\end{DoxyCompactItemize}
\doxysubsection*{Protected Member Functions inherited from \doxymbox{\hyperlink{class_beam_1_1_flux_node}{Beam::\+\+Flux\+Node}}}
\begin{DoxyCompactItemize}
\item 
void \doxymbox{\hyperlink{class_beam_1_1_flux_node_ae3bafc1c5a1aa545167256172b3d3688}{setup\+Buffers}} (int num\+Inputs, int num\+Outputs, int buffer\+Size, int channels)
\begin{DoxyCompactList}\small\item\em Pre-\/allocates buffers for inputs and outputs. \end{DoxyCompactList}\end{DoxyCompactItemize}
\doxysubsection*{Protected Attributes inherited from \doxymbox{\hyperlink{class_beam_1_1_flux_node}{Beam::\+\+Flux\+Node}}}
\begin{DoxyCompactItemize}
\item 
std::\+vector$<$ std::\+vector$<$ float $>$ $>$ \doxymbox{\hyperlink{class_beam_1_1_flux_node_a8edab1c9ebd83e73bbfd92af29d6e92c}{m\+\_\+inputs}}
\item 
std::\+vector$<$ std::\+vector$<$ float $>$ $>$ \doxymbox{\hyperlink{class_beam_1_1_flux_node_a496905f0ff42c432eb38e19bd6135383}{m\+\_\+outputs}}
\item 
std::\+map$<$ std::\+string, std::\+shared\+\_\+ptr$<$ \doxymbox{\hyperlink{class_beam_1_1_parameter}{Parameter}} $>$ $>$ \doxymbox{\hyperlink{class_beam_1_1_flux_node_a65628a37cd2dd2832eda60e74ec1aed3}{m\+\_\+parameters}}
\item 
std::\+atomic$<$ bool $>$ \doxymbox{\hyperlink{class_beam_1_1_flux_node_a6116dcdcfa20998fe90dc75a74f25d9b}{m\+\_\+bypassed}} \{false\}
\item 
size\+\_\+t \doxymbox{\hyperlink{class_beam_1_1_flux_node_a7d8556ddb1482f997cda7749d737668b}{m\+\_\+current\+Frame}} = 0
\end{DoxyCompactItemize}


\label{doc-constructors}
\Hypertarget{class_beam_1_1_console_e___e_q_doc-constructors}
\doxysubsection{Constructor \& Destructor Documentation}
\Hypertarget{class_beam_1_1_console_e___e_q_a5d5eba473f0987818112706308ecec41}\index{Beam::ConsoleE\_EQ@{Beam::ConsoleE\_EQ}!ConsoleE\_EQ@{ConsoleE\_EQ}}
\index{ConsoleE\_EQ@{ConsoleE\_EQ}!Beam::ConsoleE\_EQ@{Beam::ConsoleE\_EQ}}
\doxysubsubsection{\texorpdfstring{ConsoleE\_EQ()}{ConsoleE\_EQ()}}
{\footnotesize\ttfamily \label{class_beam_1_1_console_e___e_q_a5d5eba473f0987818112706308ecec41} 
Beam::\+\+Console\+E\+\_\+\+EQ::\+\+Console\+E\+\_\+\+EQ (\begin{DoxyParamCaption}\item[{int}]{buf}{, }\item[{float}]{sr}{}\end{DoxyParamCaption})\hspace{0.3cm}{\ttfamily [inline]}}



\label{doc-func-members}
\Hypertarget{class_beam_1_1_console_e___e_q_doc-func-members}
\doxysubsection{Member Function Documentation}
\Hypertarget{class_beam_1_1_console_e___e_q_a957a65f7ecdbe08cd5f6ae9c5c8796da}\index{Beam::ConsoleE\_EQ@{Beam::ConsoleE\_EQ}!processBlock@{processBlock}}
\index{processBlock@{processBlock}!Beam::ConsoleE\_EQ@{Beam::ConsoleE\_EQ}}
\doxysubsubsection{\texorpdfstring{processBlock()}{processBlock()}}
{\footnotesize\ttfamily \label{class_beam_1_1_console_e___e_q_a957a65f7ecdbe08cd5f6ae9c5c8796da} 
void Beam::\+\+Console\+E\+\_\+\+EQ::\+process\+Block (\begin{DoxyParamCaption}\item[{const float \texorpdfstring{$\ast$}{*}}]{in}{, }\item[{float \texorpdfstring{$\ast$}{*}}]{out}{, }\item[{int}]{total}{}\end{DoxyParamCaption})\hspace{0.3cm}{\ttfamily [inline]}, {\ttfamily [override]}, {\ttfamily [virtual]}}



Implements \doxymbox{\hyperlink{class_beam_1_1_flux_plugin_ab10324716cec75feee93b6a3159c7912}{Beam::\+\+Flux\+Plugin}}.



The documentation for this class was generated from the following file:\+\begin{DoxyCompactItemize}
\item 
src/\+engine/\+\doxymbox{\hyperlink{analog__suite_8hpp}{analog\+\_\+suite.\+hpp}}\end{DoxyCompactItemize}

\doxysection{Beam::\+Custom\+Filter Class Reference}
\hypertarget{class_beam_1_1_custom_filter}{}\label{class_beam_1_1_custom_filter}\index{Beam::CustomFilter@{Beam::CustomFilter}}


{\ttfamily \+\#include $<$custom\+\_\+filter.\+hpp$>$}

Inheritance diagram for Beam::\+Custom\+Filter:\+\begin{figure}[H]
\begin{center}
\leavevmode
\includegraphics[height=3.000000cm]{class_beam_1_1_custom_filter}
\end{center}
\end{figure}
\doxysubsubsection*{Public Member Functions}
\begin{DoxyCompactItemize}
\item 
\doxymbox{\hyperlink{class_beam_1_1_custom_filter_a9e80d7f6addabbb786d562eb45709fc7}{Custom\+Filter}} (int buffer\+Size, float sample\+Rate)
\item 
void \doxymbox{\hyperlink{class_beam_1_1_custom_filter_ae561b7a4fb99126dda6d9adb7c12c80a}{process\+Block}} (const float \texorpdfstring{$\ast$}{*}input, float \texorpdfstring{$\ast$}{*}output, int total\+Samples) override
\end{DoxyCompactItemize}
\doxysubsection*{Public Member Functions inherited from \doxymbox{\hyperlink{class_beam_1_1_flux_plugin}{Beam::\+\+Flux\+Plugin}}}
\begin{DoxyCompactItemize}
\item 
\doxymbox{\hyperlink{class_beam_1_1_flux_plugin_a9d91b56960799fdbfc4575e2fcfa6689}{Flux\+Plugin}} (const std::\+string \&name, int buffer\+Size, float sample\+Rate)
\item 
void \doxymbox{\hyperlink{class_beam_1_1_flux_plugin_a181430e1cbf129891fe3ed72f3905a61}{process}} (int frames) override
\item 
std::\+string \doxymbox{\hyperlink{class_beam_1_1_flux_plugin_a450563af4d65a25b8a8e896dab77a3c6}{get\+Name}} () const override
\item 
std::\+vector$<$ \doxymbox{\hyperlink{struct_beam_1_1_flux_node_1_1_port}{Port}} $>$ \doxymbox{\hyperlink{class_beam_1_1_flux_plugin_ad231db67f900e8e7dd853936ad2e866a}{get\+Input\+Ports}} () const override
\item 
std::\+vector$<$ \doxymbox{\hyperlink{struct_beam_1_1_flux_node_1_1_port}{Port}} $>$ \doxymbox{\hyperlink{class_beam_1_1_flux_plugin_a4ea312de74047e127e818a26e715d8bb}{get\+Output\+Ports}} () const override
\end{DoxyCompactItemize}
\doxysubsection*{Public Member Functions inherited from \doxymbox{\hyperlink{class_beam_1_1_flux_node}{Beam::\+\+Flux\+Node}}}
\begin{DoxyCompactItemize}
\item 
virtual \doxymbox{\hyperlink{class_beam_1_1_flux_node_a708c135cdb61e8838469998cd8a84e65}{\texorpdfstring{$\sim$}{\string~}\+Flux\+Node}} ()=default
\item 
float \texorpdfstring{$\ast$}{*} \doxymbox{\hyperlink{class_beam_1_1_flux_node_ac90bd1a05b5bed3d68978f532386ed29}{get\+Input\+Buffer}} (int port\+Idx)
\item 
float \texorpdfstring{$\ast$}{*} \doxymbox{\hyperlink{class_beam_1_1_flux_node_abf11cfd4f2346ee0cd46d4345f1ed7d4}{get\+Output\+Buffer}} (int port\+Idx)
\item 
void \doxymbox{\hyperlink{class_beam_1_1_flux_node_af37f8c1b6b825da2ce7e35011d6f8253}{set\+Bypass}} (bool bypass)
\item 
bool \doxymbox{\hyperlink{class_beam_1_1_flux_node_a4bd30f3c8d311afdcd5c0d208e3bbf0f}{is\+Bypassed}} () const
\item 
void \doxymbox{\hyperlink{class_beam_1_1_flux_node_ad53f3fcaa5737f46d88530f40dbfbe32}{add\+Parameter}} (std::\+shared\+\_\+ptr$<$ \doxymbox{\hyperlink{class_beam_1_1_parameter}{Parameter}} $>$ param)
\item 
std::\+shared\+\_\+ptr$<$ \doxymbox{\hyperlink{class_beam_1_1_parameter}{Parameter}} $>$ \doxymbox{\hyperlink{class_beam_1_1_flux_node_a59a32442eec144010741b9f2086c516e}{get\+Parameter}} (const std::\+string \&name)
\item 
const std::\+map$<$ std::\+string, std::\+shared\+\_\+ptr$<$ \doxymbox{\hyperlink{class_beam_1_1_parameter}{Parameter}} $>$ $>$ \& \doxymbox{\hyperlink{class_beam_1_1_flux_node_a6296c79b1ba77aa8b9526ace4a109529}{get\+Parameters}} () const
\end{DoxyCompactItemize}
\doxysubsubsection*{Private Attributes}
\begin{DoxyCompactItemize}
\item 
float \doxymbox{\hyperlink{class_beam_1_1_custom_filter_a1c8531ae8764996a5c95bffa5f2f2c47}{m\+\_\+z1}} = 0.\+0f
\item 
float \doxymbox{\hyperlink{class_beam_1_1_custom_filter_a1fd2bc925fa36ca245925a4bd187d6c1}{m\+\_\+last\+Out}} = 0.\+0f
\end{DoxyCompactItemize}
\doxysubsubsection*{Additional Inherited Members}
\doxysubsection*{Protected Member Functions inherited from \doxymbox{\hyperlink{class_beam_1_1_flux_plugin}{Beam::\+\+Flux\+Plugin}}}
\begin{DoxyCompactItemize}
\item 
void \doxymbox{\hyperlink{class_beam_1_1_flux_plugin_a1768cc84018f8de19bcbf781e9b7ac3f}{add\+Param}} (const std::\+string \&name, float min, float max, float initial)
\item 
float \doxymbox{\hyperlink{class_beam_1_1_flux_plugin_a1b292b033caaaf9ec67154dfee2e577b}{get\+Param}} (const std::\+string \&name)
\item 
float \doxymbox{\hyperlink{class_beam_1_1_flux_plugin_a3e65f35944360e4e6ac370a967cf5eb3}{get\+Sample\+Rate}} () const
\end{DoxyCompactItemize}
\doxysubsection*{Protected Member Functions inherited from \doxymbox{\hyperlink{class_beam_1_1_flux_node}{Beam::\+\+Flux\+Node}}}
\begin{DoxyCompactItemize}
\item 
void \doxymbox{\hyperlink{class_beam_1_1_flux_node_ae3bafc1c5a1aa545167256172b3d3688}{setup\+Buffers}} (int num\+Inputs, int num\+Outputs, int buffer\+Size, int channels)
\end{DoxyCompactItemize}
\doxysubsection*{Protected Attributes inherited from \doxymbox{\hyperlink{class_beam_1_1_flux_node}{Beam::\+\+Flux\+Node}}}
\begin{DoxyCompactItemize}
\item 
std::\+vector$<$ std::\+vector$<$ float $>$ $>$ \doxymbox{\hyperlink{class_beam_1_1_flux_node_a8edab1c9ebd83e73bbfd92af29d6e92c}{m\+\_\+inputs}}
\item 
std::\+vector$<$ std::\+vector$<$ float $>$ $>$ \doxymbox{\hyperlink{class_beam_1_1_flux_node_a496905f0ff42c432eb38e19bd6135383}{m\+\_\+outputs}}
\item 
std::\+map$<$ std::\+string, std::\+shared\+\_\+ptr$<$ \doxymbox{\hyperlink{class_beam_1_1_parameter}{Parameter}} $>$ $>$ \doxymbox{\hyperlink{class_beam_1_1_flux_node_a65628a37cd2dd2832eda60e74ec1aed3}{m\+\_\+parameters}}
\item 
std::\+atomic$<$ bool $>$ \doxymbox{\hyperlink{class_beam_1_1_flux_node_a6116dcdcfa20998fe90dc75a74f25d9b}{m\+\_\+bypassed}} \{false\}
\end{DoxyCompactItemize}


\doxysubsection{Detailed Description}
Example of a User-\/\+Designed Filter using the Flux SDK. This is a simple One-\/\+Pole Low Pass filter. 

\label{doc-constructors}
\Hypertarget{class_beam_1_1_custom_filter_doc-constructors}
\doxysubsection{Constructor \& Destructor Documentation}
\Hypertarget{class_beam_1_1_custom_filter_a9e80d7f6addabbb786d562eb45709fc7}\index{Beam::CustomFilter@{Beam::CustomFilter}!CustomFilter@{CustomFilter}}
\index{CustomFilter@{CustomFilter}!Beam::CustomFilter@{Beam::CustomFilter}}
\doxysubsubsection{\texorpdfstring{CustomFilter()}{CustomFilter()}}
{\footnotesize\ttfamily \label{class_beam_1_1_custom_filter_a9e80d7f6addabbb786d562eb45709fc7} 
Beam::\+\+Custom\+Filter::\+\+Custom\+Filter (\begin{DoxyParamCaption}\item[{int}]{buffer\+Size}{, }\item[{float}]{sample\+Rate}{}\end{DoxyParamCaption})\hspace{0.3cm}{\ttfamily [inline]}}



\label{doc-func-members}
\Hypertarget{class_beam_1_1_custom_filter_doc-func-members}
\doxysubsection{Member Function Documentation}
\Hypertarget{class_beam_1_1_custom_filter_ae561b7a4fb99126dda6d9adb7c12c80a}\index{Beam::CustomFilter@{Beam::CustomFilter}!processBlock@{processBlock}}
\index{processBlock@{processBlock}!Beam::CustomFilter@{Beam::CustomFilter}}
\doxysubsubsection{\texorpdfstring{processBlock()}{processBlock()}}
{\footnotesize\ttfamily \label{class_beam_1_1_custom_filter_ae561b7a4fb99126dda6d9adb7c12c80a} 
void Beam::\+\+Custom\+Filter::\+process\+Block (\begin{DoxyParamCaption}\item[{const float \texorpdfstring{$\ast$}{*}}]{input}{, }\item[{float \texorpdfstring{$\ast$}{*}}]{output}{, }\item[{int}]{total\+Samples}{}\end{DoxyParamCaption})\hspace{0.3cm}{\ttfamily [inline]}, {\ttfamily [override]}, {\ttfamily [virtual]}}



Implements \doxymbox{\hyperlink{class_beam_1_1_flux_plugin_ab10324716cec75feee93b6a3159c7912}{Beam::\+\+Flux\+Plugin}}.



\label{doc-variable-members}
\Hypertarget{class_beam_1_1_custom_filter_doc-variable-members}
\doxysubsection{Member Data Documentation}
\Hypertarget{class_beam_1_1_custom_filter_a1fd2bc925fa36ca245925a4bd187d6c1}\index{Beam::CustomFilter@{Beam::CustomFilter}!m\_lastOut@{m\_lastOut}}
\index{m\_lastOut@{m\_lastOut}!Beam::CustomFilter@{Beam::CustomFilter}}
\doxysubsubsection{\texorpdfstring{m\_lastOut}{m\_lastOut}}
{\footnotesize\ttfamily \label{class_beam_1_1_custom_filter_a1fd2bc925fa36ca245925a4bd187d6c1} 
float Beam::\+\+Custom\+Filter::\+m\+\_\+last\+Out = 0.\+0f\hspace{0.3cm}{\ttfamily [private]}}

\Hypertarget{class_beam_1_1_custom_filter_a1c8531ae8764996a5c95bffa5f2f2c47}\index{Beam::CustomFilter@{Beam::CustomFilter}!m\_z1@{m\_z1}}
\index{m\_z1@{m\_z1}!Beam::CustomFilter@{Beam::CustomFilter}}
\doxysubsubsection{\texorpdfstring{m\_z1}{m\_z1}}
{\footnotesize\ttfamily \label{class_beam_1_1_custom_filter_a1c8531ae8764996a5c95bffa5f2f2c47} 
float Beam::\+\+Custom\+Filter::\+m\+\_\+z1 = 0.\+0f\hspace{0.3cm}{\ttfamily [private]}}



The documentation for this class was generated from the following file:\+\begin{DoxyCompactItemize}
\item 
src/\+dsp/\+\doxymbox{\hyperlink{custom__filter_8hpp}{custom\+\_\+filter.\+hpp}}\end{DoxyCompactItemize}

\doxysection{Beam::\+Delay\+Node Class Reference}
\hypertarget{class_beam_1_1_delay_node}{}\label{class_beam_1_1_delay_node}\index{Beam::DelayNode@{Beam::DelayNode}}


{\ttfamily \+\#include $<$delay\+\_\+node.\+hpp$>$}

Inheritance diagram for Beam::\+Delay\+Node:\+\begin{figure}[H]
\begin{center}
\leavevmode
\includegraphics[height=2.000000cm]{class_beam_1_1_delay_node}
\end{center}
\end{figure}
\doxysubsubsection*{Public Member Functions}
\begin{DoxyCompactItemize}
\item 
\doxymbox{\hyperlink{class_beam_1_1_delay_node_af4f9dc6cc6e26c7bb963f17c85c3858c}{Delay\+Node}} (float delay\+Seconds, float feedback, float sample\+Rate)
\item 
void \doxymbox{\hyperlink{class_beam_1_1_delay_node_a44fdc2abfe4112bf5e89f97d1c9ebe20}{process}} (float \texorpdfstring{$\ast$}{*}buffer, int frames, int channels, size\+\_\+t start\+Frame=0) override
\item 
std::\+string \doxymbox{\hyperlink{class_beam_1_1_delay_node_aac469a6ff6f2ea95f5ffbb06f60dbf2e}{get\+Name}} () const override
\end{DoxyCompactItemize}
\doxysubsection*{Public Member Functions inherited from \doxymbox{\hyperlink{class_beam_1_1_audio_node}{Beam::\+\+Audio\+Node}}}
\begin{DoxyCompactItemize}
\item 
virtual \doxymbox{\hyperlink{class_beam_1_1_audio_node_afbea31954b50918131d31fc0d1f6de8c}{\texorpdfstring{$\sim$}{\string~}\+Audio\+Node}} ()=default
\item 
void \doxymbox{\hyperlink{class_beam_1_1_audio_node_a3fcc68eab5b1adf547a4205f258b212c}{set\+Bypass}} (bool bypass)
\item 
bool \doxymbox{\hyperlink{class_beam_1_1_audio_node_a6ba1724cff34b5bc0f811ee2537caae5}{is\+Bypassed}} () const
\end{DoxyCompactItemize}
\doxysubsubsection*{Private Attributes}
\begin{DoxyCompactItemize}
\item 
std::\+vector$<$ float $>$ \doxymbox{\hyperlink{class_beam_1_1_delay_node_a5b84198373b6956ed0a207cff4bbe89e}{m\+\_\+buffer}}
\item 
float \doxymbox{\hyperlink{class_beam_1_1_delay_node_aea926261f1f2685e8e626cad0eff131b}{m\+\_\+feedback}}
\item 
float \doxymbox{\hyperlink{class_beam_1_1_delay_node_a00520ac5853ff38e3f68cd5c64d635b2}{m\+\_\+sample\+Rate}}
\item 
size\+\_\+t \doxymbox{\hyperlink{class_beam_1_1_delay_node_a418ca0782552901f5dc8b8910ea9e809}{m\+\_\+write\+Ptr}}
\end{DoxyCompactItemize}
\doxysubsubsection*{Additional Inherited Members}
\doxysubsection*{Protected Attributes inherited from \doxymbox{\hyperlink{class_beam_1_1_audio_node}{Beam::\+\+Audio\+Node}}}
\begin{DoxyCompactItemize}
\item 
bool \doxymbox{\hyperlink{class_beam_1_1_audio_node_ac5ad81de4a5d0abe555fe9f06219b09f}{m\+\_\+is\+Bypassed}} = false
\end{DoxyCompactItemize}


\label{doc-constructors}
\Hypertarget{class_beam_1_1_delay_node_doc-constructors}
\doxysubsection{Constructor \& Destructor Documentation}
\Hypertarget{class_beam_1_1_delay_node_af4f9dc6cc6e26c7bb963f17c85c3858c}\index{Beam::DelayNode@{Beam::DelayNode}!DelayNode@{DelayNode}}
\index{DelayNode@{DelayNode}!Beam::DelayNode@{Beam::DelayNode}}
\doxysubsubsection{\texorpdfstring{DelayNode()}{DelayNode()}}
{\footnotesize\ttfamily \label{class_beam_1_1_delay_node_af4f9dc6cc6e26c7bb963f17c85c3858c} 
Beam::\+\+Delay\+Node::\+\+Delay\+Node (\begin{DoxyParamCaption}\item[{float}]{delay\+Seconds}{, }\item[{float}]{feedback}{, }\item[{float}]{sample\+Rate}{}\end{DoxyParamCaption})\hspace{0.3cm}{\ttfamily [inline]}}



\label{doc-func-members}
\Hypertarget{class_beam_1_1_delay_node_doc-func-members}
\doxysubsection{Member Function Documentation}
\Hypertarget{class_beam_1_1_delay_node_aac469a6ff6f2ea95f5ffbb06f60dbf2e}\index{Beam::DelayNode@{Beam::DelayNode}!getName@{getName}}
\index{getName@{getName}!Beam::DelayNode@{Beam::DelayNode}}
\doxysubsubsection{\texorpdfstring{getName()}{getName()}}
{\footnotesize\ttfamily \label{class_beam_1_1_delay_node_aac469a6ff6f2ea95f5ffbb06f60dbf2e} 
std::\+string Beam::\+\+Delay\+Node::\+get\+Name (\begin{DoxyParamCaption}{}{}\end{DoxyParamCaption}) const\hspace{0.3cm}{\ttfamily [inline]}, {\ttfamily [override]}, {\ttfamily [virtual]}}



Implements \doxymbox{\hyperlink{class_beam_1_1_audio_node_a864b3bf9095638e43ad334b7b3706bec}{Beam::\+\+Audio\+Node}}.

\Hypertarget{class_beam_1_1_delay_node_a44fdc2abfe4112bf5e89f97d1c9ebe20}\index{Beam::DelayNode@{Beam::DelayNode}!process@{process}}
\index{process@{process}!Beam::DelayNode@{Beam::DelayNode}}
\doxysubsubsection{\texorpdfstring{process()}{process()}}
{\footnotesize\ttfamily \label{class_beam_1_1_delay_node_a44fdc2abfe4112bf5e89f97d1c9ebe20} 
void Beam::\+\+Delay\+Node::\+process (\begin{DoxyParamCaption}\item[{float \texorpdfstring{$\ast$}{*}}]{buffer}{, }\item[{int}]{frames}{, }\item[{int}]{channels}{, }\item[{size\+\_\+t}]{start\+Frame}{ = {\ttfamily 0}}\end{DoxyParamCaption})\hspace{0.3cm}{\ttfamily [inline]}, {\ttfamily [override]}, {\ttfamily [virtual]}}



Implements \doxymbox{\hyperlink{class_beam_1_1_audio_node_ab0cb6fa1aba031e703be16be26e0d6b7}{Beam::\+\+Audio\+Node}}.



\label{doc-variable-members}
\Hypertarget{class_beam_1_1_delay_node_doc-variable-members}
\doxysubsection{Member Data Documentation}
\Hypertarget{class_beam_1_1_delay_node_a5b84198373b6956ed0a207cff4bbe89e}\index{Beam::DelayNode@{Beam::DelayNode}!m\_buffer@{m\_buffer}}
\index{m\_buffer@{m\_buffer}!Beam::DelayNode@{Beam::DelayNode}}
\doxysubsubsection{\texorpdfstring{m\_buffer}{m\_buffer}}
{\footnotesize\ttfamily \label{class_beam_1_1_delay_node_a5b84198373b6956ed0a207cff4bbe89e} 
std::\+vector$<$float$>$ Beam::\+\+Delay\+Node::\+m\+\_\+buffer\hspace{0.3cm}{\ttfamily [private]}}

\Hypertarget{class_beam_1_1_delay_node_aea926261f1f2685e8e626cad0eff131b}\index{Beam::DelayNode@{Beam::DelayNode}!m\_feedback@{m\_feedback}}
\index{m\_feedback@{m\_feedback}!Beam::DelayNode@{Beam::DelayNode}}
\doxysubsubsection{\texorpdfstring{m\_feedback}{m\_feedback}}
{\footnotesize\ttfamily \label{class_beam_1_1_delay_node_aea926261f1f2685e8e626cad0eff131b} 
float Beam::\+\+Delay\+Node::\+m\+\_\+feedback\hspace{0.3cm}{\ttfamily [private]}}

\Hypertarget{class_beam_1_1_delay_node_a00520ac5853ff38e3f68cd5c64d635b2}\index{Beam::DelayNode@{Beam::DelayNode}!m\_sampleRate@{m\_sampleRate}}
\index{m\_sampleRate@{m\_sampleRate}!Beam::DelayNode@{Beam::DelayNode}}
\doxysubsubsection{\texorpdfstring{m\_sampleRate}{m\_sampleRate}}
{\footnotesize\ttfamily \label{class_beam_1_1_delay_node_a00520ac5853ff38e3f68cd5c64d635b2} 
float Beam::\+\+Delay\+Node::\+m\+\_\+sample\+Rate\hspace{0.3cm}{\ttfamily [private]}}

\Hypertarget{class_beam_1_1_delay_node_a418ca0782552901f5dc8b8910ea9e809}\index{Beam::DelayNode@{Beam::DelayNode}!m\_writePtr@{m\_writePtr}}
\index{m\_writePtr@{m\_writePtr}!Beam::DelayNode@{Beam::DelayNode}}
\doxysubsubsection{\texorpdfstring{m\_writePtr}{m\_writePtr}}
{\footnotesize\ttfamily \label{class_beam_1_1_delay_node_a418ca0782552901f5dc8b8910ea9e809} 
size\+\_\+t Beam::\+\+Delay\+Node::\+m\+\_\+write\+Ptr\hspace{0.3cm}{\ttfamily [private]}}



The documentation for this class was generated from the following file:\+\begin{DoxyCompactItemize}
\item 
src/\+engine/\+\doxymbox{\hyperlink{delay__node_8hpp}{delay\+\_\+node.\+hpp}}\end{DoxyCompactItemize}

\doxysection{Beam::\+Disk\+Streamer Class Reference}
\hypertarget{class_beam_1_1_disk_streamer}{}\label{class_beam_1_1_disk_streamer}\index{Beam::DiskStreamer@{Beam::DiskStreamer}}


{\ttfamily \+\#include $<$disk\+\_\+streamer.\+hpp$>$}

\doxysubsubsection*{Public Member Functions}
\begin{DoxyCompactItemize}
\item 
\doxymbox{\hyperlink{class_beam_1_1_disk_streamer_a93d68a5758f8220ce8c7da332493d6db}{Disk\+Streamer}} (size\+\_\+t buffer\+Size=4096 \texorpdfstring{$\ast$}{*}4)
\item 
\doxymbox{\hyperlink{class_beam_1_1_disk_streamer_a93a8d7b377c73d317edac3a4d6f1b32b}{\texorpdfstring{$\sim$}{\string~}\+Disk\+Streamer}} ()
\item 
bool \doxymbox{\hyperlink{class_beam_1_1_disk_streamer_a71d0fbf4718a1e5aad5053ce758581e2}{open}} (const std::\+string \&file\+Path)
\item 
void \doxymbox{\hyperlink{class_beam_1_1_disk_streamer_a122892dcfb53b85476db17ed3b5b1682}{close}} ()
\item 
size\+\_\+t \doxymbox{\hyperlink{class_beam_1_1_disk_streamer_aafe6893c5cc132d4f09a298c6b833b0e}{read}} (float \texorpdfstring{$\ast$}{*}output, size\+\_\+t frames, int channels)
\item 
void \doxymbox{\hyperlink{class_beam_1_1_disk_streamer_a27a3c695476c10762819099d858aba0a}{seek}} (size\+\_\+t frame)
\end{DoxyCompactItemize}
\doxysubsubsection*{Private Member Functions}
\begin{DoxyCompactItemize}
\item 
void \doxymbox{\hyperlink{class_beam_1_1_disk_streamer_aa0ee671fa3e819b9cf4a85a76a0e3af7}{stream\+Loop}} ()
\end{DoxyCompactItemize}
\doxysubsubsection*{Private Attributes}
\begin{DoxyCompactItemize}
\item 
std::\+string \doxymbox{\hyperlink{class_beam_1_1_disk_streamer_a5e2c3ec0da0e5b6559309cb10c23c2b6}{m\+\_\+file\+Path}}
\item 
std::\+atomic$<$ bool $>$ \doxymbox{\hyperlink{class_beam_1_1_disk_streamer_a0449260006d5de515f80bbf5fef28354}{m\+\_\+keep\+Streaming}} \{false\}
\item 
std::\+unique\+\_\+ptr$<$ \doxymbox{\hyperlink{class_beam_1_1_wav_reader}{Wav\+Reader}} $>$ \doxymbox{\hyperlink{class_beam_1_1_disk_streamer_aaefa6abf914cceb4efe0859a4ea462c2}{m\+\_\+reader}}
\item 
size\+\_\+t \doxymbox{\hyperlink{class_beam_1_1_disk_streamer_a2bfdd24e98745a5b8cc74863d2c04d38}{m\+\_\+buffer\+Size}}
\end{DoxyCompactItemize}


\label{doc-constructors}
\Hypertarget{class_beam_1_1_disk_streamer_doc-constructors}
\doxysubsection{Constructor \& Destructor Documentation}
\Hypertarget{class_beam_1_1_disk_streamer_a93d68a5758f8220ce8c7da332493d6db}\index{Beam::DiskStreamer@{Beam::DiskStreamer}!DiskStreamer@{DiskStreamer}}
\index{DiskStreamer@{DiskStreamer}!Beam::DiskStreamer@{Beam::DiskStreamer}}
\doxysubsubsection{\texorpdfstring{DiskStreamer()}{DiskStreamer()}}
{\footnotesize\ttfamily \label{class_beam_1_1_disk_streamer_a93d68a5758f8220ce8c7da332493d6db} 
Beam::\+\+Disk\+Streamer::\+\+Disk\+Streamer (\begin{DoxyParamCaption}\item[{size\+\_\+t}]{buffer\+Size}{ = {\ttfamily 4096~\texorpdfstring{$\ast$}{*}~4}}\end{DoxyParamCaption})}

\Hypertarget{class_beam_1_1_disk_streamer_a93a8d7b377c73d317edac3a4d6f1b32b}\index{Beam::DiskStreamer@{Beam::DiskStreamer}!````~DiskStreamer@{\texorpdfstring{$\sim$}{\string~}DiskStreamer}}
\index{````~DiskStreamer@{\texorpdfstring{$\sim$}{\string~}DiskStreamer}!Beam::DiskStreamer@{Beam::DiskStreamer}}
\doxysubsubsection{\texorpdfstring{\texorpdfstring{$\sim$}{\string~}DiskStreamer()}{\string~DiskStreamer()}}
{\footnotesize\ttfamily \label{class_beam_1_1_disk_streamer_a93a8d7b377c73d317edac3a4d6f1b32b} 
Beam::\+\+Disk\+Streamer::\+\texorpdfstring{$\sim$}{\string~}\+Disk\+Streamer (\begin{DoxyParamCaption}{}{}\end{DoxyParamCaption})}



\label{doc-func-members}
\Hypertarget{class_beam_1_1_disk_streamer_doc-func-members}
\doxysubsection{Member Function Documentation}
\Hypertarget{class_beam_1_1_disk_streamer_a122892dcfb53b85476db17ed3b5b1682}\index{Beam::DiskStreamer@{Beam::DiskStreamer}!close@{close}}
\index{close@{close}!Beam::DiskStreamer@{Beam::DiskStreamer}}
\doxysubsubsection{\texorpdfstring{close()}{close()}}
{\footnotesize\ttfamily \label{class_beam_1_1_disk_streamer_a122892dcfb53b85476db17ed3b5b1682} 
void Beam::\+\+Disk\+Streamer::\+close (\begin{DoxyParamCaption}{}{}\end{DoxyParamCaption})}

\Hypertarget{class_beam_1_1_disk_streamer_a71d0fbf4718a1e5aad5053ce758581e2}\index{Beam::DiskStreamer@{Beam::DiskStreamer}!open@{open}}
\index{open@{open}!Beam::DiskStreamer@{Beam::DiskStreamer}}
\doxysubsubsection{\texorpdfstring{open()}{open()}}
{\footnotesize\ttfamily \label{class_beam_1_1_disk_streamer_a71d0fbf4718a1e5aad5053ce758581e2} 
bool Beam::\+\+Disk\+Streamer::\+open (\begin{DoxyParamCaption}\item[{const std::\+string \&}]{file\+Path}{}\end{DoxyParamCaption})}

\Hypertarget{class_beam_1_1_disk_streamer_aafe6893c5cc132d4f09a298c6b833b0e}\index{Beam::DiskStreamer@{Beam::DiskStreamer}!read@{read}}
\index{read@{read}!Beam::DiskStreamer@{Beam::DiskStreamer}}
\doxysubsubsection{\texorpdfstring{read()}{read()}}
{\footnotesize\ttfamily \label{class_beam_1_1_disk_streamer_aafe6893c5cc132d4f09a298c6b833b0e} 
size\+\_\+t Beam::\+\+Disk\+Streamer::\+read (\begin{DoxyParamCaption}\item[{float \texorpdfstring{$\ast$}{*}}]{output}{, }\item[{size\+\_\+t}]{frames}{, }\item[{int}]{channels}{}\end{DoxyParamCaption})}

\Hypertarget{class_beam_1_1_disk_streamer_a27a3c695476c10762819099d858aba0a}\index{Beam::DiskStreamer@{Beam::DiskStreamer}!seek@{seek}}
\index{seek@{seek}!Beam::DiskStreamer@{Beam::DiskStreamer}}
\doxysubsubsection{\texorpdfstring{seek()}{seek()}}
{\footnotesize\ttfamily \label{class_beam_1_1_disk_streamer_a27a3c695476c10762819099d858aba0a} 
void Beam::\+\+Disk\+Streamer::\+seek (\begin{DoxyParamCaption}\item[{size\+\_\+t}]{frame}{}\end{DoxyParamCaption})}

\Hypertarget{class_beam_1_1_disk_streamer_aa0ee671fa3e819b9cf4a85a76a0e3af7}\index{Beam::DiskStreamer@{Beam::DiskStreamer}!streamLoop@{streamLoop}}
\index{streamLoop@{streamLoop}!Beam::DiskStreamer@{Beam::DiskStreamer}}
\doxysubsubsection{\texorpdfstring{streamLoop()}{streamLoop()}}
{\footnotesize\ttfamily \label{class_beam_1_1_disk_streamer_aa0ee671fa3e819b9cf4a85a76a0e3af7} 
void Beam::\+\+Disk\+Streamer::\+stream\+Loop (\begin{DoxyParamCaption}{}{}\end{DoxyParamCaption})\hspace{0.3cm}{\ttfamily [private]}}



\label{doc-variable-members}
\Hypertarget{class_beam_1_1_disk_streamer_doc-variable-members}
\doxysubsection{Member Data Documentation}
\Hypertarget{class_beam_1_1_disk_streamer_a2bfdd24e98745a5b8cc74863d2c04d38}\index{Beam::DiskStreamer@{Beam::DiskStreamer}!m\_bufferSize@{m\_bufferSize}}
\index{m\_bufferSize@{m\_bufferSize}!Beam::DiskStreamer@{Beam::DiskStreamer}}
\doxysubsubsection{\texorpdfstring{m\_bufferSize}{m\_bufferSize}}
{\footnotesize\ttfamily \label{class_beam_1_1_disk_streamer_a2bfdd24e98745a5b8cc74863d2c04d38} 
size\+\_\+t Beam::\+\+Disk\+Streamer::\+m\+\_\+buffer\+Size\hspace{0.3cm}{\ttfamily [private]}}

\Hypertarget{class_beam_1_1_disk_streamer_a5e2c3ec0da0e5b6559309cb10c23c2b6}\index{Beam::DiskStreamer@{Beam::DiskStreamer}!m\_filePath@{m\_filePath}}
\index{m\_filePath@{m\_filePath}!Beam::DiskStreamer@{Beam::DiskStreamer}}
\doxysubsubsection{\texorpdfstring{m\_filePath}{m\_filePath}}
{\footnotesize\ttfamily \label{class_beam_1_1_disk_streamer_a5e2c3ec0da0e5b6559309cb10c23c2b6} 
std::\+string Beam::\+\+Disk\+Streamer::\+m\+\_\+file\+Path\hspace{0.3cm}{\ttfamily [private]}}

\Hypertarget{class_beam_1_1_disk_streamer_a0449260006d5de515f80bbf5fef28354}\index{Beam::DiskStreamer@{Beam::DiskStreamer}!m\_keepStreaming@{m\_keepStreaming}}
\index{m\_keepStreaming@{m\_keepStreaming}!Beam::DiskStreamer@{Beam::DiskStreamer}}
\doxysubsubsection{\texorpdfstring{m\_keepStreaming}{m\_keepStreaming}}
{\footnotesize\ttfamily \label{class_beam_1_1_disk_streamer_a0449260006d5de515f80bbf5fef28354} 
std::\+atomic$<$bool$>$ Beam::\+\+Disk\+Streamer::\+m\+\_\+keep\+Streaming \{false\}\hspace{0.3cm}{\ttfamily [private]}}

\Hypertarget{class_beam_1_1_disk_streamer_aaefa6abf914cceb4efe0859a4ea462c2}\index{Beam::DiskStreamer@{Beam::DiskStreamer}!m\_reader@{m\_reader}}
\index{m\_reader@{m\_reader}!Beam::DiskStreamer@{Beam::DiskStreamer}}
\doxysubsubsection{\texorpdfstring{m\_reader}{m\_reader}}
{\footnotesize\ttfamily \label{class_beam_1_1_disk_streamer_aaefa6abf914cceb4efe0859a4ea462c2} 
std::\+unique\+\_\+ptr$<$\doxymbox{\hyperlink{class_beam_1_1_wav_reader}{Wav\+Reader}}$>$ Beam::\+\+Disk\+Streamer::\+m\+\_\+reader\hspace{0.3cm}{\ttfamily [private]}}



The documentation for this class was generated from the following files:\+\begin{DoxyCompactItemize}
\item 
src/\+dsp/\+\doxymbox{\hyperlink{disk__streamer_8hpp}{disk\+\_\+streamer.\+hpp}}\item 
src/\+dsp/\+\doxymbox{\hyperlink{disk__streamer_8cpp}{disk\+\_\+streamer.\+cpp}}\end{DoxyCompactItemize}

\input{class_beam_1_1_dynamics_module}
\doxysection{Beam::\+Echo\+Plex Class Reference}
\hypertarget{class_beam_1_1_echo_plex}{}\label{class_beam_1_1_echo_plex}\index{Beam::EchoPlex@{Beam::EchoPlex}}


{\ttfamily \+\#include $<$analog\+\_\+suite.\+hpp$>$}

Inheritance diagram for Beam::\+Echo\+Plex:\+\begin{figure}[H]
\begin{center}
\leavevmode
\includegraphics[height=3.000000cm]{class_beam_1_1_echo_plex}
\end{center}
\end{figure}
\doxysubsubsection*{Public Member Functions}
\begin{DoxyCompactItemize}
\item 
\doxymbox{\hyperlink{class_beam_1_1_echo_plex_ae58bd4f92e13a4aadaa591423ff2b0bb}{Echo\+Plex}} (int buf, float sr)
\item 
void \doxymbox{\hyperlink{class_beam_1_1_echo_plex_a26d16b6feed01873e169f83083873c61}{process\+Block}} (const float \texorpdfstring{$\ast$}{*}in, float \texorpdfstring{$\ast$}{*}out, int total) override
\end{DoxyCompactItemize}
\doxysubsection*{Public Member Functions inherited from \doxymbox{\hyperlink{class_beam_1_1_flux_plugin}{Beam::\+\+Flux\+Plugin}}}
\begin{DoxyCompactItemize}
\item 
\doxymbox{\hyperlink{class_beam_1_1_flux_plugin_a9d91b56960799fdbfc4575e2fcfa6689}{Flux\+Plugin}} (const std::\+string \&name, int buffer\+Size, float sample\+Rate)
\item 
virtual void \doxymbox{\hyperlink{class_beam_1_1_flux_plugin_a87fb076475f20b062493efa7ca00e045}{process\+Events}} (const \doxymbox{\hyperlink{class_beam_1_1_m_i_d_i_buffer}{MIDIBuffer}} \&midi)
\begin{DoxyCompactList}\small\item\em Handle MIDI events in your plugin. \end{DoxyCompactList}\item 
void \doxymbox{\hyperlink{class_beam_1_1_flux_plugin_aa1f9c569002ec23eeb5db0af686abea7}{process\+MIDI}} (const \doxymbox{\hyperlink{class_beam_1_1_m_i_d_i_buffer}{MIDIBuffer}} \&midi) override
\begin{DoxyCompactList}\small\item\em Optional MIDI processing. Called before \doxylink{class_beam_1_1_flux_plugin_a181430e1cbf129891fe3ed72f3905a61}{process()} in the engine loop. \end{DoxyCompactList}\item 
void \doxymbox{\hyperlink{class_beam_1_1_flux_plugin_a181430e1cbf129891fe3ed72f3905a61}{process}} (int frames) override
\begin{DoxyCompactList}\small\item\em Main audio processing method. Must be implemented by subclasses. \end{DoxyCompactList}\item 
std::\+string \doxymbox{\hyperlink{class_beam_1_1_flux_plugin_a450563af4d65a25b8a8e896dab77a3c6}{get\+Name}} () const override
\item 
std::\+vector$<$ \doxymbox{\hyperlink{struct_beam_1_1_flux_node_1_1_port}{Port}} $>$ \doxymbox{\hyperlink{class_beam_1_1_flux_plugin_ad231db67f900e8e7dd853936ad2e866a}{get\+Input\+Ports}} () const override
\item 
std::\+vector$<$ \doxymbox{\hyperlink{struct_beam_1_1_flux_node_1_1_port}{Port}} $>$ \doxymbox{\hyperlink{class_beam_1_1_flux_plugin_a4ea312de74047e127e818a26e715d8bb}{get\+Output\+Ports}} () const override
\end{DoxyCompactItemize}
\doxysubsection*{Public Member Functions inherited from \doxymbox{\hyperlink{class_beam_1_1_flux_node}{Beam::\+\+Flux\+Node}}}
\begin{DoxyCompactItemize}
\item 
virtual \doxymbox{\hyperlink{class_beam_1_1_flux_node_a708c135cdb61e8838469998cd8a84e65}{\texorpdfstring{$\sim$}{\string~}\+Flux\+Node}} ()=default
\item 
virtual void \doxymbox{\hyperlink{class_beam_1_1_flux_node_ace8cc49479d8924d44bca5fd4cd955e2}{on\+Transport\+State\+Changed}} (bool playing)
\begin{DoxyCompactList}\small\item\em Responds to global transport changes (Play/\+\+Pause). \end{DoxyCompactList}\item 
virtual void \doxymbox{\hyperlink{class_beam_1_1_flux_node_adc7c4e979bf27de5bfca66815ae97a67}{on\+Transport\+Seek}} (size\+\_\+t frame)
\begin{DoxyCompactList}\small\item\em Responds to timeline seeking. \end{DoxyCompactList}\item 
void \doxymbox{\hyperlink{class_beam_1_1_flux_node_aa579ec06608fd776987bbb089f27fd94}{set\+Current\+Frame}} (size\+\_\+t frame)
\begin{DoxyCompactList}\small\item\em Sets the current playhead position for this block. \end{DoxyCompactList}\item 
float \texorpdfstring{$\ast$}{*} \doxymbox{\hyperlink{class_beam_1_1_flux_node_ac90bd1a05b5bed3d68978f532386ed29}{get\+Input\+Buffer}} (int port\+Idx)
\item 
float \texorpdfstring{$\ast$}{*} \doxymbox{\hyperlink{class_beam_1_1_flux_node_abf11cfd4f2346ee0cd46d4345f1ed7d4}{get\+Output\+Buffer}} (int port\+Idx)
\item 
void \doxymbox{\hyperlink{class_beam_1_1_flux_node_af37f8c1b6b825da2ce7e35011d6f8253}{set\+Bypass}} (bool bypass)
\item 
bool \doxymbox{\hyperlink{class_beam_1_1_flux_node_a4bd30f3c8d311afdcd5c0d208e3bbf0f}{is\+Bypassed}} () const
\item 
void \doxymbox{\hyperlink{class_beam_1_1_flux_node_ad53f3fcaa5737f46d88530f40dbfbe32}{add\+Parameter}} (std::\+shared\+\_\+ptr$<$ \doxymbox{\hyperlink{class_beam_1_1_parameter}{Parameter}} $>$ \doxymbox{\hyperlink{texture_8cpp_aaded45152436a99bb4f9bda081df9f69}{param}})
\item 
std::\+shared\+\_\+ptr$<$ \doxymbox{\hyperlink{class_beam_1_1_parameter}{Parameter}} $>$ \doxymbox{\hyperlink{class_beam_1_1_flux_node_a59a32442eec144010741b9f2086c516e}{get\+Parameter}} (const std::\+string \&name)
\item 
const std::\+map$<$ std::\+string, std::\+shared\+\_\+ptr$<$ \doxymbox{\hyperlink{class_beam_1_1_parameter}{Parameter}} $>$ $>$ \& \doxymbox{\hyperlink{class_beam_1_1_flux_node_a6296c79b1ba77aa8b9526ace4a109529}{get\+Parameters}} () const
\end{DoxyCompactItemize}
\doxysubsubsection*{Private Attributes}
\begin{DoxyCompactItemize}
\item 
std::\+vector$<$ float $>$ \doxymbox{\hyperlink{class_beam_1_1_echo_plex_a010a781d2849a816b4e73419373f515c}{m\+\_\+buffer}}
\item 
size\+\_\+t \doxymbox{\hyperlink{class_beam_1_1_echo_plex_a7f4427735451334850832919c197106c}{m\+\_\+pos}} = 0
\item 
std::\+unique\+\_\+ptr$<$ \doxymbox{\hyperlink{class_beam_1_1_analog_base_1_1_wow_flutter_generator}{Analog\+Base::\+\+Wow\+Flutter\+Generator}} $>$ \doxymbox{\hyperlink{class_beam_1_1_echo_plex_a7375a64da87d05d8b4660f8e127e007c}{m\+\_\+wf}}
\end{DoxyCompactItemize}
\doxysubsubsection*{Additional Inherited Members}
\doxysubsection*{Protected Member Functions inherited from \doxymbox{\hyperlink{class_beam_1_1_flux_plugin}{Beam::\+\+Flux\+Plugin}}}
\begin{DoxyCompactItemize}
\item 
void \doxymbox{\hyperlink{class_beam_1_1_flux_plugin_a1768cc84018f8de19bcbf781e9b7ac3f}{add\+Param}} (const std::\+string \&name, float min, float max, float initial)
\item 
float \doxymbox{\hyperlink{class_beam_1_1_flux_plugin_a1b292b033caaaf9ec67154dfee2e577b}{get\+Param}} (const std::\+string \&name)
\item 
float \doxymbox{\hyperlink{class_beam_1_1_flux_plugin_a3e65f35944360e4e6ac370a967cf5eb3}{get\+Sample\+Rate}} () const
\end{DoxyCompactItemize}
\doxysubsection*{Protected Member Functions inherited from \doxymbox{\hyperlink{class_beam_1_1_flux_node}{Beam::\+\+Flux\+Node}}}
\begin{DoxyCompactItemize}
\item 
void \doxymbox{\hyperlink{class_beam_1_1_flux_node_ae3bafc1c5a1aa545167256172b3d3688}{setup\+Buffers}} (int num\+Inputs, int num\+Outputs, int buffer\+Size, int channels)
\begin{DoxyCompactList}\small\item\em Pre-\/allocates buffers for inputs and outputs. \end{DoxyCompactList}\end{DoxyCompactItemize}
\doxysubsection*{Protected Attributes inherited from \doxymbox{\hyperlink{class_beam_1_1_flux_node}{Beam::\+\+Flux\+Node}}}
\begin{DoxyCompactItemize}
\item 
std::\+vector$<$ std::\+vector$<$ float $>$ $>$ \doxymbox{\hyperlink{class_beam_1_1_flux_node_a8edab1c9ebd83e73bbfd92af29d6e92c}{m\+\_\+inputs}}
\item 
std::\+vector$<$ std::\+vector$<$ float $>$ $>$ \doxymbox{\hyperlink{class_beam_1_1_flux_node_a496905f0ff42c432eb38e19bd6135383}{m\+\_\+outputs}}
\item 
std::\+map$<$ std::\+string, std::\+shared\+\_\+ptr$<$ \doxymbox{\hyperlink{class_beam_1_1_parameter}{Parameter}} $>$ $>$ \doxymbox{\hyperlink{class_beam_1_1_flux_node_a65628a37cd2dd2832eda60e74ec1aed3}{m\+\_\+parameters}}
\item 
std::\+atomic$<$ bool $>$ \doxymbox{\hyperlink{class_beam_1_1_flux_node_a6116dcdcfa20998fe90dc75a74f25d9b}{m\+\_\+bypassed}} \{false\}
\item 
size\+\_\+t \doxymbox{\hyperlink{class_beam_1_1_flux_node_a7d8556ddb1482f997cda7749d737668b}{m\+\_\+current\+Frame}} = 0
\end{DoxyCompactItemize}


\label{doc-constructors}
\Hypertarget{class_beam_1_1_echo_plex_doc-constructors}
\doxysubsection{Constructor \& Destructor Documentation}
\Hypertarget{class_beam_1_1_echo_plex_ae58bd4f92e13a4aadaa591423ff2b0bb}\index{Beam::EchoPlex@{Beam::EchoPlex}!EchoPlex@{EchoPlex}}
\index{EchoPlex@{EchoPlex}!Beam::EchoPlex@{Beam::EchoPlex}}
\doxysubsubsection{\texorpdfstring{EchoPlex()}{EchoPlex()}}
{\footnotesize\ttfamily \label{class_beam_1_1_echo_plex_ae58bd4f92e13a4aadaa591423ff2b0bb} 
Beam::\+\+Echo\+Plex::\+\+Echo\+Plex (\begin{DoxyParamCaption}\item[{int}]{buf}{, }\item[{float}]{sr}{}\end{DoxyParamCaption})\hspace{0.3cm}{\ttfamily [inline]}}



\label{doc-func-members}
\Hypertarget{class_beam_1_1_echo_plex_doc-func-members}
\doxysubsection{Member Function Documentation}
\Hypertarget{class_beam_1_1_echo_plex_a26d16b6feed01873e169f83083873c61}\index{Beam::EchoPlex@{Beam::EchoPlex}!processBlock@{processBlock}}
\index{processBlock@{processBlock}!Beam::EchoPlex@{Beam::EchoPlex}}
\doxysubsubsection{\texorpdfstring{processBlock()}{processBlock()}}
{\footnotesize\ttfamily \label{class_beam_1_1_echo_plex_a26d16b6feed01873e169f83083873c61} 
void Beam::\+\+Echo\+Plex::\+process\+Block (\begin{DoxyParamCaption}\item[{const float \texorpdfstring{$\ast$}{*}}]{in}{, }\item[{float \texorpdfstring{$\ast$}{*}}]{out}{, }\item[{int}]{total}{}\end{DoxyParamCaption})\hspace{0.3cm}{\ttfamily [inline]}, {\ttfamily [override]}, {\ttfamily [virtual]}}



Implements \doxymbox{\hyperlink{class_beam_1_1_flux_plugin_ab10324716cec75feee93b6a3159c7912}{Beam::\+\+Flux\+Plugin}}.



\label{doc-variable-members}
\Hypertarget{class_beam_1_1_echo_plex_doc-variable-members}
\doxysubsection{Member Data Documentation}
\Hypertarget{class_beam_1_1_echo_plex_a010a781d2849a816b4e73419373f515c}\index{Beam::EchoPlex@{Beam::EchoPlex}!m\_buffer@{m\_buffer}}
\index{m\_buffer@{m\_buffer}!Beam::EchoPlex@{Beam::EchoPlex}}
\doxysubsubsection{\texorpdfstring{m\_buffer}{m\_buffer}}
{\footnotesize\ttfamily \label{class_beam_1_1_echo_plex_a010a781d2849a816b4e73419373f515c} 
std::\+vector$<$float$>$ Beam::\+\+Echo\+Plex::\+m\+\_\+buffer\hspace{0.3cm}{\ttfamily [private]}}

\Hypertarget{class_beam_1_1_echo_plex_a7f4427735451334850832919c197106c}\index{Beam::EchoPlex@{Beam::EchoPlex}!m\_pos@{m\_pos}}
\index{m\_pos@{m\_pos}!Beam::EchoPlex@{Beam::EchoPlex}}
\doxysubsubsection{\texorpdfstring{m\_pos}{m\_pos}}
{\footnotesize\ttfamily \label{class_beam_1_1_echo_plex_a7f4427735451334850832919c197106c} 
size\+\_\+t Beam::\+\+Echo\+Plex::\+m\+\_\+pos = 0\hspace{0.3cm}{\ttfamily [private]}}

\Hypertarget{class_beam_1_1_echo_plex_a7375a64da87d05d8b4660f8e127e007c}\index{Beam::EchoPlex@{Beam::EchoPlex}!m\_wf@{m\_wf}}
\index{m\_wf@{m\_wf}!Beam::EchoPlex@{Beam::EchoPlex}}
\doxysubsubsection{\texorpdfstring{m\_wf}{m\_wf}}
{\footnotesize\ttfamily \label{class_beam_1_1_echo_plex_a7375a64da87d05d8b4660f8e127e007c} 
std::\+unique\+\_\+ptr$<$\doxymbox{\hyperlink{class_beam_1_1_analog_base_1_1_wow_flutter_generator}{Analog\+Base::\+\+Wow\+Flutter\+Generator}}$>$ Beam::\+\+Echo\+Plex::\+m\+\_\+wf\hspace{0.3cm}{\ttfamily [private]}}



The documentation for this class was generated from the following file:\+\begin{DoxyCompactItemize}
\item 
src/\+engine/\+\doxymbox{\hyperlink{analog__suite_8hpp}{analog\+\_\+suite.\+hpp}}\end{DoxyCompactItemize}

\doxysection{Beam::\+FET76 Class Reference}
\hypertarget{class_beam_1_1_f_e_t76}{}\label{class_beam_1_1_f_e_t76}\index{Beam::FET76@{Beam::FET76}}


{\ttfamily \+\#include $<$analog\+\_\+suite.\+hpp$>$}

Inheritance diagram for Beam::\+FET76:\+\begin{figure}[H]
\begin{center}
\leavevmode
\includegraphics[height=3.000000cm]{class_beam_1_1_f_e_t76}
\end{center}
\end{figure}
\doxysubsubsection*{Public Member Functions}
\begin{DoxyCompactItemize}
\item 
\doxymbox{\hyperlink{class_beam_1_1_f_e_t76_a2d58a0266c55b0c99afd18d38ea52abf}{FET76}} (int buf, float sr)
\item 
void \doxymbox{\hyperlink{class_beam_1_1_f_e_t76_a592f2490ae49223add9dbdd1d7ce23d5}{process\+Block}} (const float \texorpdfstring{$\ast$}{*}in, float \texorpdfstring{$\ast$}{*}out, int total) override
\item 
float \doxymbox{\hyperlink{class_beam_1_1_f_e_t76_a9911a5d434ef90cbfe6e4630ef443883}{get\+Latest\+GR}} () const
\end{DoxyCompactItemize}
\doxysubsection*{Public Member Functions inherited from \doxymbox{\hyperlink{class_beam_1_1_flux_plugin}{Beam::\+\+Flux\+Plugin}}}
\begin{DoxyCompactItemize}
\item 
\doxymbox{\hyperlink{class_beam_1_1_flux_plugin_a9d91b56960799fdbfc4575e2fcfa6689}{Flux\+Plugin}} (const std::\+string \&name, int buffer\+Size, float sample\+Rate)
\item 
virtual void \doxymbox{\hyperlink{class_beam_1_1_flux_plugin_a87fb076475f20b062493efa7ca00e045}{process\+Events}} (const \doxymbox{\hyperlink{class_beam_1_1_m_i_d_i_buffer}{MIDIBuffer}} \&midi)
\begin{DoxyCompactList}\small\item\em Handle MIDI events in your plugin. \end{DoxyCompactList}\item 
void \doxymbox{\hyperlink{class_beam_1_1_flux_plugin_aa1f9c569002ec23eeb5db0af686abea7}{process\+MIDI}} (const \doxymbox{\hyperlink{class_beam_1_1_m_i_d_i_buffer}{MIDIBuffer}} \&midi) override
\begin{DoxyCompactList}\small\item\em Optional MIDI processing. Called before \doxylink{class_beam_1_1_flux_plugin_a181430e1cbf129891fe3ed72f3905a61}{process()} in the engine loop. \end{DoxyCompactList}\item 
void \doxymbox{\hyperlink{class_beam_1_1_flux_plugin_a181430e1cbf129891fe3ed72f3905a61}{process}} (int frames) override
\begin{DoxyCompactList}\small\item\em Main audio processing method. Must be implemented by subclasses. \end{DoxyCompactList}\item 
std::\+string \doxymbox{\hyperlink{class_beam_1_1_flux_plugin_a450563af4d65a25b8a8e896dab77a3c6}{get\+Name}} () const override
\item 
std::\+vector$<$ \doxymbox{\hyperlink{struct_beam_1_1_flux_node_1_1_port}{Port}} $>$ \doxymbox{\hyperlink{class_beam_1_1_flux_plugin_ad231db67f900e8e7dd853936ad2e866a}{get\+Input\+Ports}} () const override
\item 
std::\+vector$<$ \doxymbox{\hyperlink{struct_beam_1_1_flux_node_1_1_port}{Port}} $>$ \doxymbox{\hyperlink{class_beam_1_1_flux_plugin_a4ea312de74047e127e818a26e715d8bb}{get\+Output\+Ports}} () const override
\end{DoxyCompactItemize}
\doxysubsection*{Public Member Functions inherited from \doxymbox{\hyperlink{class_beam_1_1_flux_node}{Beam::\+\+Flux\+Node}}}
\begin{DoxyCompactItemize}
\item 
virtual \doxymbox{\hyperlink{class_beam_1_1_flux_node_a708c135cdb61e8838469998cd8a84e65}{\texorpdfstring{$\sim$}{\string~}\+Flux\+Node}} ()=default
\item 
virtual void \doxymbox{\hyperlink{class_beam_1_1_flux_node_ace8cc49479d8924d44bca5fd4cd955e2}{on\+Transport\+State\+Changed}} (bool playing)
\begin{DoxyCompactList}\small\item\em Responds to global transport changes (Play/\+\+Pause). \end{DoxyCompactList}\item 
virtual void \doxymbox{\hyperlink{class_beam_1_1_flux_node_adc7c4e979bf27de5bfca66815ae97a67}{on\+Transport\+Seek}} (size\+\_\+t frame)
\begin{DoxyCompactList}\small\item\em Responds to timeline seeking. \end{DoxyCompactList}\item 
void \doxymbox{\hyperlink{class_beam_1_1_flux_node_aa579ec06608fd776987bbb089f27fd94}{set\+Current\+Frame}} (size\+\_\+t frame)
\begin{DoxyCompactList}\small\item\em Sets the current playhead position for this block. \end{DoxyCompactList}\item 
float \texorpdfstring{$\ast$}{*} \doxymbox{\hyperlink{class_beam_1_1_flux_node_ac90bd1a05b5bed3d68978f532386ed29}{get\+Input\+Buffer}} (int port\+Idx)
\item 
float \texorpdfstring{$\ast$}{*} \doxymbox{\hyperlink{class_beam_1_1_flux_node_abf11cfd4f2346ee0cd46d4345f1ed7d4}{get\+Output\+Buffer}} (int port\+Idx)
\item 
void \doxymbox{\hyperlink{class_beam_1_1_flux_node_af37f8c1b6b825da2ce7e35011d6f8253}{set\+Bypass}} (bool bypass)
\item 
bool \doxymbox{\hyperlink{class_beam_1_1_flux_node_a4bd30f3c8d311afdcd5c0d208e3bbf0f}{is\+Bypassed}} () const
\item 
void \doxymbox{\hyperlink{class_beam_1_1_flux_node_ad53f3fcaa5737f46d88530f40dbfbe32}{add\+Parameter}} (std::\+shared\+\_\+ptr$<$ \doxymbox{\hyperlink{class_beam_1_1_parameter}{Parameter}} $>$ \doxymbox{\hyperlink{texture_8cpp_aaded45152436a99bb4f9bda081df9f69}{param}})
\item 
std::\+shared\+\_\+ptr$<$ \doxymbox{\hyperlink{class_beam_1_1_parameter}{Parameter}} $>$ \doxymbox{\hyperlink{class_beam_1_1_flux_node_a59a32442eec144010741b9f2086c516e}{get\+Parameter}} (const std::\+string \&name)
\item 
const std::\+map$<$ std::\+string, std::\+shared\+\_\+ptr$<$ \doxymbox{\hyperlink{class_beam_1_1_parameter}{Parameter}} $>$ $>$ \& \doxymbox{\hyperlink{class_beam_1_1_flux_node_a6296c79b1ba77aa8b9526ace4a109529}{get\+Parameters}} () const
\end{DoxyCompactItemize}
\doxysubsubsection*{Private Attributes}
\begin{DoxyCompactItemize}
\item 
float \doxymbox{\hyperlink{class_beam_1_1_f_e_t76_a5a84ba4199ec0119f70dd7564200cc92}{m\+\_\+envelope}}
\end{DoxyCompactItemize}
\doxysubsubsection*{Additional Inherited Members}
\doxysubsection*{Protected Member Functions inherited from \doxymbox{\hyperlink{class_beam_1_1_flux_plugin}{Beam::\+\+Flux\+Plugin}}}
\begin{DoxyCompactItemize}
\item 
void \doxymbox{\hyperlink{class_beam_1_1_flux_plugin_a1768cc84018f8de19bcbf781e9b7ac3f}{add\+Param}} (const std::\+string \&name, float min, float max, float initial)
\item 
float \doxymbox{\hyperlink{class_beam_1_1_flux_plugin_a1b292b033caaaf9ec67154dfee2e577b}{get\+Param}} (const std::\+string \&name)
\item 
float \doxymbox{\hyperlink{class_beam_1_1_flux_plugin_a3e65f35944360e4e6ac370a967cf5eb3}{get\+Sample\+Rate}} () const
\end{DoxyCompactItemize}
\doxysubsection*{Protected Member Functions inherited from \doxymbox{\hyperlink{class_beam_1_1_flux_node}{Beam::\+\+Flux\+Node}}}
\begin{DoxyCompactItemize}
\item 
void \doxymbox{\hyperlink{class_beam_1_1_flux_node_ae3bafc1c5a1aa545167256172b3d3688}{setup\+Buffers}} (int num\+Inputs, int num\+Outputs, int buffer\+Size, int channels)
\begin{DoxyCompactList}\small\item\em Pre-\/allocates buffers for inputs and outputs. \end{DoxyCompactList}\end{DoxyCompactItemize}
\doxysubsection*{Protected Attributes inherited from \doxymbox{\hyperlink{class_beam_1_1_flux_node}{Beam::\+\+Flux\+Node}}}
\begin{DoxyCompactItemize}
\item 
std::\+vector$<$ std::\+vector$<$ float $>$ $>$ \doxymbox{\hyperlink{class_beam_1_1_flux_node_a8edab1c9ebd83e73bbfd92af29d6e92c}{m\+\_\+inputs}}
\item 
std::\+vector$<$ std::\+vector$<$ float $>$ $>$ \doxymbox{\hyperlink{class_beam_1_1_flux_node_a496905f0ff42c432eb38e19bd6135383}{m\+\_\+outputs}}
\item 
std::\+map$<$ std::\+string, std::\+shared\+\_\+ptr$<$ \doxymbox{\hyperlink{class_beam_1_1_parameter}{Parameter}} $>$ $>$ \doxymbox{\hyperlink{class_beam_1_1_flux_node_a65628a37cd2dd2832eda60e74ec1aed3}{m\+\_\+parameters}}
\item 
std::\+atomic$<$ bool $>$ \doxymbox{\hyperlink{class_beam_1_1_flux_node_a6116dcdcfa20998fe90dc75a74f25d9b}{m\+\_\+bypassed}} \{false\}
\item 
size\+\_\+t \doxymbox{\hyperlink{class_beam_1_1_flux_node_a7d8556ddb1482f997cda7749d737668b}{m\+\_\+current\+Frame}} = 0
\end{DoxyCompactItemize}


\label{doc-constructors}
\Hypertarget{class_beam_1_1_f_e_t76_doc-constructors}
\doxysubsection{Constructor \& Destructor Documentation}
\Hypertarget{class_beam_1_1_f_e_t76_a2d58a0266c55b0c99afd18d38ea52abf}\index{Beam::FET76@{Beam::FET76}!FET76@{FET76}}
\index{FET76@{FET76}!Beam::FET76@{Beam::FET76}}
\doxysubsubsection{\texorpdfstring{FET76()}{FET76()}}
{\footnotesize\ttfamily \label{class_beam_1_1_f_e_t76_a2d58a0266c55b0c99afd18d38ea52abf} 
Beam::\+\+FET76::\+\+FET76 (\begin{DoxyParamCaption}\item[{int}]{buf}{, }\item[{float}]{sr}{}\end{DoxyParamCaption})\hspace{0.3cm}{\ttfamily [inline]}}



\label{doc-func-members}
\Hypertarget{class_beam_1_1_f_e_t76_doc-func-members}
\doxysubsection{Member Function Documentation}
\Hypertarget{class_beam_1_1_f_e_t76_a9911a5d434ef90cbfe6e4630ef443883}\index{Beam::FET76@{Beam::FET76}!getLatestGR@{getLatestGR}}
\index{getLatestGR@{getLatestGR}!Beam::FET76@{Beam::FET76}}
\doxysubsubsection{\texorpdfstring{getLatestGR()}{getLatestGR()}}
{\footnotesize\ttfamily \label{class_beam_1_1_f_e_t76_a9911a5d434ef90cbfe6e4630ef443883} 
float Beam::\+\+FET76::\+get\+Latest\+GR (\begin{DoxyParamCaption}{}{}\end{DoxyParamCaption}) const\hspace{0.3cm}{\ttfamily [inline]}}

\Hypertarget{class_beam_1_1_f_e_t76_a592f2490ae49223add9dbdd1d7ce23d5}\index{Beam::FET76@{Beam::FET76}!processBlock@{processBlock}}
\index{processBlock@{processBlock}!Beam::FET76@{Beam::FET76}}
\doxysubsubsection{\texorpdfstring{processBlock()}{processBlock()}}
{\footnotesize\ttfamily \label{class_beam_1_1_f_e_t76_a592f2490ae49223add9dbdd1d7ce23d5} 
void Beam::\+\+FET76::\+process\+Block (\begin{DoxyParamCaption}\item[{const float \texorpdfstring{$\ast$}{*}}]{in}{, }\item[{float \texorpdfstring{$\ast$}{*}}]{out}{, }\item[{int}]{total}{}\end{DoxyParamCaption})\hspace{0.3cm}{\ttfamily [inline]}, {\ttfamily [override]}, {\ttfamily [virtual]}}



Implements \doxymbox{\hyperlink{class_beam_1_1_flux_plugin_ab10324716cec75feee93b6a3159c7912}{Beam::\+\+Flux\+Plugin}}.



\label{doc-variable-members}
\Hypertarget{class_beam_1_1_f_e_t76_doc-variable-members}
\doxysubsection{Member Data Documentation}
\Hypertarget{class_beam_1_1_f_e_t76_a5a84ba4199ec0119f70dd7564200cc92}\index{Beam::FET76@{Beam::FET76}!m\_envelope@{m\_envelope}}
\index{m\_envelope@{m\_envelope}!Beam::FET76@{Beam::FET76}}
\doxysubsubsection{\texorpdfstring{m\_envelope}{m\_envelope}}
{\footnotesize\ttfamily \label{class_beam_1_1_f_e_t76_a5a84ba4199ec0119f70dd7564200cc92} 
float Beam::\+\+FET76::\+m\+\_\+envelope\hspace{0.3cm}{\ttfamily [private]}}



The documentation for this class was generated from the following file:\+\begin{DoxyCompactItemize}
\item 
src/\+engine/\+\doxymbox{\hyperlink{analog__suite_8hpp}{analog\+\_\+suite.\+hpp}}\end{DoxyCompactItemize}

\doxysection{Beam::\+Filter\+Graph Class Reference}
\hypertarget{class_beam_1_1_filter_graph}{}\label{class_beam_1_1_filter_graph}\index{Beam::FilterGraph@{Beam::FilterGraph}}


{\ttfamily \+\#include $<$filter\+\_\+graph.\+hpp$>$}

Inheritance diagram for Beam::\+Filter\+Graph:\+\begin{figure}[H]
\begin{center}
\leavevmode
\includegraphics[height=2.000000cm]{class_beam_1_1_filter_graph}
\end{center}
\end{figure}
\doxysubsubsection*{Public Member Functions}
\begin{DoxyCompactItemize}
\item 
\doxymbox{\hyperlink{class_beam_1_1_filter_graph_ad476b70c8e2e80e1bf4df64e042976b7}{Filter\+Graph}} (\doxymbox{\hyperlink{class_beam_1_1_biquad_filter_node}{Biquad\+Filter\+Node}} \texorpdfstring{$\ast$}{*}node)
\item 
void \doxymbox{\hyperlink{class_beam_1_1_filter_graph_ab12290c6fa0fe9df0bf65e4c09a623c9}{render}} (\doxymbox{\hyperlink{class_beam_1_1_quad_batcher}{Quad\+Batcher}} \&batcher, float dt, float screenW, float screenH) override
\end{DoxyCompactItemize}
\doxysubsection*{Public Member Functions inherited from \doxymbox{\hyperlink{class_beam_1_1_component}{Beam::\+\+Component}}}
\begin{DoxyCompactItemize}
\item 
virtual \doxymbox{\hyperlink{class_beam_1_1_component_af9d734d649978e027412a87bc54362cd}{\texorpdfstring{$\sim$}{\string~}\+Component}} ()=default
\item 
virtual void \doxymbox{\hyperlink{class_beam_1_1_component_ad3d3fb19d25b4371d07620567970a158}{update}} (float dt)
\item 
virtual bool \doxymbox{\hyperlink{class_beam_1_1_component_aec1da33d2d6e3d4e7dd6708309264e76}{on\+Mouse\+Down}} (float x, float y, int button)
\item 
virtual bool \doxymbox{\hyperlink{class_beam_1_1_component_ae36b8e9d70e8f9a1b9ba81c23c54d5c8}{on\+Mouse\+Up}} (float x, float y, int button)
\item 
virtual bool \doxymbox{\hyperlink{class_beam_1_1_component_a9d8e5970783d315044277a1228659e6c}{on\+Mouse\+Move}} (float x, float y)
\item 
virtual bool \doxymbox{\hyperlink{class_beam_1_1_component_ab92e884903f8a621fcd57bc00a24b041}{on\+Mouse\+Wheel}} (float x, float y, float delta)
\item 
virtual void \doxymbox{\hyperlink{class_beam_1_1_component_a6865b1f22388af467bf6c789120fac05}{set\+Bounds}} (float x, float y, float w, float h)
\item 
const \doxymbox{\hyperlink{struct_beam_1_1_rect}{Rect}} \& \doxymbox{\hyperlink{class_beam_1_1_component_a5746dbc69d5b0adb4cffbcf920936d00}{get\+Bounds}} () const
\item 
void \doxymbox{\hyperlink{class_beam_1_1_component_a00d4e2dfa7703e59d6486852321dbdf1}{set\+Draggable}} (bool draggable)
\item 
void \doxymbox{\hyperlink{class_beam_1_1_component_aca7b02d1dddf7cd20378db9e3242fb84}{start\+Dragging}} (float x, float y)
\end{DoxyCompactItemize}
\doxysubsubsection*{Private Attributes}
\begin{DoxyCompactItemize}
\item 
\doxymbox{\hyperlink{class_beam_1_1_biquad_filter_node}{Biquad\+Filter\+Node}} \texorpdfstring{$\ast$}{*} \doxymbox{\hyperlink{class_beam_1_1_filter_graph_a146771f5e8c7b1b9129c3bd884dde2d7}{m\+\_\+node}}
\end{DoxyCompactItemize}
\doxysubsubsection*{Additional Inherited Members}
\doxysubsection*{Protected Attributes inherited from \doxymbox{\hyperlink{class_beam_1_1_component}{Beam::\+\+Component}}}
\begin{DoxyCompactItemize}
\item 
\doxymbox{\hyperlink{struct_beam_1_1_rect}{Rect}} \doxymbox{\hyperlink{class_beam_1_1_component_a4f1ec4a5fb168c39a6c18f958b2b1495}{m\+\_\+bounds}} \{0, 0, 0, 0\}
\item 
bool \doxymbox{\hyperlink{class_beam_1_1_component_adc07913aed6ddadf1c730e7b3bb599cf}{m\+\_\+is\+Visible}} = true
\item 
bool \doxymbox{\hyperlink{class_beam_1_1_component_a0bf77b204ae374a14b5a6d7e5a3c13c6}{m\+\_\+is\+Enabled}} = true
\item 
bool \doxymbox{\hyperlink{class_beam_1_1_component_a9646efcaa9540a26a387f5da9aae4bde}{m\+\_\+is\+Draggable}} = false
\item 
bool \doxymbox{\hyperlink{class_beam_1_1_component_ab03af9a9743acf040f38e3fb11f8dc14}{m\+\_\+is\+Dragging}} = false
\item 
float \doxymbox{\hyperlink{class_beam_1_1_component_a7110b2b9dc235f724bf4689569266a63}{m\+\_\+last\+MouseX}} = 0
\item 
float \doxymbox{\hyperlink{class_beam_1_1_component_a768931a0f51394bf011f821f6ed2efe9}{m\+\_\+last\+MouseY}} = 0
\end{DoxyCompactItemize}


\label{doc-constructors}
\Hypertarget{class_beam_1_1_filter_graph_doc-constructors}
\doxysubsection{Constructor \& Destructor Documentation}
\Hypertarget{class_beam_1_1_filter_graph_ad476b70c8e2e80e1bf4df64e042976b7}\index{Beam::FilterGraph@{Beam::FilterGraph}!FilterGraph@{FilterGraph}}
\index{FilterGraph@{FilterGraph}!Beam::FilterGraph@{Beam::FilterGraph}}
\doxysubsubsection{\texorpdfstring{FilterGraph()}{FilterGraph()}}
{\footnotesize\ttfamily \label{class_beam_1_1_filter_graph_ad476b70c8e2e80e1bf4df64e042976b7} 
Beam::\+\+Filter\+Graph::\+\+Filter\+Graph (\begin{DoxyParamCaption}\item[{\doxymbox{\hyperlink{class_beam_1_1_biquad_filter_node}{Biquad\+Filter\+Node}} \texorpdfstring{$\ast$}{*}}]{node}{}\end{DoxyParamCaption})\hspace{0.3cm}{\ttfamily [inline]}}



\label{doc-func-members}
\Hypertarget{class_beam_1_1_filter_graph_doc-func-members}
\doxysubsection{Member Function Documentation}
\Hypertarget{class_beam_1_1_filter_graph_ab12290c6fa0fe9df0bf65e4c09a623c9}\index{Beam::FilterGraph@{Beam::FilterGraph}!render@{render}}
\index{render@{render}!Beam::FilterGraph@{Beam::FilterGraph}}
\doxysubsubsection{\texorpdfstring{render()}{render()}}
{\footnotesize\ttfamily \label{class_beam_1_1_filter_graph_ab12290c6fa0fe9df0bf65e4c09a623c9} 
void Beam::\+\+Filter\+Graph::\+render (\begin{DoxyParamCaption}\item[{\doxymbox{\hyperlink{class_beam_1_1_quad_batcher}{Quad\+Batcher}} \&}]{batcher}{, }\item[{float}]{dt}{, }\item[{float}]{screenW}{, }\item[{float}]{screenH}{}\end{DoxyParamCaption})\hspace{0.3cm}{\ttfamily [inline]}, {\ttfamily [override]}, {\ttfamily [virtual]}}



Implements \doxymbox{\hyperlink{class_beam_1_1_component_acef3496a55f0d94c8678f6049dbaa7cd}{Beam::\+\+Component}}.



\label{doc-variable-members}
\Hypertarget{class_beam_1_1_filter_graph_doc-variable-members}
\doxysubsection{Member Data Documentation}
\Hypertarget{class_beam_1_1_filter_graph_a146771f5e8c7b1b9129c3bd884dde2d7}\index{Beam::FilterGraph@{Beam::FilterGraph}!m\_node@{m\_node}}
\index{m\_node@{m\_node}!Beam::FilterGraph@{Beam::FilterGraph}}
\doxysubsubsection{\texorpdfstring{m\_node}{m\_node}}
{\footnotesize\ttfamily \label{class_beam_1_1_filter_graph_a146771f5e8c7b1b9129c3bd884dde2d7} 
\doxymbox{\hyperlink{class_beam_1_1_biquad_filter_node}{Biquad\+Filter\+Node}}\texorpdfstring{$\ast$}{*} Beam::\+\+Filter\+Graph::\+m\+\_\+node\hspace{0.3cm}{\ttfamily [private]}}



The documentation for this class was generated from the following file:\+\begin{DoxyCompactItemize}
\item 
src/\+interface/\+\doxymbox{\hyperlink{filter__graph_8hpp}{filter\+\_\+graph.\+hpp}}\end{DoxyCompactItemize}

\doxysection{Beam::\+Filter\+Module Class Reference}
\hypertarget{class_beam_1_1_filter_module}{}\label{class_beam_1_1_filter_module}\index{Beam::FilterModule@{Beam::FilterModule}}


{\ttfamily \+\#include $<$filter\+\_\+module.\+hpp$>$}

Inheritance diagram for Beam::\+Filter\+Module:\+\begin{figure}[H]
\begin{center}
\leavevmode
\includegraphics[height=3.000000cm]{class_beam_1_1_filter_module}
\end{center}
\end{figure}
\doxysubsubsection*{Public Member Functions}
\begin{DoxyCompactItemize}
\item 
\doxymbox{\hyperlink{class_beam_1_1_filter_module_a9299d6fc0134f9461e0a7ffae6ced3ed}{Filter\+Module}} (std::\+shared\+\_\+ptr$<$ \doxymbox{\hyperlink{class_beam_1_1_flux_filter_node}{Flux\+Filter\+Node}} $>$ node, size\+\_\+t node\+Id, float x, float y)
\item 
void \doxymbox{\hyperlink{class_beam_1_1_filter_module_a05cb71fbd7f6cd82181d4a2c81abc44a}{set\+Bounds}} (float x, float y, float w, float h) override
\end{DoxyCompactItemize}
\doxysubsection*{Public Member Functions inherited from \doxymbox{\hyperlink{class_beam_1_1_audio_module}{Beam::\+\+Audio\+Module}}}
\begin{DoxyCompactItemize}
\item 
\doxymbox{\hyperlink{class_beam_1_1_audio_module_a409c75189798146d7f556b1d50f4ba98}{Audio\+Module}} (std::\+shared\+\_\+ptr$<$ \doxymbox{\hyperlink{class_beam_1_1_flux_node}{Flux\+Node}} $>$ node, size\+\_\+t node\+Id, float x, float y)
\item 
size\+\_\+t \doxymbox{\hyperlink{class_beam_1_1_audio_module_a7fc32b3bfadec2badbfd067d2f37da96}{get\+Node\+Id}} () const
\item 
void \doxymbox{\hyperlink{class_beam_1_1_audio_module_a655aa14548b3d4e977294a9bdaafa879}{auto\+Generate\+UI}} ()
\item 
void \doxymbox{\hyperlink{class_beam_1_1_audio_module_a0e43bace4dcd9eb159ff2237b0c7c3fd}{render}} (\doxymbox{\hyperlink{class_beam_1_1_quad_batcher}{Quad\+Batcher}} \&batcher, float dt, float screenW, float screenH) override
\item 
bool \doxymbox{\hyperlink{class_beam_1_1_audio_module_adddaa58e40512d782c2d902917491499}{on\+Mouse\+Down}} (float x, float y, int button) override
\item 
bool \doxymbox{\hyperlink{class_beam_1_1_audio_module_a4ac204a52c61e603ae6a51daa610bc19}{on\+Mouse\+Up}} (float x, float y, int button) override
\item 
bool \doxymbox{\hyperlink{class_beam_1_1_audio_module_aff90c84092de907409b84eed2c46cbe2}{on\+Mouse\+Move}} (float x, float y) override
\item 
void \doxymbox{\hyperlink{class_beam_1_1_audio_module_a68a3d4bef3787a290f9a313b975f7925}{add\+Child}} (std::\+shared\+\_\+ptr$<$ \doxymbox{\hyperlink{class_beam_1_1_component}{Component}} $>$ child)
\item 
std::\+shared\+\_\+ptr$<$ \doxymbox{\hyperlink{class_beam_1_1_port}{Port}} $>$ \doxymbox{\hyperlink{class_beam_1_1_audio_module_a0d0da8bdbfb3a2355994bba086e0e721}{get\+Input\+Port}} ()
\item 
std::\+shared\+\_\+ptr$<$ \doxymbox{\hyperlink{class_beam_1_1_port}{Port}} $>$ \doxymbox{\hyperlink{class_beam_1_1_audio_module_a75e91aa2e7da1c9a6d31f4c90915403a}{get\+Output\+Port}} ()
\end{DoxyCompactItemize}
\doxysubsection*{Public Member Functions inherited from \doxymbox{\hyperlink{class_beam_1_1_component}{Beam::\+\+Component}}}
\begin{DoxyCompactItemize}
\item 
virtual \doxymbox{\hyperlink{class_beam_1_1_component_af9d734d649978e027412a87bc54362cd}{\texorpdfstring{$\sim$}{\string~}\+Component}} ()=default
\item 
virtual void \doxymbox{\hyperlink{class_beam_1_1_component_ad3d3fb19d25b4371d07620567970a158}{update}} (float dt)
\item 
virtual bool \doxymbox{\hyperlink{class_beam_1_1_component_ab92e884903f8a621fcd57bc00a24b041}{on\+Mouse\+Wheel}} (float x, float y, float delta)
\item 
const \doxymbox{\hyperlink{struct_beam_1_1_rect}{Rect}} \& \doxymbox{\hyperlink{class_beam_1_1_component_a5746dbc69d5b0adb4cffbcf920936d00}{get\+Bounds}} () const
\item 
void \doxymbox{\hyperlink{class_beam_1_1_component_a00d4e2dfa7703e59d6486852321dbdf1}{set\+Draggable}} (bool draggable)
\item 
void \doxymbox{\hyperlink{class_beam_1_1_component_aca7b02d1dddf7cd20378db9e3242fb84}{start\+Dragging}} (float x, float y)
\end{DoxyCompactItemize}
\doxysubsubsection*{Private Attributes}
\begin{DoxyCompactItemize}
\item 
std::\+shared\+\_\+ptr$<$ \doxymbox{\hyperlink{class_beam_1_1_flux_filter_node}{Flux\+Filter\+Node}} $>$ \doxymbox{\hyperlink{class_beam_1_1_filter_module_a080cc0de3b150eb1fd9bb8a2515a99d7}{m\+\_\+filter\+Node}}
\item 
std::\+shared\+\_\+ptr$<$ \doxymbox{\hyperlink{class_beam_1_1_filter_graph}{Filter\+Graph}} $>$ \doxymbox{\hyperlink{class_beam_1_1_filter_module_a38444bcbceb1e4bb742a1b99cbdcd0c1}{m\+\_\+graph}}
\end{DoxyCompactItemize}
\doxysubsubsection*{Additional Inherited Members}
\doxysubsection*{Public Attributes inherited from \doxymbox{\hyperlink{class_beam_1_1_audio_module}{Beam::\+\+Audio\+Module}}}
\begin{DoxyCompactItemize}
\item 
std::\+function$<$ void(\doxymbox{\hyperlink{class_beam_1_1_audio_module_a409c75189798146d7f556b1d50f4ba98}{Audio\+Module}} \texorpdfstring{$\ast$}{*})$>$ \doxymbox{\hyperlink{class_beam_1_1_audio_module_af95229e9824037c0c9916cc04bf67e90}{on\+Delete\+Requested}}
\end{DoxyCompactItemize}
\doxysubsection*{Protected Attributes inherited from \doxymbox{\hyperlink{class_beam_1_1_audio_module}{Beam::\+\+Audio\+Module}}}
\begin{DoxyCompactItemize}
\item 
std::\+vector$<$ std::\+shared\+\_\+ptr$<$ \doxymbox{\hyperlink{class_beam_1_1_component}{Component}} $>$ $>$ \doxymbox{\hyperlink{class_beam_1_1_audio_module_a1af200892e351f910e6447a50913a560}{m\+\_\+children}}
\end{DoxyCompactItemize}
\doxysubsection*{Protected Attributes inherited from \doxymbox{\hyperlink{class_beam_1_1_component}{Beam::\+\+Component}}}
\begin{DoxyCompactItemize}
\item 
\doxymbox{\hyperlink{struct_beam_1_1_rect}{Rect}} \doxymbox{\hyperlink{class_beam_1_1_component_a4f1ec4a5fb168c39a6c18f958b2b1495}{m\+\_\+bounds}} \{0, 0, 0, 0\}
\item 
bool \doxymbox{\hyperlink{class_beam_1_1_component_adc07913aed6ddadf1c730e7b3bb599cf}{m\+\_\+is\+Visible}} = true
\item 
bool \doxymbox{\hyperlink{class_beam_1_1_component_a0bf77b204ae374a14b5a6d7e5a3c13c6}{m\+\_\+is\+Enabled}} = true
\item 
bool \doxymbox{\hyperlink{class_beam_1_1_component_a9646efcaa9540a26a387f5da9aae4bde}{m\+\_\+is\+Draggable}} = false
\item 
bool \doxymbox{\hyperlink{class_beam_1_1_component_ab03af9a9743acf040f38e3fb11f8dc14}{m\+\_\+is\+Dragging}} = false
\item 
float \doxymbox{\hyperlink{class_beam_1_1_component_a7110b2b9dc235f724bf4689569266a63}{m\+\_\+last\+MouseX}} = 0
\item 
float \doxymbox{\hyperlink{class_beam_1_1_component_a768931a0f51394bf011f821f6ed2efe9}{m\+\_\+last\+MouseY}} = 0
\end{DoxyCompactItemize}


\label{doc-constructors}
\Hypertarget{class_beam_1_1_filter_module_doc-constructors}
\doxysubsection{Constructor \& Destructor Documentation}
\Hypertarget{class_beam_1_1_filter_module_a9299d6fc0134f9461e0a7ffae6ced3ed}\index{Beam::FilterModule@{Beam::FilterModule}!FilterModule@{FilterModule}}
\index{FilterModule@{FilterModule}!Beam::FilterModule@{Beam::FilterModule}}
\doxysubsubsection{\texorpdfstring{FilterModule()}{FilterModule()}}
{\footnotesize\ttfamily \label{class_beam_1_1_filter_module_a9299d6fc0134f9461e0a7ffae6ced3ed} 
Beam::\+\+Filter\+Module::\+\+Filter\+Module (\begin{DoxyParamCaption}\item[{std::\+shared\+\_\+ptr$<$ \doxymbox{\hyperlink{class_beam_1_1_flux_filter_node}{Flux\+Filter\+Node}} $>$}]{node}{, }\item[{size\+\_\+t}]{node\+Id}{, }\item[{float}]{x}{, }\item[{float}]{y}{}\end{DoxyParamCaption})\hspace{0.3cm}{\ttfamily [inline]}}



\label{doc-func-members}
\Hypertarget{class_beam_1_1_filter_module_doc-func-members}
\doxysubsection{Member Function Documentation}
\Hypertarget{class_beam_1_1_filter_module_a05cb71fbd7f6cd82181d4a2c81abc44a}\index{Beam::FilterModule@{Beam::FilterModule}!setBounds@{setBounds}}
\index{setBounds@{setBounds}!Beam::FilterModule@{Beam::FilterModule}}
\doxysubsubsection{\texorpdfstring{setBounds()}{setBounds()}}
{\footnotesize\ttfamily \label{class_beam_1_1_filter_module_a05cb71fbd7f6cd82181d4a2c81abc44a} 
void Beam::\+\+Filter\+Module::\+set\+Bounds (\begin{DoxyParamCaption}\item[{float}]{x}{, }\item[{float}]{y}{, }\item[{float}]{w}{, }\item[{float}]{h}{}\end{DoxyParamCaption})\hspace{0.3cm}{\ttfamily [inline]}, {\ttfamily [override]}, {\ttfamily [virtual]}}



Reimplemented from \doxymbox{\hyperlink{class_beam_1_1_audio_module_a75c0758091a1cec0871134babb541135}{Beam::\+\+Audio\+Module}}.



\label{doc-variable-members}
\Hypertarget{class_beam_1_1_filter_module_doc-variable-members}
\doxysubsection{Member Data Documentation}
\Hypertarget{class_beam_1_1_filter_module_a080cc0de3b150eb1fd9bb8a2515a99d7}\index{Beam::FilterModule@{Beam::FilterModule}!m\_filterNode@{m\_filterNode}}
\index{m\_filterNode@{m\_filterNode}!Beam::FilterModule@{Beam::FilterModule}}
\doxysubsubsection{\texorpdfstring{m\_filterNode}{m\_filterNode}}
{\footnotesize\ttfamily \label{class_beam_1_1_filter_module_a080cc0de3b150eb1fd9bb8a2515a99d7} 
std::\+shared\+\_\+ptr$<$\doxymbox{\hyperlink{class_beam_1_1_flux_filter_node}{Flux\+Filter\+Node}}$>$ Beam::\+\+Filter\+Module::\+m\+\_\+filter\+Node\hspace{0.3cm}{\ttfamily [private]}}

\Hypertarget{class_beam_1_1_filter_module_a38444bcbceb1e4bb742a1b99cbdcd0c1}\index{Beam::FilterModule@{Beam::FilterModule}!m\_graph@{m\_graph}}
\index{m\_graph@{m\_graph}!Beam::FilterModule@{Beam::FilterModule}}
\doxysubsubsection{\texorpdfstring{m\_graph}{m\_graph}}
{\footnotesize\ttfamily \label{class_beam_1_1_filter_module_a38444bcbceb1e4bb742a1b99cbdcd0c1} 
std::\+shared\+\_\+ptr$<$\doxymbox{\hyperlink{class_beam_1_1_filter_graph}{Filter\+Graph}}$>$ Beam::\+\+Filter\+Module::\+m\+\_\+graph\hspace{0.3cm}{\ttfamily [private]}}



The documentation for this class was generated from the following file:\+\begin{DoxyCompactItemize}
\item 
src/\+interface/\+\doxymbox{\hyperlink{filter__module_8hpp}{filter\+\_\+module.\+hpp}}\end{DoxyCompactItemize}

\doxysection{Beam::\+Flux\+Connection Struct Reference}
\hypertarget{struct_beam_1_1_flux_connection}{}\label{struct_beam_1_1_flux_connection}\index{Beam::FluxConnection@{Beam::FluxConnection}}


{\ttfamily \+\#include $<$flux\+\_\+graph.\+hpp$>$}

\doxysubsubsection*{Public Member Functions}
\begin{DoxyCompactItemize}
\item 
bool \doxymbox{\hyperlink{struct_beam_1_1_flux_connection_a377c2c2575b53b8d9eaf337b5f4161b0}{operator$<$}} (const \doxymbox{\hyperlink{struct_beam_1_1_flux_connection}{Flux\+Connection}} \&other) const
\end{DoxyCompactItemize}
\doxysubsubsection*{Public Attributes}
\begin{DoxyCompactItemize}
\item 
size\+\_\+t \doxymbox{\hyperlink{struct_beam_1_1_flux_connection_acebe798082c83291724607fa1214e2d2}{src\+Node\+Id}}
\item 
int \doxymbox{\hyperlink{struct_beam_1_1_flux_connection_a405e6d0d968693ff7a12ef40f9fd1bc2}{src\+Port\+Idx}}
\item 
size\+\_\+t \doxymbox{\hyperlink{struct_beam_1_1_flux_connection_a0f0e358a5e3ae4569afa328f726fe570}{dst\+Node\+Id}}
\item 
int \doxymbox{\hyperlink{struct_beam_1_1_flux_connection_a6fd533dee615e6cc409c0958a4a2f461}{dst\+Port\+Idx}}
\end{DoxyCompactItemize}


\label{doc-func-members}
\Hypertarget{struct_beam_1_1_flux_connection_doc-func-members}
\doxysubsection{Member Function Documentation}
\Hypertarget{struct_beam_1_1_flux_connection_a377c2c2575b53b8d9eaf337b5f4161b0}\index{Beam::FluxConnection@{Beam::FluxConnection}!operator$<$@{operator$<$}}
\index{operator$<$@{operator$<$}!Beam::FluxConnection@{Beam::FluxConnection}}
\doxysubsubsection{\texorpdfstring{operator$<$()}{operator<()}}
{\footnotesize\ttfamily \label{struct_beam_1_1_flux_connection_a377c2c2575b53b8d9eaf337b5f4161b0} 
bool Beam::\+\+Flux\+Connection::\+operator$<$ (\begin{DoxyParamCaption}\item[{const \doxymbox{\hyperlink{struct_beam_1_1_flux_connection}{Flux\+Connection}} \&}]{other}{}\end{DoxyParamCaption}) const\hspace{0.3cm}{\ttfamily [inline]}}



\label{doc-variable-members}
\Hypertarget{struct_beam_1_1_flux_connection_doc-variable-members}
\doxysubsection{Member Data Documentation}
\Hypertarget{struct_beam_1_1_flux_connection_a0f0e358a5e3ae4569afa328f726fe570}\index{Beam::FluxConnection@{Beam::FluxConnection}!dstNodeId@{dstNodeId}}
\index{dstNodeId@{dstNodeId}!Beam::FluxConnection@{Beam::FluxConnection}}
\doxysubsubsection{\texorpdfstring{dstNodeId}{dstNodeId}}
{\footnotesize\ttfamily \label{struct_beam_1_1_flux_connection_a0f0e358a5e3ae4569afa328f726fe570} 
size\+\_\+t Beam::\+\+Flux\+Connection::\+dst\+Node\+Id}

\Hypertarget{struct_beam_1_1_flux_connection_a6fd533dee615e6cc409c0958a4a2f461}\index{Beam::FluxConnection@{Beam::FluxConnection}!dstPortIdx@{dstPortIdx}}
\index{dstPortIdx@{dstPortIdx}!Beam::FluxConnection@{Beam::FluxConnection}}
\doxysubsubsection{\texorpdfstring{dstPortIdx}{dstPortIdx}}
{\footnotesize\ttfamily \label{struct_beam_1_1_flux_connection_a6fd533dee615e6cc409c0958a4a2f461} 
int Beam::\+\+Flux\+Connection::\+dst\+Port\+Idx}

\Hypertarget{struct_beam_1_1_flux_connection_acebe798082c83291724607fa1214e2d2}\index{Beam::FluxConnection@{Beam::FluxConnection}!srcNodeId@{srcNodeId}}
\index{srcNodeId@{srcNodeId}!Beam::FluxConnection@{Beam::FluxConnection}}
\doxysubsubsection{\texorpdfstring{srcNodeId}{srcNodeId}}
{\footnotesize\ttfamily \label{struct_beam_1_1_flux_connection_acebe798082c83291724607fa1214e2d2} 
size\+\_\+t Beam::\+\+Flux\+Connection::\+src\+Node\+Id}

\Hypertarget{struct_beam_1_1_flux_connection_a405e6d0d968693ff7a12ef40f9fd1bc2}\index{Beam::FluxConnection@{Beam::FluxConnection}!srcPortIdx@{srcPortIdx}}
\index{srcPortIdx@{srcPortIdx}!Beam::FluxConnection@{Beam::FluxConnection}}
\doxysubsubsection{\texorpdfstring{srcPortIdx}{srcPortIdx}}
{\footnotesize\ttfamily \label{struct_beam_1_1_flux_connection_a405e6d0d968693ff7a12ef40f9fd1bc2} 
int Beam::\+\+Flux\+Connection::\+src\+Port\+Idx}



The documentation for this struct was generated from the following file:\+\begin{DoxyCompactItemize}
\item 
src/\+engine/\+\doxymbox{\hyperlink{flux__graph_8hpp}{flux\+\_\+graph.\+hpp}}\end{DoxyCompactItemize}

\doxysection{Beam::\+Flux\+Delay\+Node Class Reference}
\hypertarget{class_beam_1_1_flux_delay_node}{}\label{class_beam_1_1_flux_delay_node}\index{Beam::FluxDelayNode@{Beam::FluxDelayNode}}


{\ttfamily \+\#include $<$flux\+\_\+fx\+\_\+nodes.\+hpp$>$}

Inheritance diagram for Beam::\+Flux\+Delay\+Node:\+\begin{figure}[H]
\begin{center}
\leavevmode
\includegraphics[height=2.000000cm]{class_beam_1_1_flux_delay_node}
\end{center}
\end{figure}
\doxysubsubsection*{Public Member Functions}
\begin{DoxyCompactItemize}
\item 
\doxymbox{\hyperlink{class_beam_1_1_flux_delay_node_ac25de6a616be31281a6303b24785125e}{Flux\+Delay\+Node}} (int buffer\+Size, float sample\+Rate)
\item 
void \doxymbox{\hyperlink{class_beam_1_1_flux_delay_node_a1ff56dfcb9d8d2780d855b5e545e2552}{process}} (int frames) override
\item 
std::\+string \doxymbox{\hyperlink{class_beam_1_1_flux_delay_node_a603094c5e37cdcca6213eb1f243520f7}{get\+Name}} () const override
\item 
std::\+vector$<$ \doxymbox{\hyperlink{struct_beam_1_1_flux_node_1_1_port}{Port}} $>$ \doxymbox{\hyperlink{class_beam_1_1_flux_delay_node_a2fe1caad4bba89af6aa8994d8bdd0130}{get\+Input\+Ports}} () const override
\item 
std::\+vector$<$ \doxymbox{\hyperlink{struct_beam_1_1_flux_node_1_1_port}{Port}} $>$ \doxymbox{\hyperlink{class_beam_1_1_flux_delay_node_aa50c030fa23cfd17e3bf82380e0aecf3}{get\+Output\+Ports}} () const override
\end{DoxyCompactItemize}
\doxysubsection*{Public Member Functions inherited from \doxymbox{\hyperlink{class_beam_1_1_flux_node}{Beam::\+\+Flux\+Node}}}
\begin{DoxyCompactItemize}
\item 
virtual \doxymbox{\hyperlink{class_beam_1_1_flux_node_a708c135cdb61e8838469998cd8a84e65}{\texorpdfstring{$\sim$}{\string~}\+Flux\+Node}} ()=default
\item 
float \texorpdfstring{$\ast$}{*} \doxymbox{\hyperlink{class_beam_1_1_flux_node_ac90bd1a05b5bed3d68978f532386ed29}{get\+Input\+Buffer}} (int port\+Idx)
\item 
float \texorpdfstring{$\ast$}{*} \doxymbox{\hyperlink{class_beam_1_1_flux_node_abf11cfd4f2346ee0cd46d4345f1ed7d4}{get\+Output\+Buffer}} (int port\+Idx)
\item 
void \doxymbox{\hyperlink{class_beam_1_1_flux_node_af37f8c1b6b825da2ce7e35011d6f8253}{set\+Bypass}} (bool bypass)
\item 
bool \doxymbox{\hyperlink{class_beam_1_1_flux_node_a4bd30f3c8d311afdcd5c0d208e3bbf0f}{is\+Bypassed}} () const
\item 
void \doxymbox{\hyperlink{class_beam_1_1_flux_node_ad53f3fcaa5737f46d88530f40dbfbe32}{add\+Parameter}} (std::\+shared\+\_\+ptr$<$ \doxymbox{\hyperlink{class_beam_1_1_parameter}{Parameter}} $>$ param)
\item 
std::\+shared\+\_\+ptr$<$ \doxymbox{\hyperlink{class_beam_1_1_parameter}{Parameter}} $>$ \doxymbox{\hyperlink{class_beam_1_1_flux_node_a59a32442eec144010741b9f2086c516e}{get\+Parameter}} (const std::\+string \&name)
\item 
const std::\+map$<$ std::\+string, std::\+shared\+\_\+ptr$<$ \doxymbox{\hyperlink{class_beam_1_1_parameter}{Parameter}} $>$ $>$ \& \doxymbox{\hyperlink{class_beam_1_1_flux_node_a6296c79b1ba77aa8b9526ace4a109529}{get\+Parameters}} () const
\end{DoxyCompactItemize}
\doxysubsubsection*{Private Attributes}
\begin{DoxyCompactItemize}
\item 
std::\+shared\+\_\+ptr$<$ \doxymbox{\hyperlink{class_beam_1_1_delay_node}{Delay\+Node}} $>$ \doxymbox{\hyperlink{class_beam_1_1_flux_delay_node_aba0f0712321df607e37fbcd4cb6c0387}{m\+\_\+delay}}
\end{DoxyCompactItemize}
\doxysubsubsection*{Additional Inherited Members}
\doxysubsection*{Protected Member Functions inherited from \doxymbox{\hyperlink{class_beam_1_1_flux_node}{Beam::\+\+Flux\+Node}}}
\begin{DoxyCompactItemize}
\item 
void \doxymbox{\hyperlink{class_beam_1_1_flux_node_ae3bafc1c5a1aa545167256172b3d3688}{setup\+Buffers}} (int num\+Inputs, int num\+Outputs, int buffer\+Size, int channels)
\end{DoxyCompactItemize}
\doxysubsection*{Protected Attributes inherited from \doxymbox{\hyperlink{class_beam_1_1_flux_node}{Beam::\+\+Flux\+Node}}}
\begin{DoxyCompactItemize}
\item 
std::\+vector$<$ std::\+vector$<$ float $>$ $>$ \doxymbox{\hyperlink{class_beam_1_1_flux_node_a8edab1c9ebd83e73bbfd92af29d6e92c}{m\+\_\+inputs}}
\item 
std::\+vector$<$ std::\+vector$<$ float $>$ $>$ \doxymbox{\hyperlink{class_beam_1_1_flux_node_a496905f0ff42c432eb38e19bd6135383}{m\+\_\+outputs}}
\item 
std::\+map$<$ std::\+string, std::\+shared\+\_\+ptr$<$ \doxymbox{\hyperlink{class_beam_1_1_parameter}{Parameter}} $>$ $>$ \doxymbox{\hyperlink{class_beam_1_1_flux_node_a65628a37cd2dd2832eda60e74ec1aed3}{m\+\_\+parameters}}
\item 
std::\+atomic$<$ bool $>$ \doxymbox{\hyperlink{class_beam_1_1_flux_node_a6116dcdcfa20998fe90dc75a74f25d9b}{m\+\_\+bypassed}} \{false\}
\end{DoxyCompactItemize}


\label{doc-constructors}
\Hypertarget{class_beam_1_1_flux_delay_node_doc-constructors}
\doxysubsection{Constructor \& Destructor Documentation}
\Hypertarget{class_beam_1_1_flux_delay_node_ac25de6a616be31281a6303b24785125e}\index{Beam::FluxDelayNode@{Beam::FluxDelayNode}!FluxDelayNode@{FluxDelayNode}}
\index{FluxDelayNode@{FluxDelayNode}!Beam::FluxDelayNode@{Beam::FluxDelayNode}}
\doxysubsubsection{\texorpdfstring{FluxDelayNode()}{FluxDelayNode()}}
{\footnotesize\ttfamily \label{class_beam_1_1_flux_delay_node_ac25de6a616be31281a6303b24785125e} 
Beam::\+\+Flux\+Delay\+Node::\+\+Flux\+Delay\+Node (\begin{DoxyParamCaption}\item[{int}]{buffer\+Size}{, }\item[{float}]{sample\+Rate}{}\end{DoxyParamCaption})\hspace{0.3cm}{\ttfamily [inline]}}



\label{doc-func-members}
\Hypertarget{class_beam_1_1_flux_delay_node_doc-func-members}
\doxysubsection{Member Function Documentation}
\Hypertarget{class_beam_1_1_flux_delay_node_a2fe1caad4bba89af6aa8994d8bdd0130}\index{Beam::FluxDelayNode@{Beam::FluxDelayNode}!getInputPorts@{getInputPorts}}
\index{getInputPorts@{getInputPorts}!Beam::FluxDelayNode@{Beam::FluxDelayNode}}
\doxysubsubsection{\texorpdfstring{getInputPorts()}{getInputPorts()}}
{\footnotesize\ttfamily \label{class_beam_1_1_flux_delay_node_a2fe1caad4bba89af6aa8994d8bdd0130} 
std::\+vector$<$ \doxymbox{\hyperlink{struct_beam_1_1_flux_node_1_1_port}{Port}} $>$ Beam::\+\+Flux\+Delay\+Node::\+get\+Input\+Ports (\begin{DoxyParamCaption}{}{}\end{DoxyParamCaption}) const\hspace{0.3cm}{\ttfamily [inline]}, {\ttfamily [override]}, {\ttfamily [virtual]}}



Implements \doxymbox{\hyperlink{class_beam_1_1_flux_node_a17eb02187925b52bf8e53fa3ebe3da66}{Beam::\+\+Flux\+Node}}.

\Hypertarget{class_beam_1_1_flux_delay_node_a603094c5e37cdcca6213eb1f243520f7}\index{Beam::FluxDelayNode@{Beam::FluxDelayNode}!getName@{getName}}
\index{getName@{getName}!Beam::FluxDelayNode@{Beam::FluxDelayNode}}
\doxysubsubsection{\texorpdfstring{getName()}{getName()}}
{\footnotesize\ttfamily \label{class_beam_1_1_flux_delay_node_a603094c5e37cdcca6213eb1f243520f7} 
std::\+string Beam::\+\+Flux\+Delay\+Node::\+get\+Name (\begin{DoxyParamCaption}{}{}\end{DoxyParamCaption}) const\hspace{0.3cm}{\ttfamily [inline]}, {\ttfamily [override]}, {\ttfamily [virtual]}}



Implements \doxymbox{\hyperlink{class_beam_1_1_flux_node_ac638d3d9bb1050d658294bc5470abeba}{Beam::\+\+Flux\+Node}}.

\Hypertarget{class_beam_1_1_flux_delay_node_aa50c030fa23cfd17e3bf82380e0aecf3}\index{Beam::FluxDelayNode@{Beam::FluxDelayNode}!getOutputPorts@{getOutputPorts}}
\index{getOutputPorts@{getOutputPorts}!Beam::FluxDelayNode@{Beam::FluxDelayNode}}
\doxysubsubsection{\texorpdfstring{getOutputPorts()}{getOutputPorts()}}
{\footnotesize\ttfamily \label{class_beam_1_1_flux_delay_node_aa50c030fa23cfd17e3bf82380e0aecf3} 
std::\+vector$<$ \doxymbox{\hyperlink{struct_beam_1_1_flux_node_1_1_port}{Port}} $>$ Beam::\+\+Flux\+Delay\+Node::\+get\+Output\+Ports (\begin{DoxyParamCaption}{}{}\end{DoxyParamCaption}) const\hspace{0.3cm}{\ttfamily [inline]}, {\ttfamily [override]}, {\ttfamily [virtual]}}



Implements \doxymbox{\hyperlink{class_beam_1_1_flux_node_a034f59d236afd7901ed84090422e3279}{Beam::\+\+Flux\+Node}}.

\Hypertarget{class_beam_1_1_flux_delay_node_a1ff56dfcb9d8d2780d855b5e545e2552}\index{Beam::FluxDelayNode@{Beam::FluxDelayNode}!process@{process}}
\index{process@{process}!Beam::FluxDelayNode@{Beam::FluxDelayNode}}
\doxysubsubsection{\texorpdfstring{process()}{process()}}
{\footnotesize\ttfamily \label{class_beam_1_1_flux_delay_node_a1ff56dfcb9d8d2780d855b5e545e2552} 
void Beam::\+\+Flux\+Delay\+Node::\+process (\begin{DoxyParamCaption}\item[{int}]{frames}{}\end{DoxyParamCaption})\hspace{0.3cm}{\ttfamily [inline]}, {\ttfamily [override]}, {\ttfamily [virtual]}}



Implements \doxymbox{\hyperlink{class_beam_1_1_flux_node_a3c263446753fa7ae5ff6928ee57bcd4d}{Beam::\+\+Flux\+Node}}.



\label{doc-variable-members}
\Hypertarget{class_beam_1_1_flux_delay_node_doc-variable-members}
\doxysubsection{Member Data Documentation}
\Hypertarget{class_beam_1_1_flux_delay_node_aba0f0712321df607e37fbcd4cb6c0387}\index{Beam::FluxDelayNode@{Beam::FluxDelayNode}!m\_delay@{m\_delay}}
\index{m\_delay@{m\_delay}!Beam::FluxDelayNode@{Beam::FluxDelayNode}}
\doxysubsubsection{\texorpdfstring{m\_delay}{m\_delay}}
{\footnotesize\ttfamily \label{class_beam_1_1_flux_delay_node_aba0f0712321df607e37fbcd4cb6c0387} 
std::\+shared\+\_\+ptr$<$\doxymbox{\hyperlink{class_beam_1_1_delay_node}{Delay\+Node}}$>$ Beam::\+\+Flux\+Delay\+Node::\+m\+\_\+delay\hspace{0.3cm}{\ttfamily [private]}}



The documentation for this class was generated from the following file:\+\begin{DoxyCompactItemize}
\item 
src/\+dsp/\+\doxymbox{\hyperlink{flux__fx__nodes_8hpp}{flux\+\_\+fx\+\_\+nodes.\+hpp}}\end{DoxyCompactItemize}

\doxysection{Beam::\+Flux\+Filter\+Node Class Reference}
\hypertarget{class_beam_1_1_flux_filter_node}{}\label{class_beam_1_1_flux_filter_node}\index{Beam::FluxFilterNode@{Beam::FluxFilterNode}}


{\ttfamily \+\#include $<$flux\+\_\+fx\+\_\+nodes.\+hpp$>$}

Inheritance diagram for Beam::\+Flux\+Filter\+Node:\+\begin{figure}[H]
\begin{center}
\leavevmode
\includegraphics[height=2.000000cm]{class_beam_1_1_flux_filter_node}
\end{center}
\end{figure}
\doxysubsubsection*{Public Member Functions}
\begin{DoxyCompactItemize}
\item 
\doxymbox{\hyperlink{class_beam_1_1_flux_filter_node_a34beb77d37412cbaddefb3de6efe6247}{Flux\+Filter\+Node}} (int buffer\+Size, float sample\+Rate)
\item 
void \doxymbox{\hyperlink{class_beam_1_1_flux_filter_node_aa06a586b7ba994ed77ebfceff2570549}{process}} (int frames) override
\item 
std::\+string \doxymbox{\hyperlink{class_beam_1_1_flux_filter_node_ac4c3036266fce8b459991ea133e04263}{get\+Name}} () const override
\item 
std::\+vector$<$ \doxymbox{\hyperlink{struct_beam_1_1_flux_node_1_1_port}{Port}} $>$ \doxymbox{\hyperlink{class_beam_1_1_flux_filter_node_a05a526abb52eb8a88f79c219f7155ea8}{get\+Input\+Ports}} () const override
\item 
std::\+vector$<$ \doxymbox{\hyperlink{struct_beam_1_1_flux_node_1_1_port}{Port}} $>$ \doxymbox{\hyperlink{class_beam_1_1_flux_filter_node_a5fbda5bb2db19d441a0272fa8831aebe}{get\+Output\+Ports}} () const override
\end{DoxyCompactItemize}
\doxysubsection*{Public Member Functions inherited from \doxymbox{\hyperlink{class_beam_1_1_flux_node}{Beam::\+\+Flux\+Node}}}
\begin{DoxyCompactItemize}
\item 
virtual \doxymbox{\hyperlink{class_beam_1_1_flux_node_a708c135cdb61e8838469998cd8a84e65}{\texorpdfstring{$\sim$}{\string~}\+Flux\+Node}} ()=default
\item 
float \texorpdfstring{$\ast$}{*} \doxymbox{\hyperlink{class_beam_1_1_flux_node_ac90bd1a05b5bed3d68978f532386ed29}{get\+Input\+Buffer}} (int port\+Idx)
\item 
float \texorpdfstring{$\ast$}{*} \doxymbox{\hyperlink{class_beam_1_1_flux_node_abf11cfd4f2346ee0cd46d4345f1ed7d4}{get\+Output\+Buffer}} (int port\+Idx)
\item 
void \doxymbox{\hyperlink{class_beam_1_1_flux_node_af37f8c1b6b825da2ce7e35011d6f8253}{set\+Bypass}} (bool bypass)
\item 
bool \doxymbox{\hyperlink{class_beam_1_1_flux_node_a4bd30f3c8d311afdcd5c0d208e3bbf0f}{is\+Bypassed}} () const
\item 
void \doxymbox{\hyperlink{class_beam_1_1_flux_node_ad53f3fcaa5737f46d88530f40dbfbe32}{add\+Parameter}} (std::\+shared\+\_\+ptr$<$ \doxymbox{\hyperlink{class_beam_1_1_parameter}{Parameter}} $>$ param)
\item 
std::\+shared\+\_\+ptr$<$ \doxymbox{\hyperlink{class_beam_1_1_parameter}{Parameter}} $>$ \doxymbox{\hyperlink{class_beam_1_1_flux_node_a59a32442eec144010741b9f2086c516e}{get\+Parameter}} (const std::\+string \&name)
\item 
const std::\+map$<$ std::\+string, std::\+shared\+\_\+ptr$<$ \doxymbox{\hyperlink{class_beam_1_1_parameter}{Parameter}} $>$ $>$ \& \doxymbox{\hyperlink{class_beam_1_1_flux_node_a6296c79b1ba77aa8b9526ace4a109529}{get\+Parameters}} () const
\end{DoxyCompactItemize}
\doxysubsubsection*{Private Attributes}
\begin{DoxyCompactItemize}
\item 
std::\+shared\+\_\+ptr$<$ \doxymbox{\hyperlink{class_beam_1_1_biquad_filter_node}{Biquad\+Filter\+Node}} $>$ \doxymbox{\hyperlink{class_beam_1_1_flux_filter_node_aa75609be1d12c53002edf8ed1cbf6b6a}{m\+\_\+filter}}
\end{DoxyCompactItemize}
\doxysubsubsection*{Additional Inherited Members}
\doxysubsection*{Protected Member Functions inherited from \doxymbox{\hyperlink{class_beam_1_1_flux_node}{Beam::\+\+Flux\+Node}}}
\begin{DoxyCompactItemize}
\item 
void \doxymbox{\hyperlink{class_beam_1_1_flux_node_ae3bafc1c5a1aa545167256172b3d3688}{setup\+Buffers}} (int num\+Inputs, int num\+Outputs, int buffer\+Size, int channels)
\end{DoxyCompactItemize}
\doxysubsection*{Protected Attributes inherited from \doxymbox{\hyperlink{class_beam_1_1_flux_node}{Beam::\+\+Flux\+Node}}}
\begin{DoxyCompactItemize}
\item 
std::\+vector$<$ std::\+vector$<$ float $>$ $>$ \doxymbox{\hyperlink{class_beam_1_1_flux_node_a8edab1c9ebd83e73bbfd92af29d6e92c}{m\+\_\+inputs}}
\item 
std::\+vector$<$ std::\+vector$<$ float $>$ $>$ \doxymbox{\hyperlink{class_beam_1_1_flux_node_a496905f0ff42c432eb38e19bd6135383}{m\+\_\+outputs}}
\item 
std::\+map$<$ std::\+string, std::\+shared\+\_\+ptr$<$ \doxymbox{\hyperlink{class_beam_1_1_parameter}{Parameter}} $>$ $>$ \doxymbox{\hyperlink{class_beam_1_1_flux_node_a65628a37cd2dd2832eda60e74ec1aed3}{m\+\_\+parameters}}
\item 
std::\+atomic$<$ bool $>$ \doxymbox{\hyperlink{class_beam_1_1_flux_node_a6116dcdcfa20998fe90dc75a74f25d9b}{m\+\_\+bypassed}} \{false\}
\end{DoxyCompactItemize}


\label{doc-constructors}
\Hypertarget{class_beam_1_1_flux_filter_node_doc-constructors}
\doxysubsection{Constructor \& Destructor Documentation}
\Hypertarget{class_beam_1_1_flux_filter_node_a34beb77d37412cbaddefb3de6efe6247}\index{Beam::FluxFilterNode@{Beam::FluxFilterNode}!FluxFilterNode@{FluxFilterNode}}
\index{FluxFilterNode@{FluxFilterNode}!Beam::FluxFilterNode@{Beam::FluxFilterNode}}
\doxysubsubsection{\texorpdfstring{FluxFilterNode()}{FluxFilterNode()}}
{\footnotesize\ttfamily \label{class_beam_1_1_flux_filter_node_a34beb77d37412cbaddefb3de6efe6247} 
Beam::\+\+Flux\+Filter\+Node::\+\+Flux\+Filter\+Node (\begin{DoxyParamCaption}\item[{int}]{buffer\+Size}{, }\item[{float}]{sample\+Rate}{}\end{DoxyParamCaption})\hspace{0.3cm}{\ttfamily [inline]}}



\label{doc-func-members}
\Hypertarget{class_beam_1_1_flux_filter_node_doc-func-members}
\doxysubsection{Member Function Documentation}
\Hypertarget{class_beam_1_1_flux_filter_node_a05a526abb52eb8a88f79c219f7155ea8}\index{Beam::FluxFilterNode@{Beam::FluxFilterNode}!getInputPorts@{getInputPorts}}
\index{getInputPorts@{getInputPorts}!Beam::FluxFilterNode@{Beam::FluxFilterNode}}
\doxysubsubsection{\texorpdfstring{getInputPorts()}{getInputPorts()}}
{\footnotesize\ttfamily \label{class_beam_1_1_flux_filter_node_a05a526abb52eb8a88f79c219f7155ea8} 
std::\+vector$<$ \doxymbox{\hyperlink{struct_beam_1_1_flux_node_1_1_port}{Port}} $>$ Beam::\+\+Flux\+Filter\+Node::\+get\+Input\+Ports (\begin{DoxyParamCaption}{}{}\end{DoxyParamCaption}) const\hspace{0.3cm}{\ttfamily [inline]}, {\ttfamily [override]}, {\ttfamily [virtual]}}



Implements \doxymbox{\hyperlink{class_beam_1_1_flux_node_a17eb02187925b52bf8e53fa3ebe3da66}{Beam::\+\+Flux\+Node}}.

\Hypertarget{class_beam_1_1_flux_filter_node_ac4c3036266fce8b459991ea133e04263}\index{Beam::FluxFilterNode@{Beam::FluxFilterNode}!getName@{getName}}
\index{getName@{getName}!Beam::FluxFilterNode@{Beam::FluxFilterNode}}
\doxysubsubsection{\texorpdfstring{getName()}{getName()}}
{\footnotesize\ttfamily \label{class_beam_1_1_flux_filter_node_ac4c3036266fce8b459991ea133e04263} 
std::\+string Beam::\+\+Flux\+Filter\+Node::\+get\+Name (\begin{DoxyParamCaption}{}{}\end{DoxyParamCaption}) const\hspace{0.3cm}{\ttfamily [inline]}, {\ttfamily [override]}, {\ttfamily [virtual]}}



Implements \doxymbox{\hyperlink{class_beam_1_1_flux_node_ac638d3d9bb1050d658294bc5470abeba}{Beam::\+\+Flux\+Node}}.

\Hypertarget{class_beam_1_1_flux_filter_node_a5fbda5bb2db19d441a0272fa8831aebe}\index{Beam::FluxFilterNode@{Beam::FluxFilterNode}!getOutputPorts@{getOutputPorts}}
\index{getOutputPorts@{getOutputPorts}!Beam::FluxFilterNode@{Beam::FluxFilterNode}}
\doxysubsubsection{\texorpdfstring{getOutputPorts()}{getOutputPorts()}}
{\footnotesize\ttfamily \label{class_beam_1_1_flux_filter_node_a5fbda5bb2db19d441a0272fa8831aebe} 
std::\+vector$<$ \doxymbox{\hyperlink{struct_beam_1_1_flux_node_1_1_port}{Port}} $>$ Beam::\+\+Flux\+Filter\+Node::\+get\+Output\+Ports (\begin{DoxyParamCaption}{}{}\end{DoxyParamCaption}) const\hspace{0.3cm}{\ttfamily [inline]}, {\ttfamily [override]}, {\ttfamily [virtual]}}



Implements \doxymbox{\hyperlink{class_beam_1_1_flux_node_a034f59d236afd7901ed84090422e3279}{Beam::\+\+Flux\+Node}}.

\Hypertarget{class_beam_1_1_flux_filter_node_aa06a586b7ba994ed77ebfceff2570549}\index{Beam::FluxFilterNode@{Beam::FluxFilterNode}!process@{process}}
\index{process@{process}!Beam::FluxFilterNode@{Beam::FluxFilterNode}}
\doxysubsubsection{\texorpdfstring{process()}{process()}}
{\footnotesize\ttfamily \label{class_beam_1_1_flux_filter_node_aa06a586b7ba994ed77ebfceff2570549} 
void Beam::\+\+Flux\+Filter\+Node::\+process (\begin{DoxyParamCaption}\item[{int}]{frames}{}\end{DoxyParamCaption})\hspace{0.3cm}{\ttfamily [inline]}, {\ttfamily [override]}, {\ttfamily [virtual]}}



Implements \doxymbox{\hyperlink{class_beam_1_1_flux_node_a3c263446753fa7ae5ff6928ee57bcd4d}{Beam::\+\+Flux\+Node}}.



\label{doc-variable-members}
\Hypertarget{class_beam_1_1_flux_filter_node_doc-variable-members}
\doxysubsection{Member Data Documentation}
\Hypertarget{class_beam_1_1_flux_filter_node_aa75609be1d12c53002edf8ed1cbf6b6a}\index{Beam::FluxFilterNode@{Beam::FluxFilterNode}!m\_filter@{m\_filter}}
\index{m\_filter@{m\_filter}!Beam::FluxFilterNode@{Beam::FluxFilterNode}}
\doxysubsubsection{\texorpdfstring{m\_filter}{m\_filter}}
{\footnotesize\ttfamily \label{class_beam_1_1_flux_filter_node_aa75609be1d12c53002edf8ed1cbf6b6a} 
std::\+shared\+\_\+ptr$<$\doxymbox{\hyperlink{class_beam_1_1_biquad_filter_node}{Biquad\+Filter\+Node}}$>$ Beam::\+\+Flux\+Filter\+Node::\+m\+\_\+filter\hspace{0.3cm}{\ttfamily [private]}}



The documentation for this class was generated from the following file:\+\begin{DoxyCompactItemize}
\item 
src/\+dsp/\+\doxymbox{\hyperlink{flux__fx__nodes_8hpp}{flux\+\_\+fx\+\_\+nodes.\+hpp}}\end{DoxyCompactItemize}

\doxysection{Beam::\+Flux\+Gain\+Node Class Reference}
\hypertarget{class_beam_1_1_flux_gain_node}{}\label{class_beam_1_1_flux_gain_node}\index{Beam::FluxGainNode@{Beam::FluxGainNode}}


{\ttfamily \+\#include $<$flux\+\_\+fx\+\_\+nodes.\+hpp$>$}

Inheritance diagram for Beam::\+Flux\+Gain\+Node:\+\begin{figure}[H]
\begin{center}
\leavevmode
\includegraphics[height=2.000000cm]{class_beam_1_1_flux_gain_node}
\end{center}
\end{figure}
\doxysubsubsection*{Public Member Functions}
\begin{DoxyCompactItemize}
\item 
\doxymbox{\hyperlink{class_beam_1_1_flux_gain_node_a9141fd7342b8411b66bf1ddf62111df6}{Flux\+Gain\+Node}} (int buffer\+Size)
\item 
void \doxymbox{\hyperlink{class_beam_1_1_flux_gain_node_adeb55f996d1f48c22b674d62c27f2b8e}{process}} (int frames) override
\item 
std::\+string \doxymbox{\hyperlink{class_beam_1_1_flux_gain_node_a628da5d08479cdcfdbf4965f5b7339f2}{get\+Name}} () const override
\item 
std::\+vector$<$ \doxymbox{\hyperlink{struct_beam_1_1_flux_node_1_1_port}{Port}} $>$ \doxymbox{\hyperlink{class_beam_1_1_flux_gain_node_a8233991523774e343143415f5d314ca4}{get\+Input\+Ports}} () const override
\item 
std::\+vector$<$ \doxymbox{\hyperlink{struct_beam_1_1_flux_node_1_1_port}{Port}} $>$ \doxymbox{\hyperlink{class_beam_1_1_flux_gain_node_a8c69847df47b060cd8027c4e85f74a8f}{get\+Output\+Ports}} () const override
\end{DoxyCompactItemize}
\doxysubsection*{Public Member Functions inherited from \doxymbox{\hyperlink{class_beam_1_1_flux_node}{Beam::\+\+Flux\+Node}}}
\begin{DoxyCompactItemize}
\item 
virtual \doxymbox{\hyperlink{class_beam_1_1_flux_node_a708c135cdb61e8838469998cd8a84e65}{\texorpdfstring{$\sim$}{\string~}\+Flux\+Node}} ()=default
\item 
float \texorpdfstring{$\ast$}{*} \doxymbox{\hyperlink{class_beam_1_1_flux_node_ac90bd1a05b5bed3d68978f532386ed29}{get\+Input\+Buffer}} (int port\+Idx)
\item 
float \texorpdfstring{$\ast$}{*} \doxymbox{\hyperlink{class_beam_1_1_flux_node_abf11cfd4f2346ee0cd46d4345f1ed7d4}{get\+Output\+Buffer}} (int port\+Idx)
\item 
void \doxymbox{\hyperlink{class_beam_1_1_flux_node_af37f8c1b6b825da2ce7e35011d6f8253}{set\+Bypass}} (bool bypass)
\item 
bool \doxymbox{\hyperlink{class_beam_1_1_flux_node_a4bd30f3c8d311afdcd5c0d208e3bbf0f}{is\+Bypassed}} () const
\item 
void \doxymbox{\hyperlink{class_beam_1_1_flux_node_ad53f3fcaa5737f46d88530f40dbfbe32}{add\+Parameter}} (std::\+shared\+\_\+ptr$<$ \doxymbox{\hyperlink{class_beam_1_1_parameter}{Parameter}} $>$ param)
\item 
std::\+shared\+\_\+ptr$<$ \doxymbox{\hyperlink{class_beam_1_1_parameter}{Parameter}} $>$ \doxymbox{\hyperlink{class_beam_1_1_flux_node_a59a32442eec144010741b9f2086c516e}{get\+Parameter}} (const std::\+string \&name)
\item 
const std::\+map$<$ std::\+string, std::\+shared\+\_\+ptr$<$ \doxymbox{\hyperlink{class_beam_1_1_parameter}{Parameter}} $>$ $>$ \& \doxymbox{\hyperlink{class_beam_1_1_flux_node_a6296c79b1ba77aa8b9526ace4a109529}{get\+Parameters}} () const
\end{DoxyCompactItemize}
\doxysubsubsection*{Private Attributes}
\begin{DoxyCompactItemize}
\item 
std::\+shared\+\_\+ptr$<$ \doxymbox{\hyperlink{class_beam_1_1_gain_node}{Gain\+Node}} $>$ \doxymbox{\hyperlink{class_beam_1_1_flux_gain_node_a67388972db423ad08f7230c627654f6b}{m\+\_\+gain}}
\end{DoxyCompactItemize}
\doxysubsubsection*{Additional Inherited Members}
\doxysubsection*{Protected Member Functions inherited from \doxymbox{\hyperlink{class_beam_1_1_flux_node}{Beam::\+\+Flux\+Node}}}
\begin{DoxyCompactItemize}
\item 
void \doxymbox{\hyperlink{class_beam_1_1_flux_node_ae3bafc1c5a1aa545167256172b3d3688}{setup\+Buffers}} (int num\+Inputs, int num\+Outputs, int buffer\+Size, int channels)
\end{DoxyCompactItemize}
\doxysubsection*{Protected Attributes inherited from \doxymbox{\hyperlink{class_beam_1_1_flux_node}{Beam::\+\+Flux\+Node}}}
\begin{DoxyCompactItemize}
\item 
std::\+vector$<$ std::\+vector$<$ float $>$ $>$ \doxymbox{\hyperlink{class_beam_1_1_flux_node_a8edab1c9ebd83e73bbfd92af29d6e92c}{m\+\_\+inputs}}
\item 
std::\+vector$<$ std::\+vector$<$ float $>$ $>$ \doxymbox{\hyperlink{class_beam_1_1_flux_node_a496905f0ff42c432eb38e19bd6135383}{m\+\_\+outputs}}
\item 
std::\+map$<$ std::\+string, std::\+shared\+\_\+ptr$<$ \doxymbox{\hyperlink{class_beam_1_1_parameter}{Parameter}} $>$ $>$ \doxymbox{\hyperlink{class_beam_1_1_flux_node_a65628a37cd2dd2832eda60e74ec1aed3}{m\+\_\+parameters}}
\item 
std::\+atomic$<$ bool $>$ \doxymbox{\hyperlink{class_beam_1_1_flux_node_a6116dcdcfa20998fe90dc75a74f25d9b}{m\+\_\+bypassed}} \{false\}
\end{DoxyCompactItemize}


\label{doc-constructors}
\Hypertarget{class_beam_1_1_flux_gain_node_doc-constructors}
\doxysubsection{Constructor \& Destructor Documentation}
\Hypertarget{class_beam_1_1_flux_gain_node_a9141fd7342b8411b66bf1ddf62111df6}\index{Beam::FluxGainNode@{Beam::FluxGainNode}!FluxGainNode@{FluxGainNode}}
\index{FluxGainNode@{FluxGainNode}!Beam::FluxGainNode@{Beam::FluxGainNode}}
\doxysubsubsection{\texorpdfstring{FluxGainNode()}{FluxGainNode()}}
{\footnotesize\ttfamily \label{class_beam_1_1_flux_gain_node_a9141fd7342b8411b66bf1ddf62111df6} 
Beam::\+\+Flux\+Gain\+Node::\+\+Flux\+Gain\+Node (\begin{DoxyParamCaption}\item[{int}]{buffer\+Size}{}\end{DoxyParamCaption})\hspace{0.3cm}{\ttfamily [inline]}}



\label{doc-func-members}
\Hypertarget{class_beam_1_1_flux_gain_node_doc-func-members}
\doxysubsection{Member Function Documentation}
\Hypertarget{class_beam_1_1_flux_gain_node_a8233991523774e343143415f5d314ca4}\index{Beam::FluxGainNode@{Beam::FluxGainNode}!getInputPorts@{getInputPorts}}
\index{getInputPorts@{getInputPorts}!Beam::FluxGainNode@{Beam::FluxGainNode}}
\doxysubsubsection{\texorpdfstring{getInputPorts()}{getInputPorts()}}
{\footnotesize\ttfamily \label{class_beam_1_1_flux_gain_node_a8233991523774e343143415f5d314ca4} 
std::\+vector$<$ \doxymbox{\hyperlink{struct_beam_1_1_flux_node_1_1_port}{Port}} $>$ Beam::\+\+Flux\+Gain\+Node::\+get\+Input\+Ports (\begin{DoxyParamCaption}{}{}\end{DoxyParamCaption}) const\hspace{0.3cm}{\ttfamily [inline]}, {\ttfamily [override]}, {\ttfamily [virtual]}}



Implements \doxymbox{\hyperlink{class_beam_1_1_flux_node_a17eb02187925b52bf8e53fa3ebe3da66}{Beam::\+\+Flux\+Node}}.

\Hypertarget{class_beam_1_1_flux_gain_node_a628da5d08479cdcfdbf4965f5b7339f2}\index{Beam::FluxGainNode@{Beam::FluxGainNode}!getName@{getName}}
\index{getName@{getName}!Beam::FluxGainNode@{Beam::FluxGainNode}}
\doxysubsubsection{\texorpdfstring{getName()}{getName()}}
{\footnotesize\ttfamily \label{class_beam_1_1_flux_gain_node_a628da5d08479cdcfdbf4965f5b7339f2} 
std::\+string Beam::\+\+Flux\+Gain\+Node::\+get\+Name (\begin{DoxyParamCaption}{}{}\end{DoxyParamCaption}) const\hspace{0.3cm}{\ttfamily [inline]}, {\ttfamily [override]}, {\ttfamily [virtual]}}



Implements \doxymbox{\hyperlink{class_beam_1_1_flux_node_ac638d3d9bb1050d658294bc5470abeba}{Beam::\+\+Flux\+Node}}.

\Hypertarget{class_beam_1_1_flux_gain_node_a8c69847df47b060cd8027c4e85f74a8f}\index{Beam::FluxGainNode@{Beam::FluxGainNode}!getOutputPorts@{getOutputPorts}}
\index{getOutputPorts@{getOutputPorts}!Beam::FluxGainNode@{Beam::FluxGainNode}}
\doxysubsubsection{\texorpdfstring{getOutputPorts()}{getOutputPorts()}}
{\footnotesize\ttfamily \label{class_beam_1_1_flux_gain_node_a8c69847df47b060cd8027c4e85f74a8f} 
std::\+vector$<$ \doxymbox{\hyperlink{struct_beam_1_1_flux_node_1_1_port}{Port}} $>$ Beam::\+\+Flux\+Gain\+Node::\+get\+Output\+Ports (\begin{DoxyParamCaption}{}{}\end{DoxyParamCaption}) const\hspace{0.3cm}{\ttfamily [inline]}, {\ttfamily [override]}, {\ttfamily [virtual]}}



Implements \doxymbox{\hyperlink{class_beam_1_1_flux_node_a034f59d236afd7901ed84090422e3279}{Beam::\+\+Flux\+Node}}.

\Hypertarget{class_beam_1_1_flux_gain_node_adeb55f996d1f48c22b674d62c27f2b8e}\index{Beam::FluxGainNode@{Beam::FluxGainNode}!process@{process}}
\index{process@{process}!Beam::FluxGainNode@{Beam::FluxGainNode}}
\doxysubsubsection{\texorpdfstring{process()}{process()}}
{\footnotesize\ttfamily \label{class_beam_1_1_flux_gain_node_adeb55f996d1f48c22b674d62c27f2b8e} 
void Beam::\+\+Flux\+Gain\+Node::\+process (\begin{DoxyParamCaption}\item[{int}]{frames}{}\end{DoxyParamCaption})\hspace{0.3cm}{\ttfamily [inline]}, {\ttfamily [override]}, {\ttfamily [virtual]}}



Implements \doxymbox{\hyperlink{class_beam_1_1_flux_node_a3c263446753fa7ae5ff6928ee57bcd4d}{Beam::\+\+Flux\+Node}}.



\label{doc-variable-members}
\Hypertarget{class_beam_1_1_flux_gain_node_doc-variable-members}
\doxysubsection{Member Data Documentation}
\Hypertarget{class_beam_1_1_flux_gain_node_a67388972db423ad08f7230c627654f6b}\index{Beam::FluxGainNode@{Beam::FluxGainNode}!m\_gain@{m\_gain}}
\index{m\_gain@{m\_gain}!Beam::FluxGainNode@{Beam::FluxGainNode}}
\doxysubsubsection{\texorpdfstring{m\_gain}{m\_gain}}
{\footnotesize\ttfamily \label{class_beam_1_1_flux_gain_node_a67388972db423ad08f7230c627654f6b} 
std::\+shared\+\_\+ptr$<$\doxymbox{\hyperlink{class_beam_1_1_gain_node}{Gain\+Node}}$>$ Beam::\+\+Flux\+Gain\+Node::\+m\+\_\+gain\hspace{0.3cm}{\ttfamily [private]}}



The documentation for this class was generated from the following file:\+\begin{DoxyCompactItemize}
\item 
src/\+dsp/\+\doxymbox{\hyperlink{flux__fx__nodes_8hpp}{flux\+\_\+fx\+\_\+nodes.\+hpp}}\end{DoxyCompactItemize}

\doxysection{Beam::\+Flux\+Node\+Audio\+Processor\+Wrapper Class Reference}
\hypertarget{class_beam_1_1_flux_node_audio_processor_wrapper}{}\label{class_beam_1_1_flux_node_audio_processor_wrapper}\index{Beam::FluxNodeAudioProcessorWrapper@{Beam::FluxNodeAudioProcessorWrapper}}


Wrapper to expose \doxylink{class_beam_1_1_flux_node}{Flux\+Node} as an \doxylink{class_beam_1_1_audio_processor}{Audio\+Processor} for JUCE-\/like API compatibility.  




{\ttfamily \+\#include $<$flux\+\_\+node\+\_\+audio\+\_\+processor\+\_\+wrapper.\+hpp$>$}

Inheritance diagram for Beam::\+Flux\+Node\+Audio\+Processor\+Wrapper:\+\begin{figure}[H]
\begin{center}
\leavevmode
\includegraphics[height=2.000000cm]{class_beam_1_1_flux_node_audio_processor_wrapper}
\end{center}
\end{figure}
\doxysubsubsection*{Public Member Functions}
\begin{DoxyCompactItemize}
\item 
\doxymbox{\hyperlink{class_beam_1_1_flux_node_audio_processor_wrapper_ab76d80d3acccaf5adb860dd57af02292}{Flux\+Node\+Audio\+Processor\+Wrapper}} (std::\+shared\+\_\+ptr$<$ \doxymbox{\hyperlink{class_beam_1_1_flux_node}{Flux\+Node}} $>$ node)
\item 
void \doxymbox{\hyperlink{class_beam_1_1_flux_node_audio_processor_wrapper_a26b885175501fb2cad51c69ef85489c1}{prepare\+To\+Play}} (double sample\+Rate, int samples\+Per\+Block) override
\begin{DoxyCompactList}\small\item\em Called before audio processing begins. \end{DoxyCompactList}\item 
void \doxymbox{\hyperlink{class_beam_1_1_flux_node_audio_processor_wrapper_a124968314929cbbf897db8a23ef5ed3b}{release\+Resources}} () override
\begin{DoxyCompactList}\small\item\em Called after audio stops. \end{DoxyCompactList}\item 
void \doxymbox{\hyperlink{class_beam_1_1_flux_node_audio_processor_wrapper_ad4916adf7b3423d1407a4058e43d3a8f}{process\+Block}} (float \texorpdfstring{$\ast$}{*}\texorpdfstring{$\ast$}{*}audio\+Input\+Output, int num\+Input\+Channels, int num\+Output\+Channels, int num\+Samples, const \doxymbox{\hyperlink{class_beam_1_1_m_i_d_i_buffer}{MIDIBuffer}} \&midi\+Messages) override
\begin{DoxyCompactList}\small\item\em Main processing function. \end{DoxyCompactList}\item 
std::\+string \doxymbox{\hyperlink{class_beam_1_1_flux_node_audio_processor_wrapper_aa2b20d72dcf9a64018cfa617ee29df8c}{get\+Name}} () const override
\begin{DoxyCompactList}\small\item\em Returns the name of this processor. \end{DoxyCompactList}\item 
int \doxymbox{\hyperlink{class_beam_1_1_flux_node_audio_processor_wrapper_af5f7eb577be8e860cd7e15126a05d285}{get\+Num\+Input\+Channels}} () const override
\begin{DoxyCompactList}\small\item\em Returns the number of input channels. \end{DoxyCompactList}\item 
int \doxymbox{\hyperlink{class_beam_1_1_flux_node_audio_processor_wrapper_a2e3a9d06433fc35e14071301544125df}{get\+Num\+Output\+Channels}} () const override
\begin{DoxyCompactList}\small\item\em Returns the number of output channels. \end{DoxyCompactList}\item 
std::\+shared\+\_\+ptr$<$ \doxymbox{\hyperlink{class_beam_1_1_flux_node}{Flux\+Node}} $>$ \doxymbox{\hyperlink{class_beam_1_1_flux_node_audio_processor_wrapper_afd958da23f8ba0df72dd915402d2128d}{get\+Flux\+Node}} () const
\end{DoxyCompactItemize}
\doxysubsection*{Public Member Functions inherited from \doxymbox{\hyperlink{class_beam_1_1_audio_processor}{Beam::\+\+Audio\+Processor}}}
\begin{DoxyCompactItemize}
\item 
\doxymbox{\hyperlink{class_beam_1_1_audio_processor_a8f0f6af900e612f1e95407eb6c8878cd}{Audio\+Processor}} ()
\item 
virtual \doxymbox{\hyperlink{class_beam_1_1_audio_processor_aa760c6d60a8045dc32aa1578b9c74afe}{\texorpdfstring{$\sim$}{\string~}\+Audio\+Processor}} ()
\item 
int \doxymbox{\hyperlink{class_beam_1_1_audio_processor_a0e0668105ca3844b5d0947a3fcf25b7b}{get\+Num\+Parameters}} () const
\begin{DoxyCompactList}\small\item\em Returns the total number of parameters. \end{DoxyCompactList}\item 
virtual std::\+string \doxymbox{\hyperlink{class_beam_1_1_audio_processor_a8dd4277153a28700b0b043ef218a4d15}{get\+Parameter\+Name}} (int parameter\+Index) const
\begin{DoxyCompactList}\small\item\em Returns the name of a parameter. \end{DoxyCompactList}\item 
virtual float \doxymbox{\hyperlink{class_beam_1_1_audio_processor_a456045a6f38f955db222bfdbb1e54be0}{get\+Parameter}} (int parameter\+Index) const
\begin{DoxyCompactList}\small\item\em Returns the value of a parameter. \end{DoxyCompactList}\item 
virtual void \doxymbox{\hyperlink{class_beam_1_1_audio_processor_a3e9a8c5b7470f9a957d85b84efff64b1}{set\+Parameter}} (int parameter\+Index, float new\+Value)
\begin{DoxyCompactList}\small\item\em Sets the value of a parameter. \end{DoxyCompactList}\item 
virtual std::\+string \doxymbox{\hyperlink{class_beam_1_1_audio_processor_ad0c8e5cece50693d131dc9476f5a7835}{get\+Parameter\+Text}} (int parameter\+Index, float value) const
\begin{DoxyCompactList}\small\item\em Returns the text representation of a parameter\textquotesingle{}s value. \end{DoxyCompactList}\item 
void \doxymbox{\hyperlink{class_beam_1_1_audio_processor_a7e9f840858284252d4a288f4e89e0c8e}{add\+Parameter}} (std::\+shared\+\_\+ptr$<$ \doxymbox{\hyperlink{class_beam_1_1_parameter}{Parameter}} $>$ parameter)
\begin{DoxyCompactList}\small\item\em Adds a parameter to this processor. \end{DoxyCompactList}\item 
std::\+shared\+\_\+ptr$<$ \doxymbox{\hyperlink{class_beam_1_1_parameter}{Parameter}} $>$ \doxymbox{\hyperlink{class_beam_1_1_audio_processor_a7298dde8e5844676fea306d579ae1b5c}{get\+Parameter\+By\+Index}} (int index) const
\begin{DoxyCompactList}\small\item\em Gets a parameter by index. \end{DoxyCompactList}\item 
std::\+shared\+\_\+ptr$<$ \doxymbox{\hyperlink{class_beam_1_1_parameter}{Parameter}} $>$ \doxymbox{\hyperlink{class_beam_1_1_audio_processor_a3b570dc9bcbeb3208e026e436c1fe9e7}{get\+Parameter\+By\+Name}} (const std::\+string \&name) const
\begin{DoxyCompactList}\small\item\em Gets a parameter by name. \end{DoxyCompactList}\item 
virtual void \doxymbox{\hyperlink{class_beam_1_1_audio_processor_a7da84e7885fe1660401485dd1f46c8f9}{set\+Current\+Playback\+State}} (bool is\+Playing, double current\+Time\+Seconds, double tempo)
\begin{DoxyCompactList}\small\item\em Called when the play head state changes. \end{DoxyCompactList}\item 
double \doxymbox{\hyperlink{class_beam_1_1_audio_processor_a480e589f13d2a2046d7d46896c17daaa}{get\+Sample\+Rate}} () const
\begin{DoxyCompactList}\small\item\em Returns the current sample rate. \end{DoxyCompactList}\item 
int \doxymbox{\hyperlink{class_beam_1_1_audio_processor_a49a5218a84064deb3c05aad0087be7a3}{get\+Block\+Size}} () const
\begin{DoxyCompactList}\small\item\em Returns the current block size. \end{DoxyCompactList}\end{DoxyCompactItemize}
\doxysubsubsection*{Private Attributes}
\begin{DoxyCompactItemize}
\item 
std::\+shared\+\_\+ptr$<$ \doxymbox{\hyperlink{class_beam_1_1_flux_node}{Flux\+Node}} $>$ \doxymbox{\hyperlink{class_beam_1_1_flux_node_audio_processor_wrapper_a5733b304d9589c0bf2a6a61b42775572}{m\+\_\+flux\+Node}}
\end{DoxyCompactItemize}
\doxysubsubsection*{Additional Inherited Members}
\doxysubsection*{Protected Attributes inherited from \doxymbox{\hyperlink{class_beam_1_1_audio_processor}{Beam::\+\+Audio\+Processor}}}
\begin{DoxyCompactItemize}
\item 
double \doxymbox{\hyperlink{class_beam_1_1_audio_processor_add21a9644aaa71b9457b4f508e1e0840}{m\+\_\+sample\+Rate}} = 44100.\+0
\item 
int \doxymbox{\hyperlink{class_beam_1_1_audio_processor_a3d2ab6b1f0c86689d55c8e72ce428466}{m\+\_\+block\+Size}} = 512
\item 
std::\+vector$<$ std::\+shared\+\_\+ptr$<$ \doxymbox{\hyperlink{class_beam_1_1_parameter}{Parameter}} $>$ $>$ \doxymbox{\hyperlink{class_beam_1_1_audio_processor_a89bb21df73830e6b14b8bf69d3ba68a1}{m\+\_\+parameters}}
\item 
std::\+map$<$ std::\+string, std::\+shared\+\_\+ptr$<$ \doxymbox{\hyperlink{class_beam_1_1_parameter}{Parameter}} $>$ $>$ \doxymbox{\hyperlink{class_beam_1_1_audio_processor_a0430f0f07132e944c8af000cae93f321}{m\+\_\+named\+Parameters}}
\end{DoxyCompactItemize}


\doxysubsection{Detailed Description}
Wrapper to expose \doxylink{class_beam_1_1_flux_node}{Flux\+Node} as an \doxylink{class_beam_1_1_audio_processor}{Audio\+Processor} for JUCE-\/like API compatibility. 

\label{doc-constructors}
\Hypertarget{class_beam_1_1_flux_node_audio_processor_wrapper_doc-constructors}
\doxysubsection{Constructor \& Destructor Documentation}
\Hypertarget{class_beam_1_1_flux_node_audio_processor_wrapper_ab76d80d3acccaf5adb860dd57af02292}\index{Beam::FluxNodeAudioProcessorWrapper@{Beam::FluxNodeAudioProcessorWrapper}!FluxNodeAudioProcessorWrapper@{FluxNodeAudioProcessorWrapper}}
\index{FluxNodeAudioProcessorWrapper@{FluxNodeAudioProcessorWrapper}!Beam::FluxNodeAudioProcessorWrapper@{Beam::FluxNodeAudioProcessorWrapper}}
\doxysubsubsection{\texorpdfstring{FluxNodeAudioProcessorWrapper()}{FluxNodeAudioProcessorWrapper()}}
{\footnotesize\ttfamily \label{class_beam_1_1_flux_node_audio_processor_wrapper_ab76d80d3acccaf5adb860dd57af02292} 
Beam::\+\+Flux\+Node\+Audio\+Processor\+Wrapper::\+\+Flux\+Node\+Audio\+Processor\+Wrapper (\begin{DoxyParamCaption}\item[{std::\+shared\+\_\+ptr$<$ \doxymbox{\hyperlink{class_beam_1_1_flux_node}{Flux\+Node}} $>$}]{node}{}\end{DoxyParamCaption})\hspace{0.3cm}{\ttfamily [inline]}, {\ttfamily [explicit]}}



\label{doc-func-members}
\Hypertarget{class_beam_1_1_flux_node_audio_processor_wrapper_doc-func-members}
\doxysubsection{Member Function Documentation}
\Hypertarget{class_beam_1_1_flux_node_audio_processor_wrapper_afd958da23f8ba0df72dd915402d2128d}\index{Beam::FluxNodeAudioProcessorWrapper@{Beam::FluxNodeAudioProcessorWrapper}!getFluxNode@{getFluxNode}}
\index{getFluxNode@{getFluxNode}!Beam::FluxNodeAudioProcessorWrapper@{Beam::FluxNodeAudioProcessorWrapper}}
\doxysubsubsection{\texorpdfstring{getFluxNode()}{getFluxNode()}}
{\footnotesize\ttfamily \label{class_beam_1_1_flux_node_audio_processor_wrapper_afd958da23f8ba0df72dd915402d2128d} 
std::\+shared\+\_\+ptr$<$ \doxymbox{\hyperlink{class_beam_1_1_flux_node}{Flux\+Node}} $>$ Beam::\+\+Flux\+Node\+Audio\+Processor\+Wrapper::\+get\+Flux\+Node (\begin{DoxyParamCaption}{}{}\end{DoxyParamCaption}) const\hspace{0.3cm}{\ttfamily [inline]}}

\Hypertarget{class_beam_1_1_flux_node_audio_processor_wrapper_aa2b20d72dcf9a64018cfa617ee29df8c}\index{Beam::FluxNodeAudioProcessorWrapper@{Beam::FluxNodeAudioProcessorWrapper}!getName@{getName}}
\index{getName@{getName}!Beam::FluxNodeAudioProcessorWrapper@{Beam::FluxNodeAudioProcessorWrapper}}
\doxysubsubsection{\texorpdfstring{getName()}{getName()}}
{\footnotesize\ttfamily \label{class_beam_1_1_flux_node_audio_processor_wrapper_aa2b20d72dcf9a64018cfa617ee29df8c} 
std::\+string Beam::\+\+Flux\+Node\+Audio\+Processor\+Wrapper::\+get\+Name (\begin{DoxyParamCaption}{}{}\end{DoxyParamCaption}) const\hspace{0.3cm}{\ttfamily [inline]}, {\ttfamily [override]}, {\ttfamily [virtual]}}



Returns the name of this processor. 



Implements \doxymbox{\hyperlink{class_beam_1_1_audio_processor_af4f88f01a436dba1fd3db520503b68c7}{Beam::\+\+Audio\+Processor}}.

\Hypertarget{class_beam_1_1_flux_node_audio_processor_wrapper_af5f7eb577be8e860cd7e15126a05d285}\index{Beam::FluxNodeAudioProcessorWrapper@{Beam::FluxNodeAudioProcessorWrapper}!getNumInputChannels@{getNumInputChannels}}
\index{getNumInputChannels@{getNumInputChannels}!Beam::FluxNodeAudioProcessorWrapper@{Beam::FluxNodeAudioProcessorWrapper}}
\doxysubsubsection{\texorpdfstring{getNumInputChannels()}{getNumInputChannels()}}
{\footnotesize\ttfamily \label{class_beam_1_1_flux_node_audio_processor_wrapper_af5f7eb577be8e860cd7e15126a05d285} 
int Beam::\+\+Flux\+Node\+Audio\+Processor\+Wrapper::\+get\+Num\+Input\+Channels (\begin{DoxyParamCaption}{}{}\end{DoxyParamCaption}) const\hspace{0.3cm}{\ttfamily [inline]}, {\ttfamily [override]}, {\ttfamily [virtual]}}



Returns the number of input channels. 



Implements \doxymbox{\hyperlink{class_beam_1_1_audio_processor_a13c1f31b340d69fce391411402082211}{Beam::\+\+Audio\+Processor}}.

\Hypertarget{class_beam_1_1_flux_node_audio_processor_wrapper_a2e3a9d06433fc35e14071301544125df}\index{Beam::FluxNodeAudioProcessorWrapper@{Beam::FluxNodeAudioProcessorWrapper}!getNumOutputChannels@{getNumOutputChannels}}
\index{getNumOutputChannels@{getNumOutputChannels}!Beam::FluxNodeAudioProcessorWrapper@{Beam::FluxNodeAudioProcessorWrapper}}
\doxysubsubsection{\texorpdfstring{getNumOutputChannels()}{getNumOutputChannels()}}
{\footnotesize\ttfamily \label{class_beam_1_1_flux_node_audio_processor_wrapper_a2e3a9d06433fc35e14071301544125df} 
int Beam::\+\+Flux\+Node\+Audio\+Processor\+Wrapper::\+get\+Num\+Output\+Channels (\begin{DoxyParamCaption}{}{}\end{DoxyParamCaption}) const\hspace{0.3cm}{\ttfamily [inline]}, {\ttfamily [override]}, {\ttfamily [virtual]}}



Returns the number of output channels. 



Implements \doxymbox{\hyperlink{class_beam_1_1_audio_processor_a403c0cf559133347ee2753ef11766f21}{Beam::\+\+Audio\+Processor}}.

\Hypertarget{class_beam_1_1_flux_node_audio_processor_wrapper_a26b885175501fb2cad51c69ef85489c1}\index{Beam::FluxNodeAudioProcessorWrapper@{Beam::FluxNodeAudioProcessorWrapper}!prepareToPlay@{prepareToPlay}}
\index{prepareToPlay@{prepareToPlay}!Beam::FluxNodeAudioProcessorWrapper@{Beam::FluxNodeAudioProcessorWrapper}}
\doxysubsubsection{\texorpdfstring{prepareToPlay()}{prepareToPlay()}}
{\footnotesize\ttfamily \label{class_beam_1_1_flux_node_audio_processor_wrapper_a26b885175501fb2cad51c69ef85489c1} 
void Beam::\+\+Flux\+Node\+Audio\+Processor\+Wrapper::\+prepare\+To\+Play (\begin{DoxyParamCaption}\item[{double}]{sample\+Rate}{, }\item[{int}]{samples\+Per\+Block}{}\end{DoxyParamCaption})\hspace{0.3cm}{\ttfamily [inline]}, {\ttfamily [override]}, {\ttfamily [virtual]}}



Called before audio processing begins. 



Implements \doxymbox{\hyperlink{class_beam_1_1_audio_processor_a0ef7279db2dc6cef108efcad0c6edb8a}{Beam::\+\+Audio\+Processor}}.

\Hypertarget{class_beam_1_1_flux_node_audio_processor_wrapper_ad4916adf7b3423d1407a4058e43d3a8f}\index{Beam::FluxNodeAudioProcessorWrapper@{Beam::FluxNodeAudioProcessorWrapper}!processBlock@{processBlock}}
\index{processBlock@{processBlock}!Beam::FluxNodeAudioProcessorWrapper@{Beam::FluxNodeAudioProcessorWrapper}}
\doxysubsubsection{\texorpdfstring{processBlock()}{processBlock()}}
{\footnotesize\ttfamily \label{class_beam_1_1_flux_node_audio_processor_wrapper_ad4916adf7b3423d1407a4058e43d3a8f} 
void Beam::\+\+Flux\+Node\+Audio\+Processor\+Wrapper::\+process\+Block (\begin{DoxyParamCaption}\item[{float \texorpdfstring{$\ast$}{*}\texorpdfstring{$\ast$}{*}}]{audio\+Input\+Output}{, }\item[{int}]{num\+Input\+Channels}{, }\item[{int}]{num\+Output\+Channels}{, }\item[{int}]{num\+Samples}{, }\item[{const \doxymbox{\hyperlink{class_beam_1_1_m_i_d_i_buffer}{MIDIBuffer}} \&}]{midi\+Messages}{}\end{DoxyParamCaption})\hspace{0.3cm}{\ttfamily [inline]}, {\ttfamily [override]}, {\ttfamily [virtual]}}



Main processing function. 



Implements \doxymbox{\hyperlink{class_beam_1_1_audio_processor_a831818fed7ed10115cd3983addf21936}{Beam::\+\+Audio\+Processor}}.

\Hypertarget{class_beam_1_1_flux_node_audio_processor_wrapper_a124968314929cbbf897db8a23ef5ed3b}\index{Beam::FluxNodeAudioProcessorWrapper@{Beam::FluxNodeAudioProcessorWrapper}!releaseResources@{releaseResources}}
\index{releaseResources@{releaseResources}!Beam::FluxNodeAudioProcessorWrapper@{Beam::FluxNodeAudioProcessorWrapper}}
\doxysubsubsection{\texorpdfstring{releaseResources()}{releaseResources()}}
{\footnotesize\ttfamily \label{class_beam_1_1_flux_node_audio_processor_wrapper_a124968314929cbbf897db8a23ef5ed3b} 
void Beam::\+\+Flux\+Node\+Audio\+Processor\+Wrapper::\+release\+Resources (\begin{DoxyParamCaption}{}{}\end{DoxyParamCaption})\hspace{0.3cm}{\ttfamily [inline]}, {\ttfamily [override]}, {\ttfamily [virtual]}}



Called after audio stops. 



Implements \doxymbox{\hyperlink{class_beam_1_1_audio_processor_adf8005d9b36d70188aa1379a61f7f5a9}{Beam::\+\+Audio\+Processor}}.



\label{doc-variable-members}
\Hypertarget{class_beam_1_1_flux_node_audio_processor_wrapper_doc-variable-members}
\doxysubsection{Member Data Documentation}
\Hypertarget{class_beam_1_1_flux_node_audio_processor_wrapper_a5733b304d9589c0bf2a6a61b42775572}\index{Beam::FluxNodeAudioProcessorWrapper@{Beam::FluxNodeAudioProcessorWrapper}!m\_fluxNode@{m\_fluxNode}}
\index{m\_fluxNode@{m\_fluxNode}!Beam::FluxNodeAudioProcessorWrapper@{Beam::FluxNodeAudioProcessorWrapper}}
\doxysubsubsection{\texorpdfstring{m\_fluxNode}{m\_fluxNode}}
{\footnotesize\ttfamily \label{class_beam_1_1_flux_node_audio_processor_wrapper_a5733b304d9589c0bf2a6a61b42775572} 
std::\+shared\+\_\+ptr$<$\doxymbox{\hyperlink{class_beam_1_1_flux_node}{Flux\+Node}}$>$ Beam::\+\+Flux\+Node\+Audio\+Processor\+Wrapper::\+m\+\_\+flux\+Node\hspace{0.3cm}{\ttfamily [private]}}



The documentation for this class was generated from the following file:\+\begin{DoxyCompactItemize}
\item 
src/\+engine/\+\doxymbox{\hyperlink{flux__node__audio__processor__wrapper_8hpp}{flux\+\_\+node\+\_\+audio\+\_\+processor\+\_\+wrapper.\+hpp}}\end{DoxyCompactItemize}

\doxysection{Beam::\+Flux\+Plugin Class Reference}
\hypertarget{class_beam_1_1_flux_plugin}{}\label{class_beam_1_1_flux_plugin}\index{Beam::FluxPlugin@{Beam::FluxPlugin}}


{\ttfamily \+\#include $<$flux\+\_\+plugin.\+hpp$>$}

Inheritance diagram for Beam::\+Flux\+Plugin:\+\begin{figure}[H]
\begin{center}
\leavevmode
\includegraphics[height=12.000000cm]{class_beam_1_1_flux_plugin}
\end{center}
\end{figure}
\doxysubsubsection*{Public Member Functions}
\begin{DoxyCompactItemize}
\item 
\doxymbox{\hyperlink{class_beam_1_1_flux_plugin_a9d91b56960799fdbfc4575e2fcfa6689}{Flux\+Plugin}} (const std::\+string \&name, int buffer\+Size, float sample\+Rate)
\item 
virtual void \doxymbox{\hyperlink{class_beam_1_1_flux_plugin_ab10324716cec75feee93b6a3159c7912}{process\+Block}} (const float \texorpdfstring{$\ast$}{*}input, float \texorpdfstring{$\ast$}{*}output, int total\+Samples)=0
\item 
virtual void \doxymbox{\hyperlink{class_beam_1_1_flux_plugin_a87fb076475f20b062493efa7ca00e045}{process\+Events}} (const \doxymbox{\hyperlink{class_beam_1_1_m_i_d_i_buffer}{MIDIBuffer}} \&midi)
\begin{DoxyCompactList}\small\item\em Handle MIDI events in your plugin. \end{DoxyCompactList}\item 
void \doxymbox{\hyperlink{class_beam_1_1_flux_plugin_aa1f9c569002ec23eeb5db0af686abea7}{process\+MIDI}} (const \doxymbox{\hyperlink{class_beam_1_1_m_i_d_i_buffer}{MIDIBuffer}} \&midi) override
\begin{DoxyCompactList}\small\item\em Optional MIDI processing. Called before \doxylink{class_beam_1_1_flux_plugin_a181430e1cbf129891fe3ed72f3905a61}{process()} in the engine loop. \end{DoxyCompactList}\item 
void \doxymbox{\hyperlink{class_beam_1_1_flux_plugin_a181430e1cbf129891fe3ed72f3905a61}{process}} (int frames) override
\begin{DoxyCompactList}\small\item\em Main audio processing method. Must be implemented by subclasses. \end{DoxyCompactList}\item 
std::\+string \doxymbox{\hyperlink{class_beam_1_1_flux_plugin_a450563af4d65a25b8a8e896dab77a3c6}{get\+Name}} () const override
\item 
std::\+vector$<$ \doxymbox{\hyperlink{struct_beam_1_1_flux_node_1_1_port}{Port}} $>$ \doxymbox{\hyperlink{class_beam_1_1_flux_plugin_ad231db67f900e8e7dd853936ad2e866a}{get\+Input\+Ports}} () const override
\item 
std::\+vector$<$ \doxymbox{\hyperlink{struct_beam_1_1_flux_node_1_1_port}{Port}} $>$ \doxymbox{\hyperlink{class_beam_1_1_flux_plugin_a4ea312de74047e127e818a26e715d8bb}{get\+Output\+Ports}} () const override
\end{DoxyCompactItemize}
\doxysubsection*{Public Member Functions inherited from \doxymbox{\hyperlink{class_beam_1_1_flux_node}{Beam::\+\+Flux\+Node}}}
\begin{DoxyCompactItemize}
\item 
virtual \doxymbox{\hyperlink{class_beam_1_1_flux_node_a708c135cdb61e8838469998cd8a84e65}{\texorpdfstring{$\sim$}{\string~}\+Flux\+Node}} ()=default
\item 
virtual void \doxymbox{\hyperlink{class_beam_1_1_flux_node_ace8cc49479d8924d44bca5fd4cd955e2}{on\+Transport\+State\+Changed}} (bool playing)
\begin{DoxyCompactList}\small\item\em Responds to global transport changes (Play/\+\+Pause). \end{DoxyCompactList}\item 
virtual void \doxymbox{\hyperlink{class_beam_1_1_flux_node_adc7c4e979bf27de5bfca66815ae97a67}{on\+Transport\+Seek}} (size\+\_\+t frame)
\begin{DoxyCompactList}\small\item\em Responds to timeline seeking. \end{DoxyCompactList}\item 
void \doxymbox{\hyperlink{class_beam_1_1_flux_node_aa579ec06608fd776987bbb089f27fd94}{set\+Current\+Frame}} (size\+\_\+t frame)
\begin{DoxyCompactList}\small\item\em Sets the current playhead position for this block. \end{DoxyCompactList}\item 
float \texorpdfstring{$\ast$}{*} \doxymbox{\hyperlink{class_beam_1_1_flux_node_ac90bd1a05b5bed3d68978f532386ed29}{get\+Input\+Buffer}} (int port\+Idx)
\item 
float \texorpdfstring{$\ast$}{*} \doxymbox{\hyperlink{class_beam_1_1_flux_node_abf11cfd4f2346ee0cd46d4345f1ed7d4}{get\+Output\+Buffer}} (int port\+Idx)
\item 
void \doxymbox{\hyperlink{class_beam_1_1_flux_node_af37f8c1b6b825da2ce7e35011d6f8253}{set\+Bypass}} (bool bypass)
\item 
bool \doxymbox{\hyperlink{class_beam_1_1_flux_node_a4bd30f3c8d311afdcd5c0d208e3bbf0f}{is\+Bypassed}} () const
\item 
void \doxymbox{\hyperlink{class_beam_1_1_flux_node_ad53f3fcaa5737f46d88530f40dbfbe32}{add\+Parameter}} (std::\+shared\+\_\+ptr$<$ \doxymbox{\hyperlink{class_beam_1_1_parameter}{Parameter}} $>$ \doxymbox{\hyperlink{texture_8cpp_aaded45152436a99bb4f9bda081df9f69}{param}})
\item 
std::\+shared\+\_\+ptr$<$ \doxymbox{\hyperlink{class_beam_1_1_parameter}{Parameter}} $>$ \doxymbox{\hyperlink{class_beam_1_1_flux_node_a59a32442eec144010741b9f2086c516e}{get\+Parameter}} (const std::\+string \&name)
\item 
const std::\+map$<$ std::\+string, std::\+shared\+\_\+ptr$<$ \doxymbox{\hyperlink{class_beam_1_1_parameter}{Parameter}} $>$ $>$ \& \doxymbox{\hyperlink{class_beam_1_1_flux_node_a6296c79b1ba77aa8b9526ace4a109529}{get\+Parameters}} () const
\end{DoxyCompactItemize}
\doxysubsubsection*{Protected Member Functions}
\begin{DoxyCompactItemize}
\item 
void \doxymbox{\hyperlink{class_beam_1_1_flux_plugin_a1768cc84018f8de19bcbf781e9b7ac3f}{add\+Param}} (const std::\+string \&name, float min, float max, float initial)
\item 
float \doxymbox{\hyperlink{class_beam_1_1_flux_plugin_a1b292b033caaaf9ec67154dfee2e577b}{get\+Param}} (const std::\+string \&name)
\item 
float \doxymbox{\hyperlink{class_beam_1_1_flux_plugin_a3e65f35944360e4e6ac370a967cf5eb3}{get\+Sample\+Rate}} () const
\end{DoxyCompactItemize}
\doxysubsection*{Protected Member Functions inherited from \doxymbox{\hyperlink{class_beam_1_1_flux_node}{Beam::\+\+Flux\+Node}}}
\begin{DoxyCompactItemize}
\item 
void \doxymbox{\hyperlink{class_beam_1_1_flux_node_ae3bafc1c5a1aa545167256172b3d3688}{setup\+Buffers}} (int num\+Inputs, int num\+Outputs, int buffer\+Size, int channels)
\begin{DoxyCompactList}\small\item\em Pre-\/allocates buffers for inputs and outputs. \end{DoxyCompactList}\end{DoxyCompactItemize}
\doxysubsubsection*{Private Attributes}
\begin{DoxyCompactItemize}
\item 
std::\+string \doxymbox{\hyperlink{class_beam_1_1_flux_plugin_abfb4e9e64c4e94f54001247d7de58011}{m\+\_\+plugin\+Name}}
\item 
float \doxymbox{\hyperlink{class_beam_1_1_flux_plugin_a7818db9fafa231a3525e4d4004bdaea9}{m\+\_\+sample\+Rate}}
\end{DoxyCompactItemize}
\doxysubsubsection*{Additional Inherited Members}
\doxysubsection*{Protected Attributes inherited from \doxymbox{\hyperlink{class_beam_1_1_flux_node}{Beam::\+\+Flux\+Node}}}
\begin{DoxyCompactItemize}
\item 
std::\+vector$<$ std::\+vector$<$ float $>$ $>$ \doxymbox{\hyperlink{class_beam_1_1_flux_node_a8edab1c9ebd83e73bbfd92af29d6e92c}{m\+\_\+inputs}}
\item 
std::\+vector$<$ std::\+vector$<$ float $>$ $>$ \doxymbox{\hyperlink{class_beam_1_1_flux_node_a496905f0ff42c432eb38e19bd6135383}{m\+\_\+outputs}}
\item 
std::\+map$<$ std::\+string, std::\+shared\+\_\+ptr$<$ \doxymbox{\hyperlink{class_beam_1_1_parameter}{Parameter}} $>$ $>$ \doxymbox{\hyperlink{class_beam_1_1_flux_node_a65628a37cd2dd2832eda60e74ec1aed3}{m\+\_\+parameters}}
\item 
std::\+atomic$<$ bool $>$ \doxymbox{\hyperlink{class_beam_1_1_flux_node_a6116dcdcfa20998fe90dc75a74f25d9b}{m\+\_\+bypassed}} \{false\}
\item 
size\+\_\+t \doxymbox{\hyperlink{class_beam_1_1_flux_node_a7d8556ddb1482f997cda7749d737668b}{m\+\_\+current\+Frame}} = 0
\end{DoxyCompactItemize}


\doxysubsection{Detailed Description}
\doxylink{namespace_beam}{Beam} Flux SDK:\+ \doxylink{class_beam_1_1_flux_plugin}{Flux\+Plugin} A high-\/level abstraction for creating custom DSP effects. Inherit from this class to design your own filters and processors. 

\label{doc-constructors}
\Hypertarget{class_beam_1_1_flux_plugin_doc-constructors}
\doxysubsection{Constructor \& Destructor Documentation}
\Hypertarget{class_beam_1_1_flux_plugin_a9d91b56960799fdbfc4575e2fcfa6689}\index{Beam::FluxPlugin@{Beam::FluxPlugin}!FluxPlugin@{FluxPlugin}}
\index{FluxPlugin@{FluxPlugin}!Beam::FluxPlugin@{Beam::FluxPlugin}}
\doxysubsubsection{\texorpdfstring{FluxPlugin()}{FluxPlugin()}}
{\footnotesize\ttfamily \label{class_beam_1_1_flux_plugin_a9d91b56960799fdbfc4575e2fcfa6689} 
Beam::\+\+Flux\+Plugin::\+\+Flux\+Plugin (\begin{DoxyParamCaption}\item[{const std::\+string \&}]{name}{, }\item[{int}]{buffer\+Size}{, }\item[{float}]{sample\+Rate}{}\end{DoxyParamCaption})\hspace{0.3cm}{\ttfamily [inline]}}



\label{doc-func-members}
\Hypertarget{class_beam_1_1_flux_plugin_doc-func-members}
\doxysubsection{Member Function Documentation}
\Hypertarget{class_beam_1_1_flux_plugin_a1768cc84018f8de19bcbf781e9b7ac3f}\index{Beam::FluxPlugin@{Beam::FluxPlugin}!addParam@{addParam}}
\index{addParam@{addParam}!Beam::FluxPlugin@{Beam::FluxPlugin}}
\doxysubsubsection{\texorpdfstring{addParam()}{addParam()}}
{\footnotesize\ttfamily \label{class_beam_1_1_flux_plugin_a1768cc84018f8de19bcbf781e9b7ac3f} 
void Beam::\+\+Flux\+Plugin::\+add\+Param (\begin{DoxyParamCaption}\item[{const std::\+string \&}]{name}{, }\item[{float}]{min}{, }\item[{float}]{max}{, }\item[{float}]{initial}{}\end{DoxyParamCaption})\hspace{0.3cm}{\ttfamily [inline]}, {\ttfamily [protected]}}

\Hypertarget{class_beam_1_1_flux_plugin_ad231db67f900e8e7dd853936ad2e866a}\index{Beam::FluxPlugin@{Beam::FluxPlugin}!getInputPorts@{getInputPorts}}
\index{getInputPorts@{getInputPorts}!Beam::FluxPlugin@{Beam::FluxPlugin}}
\doxysubsubsection{\texorpdfstring{getInputPorts()}{getInputPorts()}}
{\footnotesize\ttfamily \label{class_beam_1_1_flux_plugin_ad231db67f900e8e7dd853936ad2e866a} 
std::\+vector$<$ \doxymbox{\hyperlink{struct_beam_1_1_flux_node_1_1_port}{Port}} $>$ Beam::\+\+Flux\+Plugin::\+get\+Input\+Ports (\begin{DoxyParamCaption}{}{}\end{DoxyParamCaption}) const\hspace{0.3cm}{\ttfamily [inline]}, {\ttfamily [override]}, {\ttfamily [virtual]}}



Implements \doxymbox{\hyperlink{class_beam_1_1_flux_node_a17eb02187925b52bf8e53fa3ebe3da66}{Beam::\+\+Flux\+Node}}.

\Hypertarget{class_beam_1_1_flux_plugin_a450563af4d65a25b8a8e896dab77a3c6}\index{Beam::FluxPlugin@{Beam::FluxPlugin}!getName@{getName}}
\index{getName@{getName}!Beam::FluxPlugin@{Beam::FluxPlugin}}
\doxysubsubsection{\texorpdfstring{getName()}{getName()}}
{\footnotesize\ttfamily \label{class_beam_1_1_flux_plugin_a450563af4d65a25b8a8e896dab77a3c6} 
std::\+string Beam::\+\+Flux\+Plugin::\+get\+Name (\begin{DoxyParamCaption}{}{}\end{DoxyParamCaption}) const\hspace{0.3cm}{\ttfamily [inline]}, {\ttfamily [override]}, {\ttfamily [virtual]}}



Implements \doxymbox{\hyperlink{class_beam_1_1_flux_node_ac638d3d9bb1050d658294bc5470abeba}{Beam::\+\+Flux\+Node}}.

\Hypertarget{class_beam_1_1_flux_plugin_a4ea312de74047e127e818a26e715d8bb}\index{Beam::FluxPlugin@{Beam::FluxPlugin}!getOutputPorts@{getOutputPorts}}
\index{getOutputPorts@{getOutputPorts}!Beam::FluxPlugin@{Beam::FluxPlugin}}
\doxysubsubsection{\texorpdfstring{getOutputPorts()}{getOutputPorts()}}
{\footnotesize\ttfamily \label{class_beam_1_1_flux_plugin_a4ea312de74047e127e818a26e715d8bb} 
std::\+vector$<$ \doxymbox{\hyperlink{struct_beam_1_1_flux_node_1_1_port}{Port}} $>$ Beam::\+\+Flux\+Plugin::\+get\+Output\+Ports (\begin{DoxyParamCaption}{}{}\end{DoxyParamCaption}) const\hspace{0.3cm}{\ttfamily [inline]}, {\ttfamily [override]}, {\ttfamily [virtual]}}



Implements \doxymbox{\hyperlink{class_beam_1_1_flux_node_a034f59d236afd7901ed84090422e3279}{Beam::\+\+Flux\+Node}}.

\Hypertarget{class_beam_1_1_flux_plugin_a1b292b033caaaf9ec67154dfee2e577b}\index{Beam::FluxPlugin@{Beam::FluxPlugin}!getParam@{getParam}}
\index{getParam@{getParam}!Beam::FluxPlugin@{Beam::FluxPlugin}}
\doxysubsubsection{\texorpdfstring{getParam()}{getParam()}}
{\footnotesize\ttfamily \label{class_beam_1_1_flux_plugin_a1b292b033caaaf9ec67154dfee2e577b} 
float Beam::\+\+Flux\+Plugin::\+get\+Param (\begin{DoxyParamCaption}\item[{const std::\+string \&}]{name}{}\end{DoxyParamCaption})\hspace{0.3cm}{\ttfamily [inline]}, {\ttfamily [protected]}}

\Hypertarget{class_beam_1_1_flux_plugin_a3e65f35944360e4e6ac370a967cf5eb3}\index{Beam::FluxPlugin@{Beam::FluxPlugin}!getSampleRate@{getSampleRate}}
\index{getSampleRate@{getSampleRate}!Beam::FluxPlugin@{Beam::FluxPlugin}}
\doxysubsubsection{\texorpdfstring{getSampleRate()}{getSampleRate()}}
{\footnotesize\ttfamily \label{class_beam_1_1_flux_plugin_a3e65f35944360e4e6ac370a967cf5eb3} 
float Beam::\+\+Flux\+Plugin::\+get\+Sample\+Rate (\begin{DoxyParamCaption}{}{}\end{DoxyParamCaption}) const\hspace{0.3cm}{\ttfamily [inline]}, {\ttfamily [protected]}}

\Hypertarget{class_beam_1_1_flux_plugin_a181430e1cbf129891fe3ed72f3905a61}\index{Beam::FluxPlugin@{Beam::FluxPlugin}!process@{process}}
\index{process@{process}!Beam::FluxPlugin@{Beam::FluxPlugin}}
\doxysubsubsection{\texorpdfstring{process()}{process()}}
{\footnotesize\ttfamily \label{class_beam_1_1_flux_plugin_a181430e1cbf129891fe3ed72f3905a61} 
void Beam::\+\+Flux\+Plugin::\+process (\begin{DoxyParamCaption}\item[{int}]{frames}{}\end{DoxyParamCaption})\hspace{0.3cm}{\ttfamily [inline]}, {\ttfamily [override]}, {\ttfamily [virtual]}}



Main audio processing method. Must be implemented by subclasses. 


\begin{DoxyParams}{Parameters}
{\em frames} & Number of frames to process in the current block. \\
\hline
\end{DoxyParams}


Implements \doxymbox{\hyperlink{class_beam_1_1_flux_node_a3c263446753fa7ae5ff6928ee57bcd4d}{Beam::\+\+Flux\+Node}}.

\Hypertarget{class_beam_1_1_flux_plugin_ab10324716cec75feee93b6a3159c7912}\index{Beam::FluxPlugin@{Beam::FluxPlugin}!processBlock@{processBlock}}
\index{processBlock@{processBlock}!Beam::FluxPlugin@{Beam::FluxPlugin}}
\doxysubsubsection{\texorpdfstring{processBlock()}{processBlock()}}
{\footnotesize\ttfamily \label{class_beam_1_1_flux_plugin_ab10324716cec75feee93b6a3159c7912} 
virtual void Beam::\+\+Flux\+Plugin::\+process\+Block (\begin{DoxyParamCaption}\item[{const float \texorpdfstring{$\ast$}{*}}]{input}{, }\item[{float \texorpdfstring{$\ast$}{*}}]{output}{, }\item[{int}]{total\+Samples}{}\end{DoxyParamCaption})\hspace{0.3cm}{\ttfamily [pure virtual]}}



Implemented in \doxymbox{\hyperlink{class_beam_1_1_console_e___e_q_a957a65f7ecdbe08cd5f6ae9c5c8796da}{Beam::\+\+Console\+E\+\_\+\+EQ}}, \doxymbox{\hyperlink{class_beam_1_1_custom_filter_ae561b7a4fb99126dda6d9adb7c12c80a}{Beam::\+\+Custom\+Filter}}, \doxymbox{\hyperlink{class_beam_1_1_echo_plex_a26d16b6feed01873e169f83083873c61}{Beam::\+\+Echo\+Plex}}, \doxymbox{\hyperlink{class_beam_1_1_f_e_t76_a592f2490ae49223add9dbdd1d7ce23d5}{Beam::\+\+FET76}}, \doxymbox{\hyperlink{class_beam_1_1_flux_delay_node_a1edabd22ff7b188d62d46aac298fd2d0}{Beam::\+\+Flux\+Delay\+Node}}, \doxymbox{\hyperlink{class_beam_1_1_flux_filter_node_a075b9fc4a15a1b51c5d948bd6ac27432}{Beam::\+\+Flux\+Filter\+Node}}, \doxymbox{\hyperlink{class_beam_1_1_flux_gain_node_af9e67f1fc2cfc0772405be5d8b820776}{Beam::\+\+Flux\+Gain\+Node}}, \doxymbox{\hyperlink{class_beam_1_1_opto2_a_a6b73dd169be22f140859578aa0f7ed40}{Beam::\+\+Opto2A}}, \doxymbox{\hyperlink{class_beam_1_1_steel_plate_a5b1b681074dbcb4f3307c235fdbe914d}{Beam::\+\+Steel\+Plate}}, \doxymbox{\hyperlink{class_beam_1_1_tube_compressor_node_a45c3b4af658365cf72b55a961abd479a}{Beam::\+\+Tube\+Compressor\+Node}}, \doxymbox{\hyperlink{class_beam_1_1_tube_limiter_a3e25de329299557cdeb25a7d03a2afdc}{Beam::\+\+Tube\+Limiter}}, and \doxymbox{\hyperlink{class_beam_1_1_tube_p___e_q_a8adbb2e60d4a51d203a942d722643b4b}{Beam::\+\+Tube\+P\+\_\+\+EQ}}.

\Hypertarget{class_beam_1_1_flux_plugin_a87fb076475f20b062493efa7ca00e045}\index{Beam::FluxPlugin@{Beam::FluxPlugin}!processEvents@{processEvents}}
\index{processEvents@{processEvents}!Beam::FluxPlugin@{Beam::FluxPlugin}}
\doxysubsubsection{\texorpdfstring{processEvents()}{processEvents()}}
{\footnotesize\ttfamily \label{class_beam_1_1_flux_plugin_a87fb076475f20b062493efa7ca00e045} 
virtual void Beam::\+\+Flux\+Plugin::\+process\+Events (\begin{DoxyParamCaption}\item[{const \doxymbox{\hyperlink{class_beam_1_1_m_i_d_i_buffer}{MIDIBuffer}} \&}]{midi}{}\end{DoxyParamCaption})\hspace{0.3cm}{\ttfamily [inline]}, {\ttfamily [virtual]}}



Handle MIDI events in your plugin. 


\begin{DoxyParams}{Parameters}
{\em midi} & The buffer containing all MIDI events for the current block. \\
\hline
\end{DoxyParams}
\Hypertarget{class_beam_1_1_flux_plugin_aa1f9c569002ec23eeb5db0af686abea7}\index{Beam::FluxPlugin@{Beam::FluxPlugin}!processMIDI@{processMIDI}}
\index{processMIDI@{processMIDI}!Beam::FluxPlugin@{Beam::FluxPlugin}}
\doxysubsubsection{\texorpdfstring{processMIDI()}{processMIDI()}}
{\footnotesize\ttfamily \label{class_beam_1_1_flux_plugin_aa1f9c569002ec23eeb5db0af686abea7} 
void Beam::\+\+Flux\+Plugin::\+process\+MIDI (\begin{DoxyParamCaption}\item[{const \doxymbox{\hyperlink{class_beam_1_1_m_i_d_i_buffer}{MIDIBuffer}} \&}]{midi}{}\end{DoxyParamCaption})\hspace{0.3cm}{\ttfamily [inline]}, {\ttfamily [override]}, {\ttfamily [virtual]}}



Optional MIDI processing. Called before \doxylink{class_beam_1_1_flux_plugin_a181430e1cbf129891fe3ed72f3905a61}{process()} in the engine loop. 



Reimplemented from \doxymbox{\hyperlink{class_beam_1_1_flux_node_ae9d1e151eff5166de969f45de06d5596}{Beam::\+\+Flux\+Node}}.



\label{doc-variable-members}
\Hypertarget{class_beam_1_1_flux_plugin_doc-variable-members}
\doxysubsection{Member Data Documentation}
\Hypertarget{class_beam_1_1_flux_plugin_abfb4e9e64c4e94f54001247d7de58011}\index{Beam::FluxPlugin@{Beam::FluxPlugin}!m\_pluginName@{m\_pluginName}}
\index{m\_pluginName@{m\_pluginName}!Beam::FluxPlugin@{Beam::FluxPlugin}}
\doxysubsubsection{\texorpdfstring{m\_pluginName}{m\_pluginName}}
{\footnotesize\ttfamily \label{class_beam_1_1_flux_plugin_abfb4e9e64c4e94f54001247d7de58011} 
std::\+string Beam::\+\+Flux\+Plugin::\+m\+\_\+plugin\+Name\hspace{0.3cm}{\ttfamily [private]}}

\Hypertarget{class_beam_1_1_flux_plugin_a7818db9fafa231a3525e4d4004bdaea9}\index{Beam::FluxPlugin@{Beam::FluxPlugin}!m\_sampleRate@{m\_sampleRate}}
\index{m\_sampleRate@{m\_sampleRate}!Beam::FluxPlugin@{Beam::FluxPlugin}}
\doxysubsubsection{\texorpdfstring{m\_sampleRate}{m\_sampleRate}}
{\footnotesize\ttfamily \label{class_beam_1_1_flux_plugin_a7818db9fafa231a3525e4d4004bdaea9} 
float Beam::\+\+Flux\+Plugin::\+m\+\_\+sample\+Rate\hspace{0.3cm}{\ttfamily [private]}}



The documentation for this class was generated from the following file:\+\begin{DoxyCompactItemize}
\item 
src/\+engine/\+\doxymbox{\hyperlink{flux__plugin_8hpp}{flux\+\_\+plugin.\+hpp}}\end{DoxyCompactItemize}

\doxysection{Beam::\+Flux\+Project Class Reference}
\hypertarget{class_beam_1_1_flux_project}{}\label{class_beam_1_1_flux_project}\index{Beam::FluxProject@{Beam::FluxProject}}


{\ttfamily \+\#include $<$flux\+\_\+project.\+hpp$>$}

\doxysubsubsection*{Public Member Functions}
\begin{DoxyCompactItemize}
\item 
\doxymbox{\hyperlink{class_beam_1_1_flux_project_ab062e3ebed4944516b8e31d547d2241b}{Flux\+Project}} ()
\item 
std::\+shared\+\_\+ptr$<$ \doxymbox{\hyperlink{class_beam_1_1_flux_graph}{Flux\+Graph}} $>$ \doxymbox{\hyperlink{class_beam_1_1_flux_project_a0b32e6b040ba09486e7a5a5110866186}{get\+Graph}} ()
\item 
void \doxymbox{\hyperlink{class_beam_1_1_flux_project_a1999a9b5e6df6714b52065c697d6d344}{add\+Track}} (\doxymbox{\hyperlink{struct_beam_1_1_track_data}{Track\+Data}} td)
\item 
std::\+vector$<$ \doxymbox{\hyperlink{struct_beam_1_1_track_data}{Track\+Data}} $>$ \& \doxymbox{\hyperlink{class_beam_1_1_flux_project_ac467fd8ce5f1a101fdd23634a7b8b990}{get\+Tracks}} ()
\item 
nlohmann::\+json \doxymbox{\hyperlink{class_beam_1_1_flux_project_af0a110ba7c7ce45506fb9e243b6784b5}{serialize}} () const
\item 
void \doxymbox{\hyperlink{class_beam_1_1_flux_project_aa4a811a39323d92c97654ccc34c09379}{deserialize}} (const nlohmann::\+json \&data)
\end{DoxyCompactItemize}
\doxysubsubsection*{Private Attributes}
\begin{DoxyCompactItemize}
\item 
std::\+string \doxymbox{\hyperlink{class_beam_1_1_flux_project_a8ce6b8f232de62bb34f903f04acc8506}{m\+\_\+name}} = "{}Untitled Flux"{}
\item 
std::\+shared\+\_\+ptr$<$ \doxymbox{\hyperlink{class_beam_1_1_flux_graph}{Flux\+Graph}} $>$ \doxymbox{\hyperlink{class_beam_1_1_flux_project_a24511ed71344a5e3028b204376aab334}{m\+\_\+graph}}
\item 
std::\+vector$<$ \doxymbox{\hyperlink{struct_beam_1_1_track_data}{Track\+Data}} $>$ \doxymbox{\hyperlink{class_beam_1_1_flux_project_a99e25b294a814550c79e1d037983094b}{m\+\_\+tracks}}
\end{DoxyCompactItemize}


\label{doc-constructors}
\Hypertarget{class_beam_1_1_flux_project_doc-constructors}
\doxysubsection{Constructor \& Destructor Documentation}
\Hypertarget{class_beam_1_1_flux_project_ab062e3ebed4944516b8e31d547d2241b}\index{Beam::FluxProject@{Beam::FluxProject}!FluxProject@{FluxProject}}
\index{FluxProject@{FluxProject}!Beam::FluxProject@{Beam::FluxProject}}
\doxysubsubsection{\texorpdfstring{FluxProject()}{FluxProject()}}
{\footnotesize\ttfamily \label{class_beam_1_1_flux_project_ab062e3ebed4944516b8e31d547d2241b} 
Beam::\+\+Flux\+Project::\+\+Flux\+Project (\begin{DoxyParamCaption}{}{}\end{DoxyParamCaption})\hspace{0.3cm}{\ttfamily [inline]}}



\label{doc-func-members}
\Hypertarget{class_beam_1_1_flux_project_doc-func-members}
\doxysubsection{Member Function Documentation}
\Hypertarget{class_beam_1_1_flux_project_a1999a9b5e6df6714b52065c697d6d344}\index{Beam::FluxProject@{Beam::FluxProject}!addTrack@{addTrack}}
\index{addTrack@{addTrack}!Beam::FluxProject@{Beam::FluxProject}}
\doxysubsubsection{\texorpdfstring{addTrack()}{addTrack()}}
{\footnotesize\ttfamily \label{class_beam_1_1_flux_project_a1999a9b5e6df6714b52065c697d6d344} 
void Beam::\+\+Flux\+Project::\+add\+Track (\begin{DoxyParamCaption}\item[{\doxymbox{\hyperlink{struct_beam_1_1_track_data}{Track\+Data}}}]{td}{}\end{DoxyParamCaption})\hspace{0.3cm}{\ttfamily [inline]}}

\Hypertarget{class_beam_1_1_flux_project_aa4a811a39323d92c97654ccc34c09379}\index{Beam::FluxProject@{Beam::FluxProject}!deserialize@{deserialize}}
\index{deserialize@{deserialize}!Beam::FluxProject@{Beam::FluxProject}}
\doxysubsubsection{\texorpdfstring{deserialize()}{deserialize()}}
{\footnotesize\ttfamily \label{class_beam_1_1_flux_project_aa4a811a39323d92c97654ccc34c09379} 
void Beam::\+\+Flux\+Project::\+deserialize (\begin{DoxyParamCaption}\item[{const nlohmann::\+json \&}]{data}{}\end{DoxyParamCaption})\hspace{0.3cm}{\ttfamily [inline]}}

\Hypertarget{class_beam_1_1_flux_project_a0b32e6b040ba09486e7a5a5110866186}\index{Beam::FluxProject@{Beam::FluxProject}!getGraph@{getGraph}}
\index{getGraph@{getGraph}!Beam::FluxProject@{Beam::FluxProject}}
\doxysubsubsection{\texorpdfstring{getGraph()}{getGraph()}}
{\footnotesize\ttfamily \label{class_beam_1_1_flux_project_a0b32e6b040ba09486e7a5a5110866186} 
std::\+shared\+\_\+ptr$<$ \doxymbox{\hyperlink{class_beam_1_1_flux_graph}{Flux\+Graph}} $>$ Beam::\+\+Flux\+Project::\+get\+Graph (\begin{DoxyParamCaption}{}{}\end{DoxyParamCaption})\hspace{0.3cm}{\ttfamily [inline]}}

\Hypertarget{class_beam_1_1_flux_project_ac467fd8ce5f1a101fdd23634a7b8b990}\index{Beam::FluxProject@{Beam::FluxProject}!getTracks@{getTracks}}
\index{getTracks@{getTracks}!Beam::FluxProject@{Beam::FluxProject}}
\doxysubsubsection{\texorpdfstring{getTracks()}{getTracks()}}
{\footnotesize\ttfamily \label{class_beam_1_1_flux_project_ac467fd8ce5f1a101fdd23634a7b8b990} 
std::\+vector$<$ \doxymbox{\hyperlink{struct_beam_1_1_track_data}{Track\+Data}} $>$ \& Beam::\+\+Flux\+Project::\+get\+Tracks (\begin{DoxyParamCaption}{}{}\end{DoxyParamCaption})\hspace{0.3cm}{\ttfamily [inline]}}

\Hypertarget{class_beam_1_1_flux_project_af0a110ba7c7ce45506fb9e243b6784b5}\index{Beam::FluxProject@{Beam::FluxProject}!serialize@{serialize}}
\index{serialize@{serialize}!Beam::FluxProject@{Beam::FluxProject}}
\doxysubsubsection{\texorpdfstring{serialize()}{serialize()}}
{\footnotesize\ttfamily \label{class_beam_1_1_flux_project_af0a110ba7c7ce45506fb9e243b6784b5} 
nlohmann::\+json Beam::\+\+Flux\+Project::\+serialize (\begin{DoxyParamCaption}{}{}\end{DoxyParamCaption}) const\hspace{0.3cm}{\ttfamily [inline]}}



\label{doc-variable-members}
\Hypertarget{class_beam_1_1_flux_project_doc-variable-members}
\doxysubsection{Member Data Documentation}
\Hypertarget{class_beam_1_1_flux_project_a24511ed71344a5e3028b204376aab334}\index{Beam::FluxProject@{Beam::FluxProject}!m\_graph@{m\_graph}}
\index{m\_graph@{m\_graph}!Beam::FluxProject@{Beam::FluxProject}}
\doxysubsubsection{\texorpdfstring{m\_graph}{m\_graph}}
{\footnotesize\ttfamily \label{class_beam_1_1_flux_project_a24511ed71344a5e3028b204376aab334} 
std::\+shared\+\_\+ptr$<$\doxymbox{\hyperlink{class_beam_1_1_flux_graph}{Flux\+Graph}}$>$ Beam::\+\+Flux\+Project::\+m\+\_\+graph\hspace{0.3cm}{\ttfamily [private]}}

\Hypertarget{class_beam_1_1_flux_project_a8ce6b8f232de62bb34f903f04acc8506}\index{Beam::FluxProject@{Beam::FluxProject}!m\_name@{m\_name}}
\index{m\_name@{m\_name}!Beam::FluxProject@{Beam::FluxProject}}
\doxysubsubsection{\texorpdfstring{m\_name}{m\_name}}
{\footnotesize\ttfamily \label{class_beam_1_1_flux_project_a8ce6b8f232de62bb34f903f04acc8506} 
std::\+string Beam::\+\+Flux\+Project::\+m\+\_\+name = "{}Untitled Flux"{}\hspace{0.3cm}{\ttfamily [private]}}

\Hypertarget{class_beam_1_1_flux_project_a99e25b294a814550c79e1d037983094b}\index{Beam::FluxProject@{Beam::FluxProject}!m\_tracks@{m\_tracks}}
\index{m\_tracks@{m\_tracks}!Beam::FluxProject@{Beam::FluxProject}}
\doxysubsubsection{\texorpdfstring{m\_tracks}{m\_tracks}}
{\footnotesize\ttfamily \label{class_beam_1_1_flux_project_a99e25b294a814550c79e1d037983094b} 
std::\+vector$<$\doxymbox{\hyperlink{struct_beam_1_1_track_data}{Track\+Data}}$>$ Beam::\+\+Flux\+Project::\+m\+\_\+tracks\hspace{0.3cm}{\ttfamily [private]}}



The documentation for this class was generated from the following file:\+\begin{DoxyCompactItemize}
\item 
src/\+session/\+\doxymbox{\hyperlink{flux__project_8hpp}{flux\+\_\+project.\+hpp}}\end{DoxyCompactItemize}

\doxysection{Beam::\+Flux\+Track\+Node Class Reference}
\hypertarget{class_beam_1_1_flux_track_node}{}\label{class_beam_1_1_flux_track_node}\index{Beam::FluxTrackNode@{Beam::FluxTrackNode}}


{\ttfamily \+\#include $<$flux\+\_\+track\+\_\+node.\+hpp$>$}

Inheritance diagram for Beam::\+Flux\+Track\+Node:\+\begin{figure}[H]
\begin{center}
\leavevmode
\includegraphics[height=2.000000cm]{class_beam_1_1_flux_track_node}
\end{center}
\end{figure}
\doxysubsubsection*{Public Member Functions}
\begin{DoxyCompactItemize}
\item 
\doxymbox{\hyperlink{class_beam_1_1_flux_track_node_a2c8b503f5fd54d45e3a4a8559755d8b5}{Flux\+Track\+Node}} (const std::\+string \&name, int buffer\+Size)
\item 
bool \doxymbox{\hyperlink{class_beam_1_1_flux_track_node_aa264ea0b55854969320c1e0a730982b4}{load}} (const std::\+string \&file\+Path)
\item 
bool \doxymbox{\hyperlink{class_beam_1_1_flux_track_node_ab59401ed18ae3cf933e7ca7a4644711b}{start\+Recording}} (const std::\+string \&file\+Path, int sample\+Rate)
\item 
void \doxymbox{\hyperlink{class_beam_1_1_flux_track_node_a6ca690381930f2a26b149fa9a4699ed3}{stop\+Recording}} ()
\item 
void \doxymbox{\hyperlink{class_beam_1_1_flux_track_node_ab29facb24a749834aff2aa3de3be437a}{process}} (int frames) override
\begin{DoxyCompactList}\small\item\em Main audio processing method. Must be implemented by subclasses. \end{DoxyCompactList}\item 
void \doxymbox{\hyperlink{class_beam_1_1_flux_track_node_a26450c6c0de5d8f7ec68837bdde18ef3}{on\+Transport\+State\+Changed}} (bool playing) override
\begin{DoxyCompactList}\small\item\em Responds to global transport changes (Play/\+\+Pause). \end{DoxyCompactList}\item 
void \doxymbox{\hyperlink{class_beam_1_1_flux_track_node_a372900a79b264e873083189295b7acf3}{on\+Transport\+Seek}} (size\+\_\+t frame) override
\begin{DoxyCompactList}\small\item\em Responds to timeline seeking. \end{DoxyCompactList}\item 
void \doxymbox{\hyperlink{class_beam_1_1_flux_track_node_afbf932790f01f79b00b4d7895c39d7dc}{seek}} (size\+\_\+t frame)
\item 
void \doxymbox{\hyperlink{class_beam_1_1_flux_track_node_ae6446258757bce3c5f58ca7ae48f31b8}{set\+State}} (\doxymbox{\hyperlink{namespace_beam_aaab526c0becfd931c9f29361daaf7e9f}{Track\+State}} state)
\item 
\doxymbox{\hyperlink{namespace_beam_aaab526c0becfd931c9f29361daaf7e9f}{Track\+State}} \doxymbox{\hyperlink{class_beam_1_1_flux_track_node_aa518ca3fcc84c77991eee501f6db0896}{get\+State}} () const
\item 
std::\+vector$<$ std::\+vector$<$ float $>$ $>$ \doxymbox{\hyperlink{class_beam_1_1_flux_track_node_aca365acdc5e93673587ad33ac29b1e05}{get\+Peak\+Data}} (int num\+Points)
\item 
std::\+shared\+\_\+ptr$<$ \doxymbox{\hyperlink{class_beam_1_1_track_node}{Track\+Node}} $>$ \doxymbox{\hyperlink{class_beam_1_1_flux_track_node_a3f89a7df54007845103280ec40bf7983}{get\+Internal\+Node}} ()
\item 
std::\+string \doxymbox{\hyperlink{class_beam_1_1_flux_track_node_aeb9ce1b2d297d6c800bae5e0ed0fd70f}{get\+Name}} () const override
\item 
std::\+vector$<$ \doxymbox{\hyperlink{struct_beam_1_1_flux_node_1_1_port}{Port}} $>$ \doxymbox{\hyperlink{class_beam_1_1_flux_track_node_a99caf56c174e00c7814a6820879ccc19}{get\+Input\+Ports}} () const override
\item 
std::\+vector$<$ \doxymbox{\hyperlink{struct_beam_1_1_flux_node_1_1_port}{Port}} $>$ \doxymbox{\hyperlink{class_beam_1_1_flux_track_node_ac77568a2dcca757d7d51486cc27b6a2f}{get\+Output\+Ports}} () const override
\end{DoxyCompactItemize}
\doxysubsection*{Public Member Functions inherited from \doxymbox{\hyperlink{class_beam_1_1_flux_node}{Beam::\+\+Flux\+Node}}}
\begin{DoxyCompactItemize}
\item 
virtual \doxymbox{\hyperlink{class_beam_1_1_flux_node_a708c135cdb61e8838469998cd8a84e65}{\texorpdfstring{$\sim$}{\string~}\+Flux\+Node}} ()=default
\item 
virtual void \doxymbox{\hyperlink{class_beam_1_1_flux_node_ae9d1e151eff5166de969f45de06d5596}{process\+MIDI}} (const \doxymbox{\hyperlink{class_beam_1_1_m_i_d_i_buffer}{MIDIBuffer}} \&midi)
\begin{DoxyCompactList}\small\item\em Optional MIDI processing. Called before \doxylink{class_beam_1_1_flux_node_a3c263446753fa7ae5ff6928ee57bcd4d}{process()} in the engine loop. \end{DoxyCompactList}\item 
void \doxymbox{\hyperlink{class_beam_1_1_flux_node_aa579ec06608fd776987bbb089f27fd94}{set\+Current\+Frame}} (size\+\_\+t frame)
\begin{DoxyCompactList}\small\item\em Sets the current playhead position for this block. \end{DoxyCompactList}\item 
float \texorpdfstring{$\ast$}{*} \doxymbox{\hyperlink{class_beam_1_1_flux_node_ac90bd1a05b5bed3d68978f532386ed29}{get\+Input\+Buffer}} (int port\+Idx)
\item 
float \texorpdfstring{$\ast$}{*} \doxymbox{\hyperlink{class_beam_1_1_flux_node_abf11cfd4f2346ee0cd46d4345f1ed7d4}{get\+Output\+Buffer}} (int port\+Idx)
\item 
void \doxymbox{\hyperlink{class_beam_1_1_flux_node_af37f8c1b6b825da2ce7e35011d6f8253}{set\+Bypass}} (bool bypass)
\item 
bool \doxymbox{\hyperlink{class_beam_1_1_flux_node_a4bd30f3c8d311afdcd5c0d208e3bbf0f}{is\+Bypassed}} () const
\item 
void \doxymbox{\hyperlink{class_beam_1_1_flux_node_ad53f3fcaa5737f46d88530f40dbfbe32}{add\+Parameter}} (std::\+shared\+\_\+ptr$<$ \doxymbox{\hyperlink{class_beam_1_1_parameter}{Parameter}} $>$ \doxymbox{\hyperlink{texture_8cpp_aaded45152436a99bb4f9bda081df9f69}{param}})
\item 
std::\+shared\+\_\+ptr$<$ \doxymbox{\hyperlink{class_beam_1_1_parameter}{Parameter}} $>$ \doxymbox{\hyperlink{class_beam_1_1_flux_node_a59a32442eec144010741b9f2086c516e}{get\+Parameter}} (const std::\+string \&name)
\item 
const std::\+map$<$ std::\+string, std::\+shared\+\_\+ptr$<$ \doxymbox{\hyperlink{class_beam_1_1_parameter}{Parameter}} $>$ $>$ \& \doxymbox{\hyperlink{class_beam_1_1_flux_node_a6296c79b1ba77aa8b9526ace4a109529}{get\+Parameters}} () const
\end{DoxyCompactItemize}
\doxysubsubsection*{Private Attributes}
\begin{DoxyCompactItemize}
\item 
std::\+string \doxymbox{\hyperlink{class_beam_1_1_flux_track_node_aae94e87d1c628a62f787bcd6147a6727}{m\+\_\+name}}
\item 
std::\+shared\+\_\+ptr$<$ \doxymbox{\hyperlink{class_beam_1_1_track_node}{Track\+Node}} $>$ \doxymbox{\hyperlink{class_beam_1_1_flux_track_node_ab61ce129ca13f351a616eaca6d9d042c}{m\+\_\+track}}
\end{DoxyCompactItemize}
\doxysubsubsection*{Additional Inherited Members}
\doxysubsection*{Protected Member Functions inherited from \doxymbox{\hyperlink{class_beam_1_1_flux_node}{Beam::\+\+Flux\+Node}}}
\begin{DoxyCompactItemize}
\item 
void \doxymbox{\hyperlink{class_beam_1_1_flux_node_ae3bafc1c5a1aa545167256172b3d3688}{setup\+Buffers}} (int num\+Inputs, int num\+Outputs, int buffer\+Size, int channels)
\begin{DoxyCompactList}\small\item\em Pre-\/allocates buffers for inputs and outputs. \end{DoxyCompactList}\end{DoxyCompactItemize}
\doxysubsection*{Protected Attributes inherited from \doxymbox{\hyperlink{class_beam_1_1_flux_node}{Beam::\+\+Flux\+Node}}}
\begin{DoxyCompactItemize}
\item 
std::\+vector$<$ std::\+vector$<$ float $>$ $>$ \doxymbox{\hyperlink{class_beam_1_1_flux_node_a8edab1c9ebd83e73bbfd92af29d6e92c}{m\+\_\+inputs}}
\item 
std::\+vector$<$ std::\+vector$<$ float $>$ $>$ \doxymbox{\hyperlink{class_beam_1_1_flux_node_a496905f0ff42c432eb38e19bd6135383}{m\+\_\+outputs}}
\item 
std::\+map$<$ std::\+string, std::\+shared\+\_\+ptr$<$ \doxymbox{\hyperlink{class_beam_1_1_parameter}{Parameter}} $>$ $>$ \doxymbox{\hyperlink{class_beam_1_1_flux_node_a65628a37cd2dd2832eda60e74ec1aed3}{m\+\_\+parameters}}
\item 
std::\+atomic$<$ bool $>$ \doxymbox{\hyperlink{class_beam_1_1_flux_node_a6116dcdcfa20998fe90dc75a74f25d9b}{m\+\_\+bypassed}} \{false\}
\item 
size\+\_\+t \doxymbox{\hyperlink{class_beam_1_1_flux_node_a7d8556ddb1482f997cda7749d737668b}{m\+\_\+current\+Frame}} = 0
\end{DoxyCompactItemize}


\label{doc-constructors}
\Hypertarget{class_beam_1_1_flux_track_node_doc-constructors}
\doxysubsection{Constructor \& Destructor Documentation}
\Hypertarget{class_beam_1_1_flux_track_node_a2c8b503f5fd54d45e3a4a8559755d8b5}\index{Beam::FluxTrackNode@{Beam::FluxTrackNode}!FluxTrackNode@{FluxTrackNode}}
\index{FluxTrackNode@{FluxTrackNode}!Beam::FluxTrackNode@{Beam::FluxTrackNode}}
\doxysubsubsection{\texorpdfstring{FluxTrackNode()}{FluxTrackNode()}}
{\footnotesize\ttfamily \label{class_beam_1_1_flux_track_node_a2c8b503f5fd54d45e3a4a8559755d8b5} 
Beam::\+\+Flux\+Track\+Node::\+\+Flux\+Track\+Node (\begin{DoxyParamCaption}\item[{const std::\+string \&}]{name}{, }\item[{int}]{buffer\+Size}{}\end{DoxyParamCaption})\hspace{0.3cm}{\ttfamily [inline]}}



\label{doc-func-members}
\Hypertarget{class_beam_1_1_flux_track_node_doc-func-members}
\doxysubsection{Member Function Documentation}
\Hypertarget{class_beam_1_1_flux_track_node_a99caf56c174e00c7814a6820879ccc19}\index{Beam::FluxTrackNode@{Beam::FluxTrackNode}!getInputPorts@{getInputPorts}}
\index{getInputPorts@{getInputPorts}!Beam::FluxTrackNode@{Beam::FluxTrackNode}}
\doxysubsubsection{\texorpdfstring{getInputPorts()}{getInputPorts()}}
{\footnotesize\ttfamily \label{class_beam_1_1_flux_track_node_a99caf56c174e00c7814a6820879ccc19} 
std::\+vector$<$ \doxymbox{\hyperlink{struct_beam_1_1_flux_node_1_1_port}{Port}} $>$ Beam::\+\+Flux\+Track\+Node::\+get\+Input\+Ports (\begin{DoxyParamCaption}{}{}\end{DoxyParamCaption}) const\hspace{0.3cm}{\ttfamily [inline]}, {\ttfamily [override]}, {\ttfamily [virtual]}}



Implements \doxymbox{\hyperlink{class_beam_1_1_flux_node_a17eb02187925b52bf8e53fa3ebe3da66}{Beam::\+\+Flux\+Node}}.

\Hypertarget{class_beam_1_1_flux_track_node_a3f89a7df54007845103280ec40bf7983}\index{Beam::FluxTrackNode@{Beam::FluxTrackNode}!getInternalNode@{getInternalNode}}
\index{getInternalNode@{getInternalNode}!Beam::FluxTrackNode@{Beam::FluxTrackNode}}
\doxysubsubsection{\texorpdfstring{getInternalNode()}{getInternalNode()}}
{\footnotesize\ttfamily \label{class_beam_1_1_flux_track_node_a3f89a7df54007845103280ec40bf7983} 
std::\+shared\+\_\+ptr$<$ \doxymbox{\hyperlink{class_beam_1_1_track_node}{Track\+Node}} $>$ Beam::\+\+Flux\+Track\+Node::\+get\+Internal\+Node (\begin{DoxyParamCaption}{}{}\end{DoxyParamCaption})\hspace{0.3cm}{\ttfamily [inline]}}

\Hypertarget{class_beam_1_1_flux_track_node_aeb9ce1b2d297d6c800bae5e0ed0fd70f}\index{Beam::FluxTrackNode@{Beam::FluxTrackNode}!getName@{getName}}
\index{getName@{getName}!Beam::FluxTrackNode@{Beam::FluxTrackNode}}
\doxysubsubsection{\texorpdfstring{getName()}{getName()}}
{\footnotesize\ttfamily \label{class_beam_1_1_flux_track_node_aeb9ce1b2d297d6c800bae5e0ed0fd70f} 
std::\+string Beam::\+\+Flux\+Track\+Node::\+get\+Name (\begin{DoxyParamCaption}{}{}\end{DoxyParamCaption}) const\hspace{0.3cm}{\ttfamily [inline]}, {\ttfamily [override]}, {\ttfamily [virtual]}}



Implements \doxymbox{\hyperlink{class_beam_1_1_flux_node_ac638d3d9bb1050d658294bc5470abeba}{Beam::\+\+Flux\+Node}}.

\Hypertarget{class_beam_1_1_flux_track_node_ac77568a2dcca757d7d51486cc27b6a2f}\index{Beam::FluxTrackNode@{Beam::FluxTrackNode}!getOutputPorts@{getOutputPorts}}
\index{getOutputPorts@{getOutputPorts}!Beam::FluxTrackNode@{Beam::FluxTrackNode}}
\doxysubsubsection{\texorpdfstring{getOutputPorts()}{getOutputPorts()}}
{\footnotesize\ttfamily \label{class_beam_1_1_flux_track_node_ac77568a2dcca757d7d51486cc27b6a2f} 
std::\+vector$<$ \doxymbox{\hyperlink{struct_beam_1_1_flux_node_1_1_port}{Port}} $>$ Beam::\+\+Flux\+Track\+Node::\+get\+Output\+Ports (\begin{DoxyParamCaption}{}{}\end{DoxyParamCaption}) const\hspace{0.3cm}{\ttfamily [inline]}, {\ttfamily [override]}, {\ttfamily [virtual]}}



Implements \doxymbox{\hyperlink{class_beam_1_1_flux_node_a034f59d236afd7901ed84090422e3279}{Beam::\+\+Flux\+Node}}.

\Hypertarget{class_beam_1_1_flux_track_node_aca365acdc5e93673587ad33ac29b1e05}\index{Beam::FluxTrackNode@{Beam::FluxTrackNode}!getPeakData@{getPeakData}}
\index{getPeakData@{getPeakData}!Beam::FluxTrackNode@{Beam::FluxTrackNode}}
\doxysubsubsection{\texorpdfstring{getPeakData()}{getPeakData()}}
{\footnotesize\ttfamily \label{class_beam_1_1_flux_track_node_aca365acdc5e93673587ad33ac29b1e05} 
std::\+vector$<$ std::\+vector$<$ float $>$ $>$ Beam::\+\+Flux\+Track\+Node::\+get\+Peak\+Data (\begin{DoxyParamCaption}\item[{int}]{num\+Points}{}\end{DoxyParamCaption})\hspace{0.3cm}{\ttfamily [inline]}}

\Hypertarget{class_beam_1_1_flux_track_node_aa518ca3fcc84c77991eee501f6db0896}\index{Beam::FluxTrackNode@{Beam::FluxTrackNode}!getState@{getState}}
\index{getState@{getState}!Beam::FluxTrackNode@{Beam::FluxTrackNode}}
\doxysubsubsection{\texorpdfstring{getState()}{getState()}}
{\footnotesize\ttfamily \label{class_beam_1_1_flux_track_node_aa518ca3fcc84c77991eee501f6db0896} 
\doxymbox{\hyperlink{namespace_beam_aaab526c0becfd931c9f29361daaf7e9f}{Track\+State}} Beam::\+\+Flux\+Track\+Node::\+get\+State (\begin{DoxyParamCaption}{}{}\end{DoxyParamCaption}) const\hspace{0.3cm}{\ttfamily [inline]}}

\Hypertarget{class_beam_1_1_flux_track_node_aa264ea0b55854969320c1e0a730982b4}\index{Beam::FluxTrackNode@{Beam::FluxTrackNode}!load@{load}}
\index{load@{load}!Beam::FluxTrackNode@{Beam::FluxTrackNode}}
\doxysubsubsection{\texorpdfstring{load()}{load()}}
{\footnotesize\ttfamily \label{class_beam_1_1_flux_track_node_aa264ea0b55854969320c1e0a730982b4} 
bool Beam::\+\+Flux\+Track\+Node::\+load (\begin{DoxyParamCaption}\item[{const std::\+string \&}]{file\+Path}{}\end{DoxyParamCaption})\hspace{0.3cm}{\ttfamily [inline]}}

\Hypertarget{class_beam_1_1_flux_track_node_a372900a79b264e873083189295b7acf3}\index{Beam::FluxTrackNode@{Beam::FluxTrackNode}!onTransportSeek@{onTransportSeek}}
\index{onTransportSeek@{onTransportSeek}!Beam::FluxTrackNode@{Beam::FluxTrackNode}}
\doxysubsubsection{\texorpdfstring{onTransportSeek()}{onTransportSeek()}}
{\footnotesize\ttfamily \label{class_beam_1_1_flux_track_node_a372900a79b264e873083189295b7acf3} 
void Beam::\+\+Flux\+Track\+Node::\+on\+Transport\+Seek (\begin{DoxyParamCaption}\item[{size\+\_\+t}]{frame}{}\end{DoxyParamCaption})\hspace{0.3cm}{\ttfamily [inline]}, {\ttfamily [override]}, {\ttfamily [virtual]}}



Responds to timeline seeking. 



Reimplemented from \doxymbox{\hyperlink{class_beam_1_1_flux_node_adc7c4e979bf27de5bfca66815ae97a67}{Beam::\+\+Flux\+Node}}.

\Hypertarget{class_beam_1_1_flux_track_node_a26450c6c0de5d8f7ec68837bdde18ef3}\index{Beam::FluxTrackNode@{Beam::FluxTrackNode}!onTransportStateChanged@{onTransportStateChanged}}
\index{onTransportStateChanged@{onTransportStateChanged}!Beam::FluxTrackNode@{Beam::FluxTrackNode}}
\doxysubsubsection{\texorpdfstring{onTransportStateChanged()}{onTransportStateChanged()}}
{\footnotesize\ttfamily \label{class_beam_1_1_flux_track_node_a26450c6c0de5d8f7ec68837bdde18ef3} 
void Beam::\+\+Flux\+Track\+Node::\+on\+Transport\+State\+Changed (\begin{DoxyParamCaption}\item[{bool}]{playing}{}\end{DoxyParamCaption})\hspace{0.3cm}{\ttfamily [inline]}, {\ttfamily [override]}, {\ttfamily [virtual]}}



Responds to global transport changes (Play/\+\+Pause). 



Reimplemented from \doxymbox{\hyperlink{class_beam_1_1_flux_node_ace8cc49479d8924d44bca5fd4cd955e2}{Beam::\+\+Flux\+Node}}.

\Hypertarget{class_beam_1_1_flux_track_node_ab29facb24a749834aff2aa3de3be437a}\index{Beam::FluxTrackNode@{Beam::FluxTrackNode}!process@{process}}
\index{process@{process}!Beam::FluxTrackNode@{Beam::FluxTrackNode}}
\doxysubsubsection{\texorpdfstring{process()}{process()}}
{\footnotesize\ttfamily \label{class_beam_1_1_flux_track_node_ab29facb24a749834aff2aa3de3be437a} 
void Beam::\+\+Flux\+Track\+Node::\+process (\begin{DoxyParamCaption}\item[{int}]{frames}{}\end{DoxyParamCaption})\hspace{0.3cm}{\ttfamily [inline]}, {\ttfamily [override]}, {\ttfamily [virtual]}}



Main audio processing method. Must be implemented by subclasses. 


\begin{DoxyParams}{Parameters}
{\em frames} & Number of frames to process in the current block. \\
\hline
\end{DoxyParams}


Implements \doxymbox{\hyperlink{class_beam_1_1_flux_node_a3c263446753fa7ae5ff6928ee57bcd4d}{Beam::\+\+Flux\+Node}}.

\Hypertarget{class_beam_1_1_flux_track_node_afbf932790f01f79b00b4d7895c39d7dc}\index{Beam::FluxTrackNode@{Beam::FluxTrackNode}!seek@{seek}}
\index{seek@{seek}!Beam::FluxTrackNode@{Beam::FluxTrackNode}}
\doxysubsubsection{\texorpdfstring{seek()}{seek()}}
{\footnotesize\ttfamily \label{class_beam_1_1_flux_track_node_afbf932790f01f79b00b4d7895c39d7dc} 
void Beam::\+\+Flux\+Track\+Node::\+seek (\begin{DoxyParamCaption}\item[{size\+\_\+t}]{frame}{}\end{DoxyParamCaption})\hspace{0.3cm}{\ttfamily [inline]}}

\Hypertarget{class_beam_1_1_flux_track_node_ae6446258757bce3c5f58ca7ae48f31b8}\index{Beam::FluxTrackNode@{Beam::FluxTrackNode}!setState@{setState}}
\index{setState@{setState}!Beam::FluxTrackNode@{Beam::FluxTrackNode}}
\doxysubsubsection{\texorpdfstring{setState()}{setState()}}
{\footnotesize\ttfamily \label{class_beam_1_1_flux_track_node_ae6446258757bce3c5f58ca7ae48f31b8} 
void Beam::\+\+Flux\+Track\+Node::\+set\+State (\begin{DoxyParamCaption}\item[{\doxymbox{\hyperlink{namespace_beam_aaab526c0becfd931c9f29361daaf7e9f}{Track\+State}}}]{state}{}\end{DoxyParamCaption})\hspace{0.3cm}{\ttfamily [inline]}}

\Hypertarget{class_beam_1_1_flux_track_node_ab59401ed18ae3cf933e7ca7a4644711b}\index{Beam::FluxTrackNode@{Beam::FluxTrackNode}!startRecording@{startRecording}}
\index{startRecording@{startRecording}!Beam::FluxTrackNode@{Beam::FluxTrackNode}}
\doxysubsubsection{\texorpdfstring{startRecording()}{startRecording()}}
{\footnotesize\ttfamily \label{class_beam_1_1_flux_track_node_ab59401ed18ae3cf933e7ca7a4644711b} 
bool Beam::\+\+Flux\+Track\+Node::\+start\+Recording (\begin{DoxyParamCaption}\item[{const std::\+string \&}]{file\+Path}{, }\item[{int}]{sample\+Rate}{}\end{DoxyParamCaption})\hspace{0.3cm}{\ttfamily [inline]}}

\Hypertarget{class_beam_1_1_flux_track_node_a6ca690381930f2a26b149fa9a4699ed3}\index{Beam::FluxTrackNode@{Beam::FluxTrackNode}!stopRecording@{stopRecording}}
\index{stopRecording@{stopRecording}!Beam::FluxTrackNode@{Beam::FluxTrackNode}}
\doxysubsubsection{\texorpdfstring{stopRecording()}{stopRecording()}}
{\footnotesize\ttfamily \label{class_beam_1_1_flux_track_node_a6ca690381930f2a26b149fa9a4699ed3} 
void Beam::\+\+Flux\+Track\+Node::\+stop\+Recording (\begin{DoxyParamCaption}{}{}\end{DoxyParamCaption})\hspace{0.3cm}{\ttfamily [inline]}}



\label{doc-variable-members}
\Hypertarget{class_beam_1_1_flux_track_node_doc-variable-members}
\doxysubsection{Member Data Documentation}
\Hypertarget{class_beam_1_1_flux_track_node_aae94e87d1c628a62f787bcd6147a6727}\index{Beam::FluxTrackNode@{Beam::FluxTrackNode}!m\_name@{m\_name}}
\index{m\_name@{m\_name}!Beam::FluxTrackNode@{Beam::FluxTrackNode}}
\doxysubsubsection{\texorpdfstring{m\_name}{m\_name}}
{\footnotesize\ttfamily \label{class_beam_1_1_flux_track_node_aae94e87d1c628a62f787bcd6147a6727} 
std::\+string Beam::\+\+Flux\+Track\+Node::\+m\+\_\+name\hspace{0.3cm}{\ttfamily [private]}}

\Hypertarget{class_beam_1_1_flux_track_node_ab61ce129ca13f351a616eaca6d9d042c}\index{Beam::FluxTrackNode@{Beam::FluxTrackNode}!m\_track@{m\_track}}
\index{m\_track@{m\_track}!Beam::FluxTrackNode@{Beam::FluxTrackNode}}
\doxysubsubsection{\texorpdfstring{m\_track}{m\_track}}
{\footnotesize\ttfamily \label{class_beam_1_1_flux_track_node_ab61ce129ca13f351a616eaca6d9d042c} 
std::\+shared\+\_\+ptr$<$\doxymbox{\hyperlink{class_beam_1_1_track_node}{Track\+Node}}$>$ Beam::\+\+Flux\+Track\+Node::\+m\+\_\+track\hspace{0.3cm}{\ttfamily [private]}}



The documentation for this class was generated from the following file:\+\begin{DoxyCompactItemize}
\item 
src/\+engine/\+\doxymbox{\hyperlink{flux__track__node_8hpp}{flux\+\_\+track\+\_\+node.\+hpp}}\end{DoxyCompactItemize}

\doxysection{Beam::\+Gui\+Component Class Reference}
\hypertarget{class_beam_1_1_gui_component}{}\label{class_beam_1_1_gui_component}\index{Beam::GuiComponent@{Beam::GuiComponent}}


Base class for GUI components, similar to JUCE\textquotesingle{}s \doxylink{class_beam_1_1_component}{Component}.  




{\ttfamily \+\#include $<$gui\+\_\+component.\+hpp$>$}

Inheritance diagram for Beam::\+Gui\+Component:\+\begin{figure}[H]
\begin{center}
\leavevmode
\includegraphics[height=2.000000cm]{class_beam_1_1_gui_component}
\end{center}
\end{figure}
\doxysubsubsection*{Public Member Functions}
\begin{DoxyCompactItemize}
\item 
\doxymbox{\hyperlink{class_beam_1_1_gui_component_a3a2d448a1f30616384f16a41102c613a}{Gui\+Component}} ()
\item 
virtual \doxymbox{\hyperlink{class_beam_1_1_gui_component_a255a25fc9e1a2abdfdfcb7c11da1e66e}{\texorpdfstring{$\sim$}{\string~}\+Gui\+Component}} ()
\item 
virtual void \doxymbox{\hyperlink{class_beam_1_1_gui_component_a6ecaca5ca83fc45c232093a1012191d2}{set\+Bounds}} (float x, float y, float \doxymbox{\hyperlink{texture_8cpp_a8710f3c5c66c09e158c8619b3fca614a}{width}}, float \doxymbox{\hyperlink{texture_8cpp_a1055637f17e35a0ca82b396bb94914e5}{height}})
\begin{DoxyCompactList}\small\item\em Sets the bounds of this component. \end{DoxyCompactList}\item 
const \doxymbox{\hyperlink{struct_beam_1_1_rect}{Rect}} \& \doxymbox{\hyperlink{class_beam_1_1_gui_component_a2de58576d7e55fd82acbdb2f18207982}{get\+Bounds}} () const
\begin{DoxyCompactList}\small\item\em Gets the bounds of this component. \end{DoxyCompactList}\item 
void \doxymbox{\hyperlink{class_beam_1_1_gui_component_abc37535b6bef8b996c41849df8c95e96}{set\+Visible}} (bool should\+Be\+Visible)
\begin{DoxyCompactList}\small\item\em Sets the component\textquotesingle{}s visibility. \end{DoxyCompactList}\item 
bool \doxymbox{\hyperlink{class_beam_1_1_gui_component_ae4752b8375293063779e88a83c1a799a}{is\+Visible}} () const
\begin{DoxyCompactList}\small\item\em Checks if the component is visible. \end{DoxyCompactList}\item 
virtual void \doxymbox{\hyperlink{class_beam_1_1_gui_component_afd0b01a0cf776f3e5746622e7a4e7c5c}{paint}} (\doxymbox{\hyperlink{class_beam_1_1_quad_batcher}{Quad\+Batcher}} \&g)
\begin{DoxyCompactList}\small\item\em Called when the component needs to be painted. \end{DoxyCompactList}\item 
virtual void \doxymbox{\hyperlink{class_beam_1_1_gui_component_a683989837c3ab83ffb1148e0b473e573}{resized}} ()
\begin{DoxyCompactList}\small\item\em Called when the component\textquotesingle{}s size changes. \end{DoxyCompactList}\item 
virtual void \doxymbox{\hyperlink{class_beam_1_1_gui_component_a357d8829f546299e6fda6ff119c582c1}{mouse\+Enter}} (const \doxymbox{\hyperlink{class_beam_1_1_mouse_event}{Mouse\+Event}} \&event)
\begin{DoxyCompactList}\small\item\em Called when the mouse enters the component. \end{DoxyCompactList}\item 
virtual void \doxymbox{\hyperlink{class_beam_1_1_gui_component_a9c580600fa3396a3085b2c7ca9bd6869}{mouse\+Exit}} (const \doxymbox{\hyperlink{class_beam_1_1_mouse_event}{Mouse\+Event}} \&event)
\begin{DoxyCompactList}\small\item\em Called when the mouse exits the component. \end{DoxyCompactList}\item 
virtual void \doxymbox{\hyperlink{class_beam_1_1_gui_component_a3f1bac930389048f3ab5217ece24e032}{mouse\+Down}} (const \doxymbox{\hyperlink{class_beam_1_1_mouse_event}{Mouse\+Event}} \&event)
\begin{DoxyCompactList}\small\item\em Called when a mouse button is pressed. \end{DoxyCompactList}\item 
virtual void \doxymbox{\hyperlink{class_beam_1_1_gui_component_aeaa0f5b76f80ee669de3a9bf06158a04}{mouse\+Up}} (const \doxymbox{\hyperlink{class_beam_1_1_mouse_event}{Mouse\+Event}} \&event)
\begin{DoxyCompactList}\small\item\em Called when a mouse button is released. \end{DoxyCompactList}\item 
virtual void \doxymbox{\hyperlink{class_beam_1_1_gui_component_abb1547968352dd7ed45ec761e4ceac07}{mouse\+Move}} (const \doxymbox{\hyperlink{class_beam_1_1_mouse_event}{Mouse\+Event}} \&event)
\begin{DoxyCompactList}\small\item\em Called when the mouse is moved. \end{DoxyCompactList}\item 
virtual void \doxymbox{\hyperlink{class_beam_1_1_gui_component_a5f9e28ec7aea30422982fcb748bd54c2}{mouse\+Drag}} (const \doxymbox{\hyperlink{class_beam_1_1_mouse_event}{Mouse\+Event}} \&event)
\begin{DoxyCompactList}\small\item\em Called when the mouse is dragged. \end{DoxyCompactList}\item 
virtual bool \doxymbox{\hyperlink{class_beam_1_1_gui_component_a2459c6228fcbc914614e6252a43b016f}{key\+Pressed}} (const \doxymbox{\hyperlink{class_beam_1_1_key_press}{Key\+Press}} \&key)
\begin{DoxyCompactList}\small\item\em Called when a key is pressed. \end{DoxyCompactList}\item 
void \doxymbox{\hyperlink{class_beam_1_1_gui_component_af85677b7e220a2b1bf69659d4167bce5}{add\+Child\+Component}} (std::\+shared\+\_\+ptr$<$ \doxymbox{\hyperlink{class_beam_1_1_gui_component_a3a2d448a1f30616384f16a41102c613a}{Gui\+Component}} $>$ child)
\begin{DoxyCompactList}\small\item\em Adds a child component. \end{DoxyCompactList}\item 
void \doxymbox{\hyperlink{class_beam_1_1_gui_component_a2b68395b5a1ebdfe673053b2aa83122b}{remove\+Child\+Component}} (std::\+shared\+\_\+ptr$<$ \doxymbox{\hyperlink{class_beam_1_1_gui_component_a3a2d448a1f30616384f16a41102c613a}{Gui\+Component}} $>$ child)
\begin{DoxyCompactList}\small\item\em Removes a child component. \end{DoxyCompactList}\item 
void \doxymbox{\hyperlink{class_beam_1_1_gui_component_a7dbd4e5e0b8955748fe0962bcb476089}{paint\+Entire\+Component}} (\doxymbox{\hyperlink{class_beam_1_1_quad_batcher}{Quad\+Batcher}} \&g)
\begin{DoxyCompactList}\small\item\em Paints this component and all its children. \end{DoxyCompactList}\item 
bool \doxymbox{\hyperlink{class_beam_1_1_gui_component_abad300b88267731a68afd9a94e5225c4}{contains}} (float x, float y) const
\begin{DoxyCompactList}\small\item\em Checks if a point is inside this component. \end{DoxyCompactList}\item 
void \doxymbox{\hyperlink{class_beam_1_1_gui_component_ae7e5d9dde8c3bae347fdaaac29b5d96a}{set\+Name}} (const std::\+string \&name)
\begin{DoxyCompactList}\small\item\em Sets the component\textquotesingle{}s name. \end{DoxyCompactList}\item 
const std::\+string \& \doxymbox{\hyperlink{class_beam_1_1_gui_component_ac21fae6abb5616da7b0882c924961ae9}{get\+Name}} () const
\begin{DoxyCompactList}\small\item\em Gets the component\textquotesingle{}s name. \end{DoxyCompactList}\end{DoxyCompactItemize}
\doxysubsubsection*{Protected Attributes}
\begin{DoxyCompactItemize}
\item 
\doxymbox{\hyperlink{struct_beam_1_1_rect}{Rect}} \doxymbox{\hyperlink{class_beam_1_1_gui_component_ad3c42f55d6a7e47f65bd1d0f3ffb5291}{m\+\_\+bounds}} \{0, 0, 0, 0\}
\item 
bool \doxymbox{\hyperlink{class_beam_1_1_gui_component_afbfe066e00bfffc064d80082c839ebe6}{m\+\_\+visible}} = true
\item 
std::\+vector$<$ std::\+shared\+\_\+ptr$<$ \doxymbox{\hyperlink{class_beam_1_1_gui_component_a3a2d448a1f30616384f16a41102c613a}{Gui\+Component}} $>$ $>$ \doxymbox{\hyperlink{class_beam_1_1_gui_component_ada191fb579394c85ec4fa977b1b108b0}{m\+\_\+children}}
\item 
std::\+string \doxymbox{\hyperlink{class_beam_1_1_gui_component_a5e462cb9a299d364cb53b1a768369246}{m\+\_\+name}}
\item 
std::\+function$<$ void()$>$ \doxymbox{\hyperlink{class_beam_1_1_gui_component_a46e1692a3cd91b4464cce1680c759a1e}{m\+\_\+paint\+Callback}}
\item 
std::\+function$<$ void()$>$ \doxymbox{\hyperlink{class_beam_1_1_gui_component_a5ea91ef234ce12ce8dc7fcc143178072}{m\+\_\+resized\+Callback}}
\end{DoxyCompactItemize}


\doxysubsection{Detailed Description}
Base class for GUI components, similar to JUCE\textquotesingle{}s \doxylink{class_beam_1_1_component}{Component}. 

\label{doc-constructors}
\Hypertarget{class_beam_1_1_gui_component_doc-constructors}
\doxysubsection{Constructor \& Destructor Documentation}
\Hypertarget{class_beam_1_1_gui_component_a3a2d448a1f30616384f16a41102c613a}\index{Beam::GuiComponent@{Beam::GuiComponent}!GuiComponent@{GuiComponent}}
\index{GuiComponent@{GuiComponent}!Beam::GuiComponent@{Beam::GuiComponent}}
\doxysubsubsection{\texorpdfstring{GuiComponent()}{GuiComponent()}}
{\footnotesize\ttfamily \label{class_beam_1_1_gui_component_a3a2d448a1f30616384f16a41102c613a} 
Beam::\+\+Gui\+Component::\+\+Gui\+Component (\begin{DoxyParamCaption}{}{}\end{DoxyParamCaption})}

\Hypertarget{class_beam_1_1_gui_component_a255a25fc9e1a2abdfdfcb7c11da1e66e}\index{Beam::GuiComponent@{Beam::GuiComponent}!````~GuiComponent@{\texorpdfstring{$\sim$}{\string~}GuiComponent}}
\index{````~GuiComponent@{\texorpdfstring{$\sim$}{\string~}GuiComponent}!Beam::GuiComponent@{Beam::GuiComponent}}
\doxysubsubsection{\texorpdfstring{\texorpdfstring{$\sim$}{\string~}GuiComponent()}{\string~GuiComponent()}}
{\footnotesize\ttfamily \label{class_beam_1_1_gui_component_a255a25fc9e1a2abdfdfcb7c11da1e66e} 
Beam::\+\+Gui\+Component::\+\texorpdfstring{$\sim$}{\string~}\+Gui\+Component (\begin{DoxyParamCaption}{}{}\end{DoxyParamCaption})\hspace{0.3cm}{\ttfamily [virtual]}}



\label{doc-func-members}
\Hypertarget{class_beam_1_1_gui_component_doc-func-members}
\doxysubsection{Member Function Documentation}
\Hypertarget{class_beam_1_1_gui_component_af85677b7e220a2b1bf69659d4167bce5}\index{Beam::GuiComponent@{Beam::GuiComponent}!addChildComponent@{addChildComponent}}
\index{addChildComponent@{addChildComponent}!Beam::GuiComponent@{Beam::GuiComponent}}
\doxysubsubsection{\texorpdfstring{addChildComponent()}{addChildComponent()}}
{\footnotesize\ttfamily \label{class_beam_1_1_gui_component_af85677b7e220a2b1bf69659d4167bce5} 
void Beam::\+\+Gui\+Component::\+add\+Child\+Component (\begin{DoxyParamCaption}\item[{std::\+shared\+\_\+ptr$<$ \doxymbox{\hyperlink{class_beam_1_1_gui_component_a3a2d448a1f30616384f16a41102c613a}{Gui\+Component}} $>$}]{child}{}\end{DoxyParamCaption})}



Adds a child component. 

\Hypertarget{class_beam_1_1_gui_component_abad300b88267731a68afd9a94e5225c4}\index{Beam::GuiComponent@{Beam::GuiComponent}!contains@{contains}}
\index{contains@{contains}!Beam::GuiComponent@{Beam::GuiComponent}}
\doxysubsubsection{\texorpdfstring{contains()}{contains()}}
{\footnotesize\ttfamily \label{class_beam_1_1_gui_component_abad300b88267731a68afd9a94e5225c4} 
bool Beam::\+\+Gui\+Component::\+contains (\begin{DoxyParamCaption}\item[{float}]{x}{, }\item[{float}]{y}{}\end{DoxyParamCaption}) const}



Checks if a point is inside this component. 

\Hypertarget{class_beam_1_1_gui_component_a2de58576d7e55fd82acbdb2f18207982}\index{Beam::GuiComponent@{Beam::GuiComponent}!getBounds@{getBounds}}
\index{getBounds@{getBounds}!Beam::GuiComponent@{Beam::GuiComponent}}
\doxysubsubsection{\texorpdfstring{getBounds()}{getBounds()}}
{\footnotesize\ttfamily \label{class_beam_1_1_gui_component_a2de58576d7e55fd82acbdb2f18207982} 
const \doxymbox{\hyperlink{struct_beam_1_1_rect}{Rect}} \& Beam::\+\+Gui\+Component::\+get\+Bounds (\begin{DoxyParamCaption}{}{}\end{DoxyParamCaption}) const\hspace{0.3cm}{\ttfamily [inline]}}



Gets the bounds of this component. 

\Hypertarget{class_beam_1_1_gui_component_ac21fae6abb5616da7b0882c924961ae9}\index{Beam::GuiComponent@{Beam::GuiComponent}!getName@{getName}}
\index{getName@{getName}!Beam::GuiComponent@{Beam::GuiComponent}}
\doxysubsubsection{\texorpdfstring{getName()}{getName()}}
{\footnotesize\ttfamily \label{class_beam_1_1_gui_component_ac21fae6abb5616da7b0882c924961ae9} 
const std::\+string \& Beam::\+\+Gui\+Component::\+get\+Name (\begin{DoxyParamCaption}{}{}\end{DoxyParamCaption}) const\hspace{0.3cm}{\ttfamily [inline]}}



Gets the component\textquotesingle{}s name. 

\Hypertarget{class_beam_1_1_gui_component_ae4752b8375293063779e88a83c1a799a}\index{Beam::GuiComponent@{Beam::GuiComponent}!isVisible@{isVisible}}
\index{isVisible@{isVisible}!Beam::GuiComponent@{Beam::GuiComponent}}
\doxysubsubsection{\texorpdfstring{isVisible()}{isVisible()}}
{\footnotesize\ttfamily \label{class_beam_1_1_gui_component_ae4752b8375293063779e88a83c1a799a} 
bool Beam::\+\+Gui\+Component::\+is\+Visible (\begin{DoxyParamCaption}{}{}\end{DoxyParamCaption}) const\hspace{0.3cm}{\ttfamily [inline]}}



Checks if the component is visible. 

\Hypertarget{class_beam_1_1_gui_component_a2459c6228fcbc914614e6252a43b016f}\index{Beam::GuiComponent@{Beam::GuiComponent}!keyPressed@{keyPressed}}
\index{keyPressed@{keyPressed}!Beam::GuiComponent@{Beam::GuiComponent}}
\doxysubsubsection{\texorpdfstring{keyPressed()}{keyPressed()}}
{\footnotesize\ttfamily \label{class_beam_1_1_gui_component_a2459c6228fcbc914614e6252a43b016f} 
bool Beam::\+\+Gui\+Component::\+key\+Pressed (\begin{DoxyParamCaption}\item[{const \doxymbox{\hyperlink{class_beam_1_1_key_press}{Key\+Press}} \&}]{key}{}\end{DoxyParamCaption})\hspace{0.3cm}{\ttfamily [virtual]}}



Called when a key is pressed. 

\Hypertarget{class_beam_1_1_gui_component_a3f1bac930389048f3ab5217ece24e032}\index{Beam::GuiComponent@{Beam::GuiComponent}!mouseDown@{mouseDown}}
\index{mouseDown@{mouseDown}!Beam::GuiComponent@{Beam::GuiComponent}}
\doxysubsubsection{\texorpdfstring{mouseDown()}{mouseDown()}}
{\footnotesize\ttfamily \label{class_beam_1_1_gui_component_a3f1bac930389048f3ab5217ece24e032} 
void Beam::\+\+Gui\+Component::\+mouse\+Down (\begin{DoxyParamCaption}\item[{const \doxymbox{\hyperlink{class_beam_1_1_mouse_event}{Mouse\+Event}} \&}]{event}{}\end{DoxyParamCaption})\hspace{0.3cm}{\ttfamily [virtual]}}



Called when a mouse button is pressed. 



Reimplemented in \doxymbox{\hyperlink{class_beam_1_1_button_a82487a148cdbd4f66bd401320b5032a2}{Beam::\+\+Button}}, and \doxymbox{\hyperlink{class_beam_1_1_slider_a4a2744e42c43c6568720baa27ecdffeb}{Beam::\+\+Slider}}.

\Hypertarget{class_beam_1_1_gui_component_a5f9e28ec7aea30422982fcb748bd54c2}\index{Beam::GuiComponent@{Beam::GuiComponent}!mouseDrag@{mouseDrag}}
\index{mouseDrag@{mouseDrag}!Beam::GuiComponent@{Beam::GuiComponent}}
\doxysubsubsection{\texorpdfstring{mouseDrag()}{mouseDrag()}}
{\footnotesize\ttfamily \label{class_beam_1_1_gui_component_a5f9e28ec7aea30422982fcb748bd54c2} 
void Beam::\+\+Gui\+Component::\+mouse\+Drag (\begin{DoxyParamCaption}\item[{const \doxymbox{\hyperlink{class_beam_1_1_mouse_event}{Mouse\+Event}} \&}]{event}{}\end{DoxyParamCaption})\hspace{0.3cm}{\ttfamily [virtual]}}



Called when the mouse is dragged. 



Reimplemented in \doxymbox{\hyperlink{class_beam_1_1_slider_adb038e50182312cdb6d1d96e60f6dfea}{Beam::\+\+Slider}}.

\Hypertarget{class_beam_1_1_gui_component_a357d8829f546299e6fda6ff119c582c1}\index{Beam::GuiComponent@{Beam::GuiComponent}!mouseEnter@{mouseEnter}}
\index{mouseEnter@{mouseEnter}!Beam::GuiComponent@{Beam::GuiComponent}}
\doxysubsubsection{\texorpdfstring{mouseEnter()}{mouseEnter()}}
{\footnotesize\ttfamily \label{class_beam_1_1_gui_component_a357d8829f546299e6fda6ff119c582c1} 
void Beam::\+\+Gui\+Component::\+mouse\+Enter (\begin{DoxyParamCaption}\item[{const \doxymbox{\hyperlink{class_beam_1_1_mouse_event}{Mouse\+Event}} \&}]{event}{}\end{DoxyParamCaption})\hspace{0.3cm}{\ttfamily [virtual]}}



Called when the mouse enters the component. 



Reimplemented in \doxymbox{\hyperlink{class_beam_1_1_button_acd3724e0cdc77bcf9c25d1958dd9c5c3}{Beam::\+\+Button}}.

\Hypertarget{class_beam_1_1_gui_component_a9c580600fa3396a3085b2c7ca9bd6869}\index{Beam::GuiComponent@{Beam::GuiComponent}!mouseExit@{mouseExit}}
\index{mouseExit@{mouseExit}!Beam::GuiComponent@{Beam::GuiComponent}}
\doxysubsubsection{\texorpdfstring{mouseExit()}{mouseExit()}}
{\footnotesize\ttfamily \label{class_beam_1_1_gui_component_a9c580600fa3396a3085b2c7ca9bd6869} 
void Beam::\+\+Gui\+Component::\+mouse\+Exit (\begin{DoxyParamCaption}\item[{const \doxymbox{\hyperlink{class_beam_1_1_mouse_event}{Mouse\+Event}} \&}]{event}{}\end{DoxyParamCaption})\hspace{0.3cm}{\ttfamily [virtual]}}



Called when the mouse exits the component. 



Reimplemented in \doxymbox{\hyperlink{class_beam_1_1_button_a8199f40b2ff5391df56fbe339649c4cb}{Beam::\+\+Button}}.

\Hypertarget{class_beam_1_1_gui_component_abb1547968352dd7ed45ec761e4ceac07}\index{Beam::GuiComponent@{Beam::GuiComponent}!mouseMove@{mouseMove}}
\index{mouseMove@{mouseMove}!Beam::GuiComponent@{Beam::GuiComponent}}
\doxysubsubsection{\texorpdfstring{mouseMove()}{mouseMove()}}
{\footnotesize\ttfamily \label{class_beam_1_1_gui_component_abb1547968352dd7ed45ec761e4ceac07} 
void Beam::\+\+Gui\+Component::\+mouse\+Move (\begin{DoxyParamCaption}\item[{const \doxymbox{\hyperlink{class_beam_1_1_mouse_event}{Mouse\+Event}} \&}]{event}{}\end{DoxyParamCaption})\hspace{0.3cm}{\ttfamily [virtual]}}



Called when the mouse is moved. 

\Hypertarget{class_beam_1_1_gui_component_aeaa0f5b76f80ee669de3a9bf06158a04}\index{Beam::GuiComponent@{Beam::GuiComponent}!mouseUp@{mouseUp}}
\index{mouseUp@{mouseUp}!Beam::GuiComponent@{Beam::GuiComponent}}
\doxysubsubsection{\texorpdfstring{mouseUp()}{mouseUp()}}
{\footnotesize\ttfamily \label{class_beam_1_1_gui_component_aeaa0f5b76f80ee669de3a9bf06158a04} 
void Beam::\+\+Gui\+Component::\+mouse\+Up (\begin{DoxyParamCaption}\item[{const \doxymbox{\hyperlink{class_beam_1_1_mouse_event}{Mouse\+Event}} \&}]{event}{}\end{DoxyParamCaption})\hspace{0.3cm}{\ttfamily [virtual]}}



Called when a mouse button is released. 



Reimplemented in \doxymbox{\hyperlink{class_beam_1_1_button_a19bc963ab2ef912af08bcf75701b25b8}{Beam::\+\+Button}}, and \doxymbox{\hyperlink{class_beam_1_1_slider_a454bd346772535c3fa29e7240fc2166e}{Beam::\+\+Slider}}.

\Hypertarget{class_beam_1_1_gui_component_afd0b01a0cf776f3e5746622e7a4e7c5c}\index{Beam::GuiComponent@{Beam::GuiComponent}!paint@{paint}}
\index{paint@{paint}!Beam::GuiComponent@{Beam::GuiComponent}}
\doxysubsubsection{\texorpdfstring{paint()}{paint()}}
{\footnotesize\ttfamily \label{class_beam_1_1_gui_component_afd0b01a0cf776f3e5746622e7a4e7c5c} 
void Beam::\+\+Gui\+Component::\+paint (\begin{DoxyParamCaption}\item[{\doxymbox{\hyperlink{class_beam_1_1_quad_batcher}{Quad\+Batcher}} \&}]{g}{}\end{DoxyParamCaption})\hspace{0.3cm}{\ttfamily [virtual]}}



Called when the component needs to be painted. 



Reimplemented in \doxymbox{\hyperlink{class_beam_1_1_button_a5a5ec975b5bac74a85126cfe41563a26}{Beam::\+\+Button}}, and \doxymbox{\hyperlink{class_beam_1_1_slider_a6be4631112646615e8d4f7f722fea46b}{Beam::\+\+Slider}}.

\Hypertarget{class_beam_1_1_gui_component_a7dbd4e5e0b8955748fe0962bcb476089}\index{Beam::GuiComponent@{Beam::GuiComponent}!paintEntireComponent@{paintEntireComponent}}
\index{paintEntireComponent@{paintEntireComponent}!Beam::GuiComponent@{Beam::GuiComponent}}
\doxysubsubsection{\texorpdfstring{paintEntireComponent()}{paintEntireComponent()}}
{\footnotesize\ttfamily \label{class_beam_1_1_gui_component_a7dbd4e5e0b8955748fe0962bcb476089} 
void Beam::\+\+Gui\+Component::\+paint\+Entire\+Component (\begin{DoxyParamCaption}\item[{\doxymbox{\hyperlink{class_beam_1_1_quad_batcher}{Quad\+Batcher}} \&}]{g}{}\end{DoxyParamCaption})}



Paints this component and all its children. 

\Hypertarget{class_beam_1_1_gui_component_a2b68395b5a1ebdfe673053b2aa83122b}\index{Beam::GuiComponent@{Beam::GuiComponent}!removeChildComponent@{removeChildComponent}}
\index{removeChildComponent@{removeChildComponent}!Beam::GuiComponent@{Beam::GuiComponent}}
\doxysubsubsection{\texorpdfstring{removeChildComponent()}{removeChildComponent()}}
{\footnotesize\ttfamily \label{class_beam_1_1_gui_component_a2b68395b5a1ebdfe673053b2aa83122b} 
void Beam::\+\+Gui\+Component::\+remove\+Child\+Component (\begin{DoxyParamCaption}\item[{std::\+shared\+\_\+ptr$<$ \doxymbox{\hyperlink{class_beam_1_1_gui_component_a3a2d448a1f30616384f16a41102c613a}{Gui\+Component}} $>$}]{child}{}\end{DoxyParamCaption})}



Removes a child component. 

\Hypertarget{class_beam_1_1_gui_component_a683989837c3ab83ffb1148e0b473e573}\index{Beam::GuiComponent@{Beam::GuiComponent}!resized@{resized}}
\index{resized@{resized}!Beam::GuiComponent@{Beam::GuiComponent}}
\doxysubsubsection{\texorpdfstring{resized()}{resized()}}
{\footnotesize\ttfamily \label{class_beam_1_1_gui_component_a683989837c3ab83ffb1148e0b473e573} 
void Beam::\+\+Gui\+Component::\+resized (\begin{DoxyParamCaption}{}{}\end{DoxyParamCaption})\hspace{0.3cm}{\ttfamily [virtual]}}



Called when the component\textquotesingle{}s size changes. 

\Hypertarget{class_beam_1_1_gui_component_a6ecaca5ca83fc45c232093a1012191d2}\index{Beam::GuiComponent@{Beam::GuiComponent}!setBounds@{setBounds}}
\index{setBounds@{setBounds}!Beam::GuiComponent@{Beam::GuiComponent}}
\doxysubsubsection{\texorpdfstring{setBounds()}{setBounds()}}
{\footnotesize\ttfamily \label{class_beam_1_1_gui_component_a6ecaca5ca83fc45c232093a1012191d2} 
void Beam::\+\+Gui\+Component::\+set\+Bounds (\begin{DoxyParamCaption}\item[{float}]{x}{, }\item[{float}]{y}{, }\item[{float}]{width}{, }\item[{float}]{height}{}\end{DoxyParamCaption})\hspace{0.3cm}{\ttfamily [virtual]}}



Sets the bounds of this component. 

\Hypertarget{class_beam_1_1_gui_component_ae7e5d9dde8c3bae347fdaaac29b5d96a}\index{Beam::GuiComponent@{Beam::GuiComponent}!setName@{setName}}
\index{setName@{setName}!Beam::GuiComponent@{Beam::GuiComponent}}
\doxysubsubsection{\texorpdfstring{setName()}{setName()}}
{\footnotesize\ttfamily \label{class_beam_1_1_gui_component_ae7e5d9dde8c3bae347fdaaac29b5d96a} 
void Beam::\+\+Gui\+Component::\+set\+Name (\begin{DoxyParamCaption}\item[{const std::\+string \&}]{name}{}\end{DoxyParamCaption})\hspace{0.3cm}{\ttfamily [inline]}}



Sets the component\textquotesingle{}s name. 

\Hypertarget{class_beam_1_1_gui_component_abc37535b6bef8b996c41849df8c95e96}\index{Beam::GuiComponent@{Beam::GuiComponent}!setVisible@{setVisible}}
\index{setVisible@{setVisible}!Beam::GuiComponent@{Beam::GuiComponent}}
\doxysubsubsection{\texorpdfstring{setVisible()}{setVisible()}}
{\footnotesize\ttfamily \label{class_beam_1_1_gui_component_abc37535b6bef8b996c41849df8c95e96} 
void Beam::\+\+Gui\+Component::\+set\+Visible (\begin{DoxyParamCaption}\item[{bool}]{should\+Be\+Visible}{}\end{DoxyParamCaption})}



Sets the component\textquotesingle{}s visibility. 



\label{doc-variable-members}
\Hypertarget{class_beam_1_1_gui_component_doc-variable-members}
\doxysubsection{Member Data Documentation}
\Hypertarget{class_beam_1_1_gui_component_ad3c42f55d6a7e47f65bd1d0f3ffb5291}\index{Beam::GuiComponent@{Beam::GuiComponent}!m\_bounds@{m\_bounds}}
\index{m\_bounds@{m\_bounds}!Beam::GuiComponent@{Beam::GuiComponent}}
\doxysubsubsection{\texorpdfstring{m\_bounds}{m\_bounds}}
{\footnotesize\ttfamily \label{class_beam_1_1_gui_component_ad3c42f55d6a7e47f65bd1d0f3ffb5291} 
\doxymbox{\hyperlink{struct_beam_1_1_rect}{Rect}} Beam::\+\+Gui\+Component::\+m\+\_\+bounds \{0, 0, 0, 0\}\hspace{0.3cm}{\ttfamily [protected]}}

\Hypertarget{class_beam_1_1_gui_component_ada191fb579394c85ec4fa977b1b108b0}\index{Beam::GuiComponent@{Beam::GuiComponent}!m\_children@{m\_children}}
\index{m\_children@{m\_children}!Beam::GuiComponent@{Beam::GuiComponent}}
\doxysubsubsection{\texorpdfstring{m\_children}{m\_children}}
{\footnotesize\ttfamily \label{class_beam_1_1_gui_component_ada191fb579394c85ec4fa977b1b108b0} 
std::\+vector$<$std::\+shared\+\_\+ptr$<$\doxymbox{\hyperlink{class_beam_1_1_gui_component_a3a2d448a1f30616384f16a41102c613a}{Gui\+Component}}$>$ $>$ Beam::\+\+Gui\+Component::\+m\+\_\+children\hspace{0.3cm}{\ttfamily [protected]}}

\Hypertarget{class_beam_1_1_gui_component_a5e462cb9a299d364cb53b1a768369246}\index{Beam::GuiComponent@{Beam::GuiComponent}!m\_name@{m\_name}}
\index{m\_name@{m\_name}!Beam::GuiComponent@{Beam::GuiComponent}}
\doxysubsubsection{\texorpdfstring{m\_name}{m\_name}}
{\footnotesize\ttfamily \label{class_beam_1_1_gui_component_a5e462cb9a299d364cb53b1a768369246} 
std::\+string Beam::\+\+Gui\+Component::\+m\+\_\+name\hspace{0.3cm}{\ttfamily [protected]}}

\Hypertarget{class_beam_1_1_gui_component_a46e1692a3cd91b4464cce1680c759a1e}\index{Beam::GuiComponent@{Beam::GuiComponent}!m\_paintCallback@{m\_paintCallback}}
\index{m\_paintCallback@{m\_paintCallback}!Beam::GuiComponent@{Beam::GuiComponent}}
\doxysubsubsection{\texorpdfstring{m\_paintCallback}{m\_paintCallback}}
{\footnotesize\ttfamily \label{class_beam_1_1_gui_component_a46e1692a3cd91b4464cce1680c759a1e} 
std::\+function$<$void()$>$ Beam::\+\+Gui\+Component::\+m\+\_\+paint\+Callback\hspace{0.3cm}{\ttfamily [protected]}}

\Hypertarget{class_beam_1_1_gui_component_a5ea91ef234ce12ce8dc7fcc143178072}\index{Beam::GuiComponent@{Beam::GuiComponent}!m\_resizedCallback@{m\_resizedCallback}}
\index{m\_resizedCallback@{m\_resizedCallback}!Beam::GuiComponent@{Beam::GuiComponent}}
\doxysubsubsection{\texorpdfstring{m\_resizedCallback}{m\_resizedCallback}}
{\footnotesize\ttfamily \label{class_beam_1_1_gui_component_a5ea91ef234ce12ce8dc7fcc143178072} 
std::\+function$<$void()$>$ Beam::\+\+Gui\+Component::\+m\+\_\+resized\+Callback\hspace{0.3cm}{\ttfamily [protected]}}

\Hypertarget{class_beam_1_1_gui_component_afbfe066e00bfffc064d80082c839ebe6}\index{Beam::GuiComponent@{Beam::GuiComponent}!m\_visible@{m\_visible}}
\index{m\_visible@{m\_visible}!Beam::GuiComponent@{Beam::GuiComponent}}
\doxysubsubsection{\texorpdfstring{m\_visible}{m\_visible}}
{\footnotesize\ttfamily \label{class_beam_1_1_gui_component_afbfe066e00bfffc064d80082c839ebe6} 
bool Beam::\+\+Gui\+Component::\+m\+\_\+visible = true\hspace{0.3cm}{\ttfamily [protected]}}



The documentation for this class was generated from the following files:\+\begin{DoxyCompactItemize}
\item 
src/\+interface/\+\doxymbox{\hyperlink{gui__component_8hpp}{gui\+\_\+component.\+hpp}}\item 
src/\+interface/\+\doxymbox{\hyperlink{gui__component_8cpp}{gui\+\_\+component.\+cpp}}\end{DoxyCompactItemize}

\doxysection{Beam::\+Input\+Handler Class Reference}
\hypertarget{class_beam_1_1_input_handler}{}\label{class_beam_1_1_input_handler}\index{Beam::InputHandler@{Beam::InputHandler}}


{\ttfamily \+\#include $<$input\+\_\+handler.\+hpp$>$}

\doxysubsubsection*{Public Member Functions}
\begin{DoxyCompactItemize}
\item 
void \doxymbox{\hyperlink{class_beam_1_1_input_handler_a1d3cb50866d2f151e5b9ddffdd1f0c01}{add\+Component}} (std::\+shared\+\_\+ptr$<$ \doxymbox{\hyperlink{class_beam_1_1_component}{Component}} $>$ component)
\item 
void \doxymbox{\hyperlink{class_beam_1_1_input_handler_a89d5254f25fa9d103a8b1a05332d5a5f}{handle\+Mouse\+Down}} (float x, float y, int button)
\item 
void \doxymbox{\hyperlink{class_beam_1_1_input_handler_a200720149cc210bc1b2fae9f02872e86}{handle\+Mouse\+Up}} (float x, float y, int button)
\item 
void \doxymbox{\hyperlink{class_beam_1_1_input_handler_ab6ae366e86174358931c66a97f002f96}{handle\+Mouse\+Move}} (float x, float y)
\item 
void \doxymbox{\hyperlink{class_beam_1_1_input_handler_acd93a9973f465ad5a6d9f0ccdc72636a}{update}} (float dt)
\item 
void \doxymbox{\hyperlink{class_beam_1_1_input_handler_a2d82861b1db962c08f7bf011fe6bc8ef}{render}} (\doxymbox{\hyperlink{class_beam_1_1_quad_batcher}{Quad\+Batcher}} \&batcher)
\end{DoxyCompactItemize}
\doxysubsubsection*{Private Attributes}
\begin{DoxyCompactItemize}
\item 
std::\+vector$<$ std::\+shared\+\_\+ptr$<$ \doxymbox{\hyperlink{class_beam_1_1_component}{Component}} $>$ $>$ \doxymbox{\hyperlink{class_beam_1_1_input_handler_a28f180063d61a104843c23ac24baadc3}{m\+\_\+components}}
\item 
std::\+shared\+\_\+ptr$<$ \doxymbox{\hyperlink{class_beam_1_1_component}{Component}} $>$ \doxymbox{\hyperlink{class_beam_1_1_input_handler_a78adda91495720c5f10106ce98cad1aa}{m\+\_\+focused\+Component}}
\end{DoxyCompactItemize}


\label{doc-func-members}
\Hypertarget{class_beam_1_1_input_handler_doc-func-members}
\doxysubsection{Member Function Documentation}
\Hypertarget{class_beam_1_1_input_handler_a1d3cb50866d2f151e5b9ddffdd1f0c01}\index{Beam::InputHandler@{Beam::InputHandler}!addComponent@{addComponent}}
\index{addComponent@{addComponent}!Beam::InputHandler@{Beam::InputHandler}}
\doxysubsubsection{\texorpdfstring{addComponent()}{addComponent()}}
{\footnotesize\ttfamily \label{class_beam_1_1_input_handler_a1d3cb50866d2f151e5b9ddffdd1f0c01} 
void Beam::\+\+Input\+Handler::\+add\+Component (\begin{DoxyParamCaption}\item[{std::\+shared\+\_\+ptr$<$ \doxymbox{\hyperlink{class_beam_1_1_component}{Component}} $>$}]{component}{}\end{DoxyParamCaption})}

\Hypertarget{class_beam_1_1_input_handler_a89d5254f25fa9d103a8b1a05332d5a5f}\index{Beam::InputHandler@{Beam::InputHandler}!handleMouseDown@{handleMouseDown}}
\index{handleMouseDown@{handleMouseDown}!Beam::InputHandler@{Beam::InputHandler}}
\doxysubsubsection{\texorpdfstring{handleMouseDown()}{handleMouseDown()}}
{\footnotesize\ttfamily \label{class_beam_1_1_input_handler_a89d5254f25fa9d103a8b1a05332d5a5f} 
void Beam::\+\+Input\+Handler::\+handle\+Mouse\+Down (\begin{DoxyParamCaption}\item[{float}]{x}{, }\item[{float}]{y}{, }\item[{int}]{button}{}\end{DoxyParamCaption})}

\Hypertarget{class_beam_1_1_input_handler_ab6ae366e86174358931c66a97f002f96}\index{Beam::InputHandler@{Beam::InputHandler}!handleMouseMove@{handleMouseMove}}
\index{handleMouseMove@{handleMouseMove}!Beam::InputHandler@{Beam::InputHandler}}
\doxysubsubsection{\texorpdfstring{handleMouseMove()}{handleMouseMove()}}
{\footnotesize\ttfamily \label{class_beam_1_1_input_handler_ab6ae366e86174358931c66a97f002f96} 
void Beam::\+\+Input\+Handler::\+handle\+Mouse\+Move (\begin{DoxyParamCaption}\item[{float}]{x}{, }\item[{float}]{y}{}\end{DoxyParamCaption})}

\Hypertarget{class_beam_1_1_input_handler_a200720149cc210bc1b2fae9f02872e86}\index{Beam::InputHandler@{Beam::InputHandler}!handleMouseUp@{handleMouseUp}}
\index{handleMouseUp@{handleMouseUp}!Beam::InputHandler@{Beam::InputHandler}}
\doxysubsubsection{\texorpdfstring{handleMouseUp()}{handleMouseUp()}}
{\footnotesize\ttfamily \label{class_beam_1_1_input_handler_a200720149cc210bc1b2fae9f02872e86} 
void Beam::\+\+Input\+Handler::\+handle\+Mouse\+Up (\begin{DoxyParamCaption}\item[{float}]{x}{, }\item[{float}]{y}{, }\item[{int}]{button}{}\end{DoxyParamCaption})}

\Hypertarget{class_beam_1_1_input_handler_a2d82861b1db962c08f7bf011fe6bc8ef}\index{Beam::InputHandler@{Beam::InputHandler}!render@{render}}
\index{render@{render}!Beam::InputHandler@{Beam::InputHandler}}
\doxysubsubsection{\texorpdfstring{render()}{render()}}
{\footnotesize\ttfamily \label{class_beam_1_1_input_handler_a2d82861b1db962c08f7bf011fe6bc8ef} 
void Beam::\+\+Input\+Handler::\+render (\begin{DoxyParamCaption}\item[{\doxymbox{\hyperlink{class_beam_1_1_quad_batcher}{Quad\+Batcher}} \&}]{batcher}{}\end{DoxyParamCaption})}

\Hypertarget{class_beam_1_1_input_handler_acd93a9973f465ad5a6d9f0ccdc72636a}\index{Beam::InputHandler@{Beam::InputHandler}!update@{update}}
\index{update@{update}!Beam::InputHandler@{Beam::InputHandler}}
\doxysubsubsection{\texorpdfstring{update()}{update()}}
{\footnotesize\ttfamily \label{class_beam_1_1_input_handler_acd93a9973f465ad5a6d9f0ccdc72636a} 
void Beam::\+\+Input\+Handler::\+update (\begin{DoxyParamCaption}\item[{float}]{dt}{}\end{DoxyParamCaption})}



\label{doc-variable-members}
\Hypertarget{class_beam_1_1_input_handler_doc-variable-members}
\doxysubsection{Member Data Documentation}
\Hypertarget{class_beam_1_1_input_handler_a28f180063d61a104843c23ac24baadc3}\index{Beam::InputHandler@{Beam::InputHandler}!m\_components@{m\_components}}
\index{m\_components@{m\_components}!Beam::InputHandler@{Beam::InputHandler}}
\doxysubsubsection{\texorpdfstring{m\_components}{m\_components}}
{\footnotesize\ttfamily \label{class_beam_1_1_input_handler_a28f180063d61a104843c23ac24baadc3} 
std::\+vector$<$std::\+shared\+\_\+ptr$<$\doxymbox{\hyperlink{class_beam_1_1_component}{Component}}$>$ $>$ Beam::\+\+Input\+Handler::\+m\+\_\+components\hspace{0.3cm}{\ttfamily [private]}}

\Hypertarget{class_beam_1_1_input_handler_a78adda91495720c5f10106ce98cad1aa}\index{Beam::InputHandler@{Beam::InputHandler}!m\_focusedComponent@{m\_focusedComponent}}
\index{m\_focusedComponent@{m\_focusedComponent}!Beam::InputHandler@{Beam::InputHandler}}
\doxysubsubsection{\texorpdfstring{m\_focusedComponent}{m\_focusedComponent}}
{\footnotesize\ttfamily \label{class_beam_1_1_input_handler_a78adda91495720c5f10106ce98cad1aa} 
std::\+shared\+\_\+ptr$<$\doxymbox{\hyperlink{class_beam_1_1_component}{Component}}$>$ Beam::\+\+Input\+Handler::\+m\+\_\+focused\+Component\hspace{0.3cm}{\ttfamily [private]}}



The documentation for this class was generated from the following files:\+\begin{DoxyCompactItemize}
\item 
src/\+ui/\+\doxymbox{\hyperlink{input__handler_8hpp}{input\+\_\+handler.\+hpp}}\item 
src/\+ui/\+\doxymbox{\hyperlink{input__handler_8cpp}{input\+\_\+handler.\+cpp}}\end{DoxyCompactItemize}

\doxysection{Beam::\+Input\+Node Class Reference}
\hypertarget{class_beam_1_1_input_node}{}\label{class_beam_1_1_input_node}\index{Beam::InputNode@{Beam::InputNode}}


Provides real-\/time audio input from the hardware to the Flux Graph.  




{\ttfamily \+\#include $<$input\+\_\+node.\+hpp$>$}

Inheritance diagram for Beam::\+Input\+Node:\+\begin{figure}[H]
\begin{center}
\leavevmode
\includegraphics[height=2.000000cm]{class_beam_1_1_input_node}
\end{center}
\end{figure}
\doxysubsubsection*{Public Member Functions}
\begin{DoxyCompactItemize}
\item 
\doxymbox{\hyperlink{class_beam_1_1_input_node_a86ab1c7bdda3f4cf03c36425f1d80148}{Input\+Node}} (int buffer\+Size)
\item 
void \doxymbox{\hyperlink{class_beam_1_1_input_node_a132e1ea5e27d21645ef371a45f377dd3}{process}} (int frames) override
\begin{DoxyCompactList}\small\item\em Main audio processing method. Must be implemented by subclasses. \end{DoxyCompactList}\item 
void \doxymbox{\hyperlink{class_beam_1_1_input_node_ac72e154f19216f9a348a0b7a2cabefff}{push\+Data}} (const float \texorpdfstring{$\ast$}{*}data, int samples)
\item 
std::\+string \doxymbox{\hyperlink{class_beam_1_1_input_node_aeef98292e43008c4e69525146f6e4050}{get\+Name}} () const override
\item 
std::\+vector$<$ \doxymbox{\hyperlink{struct_beam_1_1_flux_node_1_1_port}{Port}} $>$ \doxymbox{\hyperlink{class_beam_1_1_input_node_a9f91b9b33f0c719360f57e012cbb1250}{get\+Input\+Ports}} () const override
\item 
std::\+vector$<$ \doxymbox{\hyperlink{struct_beam_1_1_flux_node_1_1_port}{Port}} $>$ \doxymbox{\hyperlink{class_beam_1_1_input_node_a7a11ca9c74a2e08d3200feb375aa57bf}{get\+Output\+Ports}} () const override
\end{DoxyCompactItemize}
\doxysubsection*{Public Member Functions inherited from \doxymbox{\hyperlink{class_beam_1_1_flux_node}{Beam::\+\+Flux\+Node}}}
\begin{DoxyCompactItemize}
\item 
virtual \doxymbox{\hyperlink{class_beam_1_1_flux_node_a708c135cdb61e8838469998cd8a84e65}{\texorpdfstring{$\sim$}{\string~}\+Flux\+Node}} ()=default
\item 
virtual void \doxymbox{\hyperlink{class_beam_1_1_flux_node_ae9d1e151eff5166de969f45de06d5596}{process\+MIDI}} (const \doxymbox{\hyperlink{class_beam_1_1_m_i_d_i_buffer}{MIDIBuffer}} \&midi)
\begin{DoxyCompactList}\small\item\em Optional MIDI processing. Called before \doxylink{class_beam_1_1_flux_node_a3c263446753fa7ae5ff6928ee57bcd4d}{process()} in the engine loop. \end{DoxyCompactList}\item 
virtual void \doxymbox{\hyperlink{class_beam_1_1_flux_node_ace8cc49479d8924d44bca5fd4cd955e2}{on\+Transport\+State\+Changed}} (bool playing)
\begin{DoxyCompactList}\small\item\em Responds to global transport changes (Play/\+\+Pause). \end{DoxyCompactList}\item 
virtual void \doxymbox{\hyperlink{class_beam_1_1_flux_node_adc7c4e979bf27de5bfca66815ae97a67}{on\+Transport\+Seek}} (size\+\_\+t frame)
\begin{DoxyCompactList}\small\item\em Responds to timeline seeking. \end{DoxyCompactList}\item 
void \doxymbox{\hyperlink{class_beam_1_1_flux_node_aa579ec06608fd776987bbb089f27fd94}{set\+Current\+Frame}} (size\+\_\+t frame)
\begin{DoxyCompactList}\small\item\em Sets the current playhead position for this block. \end{DoxyCompactList}\item 
float \texorpdfstring{$\ast$}{*} \doxymbox{\hyperlink{class_beam_1_1_flux_node_ac90bd1a05b5bed3d68978f532386ed29}{get\+Input\+Buffer}} (int port\+Idx)
\item 
float \texorpdfstring{$\ast$}{*} \doxymbox{\hyperlink{class_beam_1_1_flux_node_abf11cfd4f2346ee0cd46d4345f1ed7d4}{get\+Output\+Buffer}} (int port\+Idx)
\item 
void \doxymbox{\hyperlink{class_beam_1_1_flux_node_af37f8c1b6b825da2ce7e35011d6f8253}{set\+Bypass}} (bool bypass)
\item 
bool \doxymbox{\hyperlink{class_beam_1_1_flux_node_a4bd30f3c8d311afdcd5c0d208e3bbf0f}{is\+Bypassed}} () const
\item 
void \doxymbox{\hyperlink{class_beam_1_1_flux_node_ad53f3fcaa5737f46d88530f40dbfbe32}{add\+Parameter}} (std::\+shared\+\_\+ptr$<$ \doxymbox{\hyperlink{class_beam_1_1_parameter}{Parameter}} $>$ \doxymbox{\hyperlink{texture_8cpp_aaded45152436a99bb4f9bda081df9f69}{param}})
\item 
std::\+shared\+\_\+ptr$<$ \doxymbox{\hyperlink{class_beam_1_1_parameter}{Parameter}} $>$ \doxymbox{\hyperlink{class_beam_1_1_flux_node_a59a32442eec144010741b9f2086c516e}{get\+Parameter}} (const std::\+string \&name)
\item 
const std::\+map$<$ std::\+string, std::\+shared\+\_\+ptr$<$ \doxymbox{\hyperlink{class_beam_1_1_parameter}{Parameter}} $>$ $>$ \& \doxymbox{\hyperlink{class_beam_1_1_flux_node_a6296c79b1ba77aa8b9526ace4a109529}{get\+Parameters}} () const
\end{DoxyCompactItemize}
\doxysubsubsection*{Private Attributes}
\begin{DoxyCompactItemize}
\item 
std::\+mutex \doxymbox{\hyperlink{class_beam_1_1_input_node_a2245d4f438413452c3569373721342cb}{m\+\_\+buffer\+Mutex}}
\item 
std::\+vector$<$ float $>$ \doxymbox{\hyperlink{class_beam_1_1_input_node_ae8c1b8ca2f4168ec8e043ee31c3bdeb2}{m\+\_\+captured\+Data}}
\end{DoxyCompactItemize}
\doxysubsubsection*{Additional Inherited Members}
\doxysubsection*{Protected Member Functions inherited from \doxymbox{\hyperlink{class_beam_1_1_flux_node}{Beam::\+\+Flux\+Node}}}
\begin{DoxyCompactItemize}
\item 
void \doxymbox{\hyperlink{class_beam_1_1_flux_node_ae3bafc1c5a1aa545167256172b3d3688}{setup\+Buffers}} (int num\+Inputs, int num\+Outputs, int buffer\+Size, int channels)
\begin{DoxyCompactList}\small\item\em Pre-\/allocates buffers for inputs and outputs. \end{DoxyCompactList}\end{DoxyCompactItemize}
\doxysubsection*{Protected Attributes inherited from \doxymbox{\hyperlink{class_beam_1_1_flux_node}{Beam::\+\+Flux\+Node}}}
\begin{DoxyCompactItemize}
\item 
std::\+vector$<$ std::\+vector$<$ float $>$ $>$ \doxymbox{\hyperlink{class_beam_1_1_flux_node_a8edab1c9ebd83e73bbfd92af29d6e92c}{m\+\_\+inputs}}
\item 
std::\+vector$<$ std::\+vector$<$ float $>$ $>$ \doxymbox{\hyperlink{class_beam_1_1_flux_node_a496905f0ff42c432eb38e19bd6135383}{m\+\_\+outputs}}
\item 
std::\+map$<$ std::\+string, std::\+shared\+\_\+ptr$<$ \doxymbox{\hyperlink{class_beam_1_1_parameter}{Parameter}} $>$ $>$ \doxymbox{\hyperlink{class_beam_1_1_flux_node_a65628a37cd2dd2832eda60e74ec1aed3}{m\+\_\+parameters}}
\item 
std::\+atomic$<$ bool $>$ \doxymbox{\hyperlink{class_beam_1_1_flux_node_a6116dcdcfa20998fe90dc75a74f25d9b}{m\+\_\+bypassed}} \{false\}
\item 
size\+\_\+t \doxymbox{\hyperlink{class_beam_1_1_flux_node_a7d8556ddb1482f997cda7749d737668b}{m\+\_\+current\+Frame}} = 0
\end{DoxyCompactItemize}


\doxysubsection{Detailed Description}
Provides real-\/time audio input from the hardware to the Flux Graph. 

\label{doc-constructors}
\Hypertarget{class_beam_1_1_input_node_doc-constructors}
\doxysubsection{Constructor \& Destructor Documentation}
\Hypertarget{class_beam_1_1_input_node_a86ab1c7bdda3f4cf03c36425f1d80148}\index{Beam::InputNode@{Beam::InputNode}!InputNode@{InputNode}}
\index{InputNode@{InputNode}!Beam::InputNode@{Beam::InputNode}}
\doxysubsubsection{\texorpdfstring{InputNode()}{InputNode()}}
{\footnotesize\ttfamily \label{class_beam_1_1_input_node_a86ab1c7bdda3f4cf03c36425f1d80148} 
Beam::\+\+Input\+Node::\+\+Input\+Node (\begin{DoxyParamCaption}\item[{int}]{buffer\+Size}{}\end{DoxyParamCaption})\hspace{0.3cm}{\ttfamily [inline]}}



\label{doc-func-members}
\Hypertarget{class_beam_1_1_input_node_doc-func-members}
\doxysubsection{Member Function Documentation}
\Hypertarget{class_beam_1_1_input_node_a9f91b9b33f0c719360f57e012cbb1250}\index{Beam::InputNode@{Beam::InputNode}!getInputPorts@{getInputPorts}}
\index{getInputPorts@{getInputPorts}!Beam::InputNode@{Beam::InputNode}}
\doxysubsubsection{\texorpdfstring{getInputPorts()}{getInputPorts()}}
{\footnotesize\ttfamily \label{class_beam_1_1_input_node_a9f91b9b33f0c719360f57e012cbb1250} 
std::\+vector$<$ \doxymbox{\hyperlink{struct_beam_1_1_flux_node_1_1_port}{Port}} $>$ Beam::\+\+Input\+Node::\+get\+Input\+Ports (\begin{DoxyParamCaption}{}{}\end{DoxyParamCaption}) const\hspace{0.3cm}{\ttfamily [inline]}, {\ttfamily [override]}, {\ttfamily [virtual]}}



Implements \doxymbox{\hyperlink{class_beam_1_1_flux_node_a17eb02187925b52bf8e53fa3ebe3da66}{Beam::\+\+Flux\+Node}}.

\Hypertarget{class_beam_1_1_input_node_aeef98292e43008c4e69525146f6e4050}\index{Beam::InputNode@{Beam::InputNode}!getName@{getName}}
\index{getName@{getName}!Beam::InputNode@{Beam::InputNode}}
\doxysubsubsection{\texorpdfstring{getName()}{getName()}}
{\footnotesize\ttfamily \label{class_beam_1_1_input_node_aeef98292e43008c4e69525146f6e4050} 
std::\+string Beam::\+\+Input\+Node::\+get\+Name (\begin{DoxyParamCaption}{}{}\end{DoxyParamCaption}) const\hspace{0.3cm}{\ttfamily [inline]}, {\ttfamily [override]}, {\ttfamily [virtual]}}



Implements \doxymbox{\hyperlink{class_beam_1_1_flux_node_ac638d3d9bb1050d658294bc5470abeba}{Beam::\+\+Flux\+Node}}.

\Hypertarget{class_beam_1_1_input_node_a7a11ca9c74a2e08d3200feb375aa57bf}\index{Beam::InputNode@{Beam::InputNode}!getOutputPorts@{getOutputPorts}}
\index{getOutputPorts@{getOutputPorts}!Beam::InputNode@{Beam::InputNode}}
\doxysubsubsection{\texorpdfstring{getOutputPorts()}{getOutputPorts()}}
{\footnotesize\ttfamily \label{class_beam_1_1_input_node_a7a11ca9c74a2e08d3200feb375aa57bf} 
std::\+vector$<$ \doxymbox{\hyperlink{struct_beam_1_1_flux_node_1_1_port}{Port}} $>$ Beam::\+\+Input\+Node::\+get\+Output\+Ports (\begin{DoxyParamCaption}{}{}\end{DoxyParamCaption}) const\hspace{0.3cm}{\ttfamily [inline]}, {\ttfamily [override]}, {\ttfamily [virtual]}}



Implements \doxymbox{\hyperlink{class_beam_1_1_flux_node_a034f59d236afd7901ed84090422e3279}{Beam::\+\+Flux\+Node}}.

\Hypertarget{class_beam_1_1_input_node_a132e1ea5e27d21645ef371a45f377dd3}\index{Beam::InputNode@{Beam::InputNode}!process@{process}}
\index{process@{process}!Beam::InputNode@{Beam::InputNode}}
\doxysubsubsection{\texorpdfstring{process()}{process()}}
{\footnotesize\ttfamily \label{class_beam_1_1_input_node_a132e1ea5e27d21645ef371a45f377dd3} 
void Beam::\+\+Input\+Node::\+process (\begin{DoxyParamCaption}\item[{int}]{frames}{}\end{DoxyParamCaption})\hspace{0.3cm}{\ttfamily [inline]}, {\ttfamily [override]}, {\ttfamily [virtual]}}



Main audio processing method. Must be implemented by subclasses. 


\begin{DoxyParams}{Parameters}
{\em frames} & Number of frames to process in the current block. \\
\hline
\end{DoxyParams}


Implements \doxymbox{\hyperlink{class_beam_1_1_flux_node_a3c263446753fa7ae5ff6928ee57bcd4d}{Beam::\+\+Flux\+Node}}.

\Hypertarget{class_beam_1_1_input_node_ac72e154f19216f9a348a0b7a2cabefff}\index{Beam::InputNode@{Beam::InputNode}!pushData@{pushData}}
\index{pushData@{pushData}!Beam::InputNode@{Beam::InputNode}}
\doxysubsubsection{\texorpdfstring{pushData()}{pushData()}}
{\footnotesize\ttfamily \label{class_beam_1_1_input_node_ac72e154f19216f9a348a0b7a2cabefff} 
void Beam::\+\+Input\+Node::\+push\+Data (\begin{DoxyParamCaption}\item[{const float \texorpdfstring{$\ast$}{*}}]{data}{, }\item[{int}]{samples}{}\end{DoxyParamCaption})\hspace{0.3cm}{\ttfamily [inline]}}



\label{doc-variable-members}
\Hypertarget{class_beam_1_1_input_node_doc-variable-members}
\doxysubsection{Member Data Documentation}
\Hypertarget{class_beam_1_1_input_node_a2245d4f438413452c3569373721342cb}\index{Beam::InputNode@{Beam::InputNode}!m\_bufferMutex@{m\_bufferMutex}}
\index{m\_bufferMutex@{m\_bufferMutex}!Beam::InputNode@{Beam::InputNode}}
\doxysubsubsection{\texorpdfstring{m\_bufferMutex}{m\_bufferMutex}}
{\footnotesize\ttfamily \label{class_beam_1_1_input_node_a2245d4f438413452c3569373721342cb} 
std::\+mutex Beam::\+\+Input\+Node::\+m\+\_\+buffer\+Mutex\hspace{0.3cm}{\ttfamily [private]}}

\Hypertarget{class_beam_1_1_input_node_ae8c1b8ca2f4168ec8e043ee31c3bdeb2}\index{Beam::InputNode@{Beam::InputNode}!m\_capturedData@{m\_capturedData}}
\index{m\_capturedData@{m\_capturedData}!Beam::InputNode@{Beam::InputNode}}
\doxysubsubsection{\texorpdfstring{m\_capturedData}{m\_capturedData}}
{\footnotesize\ttfamily \label{class_beam_1_1_input_node_ae8c1b8ca2f4168ec8e043ee31c3bdeb2} 
std::\+vector$<$float$>$ Beam::\+\+Input\+Node::\+m\+\_\+captured\+Data\hspace{0.3cm}{\ttfamily [private]}}



The documentation for this class was generated from the following file:\+\begin{DoxyCompactItemize}
\item 
src/\+engine/\+\doxymbox{\hyperlink{input__node_8hpp}{input\+\_\+node.\+hpp}}\end{DoxyCompactItemize}

\doxysection{Beam::\+Key\+Press Class Reference}
\hypertarget{class_beam_1_1_key_press}{}\label{class_beam_1_1_key_press}\index{Beam::KeyPress@{Beam::KeyPress}}


{\ttfamily \+\#include $<$gui\+\_\+component.\+hpp$>$}

\doxysubsubsection*{Public Member Functions}
\begin{DoxyCompactItemize}
\item 
\doxymbox{\hyperlink{class_beam_1_1_key_press_a38e55e4227edd5bdb5eb6765d329424d}{Key\+Press}} (int code, const std::\+string \&text)
\end{DoxyCompactItemize}
\doxysubsubsection*{Public Attributes}
\begin{DoxyCompactItemize}
\item 
int \doxymbox{\hyperlink{class_beam_1_1_key_press_a279fdfe0ca146e7078ad59fc035b5265}{key\+Code}}
\item 
std::\+string \doxymbox{\hyperlink{class_beam_1_1_key_press_afe78946cf7350b5028dbfbd517605406}{key\+Text}}
\end{DoxyCompactItemize}


\label{doc-constructors}
\Hypertarget{class_beam_1_1_key_press_doc-constructors}
\doxysubsection{Constructor \& Destructor Documentation}
\Hypertarget{class_beam_1_1_key_press_a38e55e4227edd5bdb5eb6765d329424d}\index{Beam::KeyPress@{Beam::KeyPress}!KeyPress@{KeyPress}}
\index{KeyPress@{KeyPress}!Beam::KeyPress@{Beam::KeyPress}}
\doxysubsubsection{\texorpdfstring{KeyPress()}{KeyPress()}}
{\footnotesize\ttfamily \label{class_beam_1_1_key_press_a38e55e4227edd5bdb5eb6765d329424d} 
Beam::\+\+Key\+Press::\+\+Key\+Press (\begin{DoxyParamCaption}\item[{int}]{code}{, }\item[{const std::\+string \&}]{text}{}\end{DoxyParamCaption})\hspace{0.3cm}{\ttfamily [inline]}}



\label{doc-variable-members}
\Hypertarget{class_beam_1_1_key_press_doc-variable-members}
\doxysubsection{Member Data Documentation}
\Hypertarget{class_beam_1_1_key_press_a279fdfe0ca146e7078ad59fc035b5265}\index{Beam::KeyPress@{Beam::KeyPress}!keyCode@{keyCode}}
\index{keyCode@{keyCode}!Beam::KeyPress@{Beam::KeyPress}}
\doxysubsubsection{\texorpdfstring{keyCode}{keyCode}}
{\footnotesize\ttfamily \label{class_beam_1_1_key_press_a279fdfe0ca146e7078ad59fc035b5265} 
int Beam::\+\+Key\+Press::\+key\+Code}

\Hypertarget{class_beam_1_1_key_press_afe78946cf7350b5028dbfbd517605406}\index{Beam::KeyPress@{Beam::KeyPress}!keyText@{keyText}}
\index{keyText@{keyText}!Beam::KeyPress@{Beam::KeyPress}}
\doxysubsubsection{\texorpdfstring{keyText}{keyText}}
{\footnotesize\ttfamily \label{class_beam_1_1_key_press_afe78946cf7350b5028dbfbd517605406} 
std::\+string Beam::\+\+Key\+Press::\+key\+Text}



The documentation for this class was generated from the following file:\+\begin{DoxyCompactItemize}
\item 
src/\+interface/\+\doxymbox{\hyperlink{gui__component_8hpp}{gui\+\_\+component.\+hpp}}\end{DoxyCompactItemize}

\doxysection{Beam::\+Knob Class Reference}
\hypertarget{class_beam_1_1_knob}{}\label{class_beam_1_1_knob}\index{Beam::Knob@{Beam::Knob}}


{\ttfamily \+\#include $<$knob.\+hpp$>$}

Inheritance diagram for Beam::\+Knob:\+\begin{figure}[H]
\begin{center}
\leavevmode
\includegraphics[height=2.000000cm]{class_beam_1_1_knob}
\end{center}
\end{figure}
\doxysubsubsection*{Public Member Functions}
\begin{DoxyCompactItemize}
\item 
\doxymbox{\hyperlink{class_beam_1_1_knob_adf590b7511811e22abb0a1c9478fe512}{Knob}} (const std::\+string \&label, float min\+Val, float max\+Val, float initial\+Val)
\item 
void \doxymbox{\hyperlink{class_beam_1_1_knob_af53ea4ce8f7892c67225eb37955149d1}{bind\+Parameter}} (std::\+shared\+\_\+ptr$<$ \doxymbox{\hyperlink{class_beam_1_1_parameter}{Parameter}} $>$ \doxymbox{\hyperlink{texture_8cpp_aaded45152436a99bb4f9bda081df9f69}{param}})
\item 
void \doxymbox{\hyperlink{class_beam_1_1_knob_aaf28f9b129c4c9a35b5161b8cf6ca77c}{set\+Texture}} (std::\+shared\+\_\+ptr$<$ \doxymbox{\hyperlink{class_beam_1_1_texture}{Texture}} $>$ \doxymbox{\hyperlink{texture_8cpp_a0704dfe56dec926cb35f7bdc0834ecd0}{texture}}, int num\+Frames)
\item 
void \doxymbox{\hyperlink{class_beam_1_1_knob_ae0021c156924e66247674396412657aa}{render}} (\doxymbox{\hyperlink{class_beam_1_1_quad_batcher}{Quad\+Batcher}} \&batcher, float dt, float screenW, float screenH) override
\item 
bool \doxymbox{\hyperlink{class_beam_1_1_knob_a85b8ee99461b9e8c1a86c5f2d9eba4e2}{on\+Mouse\+Down}} (float x, float y, int button) override
\item 
bool \doxymbox{\hyperlink{class_beam_1_1_knob_a266e448c4403e85f68a581fb81922fdd}{on\+Mouse\+Up}} (float x, float y, int button) override
\item 
bool \doxymbox{\hyperlink{class_beam_1_1_knob_a6e918f86ad28e14eb470e00938d1b212}{on\+Mouse\+Move}} (float x, float y) override
\item 
float \doxymbox{\hyperlink{class_beam_1_1_knob_a9f4c4eb1be070db3e89e3c3c7f6eba94}{get\+Value}} () const
\item 
void \doxymbox{\hyperlink{class_beam_1_1_knob_abd452cb9e12d4ade468f1f989c7cb173}{set\+Value}} (float v)
\end{DoxyCompactItemize}
\doxysubsection*{Public Member Functions inherited from \doxymbox{\hyperlink{class_beam_1_1_component}{Beam::\+\+Component}}}
\begin{DoxyCompactItemize}
\item 
virtual \doxymbox{\hyperlink{class_beam_1_1_component_af9d734d649978e027412a87bc54362cd}{\texorpdfstring{$\sim$}{\string~}\+Component}} ()=default
\item 
virtual void \doxymbox{\hyperlink{class_beam_1_1_component_ad3d3fb19d25b4371d07620567970a158}{update}} (float dt)
\item 
virtual bool \doxymbox{\hyperlink{class_beam_1_1_component_ab92e884903f8a621fcd57bc00a24b041}{on\+Mouse\+Wheel}} (float x, float y, float delta)
\item 
virtual void \doxymbox{\hyperlink{class_beam_1_1_component_a6865b1f22388af467bf6c789120fac05}{set\+Bounds}} (float x, float y, float w, float h)
\item 
const \doxymbox{\hyperlink{struct_beam_1_1_rect}{Rect}} \& \doxymbox{\hyperlink{class_beam_1_1_component_a5746dbc69d5b0adb4cffbcf920936d00}{get\+Bounds}} () const
\item 
void \doxymbox{\hyperlink{class_beam_1_1_component_a00d4e2dfa7703e59d6486852321dbdf1}{set\+Draggable}} (bool draggable)
\item 
void \doxymbox{\hyperlink{class_beam_1_1_component_aca7b02d1dddf7cd20378db9e3242fb84}{start\+Dragging}} (float x, float y)
\end{DoxyCompactItemize}
\doxysubsubsection*{Public Attributes}
\begin{DoxyCompactItemize}
\item 
std::\+function$<$ void(float)$>$ \doxymbox{\hyperlink{class_beam_1_1_knob_a48cab0ed660cc222279a8b4694f44be4}{on\+Value\+Changed}}
\end{DoxyCompactItemize}
\doxysubsubsection*{Private Attributes}
\begin{DoxyCompactItemize}
\item 
std::\+string \doxymbox{\hyperlink{class_beam_1_1_knob_a88c46e1bd78283c6c56202736f7cb42c}{m\+\_\+label}}
\item 
float \doxymbox{\hyperlink{class_beam_1_1_knob_a73e94bcd1b1c898fda1842664d6736c8}{m\+\_\+min}}
\item 
float \doxymbox{\hyperlink{class_beam_1_1_knob_a3bc74a959b98e95859a49662422d3d00}{m\+\_\+max}}
\item 
float \doxymbox{\hyperlink{class_beam_1_1_knob_a74c5d8690c1a01eab43c84e310613600}{m\+\_\+value}}
\item 
bool \doxymbox{\hyperlink{class_beam_1_1_knob_a57b169ca15b4575bc3f0106ef02c62b9}{m\+\_\+is\+Dragging}} = false
\item 
float \doxymbox{\hyperlink{class_beam_1_1_knob_a847d67ee3d10f7d52433183029be6c83}{m\+\_\+lastY}} = 0
\item 
std::\+shared\+\_\+ptr$<$ \doxymbox{\hyperlink{class_beam_1_1_parameter}{Parameter}} $>$ \doxymbox{\hyperlink{class_beam_1_1_knob_ae21386067542a5f216e45371ac9382a7}{m\+\_\+parameter}}
\item 
std::\+shared\+\_\+ptr$<$ \doxymbox{\hyperlink{class_beam_1_1_texture}{Texture}} $>$ \doxymbox{\hyperlink{class_beam_1_1_knob_ab14166db224ccf366bf58d0eb58d2856}{m\+\_\+texture}}
\item 
int \doxymbox{\hyperlink{class_beam_1_1_knob_a8129838d229fe6c1ae6d7e4810f819d7}{m\+\_\+num\+Frames}} = 0
\end{DoxyCompactItemize}
\doxysubsubsection*{Additional Inherited Members}
\doxysubsection*{Protected Attributes inherited from \doxymbox{\hyperlink{class_beam_1_1_component}{Beam::\+\+Component}}}
\begin{DoxyCompactItemize}
\item 
\doxymbox{\hyperlink{struct_beam_1_1_rect}{Rect}} \doxymbox{\hyperlink{class_beam_1_1_component_a4f1ec4a5fb168c39a6c18f958b2b1495}{m\+\_\+bounds}} \{0, 0, 0, 0\}
\item 
bool \doxymbox{\hyperlink{class_beam_1_1_component_adc07913aed6ddadf1c730e7b3bb599cf}{m\+\_\+is\+Visible}} = true
\item 
bool \doxymbox{\hyperlink{class_beam_1_1_component_a0bf77b204ae374a14b5a6d7e5a3c13c6}{m\+\_\+is\+Enabled}} = true
\item 
bool \doxymbox{\hyperlink{class_beam_1_1_component_a9646efcaa9540a26a387f5da9aae4bde}{m\+\_\+is\+Draggable}} = false
\item 
bool \doxymbox{\hyperlink{class_beam_1_1_component_ab03af9a9743acf040f38e3fb11f8dc14}{m\+\_\+is\+Dragging}} = false
\item 
float \doxymbox{\hyperlink{class_beam_1_1_component_a7110b2b9dc235f724bf4689569266a63}{m\+\_\+last\+MouseX}} = 0
\item 
float \doxymbox{\hyperlink{class_beam_1_1_component_a768931a0f51394bf011f821f6ed2efe9}{m\+\_\+last\+MouseY}} = 0
\end{DoxyCompactItemize}


\label{doc-constructors}
\Hypertarget{class_beam_1_1_knob_doc-constructors}
\doxysubsection{Constructor \& Destructor Documentation}
\Hypertarget{class_beam_1_1_knob_adf590b7511811e22abb0a1c9478fe512}\index{Beam::Knob@{Beam::Knob}!Knob@{Knob}}
\index{Knob@{Knob}!Beam::Knob@{Beam::Knob}}
\doxysubsubsection{\texorpdfstring{Knob()}{Knob()}}
{\footnotesize\ttfamily \label{class_beam_1_1_knob_adf590b7511811e22abb0a1c9478fe512} 
Beam::\+\+Knob::\+\+Knob (\begin{DoxyParamCaption}\item[{const std::\+string \&}]{label}{, }\item[{float}]{min\+Val}{, }\item[{float}]{max\+Val}{, }\item[{float}]{initial\+Val}{}\end{DoxyParamCaption})\hspace{0.3cm}{\ttfamily [inline]}}



\label{doc-func-members}
\Hypertarget{class_beam_1_1_knob_doc-func-members}
\doxysubsection{Member Function Documentation}
\Hypertarget{class_beam_1_1_knob_af53ea4ce8f7892c67225eb37955149d1}\index{Beam::Knob@{Beam::Knob}!bindParameter@{bindParameter}}
\index{bindParameter@{bindParameter}!Beam::Knob@{Beam::Knob}}
\doxysubsubsection{\texorpdfstring{bindParameter()}{bindParameter()}}
{\footnotesize\ttfamily \label{class_beam_1_1_knob_af53ea4ce8f7892c67225eb37955149d1} 
void Beam::\+\+Knob::\+bind\+Parameter (\begin{DoxyParamCaption}\item[{std::\+shared\+\_\+ptr$<$ \doxymbox{\hyperlink{class_beam_1_1_parameter}{Parameter}} $>$}]{param}{}\end{DoxyParamCaption})\hspace{0.3cm}{\ttfamily [inline]}}

\Hypertarget{class_beam_1_1_knob_a9f4c4eb1be070db3e89e3c3c7f6eba94}\index{Beam::Knob@{Beam::Knob}!getValue@{getValue}}
\index{getValue@{getValue}!Beam::Knob@{Beam::Knob}}
\doxysubsubsection{\texorpdfstring{getValue()}{getValue()}}
{\footnotesize\ttfamily \label{class_beam_1_1_knob_a9f4c4eb1be070db3e89e3c3c7f6eba94} 
float Beam::\+\+Knob::\+get\+Value (\begin{DoxyParamCaption}{}{}\end{DoxyParamCaption}) const\hspace{0.3cm}{\ttfamily [inline]}}

\Hypertarget{class_beam_1_1_knob_a85b8ee99461b9e8c1a86c5f2d9eba4e2}\index{Beam::Knob@{Beam::Knob}!onMouseDown@{onMouseDown}}
\index{onMouseDown@{onMouseDown}!Beam::Knob@{Beam::Knob}}
\doxysubsubsection{\texorpdfstring{onMouseDown()}{onMouseDown()}}
{\footnotesize\ttfamily \label{class_beam_1_1_knob_a85b8ee99461b9e8c1a86c5f2d9eba4e2} 
bool Beam::\+\+Knob::\+on\+Mouse\+Down (\begin{DoxyParamCaption}\item[{float}]{x}{, }\item[{float}]{y}{, }\item[{int}]{button}{}\end{DoxyParamCaption})\hspace{0.3cm}{\ttfamily [inline]}, {\ttfamily [override]}, {\ttfamily [virtual]}}



Reimplemented from \doxymbox{\hyperlink{class_beam_1_1_component_aec1da33d2d6e3d4e7dd6708309264e76}{Beam::\+\+Component}}.

\Hypertarget{class_beam_1_1_knob_a6e918f86ad28e14eb470e00938d1b212}\index{Beam::Knob@{Beam::Knob}!onMouseMove@{onMouseMove}}
\index{onMouseMove@{onMouseMove}!Beam::Knob@{Beam::Knob}}
\doxysubsubsection{\texorpdfstring{onMouseMove()}{onMouseMove()}}
{\footnotesize\ttfamily \label{class_beam_1_1_knob_a6e918f86ad28e14eb470e00938d1b212} 
bool Beam::\+\+Knob::\+on\+Mouse\+Move (\begin{DoxyParamCaption}\item[{float}]{x}{, }\item[{float}]{y}{}\end{DoxyParamCaption})\hspace{0.3cm}{\ttfamily [inline]}, {\ttfamily [override]}, {\ttfamily [virtual]}}



Reimplemented from \doxymbox{\hyperlink{class_beam_1_1_component_a9d8e5970783d315044277a1228659e6c}{Beam::\+\+Component}}.

\Hypertarget{class_beam_1_1_knob_a266e448c4403e85f68a581fb81922fdd}\index{Beam::Knob@{Beam::Knob}!onMouseUp@{onMouseUp}}
\index{onMouseUp@{onMouseUp}!Beam::Knob@{Beam::Knob}}
\doxysubsubsection{\texorpdfstring{onMouseUp()}{onMouseUp()}}
{\footnotesize\ttfamily \label{class_beam_1_1_knob_a266e448c4403e85f68a581fb81922fdd} 
bool Beam::\+\+Knob::\+on\+Mouse\+Up (\begin{DoxyParamCaption}\item[{float}]{x}{, }\item[{float}]{y}{, }\item[{int}]{button}{}\end{DoxyParamCaption})\hspace{0.3cm}{\ttfamily [inline]}, {\ttfamily [override]}, {\ttfamily [virtual]}}



Reimplemented from \doxymbox{\hyperlink{class_beam_1_1_component_ae36b8e9d70e8f9a1b9ba81c23c54d5c8}{Beam::\+\+Component}}.

\Hypertarget{class_beam_1_1_knob_ae0021c156924e66247674396412657aa}\index{Beam::Knob@{Beam::Knob}!render@{render}}
\index{render@{render}!Beam::Knob@{Beam::Knob}}
\doxysubsubsection{\texorpdfstring{render()}{render()}}
{\footnotesize\ttfamily \label{class_beam_1_1_knob_ae0021c156924e66247674396412657aa} 
void Beam::\+\+Knob::\+render (\begin{DoxyParamCaption}\item[{\doxymbox{\hyperlink{class_beam_1_1_quad_batcher}{Quad\+Batcher}} \&}]{batcher}{, }\item[{float}]{dt}{, }\item[{float}]{screenW}{, }\item[{float}]{screenH}{}\end{DoxyParamCaption})\hspace{0.3cm}{\ttfamily [inline]}, {\ttfamily [override]}, {\ttfamily [virtual]}}



Implements \doxymbox{\hyperlink{class_beam_1_1_component_acef3496a55f0d94c8678f6049dbaa7cd}{Beam::\+\+Component}}.

\Hypertarget{class_beam_1_1_knob_aaf28f9b129c4c9a35b5161b8cf6ca77c}\index{Beam::Knob@{Beam::Knob}!setTexture@{setTexture}}
\index{setTexture@{setTexture}!Beam::Knob@{Beam::Knob}}
\doxysubsubsection{\texorpdfstring{setTexture()}{setTexture()}}
{\footnotesize\ttfamily \label{class_beam_1_1_knob_aaf28f9b129c4c9a35b5161b8cf6ca77c} 
void Beam::\+\+Knob::\+set\+Texture (\begin{DoxyParamCaption}\item[{std::\+shared\+\_\+ptr$<$ \doxymbox{\hyperlink{class_beam_1_1_texture}{Texture}} $>$}]{texture}{, }\item[{int}]{num\+Frames}{}\end{DoxyParamCaption})\hspace{0.3cm}{\ttfamily [inline]}}

\Hypertarget{class_beam_1_1_knob_abd452cb9e12d4ade468f1f989c7cb173}\index{Beam::Knob@{Beam::Knob}!setValue@{setValue}}
\index{setValue@{setValue}!Beam::Knob@{Beam::Knob}}
\doxysubsubsection{\texorpdfstring{setValue()}{setValue()}}
{\footnotesize\ttfamily \label{class_beam_1_1_knob_abd452cb9e12d4ade468f1f989c7cb173} 
void Beam::\+\+Knob::\+set\+Value (\begin{DoxyParamCaption}\item[{float}]{v}{}\end{DoxyParamCaption})\hspace{0.3cm}{\ttfamily [inline]}}



\label{doc-variable-members}
\Hypertarget{class_beam_1_1_knob_doc-variable-members}
\doxysubsection{Member Data Documentation}
\Hypertarget{class_beam_1_1_knob_a57b169ca15b4575bc3f0106ef02c62b9}\index{Beam::Knob@{Beam::Knob}!m\_isDragging@{m\_isDragging}}
\index{m\_isDragging@{m\_isDragging}!Beam::Knob@{Beam::Knob}}
\doxysubsubsection{\texorpdfstring{m\_isDragging}{m\_isDragging}}
{\footnotesize\ttfamily \label{class_beam_1_1_knob_a57b169ca15b4575bc3f0106ef02c62b9} 
bool Beam::\+\+Knob::\+m\+\_\+is\+Dragging = false\hspace{0.3cm}{\ttfamily [private]}}

\Hypertarget{class_beam_1_1_knob_a88c46e1bd78283c6c56202736f7cb42c}\index{Beam::Knob@{Beam::Knob}!m\_label@{m\_label}}
\index{m\_label@{m\_label}!Beam::Knob@{Beam::Knob}}
\doxysubsubsection{\texorpdfstring{m\_label}{m\_label}}
{\footnotesize\ttfamily \label{class_beam_1_1_knob_a88c46e1bd78283c6c56202736f7cb42c} 
std::\+string Beam::\+\+Knob::\+m\+\_\+label\hspace{0.3cm}{\ttfamily [private]}}

\Hypertarget{class_beam_1_1_knob_a847d67ee3d10f7d52433183029be6c83}\index{Beam::Knob@{Beam::Knob}!m\_lastY@{m\_lastY}}
\index{m\_lastY@{m\_lastY}!Beam::Knob@{Beam::Knob}}
\doxysubsubsection{\texorpdfstring{m\_lastY}{m\_lastY}}
{\footnotesize\ttfamily \label{class_beam_1_1_knob_a847d67ee3d10f7d52433183029be6c83} 
float Beam::\+\+Knob::\+m\+\_\+lastY = 0\hspace{0.3cm}{\ttfamily [private]}}

\Hypertarget{class_beam_1_1_knob_a3bc74a959b98e95859a49662422d3d00}\index{Beam::Knob@{Beam::Knob}!m\_max@{m\_max}}
\index{m\_max@{m\_max}!Beam::Knob@{Beam::Knob}}
\doxysubsubsection{\texorpdfstring{m\_max}{m\_max}}
{\footnotesize\ttfamily \label{class_beam_1_1_knob_a3bc74a959b98e95859a49662422d3d00} 
float Beam::\+\+Knob::\+m\+\_\+max\hspace{0.3cm}{\ttfamily [private]}}

\Hypertarget{class_beam_1_1_knob_a73e94bcd1b1c898fda1842664d6736c8}\index{Beam::Knob@{Beam::Knob}!m\_min@{m\_min}}
\index{m\_min@{m\_min}!Beam::Knob@{Beam::Knob}}
\doxysubsubsection{\texorpdfstring{m\_min}{m\_min}}
{\footnotesize\ttfamily \label{class_beam_1_1_knob_a73e94bcd1b1c898fda1842664d6736c8} 
float Beam::\+\+Knob::\+m\+\_\+min\hspace{0.3cm}{\ttfamily [private]}}

\Hypertarget{class_beam_1_1_knob_a8129838d229fe6c1ae6d7e4810f819d7}\index{Beam::Knob@{Beam::Knob}!m\_numFrames@{m\_numFrames}}
\index{m\_numFrames@{m\_numFrames}!Beam::Knob@{Beam::Knob}}
\doxysubsubsection{\texorpdfstring{m\_numFrames}{m\_numFrames}}
{\footnotesize\ttfamily \label{class_beam_1_1_knob_a8129838d229fe6c1ae6d7e4810f819d7} 
int Beam::\+\+Knob::\+m\+\_\+num\+Frames = 0\hspace{0.3cm}{\ttfamily [private]}}

\Hypertarget{class_beam_1_1_knob_ae21386067542a5f216e45371ac9382a7}\index{Beam::Knob@{Beam::Knob}!m\_parameter@{m\_parameter}}
\index{m\_parameter@{m\_parameter}!Beam::Knob@{Beam::Knob}}
\doxysubsubsection{\texorpdfstring{m\_parameter}{m\_parameter}}
{\footnotesize\ttfamily \label{class_beam_1_1_knob_ae21386067542a5f216e45371ac9382a7} 
std::\+shared\+\_\+ptr$<$\doxymbox{\hyperlink{class_beam_1_1_parameter}{Parameter}}$>$ Beam::\+\+Knob::\+m\+\_\+parameter\hspace{0.3cm}{\ttfamily [private]}}

\Hypertarget{class_beam_1_1_knob_ab14166db224ccf366bf58d0eb58d2856}\index{Beam::Knob@{Beam::Knob}!m\_texture@{m\_texture}}
\index{m\_texture@{m\_texture}!Beam::Knob@{Beam::Knob}}
\doxysubsubsection{\texorpdfstring{m\_texture}{m\_texture}}
{\footnotesize\ttfamily \label{class_beam_1_1_knob_ab14166db224ccf366bf58d0eb58d2856} 
std::\+shared\+\_\+ptr$<$\doxymbox{\hyperlink{class_beam_1_1_texture}{Texture}}$>$ Beam::\+\+Knob::\+m\+\_\+texture\hspace{0.3cm}{\ttfamily [private]}}

\Hypertarget{class_beam_1_1_knob_a74c5d8690c1a01eab43c84e310613600}\index{Beam::Knob@{Beam::Knob}!m\_value@{m\_value}}
\index{m\_value@{m\_value}!Beam::Knob@{Beam::Knob}}
\doxysubsubsection{\texorpdfstring{m\_value}{m\_value}}
{\footnotesize\ttfamily \label{class_beam_1_1_knob_a74c5d8690c1a01eab43c84e310613600} 
float Beam::\+\+Knob::\+m\+\_\+value\hspace{0.3cm}{\ttfamily [private]}}

\Hypertarget{class_beam_1_1_knob_a48cab0ed660cc222279a8b4694f44be4}\index{Beam::Knob@{Beam::Knob}!onValueChanged@{onValueChanged}}
\index{onValueChanged@{onValueChanged}!Beam::Knob@{Beam::Knob}}
\doxysubsubsection{\texorpdfstring{onValueChanged}{onValueChanged}}
{\footnotesize\ttfamily \label{class_beam_1_1_knob_a48cab0ed660cc222279a8b4694f44be4} 
std::\+function$<$void(float)$>$ Beam::\+\+Knob::\+on\+Value\+Changed}



The documentation for this class was generated from the following file:\+\begin{DoxyCompactItemize}
\item 
src/\+interface/\+\doxymbox{\hyperlink{knob_8hpp}{knob.\+hpp}}\end{DoxyCompactItemize}

\doxysection{Beam::\+Master\+Node Class Reference}
\hypertarget{class_beam_1_1_master_node}{}\label{class_beam_1_1_master_node}\index{Beam::MasterNode@{Beam::MasterNode}}


The final sink in the audio graph. Handles global volume, metering, and transformer saturation.  




{\ttfamily \+\#include $<$master\+\_\+node.\+hpp$>$}

Inheritance diagram for Beam::\+Master\+Node:\+\begin{figure}[H]
\begin{center}
\leavevmode
\includegraphics[height=2.000000cm]{class_beam_1_1_master_node}
\end{center}
\end{figure}
\doxysubsubsection*{Public Member Functions}
\begin{DoxyCompactItemize}
\item 
\doxymbox{\hyperlink{class_beam_1_1_master_node_aa47be01f91f6c96952fb17f96661e086}{Master\+Node}} (int buffer\+Size)
\item 
void \doxymbox{\hyperlink{class_beam_1_1_master_node_a4d78fec9962d8b9fd8a34142d0a129f7}{process}} (int frames) override
\begin{DoxyCompactList}\small\item\em Main audio processing method. Must be implemented by subclasses. \end{DoxyCompactList}\item 
float \doxymbox{\hyperlink{class_beam_1_1_master_node_a2caffedb942a9e898743cdded505ff3e}{get\+Peak\+Level}} () const
\item 
std::\+string \doxymbox{\hyperlink{class_beam_1_1_master_node_ab2d3838f891f7bb46980f3eeba9ff905}{get\+Name}} () const override
\item 
std::\+vector$<$ \doxymbox{\hyperlink{struct_beam_1_1_flux_node_1_1_port}{Port}} $>$ \doxymbox{\hyperlink{class_beam_1_1_master_node_adb6c1fcb26bf9b50765a3c47c9f3841a}{get\+Input\+Ports}} () const override
\item 
std::\+vector$<$ \doxymbox{\hyperlink{struct_beam_1_1_flux_node_1_1_port}{Port}} $>$ \doxymbox{\hyperlink{class_beam_1_1_master_node_adbe0570947ffbcfcd71e77df7e844797}{get\+Output\+Ports}} () const override
\end{DoxyCompactItemize}
\doxysubsection*{Public Member Functions inherited from \doxymbox{\hyperlink{class_beam_1_1_flux_node}{Beam::\+\+Flux\+Node}}}
\begin{DoxyCompactItemize}
\item 
virtual \doxymbox{\hyperlink{class_beam_1_1_flux_node_a708c135cdb61e8838469998cd8a84e65}{\texorpdfstring{$\sim$}{\string~}\+Flux\+Node}} ()=default
\item 
virtual void \doxymbox{\hyperlink{class_beam_1_1_flux_node_ae9d1e151eff5166de969f45de06d5596}{process\+MIDI}} (const \doxymbox{\hyperlink{class_beam_1_1_m_i_d_i_buffer}{MIDIBuffer}} \&midi)
\begin{DoxyCompactList}\small\item\em Optional MIDI processing. Called before \doxylink{class_beam_1_1_flux_node_a3c263446753fa7ae5ff6928ee57bcd4d}{process()} in the engine loop. \end{DoxyCompactList}\item 
virtual void \doxymbox{\hyperlink{class_beam_1_1_flux_node_ace8cc49479d8924d44bca5fd4cd955e2}{on\+Transport\+State\+Changed}} (bool playing)
\begin{DoxyCompactList}\small\item\em Responds to global transport changes (Play/\+\+Pause). \end{DoxyCompactList}\item 
virtual void \doxymbox{\hyperlink{class_beam_1_1_flux_node_adc7c4e979bf27de5bfca66815ae97a67}{on\+Transport\+Seek}} (size\+\_\+t frame)
\begin{DoxyCompactList}\small\item\em Responds to timeline seeking. \end{DoxyCompactList}\item 
void \doxymbox{\hyperlink{class_beam_1_1_flux_node_aa579ec06608fd776987bbb089f27fd94}{set\+Current\+Frame}} (size\+\_\+t frame)
\begin{DoxyCompactList}\small\item\em Sets the current playhead position for this block. \end{DoxyCompactList}\item 
float \texorpdfstring{$\ast$}{*} \doxymbox{\hyperlink{class_beam_1_1_flux_node_ac90bd1a05b5bed3d68978f532386ed29}{get\+Input\+Buffer}} (int port\+Idx)
\item 
float \texorpdfstring{$\ast$}{*} \doxymbox{\hyperlink{class_beam_1_1_flux_node_abf11cfd4f2346ee0cd46d4345f1ed7d4}{get\+Output\+Buffer}} (int port\+Idx)
\item 
void \doxymbox{\hyperlink{class_beam_1_1_flux_node_af37f8c1b6b825da2ce7e35011d6f8253}{set\+Bypass}} (bool bypass)
\item 
bool \doxymbox{\hyperlink{class_beam_1_1_flux_node_a4bd30f3c8d311afdcd5c0d208e3bbf0f}{is\+Bypassed}} () const
\item 
void \doxymbox{\hyperlink{class_beam_1_1_flux_node_ad53f3fcaa5737f46d88530f40dbfbe32}{add\+Parameter}} (std::\+shared\+\_\+ptr$<$ \doxymbox{\hyperlink{class_beam_1_1_parameter}{Parameter}} $>$ \doxymbox{\hyperlink{texture_8cpp_aaded45152436a99bb4f9bda081df9f69}{param}})
\item 
std::\+shared\+\_\+ptr$<$ \doxymbox{\hyperlink{class_beam_1_1_parameter}{Parameter}} $>$ \doxymbox{\hyperlink{class_beam_1_1_flux_node_a59a32442eec144010741b9f2086c516e}{get\+Parameter}} (const std::\+string \&name)
\item 
const std::\+map$<$ std::\+string, std::\+shared\+\_\+ptr$<$ \doxymbox{\hyperlink{class_beam_1_1_parameter}{Parameter}} $>$ $>$ \& \doxymbox{\hyperlink{class_beam_1_1_flux_node_a6296c79b1ba77aa8b9526ace4a109529}{get\+Parameters}} () const
\end{DoxyCompactItemize}
\doxysubsubsection*{Private Attributes}
\begin{DoxyCompactItemize}
\item 
std::\+atomic$<$ float $>$ \doxymbox{\hyperlink{class_beam_1_1_master_node_a6f7c982071eab43ff658aa9c26e64a76}{m\+\_\+current\+Peak}}
\end{DoxyCompactItemize}
\doxysubsubsection*{Additional Inherited Members}
\doxysubsection*{Protected Member Functions inherited from \doxymbox{\hyperlink{class_beam_1_1_flux_node}{Beam::\+\+Flux\+Node}}}
\begin{DoxyCompactItemize}
\item 
void \doxymbox{\hyperlink{class_beam_1_1_flux_node_ae3bafc1c5a1aa545167256172b3d3688}{setup\+Buffers}} (int num\+Inputs, int num\+Outputs, int buffer\+Size, int channels)
\begin{DoxyCompactList}\small\item\em Pre-\/allocates buffers for inputs and outputs. \end{DoxyCompactList}\end{DoxyCompactItemize}
\doxysubsection*{Protected Attributes inherited from \doxymbox{\hyperlink{class_beam_1_1_flux_node}{Beam::\+\+Flux\+Node}}}
\begin{DoxyCompactItemize}
\item 
std::\+vector$<$ std::\+vector$<$ float $>$ $>$ \doxymbox{\hyperlink{class_beam_1_1_flux_node_a8edab1c9ebd83e73bbfd92af29d6e92c}{m\+\_\+inputs}}
\item 
std::\+vector$<$ std::\+vector$<$ float $>$ $>$ \doxymbox{\hyperlink{class_beam_1_1_flux_node_a496905f0ff42c432eb38e19bd6135383}{m\+\_\+outputs}}
\item 
std::\+map$<$ std::\+string, std::\+shared\+\_\+ptr$<$ \doxymbox{\hyperlink{class_beam_1_1_parameter}{Parameter}} $>$ $>$ \doxymbox{\hyperlink{class_beam_1_1_flux_node_a65628a37cd2dd2832eda60e74ec1aed3}{m\+\_\+parameters}}
\item 
std::\+atomic$<$ bool $>$ \doxymbox{\hyperlink{class_beam_1_1_flux_node_a6116dcdcfa20998fe90dc75a74f25d9b}{m\+\_\+bypassed}} \{false\}
\item 
size\+\_\+t \doxymbox{\hyperlink{class_beam_1_1_flux_node_a7d8556ddb1482f997cda7749d737668b}{m\+\_\+current\+Frame}} = 0
\end{DoxyCompactItemize}


\doxysubsection{Detailed Description}
The final sink in the audio graph. Handles global volume, metering, and transformer saturation. 

\label{doc-constructors}
\Hypertarget{class_beam_1_1_master_node_doc-constructors}
\doxysubsection{Constructor \& Destructor Documentation}
\Hypertarget{class_beam_1_1_master_node_aa47be01f91f6c96952fb17f96661e086}\index{Beam::MasterNode@{Beam::MasterNode}!MasterNode@{MasterNode}}
\index{MasterNode@{MasterNode}!Beam::MasterNode@{Beam::MasterNode}}
\doxysubsubsection{\texorpdfstring{MasterNode()}{MasterNode()}}
{\footnotesize\ttfamily \label{class_beam_1_1_master_node_aa47be01f91f6c96952fb17f96661e086} 
Beam::\+\+Master\+Node::\+\+Master\+Node (\begin{DoxyParamCaption}\item[{int}]{buffer\+Size}{}\end{DoxyParamCaption})\hspace{0.3cm}{\ttfamily [inline]}}



\label{doc-func-members}
\Hypertarget{class_beam_1_1_master_node_doc-func-members}
\doxysubsection{Member Function Documentation}
\Hypertarget{class_beam_1_1_master_node_adb6c1fcb26bf9b50765a3c47c9f3841a}\index{Beam::MasterNode@{Beam::MasterNode}!getInputPorts@{getInputPorts}}
\index{getInputPorts@{getInputPorts}!Beam::MasterNode@{Beam::MasterNode}}
\doxysubsubsection{\texorpdfstring{getInputPorts()}{getInputPorts()}}
{\footnotesize\ttfamily \label{class_beam_1_1_master_node_adb6c1fcb26bf9b50765a3c47c9f3841a} 
std::\+vector$<$ \doxymbox{\hyperlink{struct_beam_1_1_flux_node_1_1_port}{Port}} $>$ Beam::\+\+Master\+Node::\+get\+Input\+Ports (\begin{DoxyParamCaption}{}{}\end{DoxyParamCaption}) const\hspace{0.3cm}{\ttfamily [inline]}, {\ttfamily [override]}, {\ttfamily [virtual]}}



Implements \doxymbox{\hyperlink{class_beam_1_1_flux_node_a17eb02187925b52bf8e53fa3ebe3da66}{Beam::\+\+Flux\+Node}}.

\Hypertarget{class_beam_1_1_master_node_ab2d3838f891f7bb46980f3eeba9ff905}\index{Beam::MasterNode@{Beam::MasterNode}!getName@{getName}}
\index{getName@{getName}!Beam::MasterNode@{Beam::MasterNode}}
\doxysubsubsection{\texorpdfstring{getName()}{getName()}}
{\footnotesize\ttfamily \label{class_beam_1_1_master_node_ab2d3838f891f7bb46980f3eeba9ff905} 
std::\+string Beam::\+\+Master\+Node::\+get\+Name (\begin{DoxyParamCaption}{}{}\end{DoxyParamCaption}) const\hspace{0.3cm}{\ttfamily [inline]}, {\ttfamily [override]}, {\ttfamily [virtual]}}



Implements \doxymbox{\hyperlink{class_beam_1_1_flux_node_ac638d3d9bb1050d658294bc5470abeba}{Beam::\+\+Flux\+Node}}.

\Hypertarget{class_beam_1_1_master_node_adbe0570947ffbcfcd71e77df7e844797}\index{Beam::MasterNode@{Beam::MasterNode}!getOutputPorts@{getOutputPorts}}
\index{getOutputPorts@{getOutputPorts}!Beam::MasterNode@{Beam::MasterNode}}
\doxysubsubsection{\texorpdfstring{getOutputPorts()}{getOutputPorts()}}
{\footnotesize\ttfamily \label{class_beam_1_1_master_node_adbe0570947ffbcfcd71e77df7e844797} 
std::\+vector$<$ \doxymbox{\hyperlink{struct_beam_1_1_flux_node_1_1_port}{Port}} $>$ Beam::\+\+Master\+Node::\+get\+Output\+Ports (\begin{DoxyParamCaption}{}{}\end{DoxyParamCaption}) const\hspace{0.3cm}{\ttfamily [inline]}, {\ttfamily [override]}, {\ttfamily [virtual]}}



Implements \doxymbox{\hyperlink{class_beam_1_1_flux_node_a034f59d236afd7901ed84090422e3279}{Beam::\+\+Flux\+Node}}.

\Hypertarget{class_beam_1_1_master_node_a2caffedb942a9e898743cdded505ff3e}\index{Beam::MasterNode@{Beam::MasterNode}!getPeakLevel@{getPeakLevel}}
\index{getPeakLevel@{getPeakLevel}!Beam::MasterNode@{Beam::MasterNode}}
\doxysubsubsection{\texorpdfstring{getPeakLevel()}{getPeakLevel()}}
{\footnotesize\ttfamily \label{class_beam_1_1_master_node_a2caffedb942a9e898743cdded505ff3e} 
float Beam::\+\+Master\+Node::\+get\+Peak\+Level (\begin{DoxyParamCaption}{}{}\end{DoxyParamCaption}) const\hspace{0.3cm}{\ttfamily [inline]}}

\Hypertarget{class_beam_1_1_master_node_a4d78fec9962d8b9fd8a34142d0a129f7}\index{Beam::MasterNode@{Beam::MasterNode}!process@{process}}
\index{process@{process}!Beam::MasterNode@{Beam::MasterNode}}
\doxysubsubsection{\texorpdfstring{process()}{process()}}
{\footnotesize\ttfamily \label{class_beam_1_1_master_node_a4d78fec9962d8b9fd8a34142d0a129f7} 
void Beam::\+\+Master\+Node::\+process (\begin{DoxyParamCaption}\item[{int}]{frames}{}\end{DoxyParamCaption})\hspace{0.3cm}{\ttfamily [inline]}, {\ttfamily [override]}, {\ttfamily [virtual]}}



Main audio processing method. Must be implemented by subclasses. 


\begin{DoxyParams}{Parameters}
{\em frames} & Number of frames to process in the current block. \\
\hline
\end{DoxyParams}


Implements \doxymbox{\hyperlink{class_beam_1_1_flux_node_a3c263446753fa7ae5ff6928ee57bcd4d}{Beam::\+\+Flux\+Node}}.



\label{doc-variable-members}
\Hypertarget{class_beam_1_1_master_node_doc-variable-members}
\doxysubsection{Member Data Documentation}
\Hypertarget{class_beam_1_1_master_node_a6f7c982071eab43ff658aa9c26e64a76}\index{Beam::MasterNode@{Beam::MasterNode}!m\_currentPeak@{m\_currentPeak}}
\index{m\_currentPeak@{m\_currentPeak}!Beam::MasterNode@{Beam::MasterNode}}
\doxysubsubsection{\texorpdfstring{m\_currentPeak}{m\_currentPeak}}
{\footnotesize\ttfamily \label{class_beam_1_1_master_node_a6f7c982071eab43ff658aa9c26e64a76} 
std::\+atomic$<$float$>$ Beam::\+\+Master\+Node::\+m\+\_\+current\+Peak\hspace{0.3cm}{\ttfamily [private]}}



The documentation for this class was generated from the following file:\+\begin{DoxyCompactItemize}
\item 
src/\+engine/\+\doxymbox{\hyperlink{master__node_8hpp}{master\+\_\+node.\+hpp}}\end{DoxyCompactItemize}

\doxysection{Beam::\+Master\+Strip Class Reference}
\hypertarget{class_beam_1_1_master_strip}{}\label{class_beam_1_1_master_strip}\index{Beam::MasterStrip@{Beam::MasterStrip}}


{\ttfamily \+\#include $<$master\+\_\+strip.\+hpp$>$}

Inheritance diagram for Beam::\+Master\+Strip:\+\begin{figure}[H]
\begin{center}
\leavevmode
\includegraphics[height=2.000000cm]{class_beam_1_1_master_strip}
\end{center}
\end{figure}
\doxysubsubsection*{Public Member Functions}
\begin{DoxyCompactItemize}
\item 
\doxymbox{\hyperlink{class_beam_1_1_master_strip_a76488816e88c7e0665e678ad6c92d4cb}{Master\+Strip}} (std::\+shared\+\_\+ptr$<$ \doxymbox{\hyperlink{class_beam_1_1_master_node}{Master\+Node}} $>$ master\+Node)
\item 
void \doxymbox{\hyperlink{class_beam_1_1_master_strip_a021372e32a0033a9ed5f3224f136c893}{set\+Bounds}} (float x, float y, float w, float h) override
\item 
bool \doxymbox{\hyperlink{class_beam_1_1_master_strip_a81ac423990bbd8e9e726c9915fa8b3c6}{on\+Mouse\+Down}} (float x, float y, int button) override
\item 
bool \doxymbox{\hyperlink{class_beam_1_1_master_strip_a6cf43dd82f1ea4b9d5e0114f949cecb3}{on\+Mouse\+Move}} (float x, float y) override
\item 
void \doxymbox{\hyperlink{class_beam_1_1_master_strip_a7d46628ba541556ea94e8345633d813e}{render}} (\doxymbox{\hyperlink{class_beam_1_1_quad_batcher}{Quad\+Batcher}} \&batcher, float dt, float screenW, float screenH) override
\end{DoxyCompactItemize}
\doxysubsection*{Public Member Functions inherited from \doxymbox{\hyperlink{class_beam_1_1_component}{Beam::\+\+Component}}}
\begin{DoxyCompactItemize}
\item 
virtual \doxymbox{\hyperlink{class_beam_1_1_component_af9d734d649978e027412a87bc54362cd}{\texorpdfstring{$\sim$}{\string~}\+Component}} ()=default
\item 
virtual void \doxymbox{\hyperlink{class_beam_1_1_component_ad3d3fb19d25b4371d07620567970a158}{update}} (float dt)
\item 
virtual bool \doxymbox{\hyperlink{class_beam_1_1_component_ae36b8e9d70e8f9a1b9ba81c23c54d5c8}{on\+Mouse\+Up}} (float x, float y, int button)
\item 
virtual bool \doxymbox{\hyperlink{class_beam_1_1_component_ab92e884903f8a621fcd57bc00a24b041}{on\+Mouse\+Wheel}} (float x, float y, float delta)
\item 
const \doxymbox{\hyperlink{struct_beam_1_1_rect}{Rect}} \& \doxymbox{\hyperlink{class_beam_1_1_component_a5746dbc69d5b0adb4cffbcf920936d00}{get\+Bounds}} () const
\item 
void \doxymbox{\hyperlink{class_beam_1_1_component_a00d4e2dfa7703e59d6486852321dbdf1}{set\+Draggable}} (bool draggable)
\item 
void \doxymbox{\hyperlink{class_beam_1_1_component_aca7b02d1dddf7cd20378db9e3242fb84}{start\+Dragging}} (float x, float y)
\end{DoxyCompactItemize}
\doxysubsubsection*{Private Member Functions}
\begin{DoxyCompactItemize}
\item 
float \doxymbox{\hyperlink{class_beam_1_1_master_strip_a31e45bb02d409d44ed432337d0596ad9}{get\+FaderY}} ()
\end{DoxyCompactItemize}
\doxysubsubsection*{Private Attributes}
\begin{DoxyCompactItemize}
\item 
std::\+shared\+\_\+ptr$<$ \doxymbox{\hyperlink{class_beam_1_1_master_node}{Master\+Node}} $>$ \doxymbox{\hyperlink{class_beam_1_1_master_strip_a288c4b2e8a1f7b409d9bf713e3f18ae3}{m\+\_\+master\+Node}}
\item 
std::\+shared\+\_\+ptr$<$ \doxymbox{\hyperlink{class_beam_1_1_v_u_meter}{VUMeter}} $>$ \doxymbox{\hyperlink{class_beam_1_1_master_strip_ab6f5f350f02f06c97eab6bb78770787d}{m\+\_\+vu\+Meter}}
\end{DoxyCompactItemize}
\doxysubsubsection*{Additional Inherited Members}
\doxysubsection*{Protected Attributes inherited from \doxymbox{\hyperlink{class_beam_1_1_component}{Beam::\+\+Component}}}
\begin{DoxyCompactItemize}
\item 
\doxymbox{\hyperlink{struct_beam_1_1_rect}{Rect}} \doxymbox{\hyperlink{class_beam_1_1_component_a4f1ec4a5fb168c39a6c18f958b2b1495}{m\+\_\+bounds}} \{0, 0, 0, 0\}
\item 
bool \doxymbox{\hyperlink{class_beam_1_1_component_adc07913aed6ddadf1c730e7b3bb599cf}{m\+\_\+is\+Visible}} = true
\item 
bool \doxymbox{\hyperlink{class_beam_1_1_component_a0bf77b204ae374a14b5a6d7e5a3c13c6}{m\+\_\+is\+Enabled}} = true
\item 
bool \doxymbox{\hyperlink{class_beam_1_1_component_a9646efcaa9540a26a387f5da9aae4bde}{m\+\_\+is\+Draggable}} = false
\item 
bool \doxymbox{\hyperlink{class_beam_1_1_component_ab03af9a9743acf040f38e3fb11f8dc14}{m\+\_\+is\+Dragging}} = false
\item 
float \doxymbox{\hyperlink{class_beam_1_1_component_a7110b2b9dc235f724bf4689569266a63}{m\+\_\+last\+MouseX}} = 0
\item 
float \doxymbox{\hyperlink{class_beam_1_1_component_a768931a0f51394bf011f821f6ed2efe9}{m\+\_\+last\+MouseY}} = 0
\end{DoxyCompactItemize}


\label{doc-constructors}
\Hypertarget{class_beam_1_1_master_strip_doc-constructors}
\doxysubsection{Constructor \& Destructor Documentation}
\Hypertarget{class_beam_1_1_master_strip_a76488816e88c7e0665e678ad6c92d4cb}\index{Beam::MasterStrip@{Beam::MasterStrip}!MasterStrip@{MasterStrip}}
\index{MasterStrip@{MasterStrip}!Beam::MasterStrip@{Beam::MasterStrip}}
\doxysubsubsection{\texorpdfstring{MasterStrip()}{MasterStrip()}}
{\footnotesize\ttfamily \label{class_beam_1_1_master_strip_a76488816e88c7e0665e678ad6c92d4cb} 
Beam::\+\+Master\+Strip::\+\+Master\+Strip (\begin{DoxyParamCaption}\item[{std::\+shared\+\_\+ptr$<$ \doxymbox{\hyperlink{class_beam_1_1_master_node}{Master\+Node}} $>$}]{master\+Node}{}\end{DoxyParamCaption})\hspace{0.3cm}{\ttfamily [inline]}}



\label{doc-func-members}
\Hypertarget{class_beam_1_1_master_strip_doc-func-members}
\doxysubsection{Member Function Documentation}
\Hypertarget{class_beam_1_1_master_strip_a31e45bb02d409d44ed432337d0596ad9}\index{Beam::MasterStrip@{Beam::MasterStrip}!getFaderY@{getFaderY}}
\index{getFaderY@{getFaderY}!Beam::MasterStrip@{Beam::MasterStrip}}
\doxysubsubsection{\texorpdfstring{getFaderY()}{getFaderY()}}
{\footnotesize\ttfamily \label{class_beam_1_1_master_strip_a31e45bb02d409d44ed432337d0596ad9} 
float Beam::\+\+Master\+Strip::\+get\+FaderY (\begin{DoxyParamCaption}{}{}\end{DoxyParamCaption})\hspace{0.3cm}{\ttfamily [inline]}, {\ttfamily [private]}}

\Hypertarget{class_beam_1_1_master_strip_a81ac423990bbd8e9e726c9915fa8b3c6}\index{Beam::MasterStrip@{Beam::MasterStrip}!onMouseDown@{onMouseDown}}
\index{onMouseDown@{onMouseDown}!Beam::MasterStrip@{Beam::MasterStrip}}
\doxysubsubsection{\texorpdfstring{onMouseDown()}{onMouseDown()}}
{\footnotesize\ttfamily \label{class_beam_1_1_master_strip_a81ac423990bbd8e9e726c9915fa8b3c6} 
bool Beam::\+\+Master\+Strip::\+on\+Mouse\+Down (\begin{DoxyParamCaption}\item[{float}]{x}{, }\item[{float}]{y}{, }\item[{int}]{button}{}\end{DoxyParamCaption})\hspace{0.3cm}{\ttfamily [inline]}, {\ttfamily [override]}, {\ttfamily [virtual]}}



Reimplemented from \doxymbox{\hyperlink{class_beam_1_1_component_aec1da33d2d6e3d4e7dd6708309264e76}{Beam::\+\+Component}}.

\Hypertarget{class_beam_1_1_master_strip_a6cf43dd82f1ea4b9d5e0114f949cecb3}\index{Beam::MasterStrip@{Beam::MasterStrip}!onMouseMove@{onMouseMove}}
\index{onMouseMove@{onMouseMove}!Beam::MasterStrip@{Beam::MasterStrip}}
\doxysubsubsection{\texorpdfstring{onMouseMove()}{onMouseMove()}}
{\footnotesize\ttfamily \label{class_beam_1_1_master_strip_a6cf43dd82f1ea4b9d5e0114f949cecb3} 
bool Beam::\+\+Master\+Strip::\+on\+Mouse\+Move (\begin{DoxyParamCaption}\item[{float}]{x}{, }\item[{float}]{y}{}\end{DoxyParamCaption})\hspace{0.3cm}{\ttfamily [inline]}, {\ttfamily [override]}, {\ttfamily [virtual]}}



Reimplemented from \doxymbox{\hyperlink{class_beam_1_1_component_a9d8e5970783d315044277a1228659e6c}{Beam::\+\+Component}}.

\Hypertarget{class_beam_1_1_master_strip_a7d46628ba541556ea94e8345633d813e}\index{Beam::MasterStrip@{Beam::MasterStrip}!render@{render}}
\index{render@{render}!Beam::MasterStrip@{Beam::MasterStrip}}
\doxysubsubsection{\texorpdfstring{render()}{render()}}
{\footnotesize\ttfamily \label{class_beam_1_1_master_strip_a7d46628ba541556ea94e8345633d813e} 
void Beam::\+\+Master\+Strip::\+render (\begin{DoxyParamCaption}\item[{\doxymbox{\hyperlink{class_beam_1_1_quad_batcher}{Quad\+Batcher}} \&}]{batcher}{, }\item[{float}]{dt}{, }\item[{float}]{screenW}{, }\item[{float}]{screenH}{}\end{DoxyParamCaption})\hspace{0.3cm}{\ttfamily [inline]}, {\ttfamily [override]}, {\ttfamily [virtual]}}



Implements \doxymbox{\hyperlink{class_beam_1_1_component_acef3496a55f0d94c8678f6049dbaa7cd}{Beam::\+\+Component}}.

\Hypertarget{class_beam_1_1_master_strip_a021372e32a0033a9ed5f3224f136c893}\index{Beam::MasterStrip@{Beam::MasterStrip}!setBounds@{setBounds}}
\index{setBounds@{setBounds}!Beam::MasterStrip@{Beam::MasterStrip}}
\doxysubsubsection{\texorpdfstring{setBounds()}{setBounds()}}
{\footnotesize\ttfamily \label{class_beam_1_1_master_strip_a021372e32a0033a9ed5f3224f136c893} 
void Beam::\+\+Master\+Strip::\+set\+Bounds (\begin{DoxyParamCaption}\item[{float}]{x}{, }\item[{float}]{y}{, }\item[{float}]{w}{, }\item[{float}]{h}{}\end{DoxyParamCaption})\hspace{0.3cm}{\ttfamily [inline]}, {\ttfamily [override]}, {\ttfamily [virtual]}}



Reimplemented from \doxymbox{\hyperlink{class_beam_1_1_component_a6865b1f22388af467bf6c789120fac05}{Beam::\+\+Component}}.



\label{doc-variable-members}
\Hypertarget{class_beam_1_1_master_strip_doc-variable-members}
\doxysubsection{Member Data Documentation}
\Hypertarget{class_beam_1_1_master_strip_a288c4b2e8a1f7b409d9bf713e3f18ae3}\index{Beam::MasterStrip@{Beam::MasterStrip}!m\_masterNode@{m\_masterNode}}
\index{m\_masterNode@{m\_masterNode}!Beam::MasterStrip@{Beam::MasterStrip}}
\doxysubsubsection{\texorpdfstring{m\_masterNode}{m\_masterNode}}
{\footnotesize\ttfamily \label{class_beam_1_1_master_strip_a288c4b2e8a1f7b409d9bf713e3f18ae3} 
std::\+shared\+\_\+ptr$<$\doxymbox{\hyperlink{class_beam_1_1_master_node}{Master\+Node}}$>$ Beam::\+\+Master\+Strip::\+m\+\_\+master\+Node\hspace{0.3cm}{\ttfamily [private]}}

\Hypertarget{class_beam_1_1_master_strip_ab6f5f350f02f06c97eab6bb78770787d}\index{Beam::MasterStrip@{Beam::MasterStrip}!m\_vuMeter@{m\_vuMeter}}
\index{m\_vuMeter@{m\_vuMeter}!Beam::MasterStrip@{Beam::MasterStrip}}
\doxysubsubsection{\texorpdfstring{m\_vuMeter}{m\_vuMeter}}
{\footnotesize\ttfamily \label{class_beam_1_1_master_strip_ab6f5f350f02f06c97eab6bb78770787d} 
std::\+shared\+\_\+ptr$<$\doxymbox{\hyperlink{class_beam_1_1_v_u_meter}{VUMeter}}$>$ Beam::\+\+Master\+Strip::\+m\+\_\+vu\+Meter\hspace{0.3cm}{\ttfamily [private]}}



The documentation for this class was generated from the following file:\+\begin{DoxyCompactItemize}
\item 
src/\+interface/\+\doxymbox{\hyperlink{master__strip_8hpp}{master\+\_\+strip.\+hpp}}\end{DoxyCompactItemize}

\doxysection{Beam::\+MIDIBuffer Class Reference}
\hypertarget{class_beam_1_1_m_i_d_i_buffer}{}\label{class_beam_1_1_m_i_d_i_buffer}\index{Beam::MIDIBuffer@{Beam::MIDIBuffer}}


Container for MIDI events within a single processing block.  




{\ttfamily \+\#include $<$midi\+\_\+event.\+hpp$>$}

\doxysubsubsection*{Public Member Functions}
\begin{DoxyCompactItemize}
\item 
void \doxymbox{\hyperlink{class_beam_1_1_m_i_d_i_buffer_aefc427178e249e396fed1f6c557c76d4}{add\+Event}} (const \doxymbox{\hyperlink{struct_beam_1_1_m_i_d_i_event}{MIDIEvent}} \&event)
\item 
void \doxymbox{\hyperlink{class_beam_1_1_m_i_d_i_buffer_ab80798d76cc49a272ea5fe1bd814afb2}{clear}} ()
\item 
const std::\+vector$<$ \doxymbox{\hyperlink{struct_beam_1_1_m_i_d_i_event}{MIDIEvent}} $>$ \& \doxymbox{\hyperlink{class_beam_1_1_m_i_d_i_buffer_a14bc0875f35969aa61f0093ffcf320e5}{get\+Events}} () const
\end{DoxyCompactItemize}
\doxysubsubsection*{Private Attributes}
\begin{DoxyCompactItemize}
\item 
std::\+vector$<$ \doxymbox{\hyperlink{struct_beam_1_1_m_i_d_i_event}{MIDIEvent}} $>$ \doxymbox{\hyperlink{class_beam_1_1_m_i_d_i_buffer_aae67355f8fc483e47baec6d2bc8a52c0}{m\+\_\+events}}
\end{DoxyCompactItemize}


\doxysubsection{Detailed Description}
Container for MIDI events within a single processing block. 

\label{doc-func-members}
\Hypertarget{class_beam_1_1_m_i_d_i_buffer_doc-func-members}
\doxysubsection{Member Function Documentation}
\Hypertarget{class_beam_1_1_m_i_d_i_buffer_aefc427178e249e396fed1f6c557c76d4}\index{Beam::MIDIBuffer@{Beam::MIDIBuffer}!addEvent@{addEvent}}
\index{addEvent@{addEvent}!Beam::MIDIBuffer@{Beam::MIDIBuffer}}
\doxysubsubsection{\texorpdfstring{addEvent()}{addEvent()}}
{\footnotesize\ttfamily \label{class_beam_1_1_m_i_d_i_buffer_aefc427178e249e396fed1f6c557c76d4} 
void Beam::\+\+MIDIBuffer::\+add\+Event (\begin{DoxyParamCaption}\item[{const \doxymbox{\hyperlink{struct_beam_1_1_m_i_d_i_event}{MIDIEvent}} \&}]{event}{}\end{DoxyParamCaption})\hspace{0.3cm}{\ttfamily [inline]}}

\Hypertarget{class_beam_1_1_m_i_d_i_buffer_ab80798d76cc49a272ea5fe1bd814afb2}\index{Beam::MIDIBuffer@{Beam::MIDIBuffer}!clear@{clear}}
\index{clear@{clear}!Beam::MIDIBuffer@{Beam::MIDIBuffer}}
\doxysubsubsection{\texorpdfstring{clear()}{clear()}}
{\footnotesize\ttfamily \label{class_beam_1_1_m_i_d_i_buffer_ab80798d76cc49a272ea5fe1bd814afb2} 
void Beam::\+\+MIDIBuffer::\+clear (\begin{DoxyParamCaption}{}{}\end{DoxyParamCaption})\hspace{0.3cm}{\ttfamily [inline]}}

\Hypertarget{class_beam_1_1_m_i_d_i_buffer_a14bc0875f35969aa61f0093ffcf320e5}\index{Beam::MIDIBuffer@{Beam::MIDIBuffer}!getEvents@{getEvents}}
\index{getEvents@{getEvents}!Beam::MIDIBuffer@{Beam::MIDIBuffer}}
\doxysubsubsection{\texorpdfstring{getEvents()}{getEvents()}}
{\footnotesize\ttfamily \label{class_beam_1_1_m_i_d_i_buffer_a14bc0875f35969aa61f0093ffcf320e5} 
const std::\+vector$<$ \doxymbox{\hyperlink{struct_beam_1_1_m_i_d_i_event}{MIDIEvent}} $>$ \& Beam::\+\+MIDIBuffer::\+get\+Events (\begin{DoxyParamCaption}{}{}\end{DoxyParamCaption}) const\hspace{0.3cm}{\ttfamily [inline]}}



\label{doc-variable-members}
\Hypertarget{class_beam_1_1_m_i_d_i_buffer_doc-variable-members}
\doxysubsection{Member Data Documentation}
\Hypertarget{class_beam_1_1_m_i_d_i_buffer_aae67355f8fc483e47baec6d2bc8a52c0}\index{Beam::MIDIBuffer@{Beam::MIDIBuffer}!m\_events@{m\_events}}
\index{m\_events@{m\_events}!Beam::MIDIBuffer@{Beam::MIDIBuffer}}
\doxysubsubsection{\texorpdfstring{m\_events}{m\_events}}
{\footnotesize\ttfamily \label{class_beam_1_1_m_i_d_i_buffer_aae67355f8fc483e47baec6d2bc8a52c0} 
std::\+vector$<$\doxymbox{\hyperlink{struct_beam_1_1_m_i_d_i_event}{MIDIEvent}}$>$ Beam::\+\+MIDIBuffer::\+m\+\_\+events\hspace{0.3cm}{\ttfamily [private]}}



The documentation for this class was generated from the following file:\+\begin{DoxyCompactItemize}
\item 
src/\+engine/\+\doxymbox{\hyperlink{midi__event_8hpp}{midi\+\_\+event.\+hpp}}\end{DoxyCompactItemize}

\doxysection{Beam::\+MIDIEvent Struct Reference}
\hypertarget{struct_beam_1_1_m_i_d_i_event}{}\label{struct_beam_1_1_m_i_d_i_event}\index{Beam::MIDIEvent@{Beam::MIDIEvent}}


A raw MIDI event with a local frame offset.  




{\ttfamily \+\#include $<$midi\+\_\+event.\+hpp$>$}

\doxysubsubsection*{Public Attributes}
\begin{DoxyCompactItemize}
\item 
uint32\+\_\+t \doxymbox{\hyperlink{struct_beam_1_1_m_i_d_i_event_ab9b0b3b38c88dd1413ee9b60c6e3ea1d}{frame\+Offset}}
\begin{DoxyCompactList}\small\item\em Offset from the start of the current processing block. \end{DoxyCompactList}\item 
uint8\+\_\+t \doxymbox{\hyperlink{struct_beam_1_1_m_i_d_i_event_a79c5cc1da6a0fb88d12a509d43da4e89}{status}}
\begin{DoxyCompactList}\small\item\em MIDI Status byte. \end{DoxyCompactList}\item 
uint8\+\_\+t \doxymbox{\hyperlink{struct_beam_1_1_m_i_d_i_event_a83330b2aacc33a73c5f87dd9c93866b9}{data1}}
\begin{DoxyCompactList}\small\item\em MIDI Data 1 (e.\+g., Note Number). \end{DoxyCompactList}\item 
uint8\+\_\+t \doxymbox{\hyperlink{struct_beam_1_1_m_i_d_i_event_a7aa6ccf405a4ed9642b807eb77e54c04}{data2}}
\begin{DoxyCompactList}\small\item\em MIDI Data 2 (e.\+g., Velocity). \end{DoxyCompactList}\end{DoxyCompactItemize}


\doxysubsection{Detailed Description}
A raw MIDI event with a local frame offset. 

\label{doc-variable-members}
\Hypertarget{struct_beam_1_1_m_i_d_i_event_doc-variable-members}
\doxysubsection{Member Data Documentation}
\Hypertarget{struct_beam_1_1_m_i_d_i_event_a83330b2aacc33a73c5f87dd9c93866b9}\index{Beam::MIDIEvent@{Beam::MIDIEvent}!data1@{data1}}
\index{data1@{data1}!Beam::MIDIEvent@{Beam::MIDIEvent}}
\doxysubsubsection{\texorpdfstring{data1}{data1}}
{\footnotesize\ttfamily \label{struct_beam_1_1_m_i_d_i_event_a83330b2aacc33a73c5f87dd9c93866b9} 
uint8\+\_\+t Beam::\+\+MIDIEvent::\+data1}



MIDI Data 1 (e.\+g., Note Number). 

\Hypertarget{struct_beam_1_1_m_i_d_i_event_a7aa6ccf405a4ed9642b807eb77e54c04}\index{Beam::MIDIEvent@{Beam::MIDIEvent}!data2@{data2}}
\index{data2@{data2}!Beam::MIDIEvent@{Beam::MIDIEvent}}
\doxysubsubsection{\texorpdfstring{data2}{data2}}
{\footnotesize\ttfamily \label{struct_beam_1_1_m_i_d_i_event_a7aa6ccf405a4ed9642b807eb77e54c04} 
uint8\+\_\+t Beam::\+\+MIDIEvent::\+data2}



MIDI Data 2 (e.\+g., Velocity). 

\Hypertarget{struct_beam_1_1_m_i_d_i_event_ab9b0b3b38c88dd1413ee9b60c6e3ea1d}\index{Beam::MIDIEvent@{Beam::MIDIEvent}!frameOffset@{frameOffset}}
\index{frameOffset@{frameOffset}!Beam::MIDIEvent@{Beam::MIDIEvent}}
\doxysubsubsection{\texorpdfstring{frameOffset}{frameOffset}}
{\footnotesize\ttfamily \label{struct_beam_1_1_m_i_d_i_event_ab9b0b3b38c88dd1413ee9b60c6e3ea1d} 
uint32\+\_\+t Beam::\+\+MIDIEvent::\+frame\+Offset}



Offset from the start of the current processing block. 

\Hypertarget{struct_beam_1_1_m_i_d_i_event_a79c5cc1da6a0fb88d12a509d43da4e89}\index{Beam::MIDIEvent@{Beam::MIDIEvent}!status@{status}}
\index{status@{status}!Beam::MIDIEvent@{Beam::MIDIEvent}}
\doxysubsubsection{\texorpdfstring{status}{status}}
{\footnotesize\ttfamily \label{struct_beam_1_1_m_i_d_i_event_a79c5cc1da6a0fb88d12a509d43da4e89} 
uint8\+\_\+t Beam::\+\+MIDIEvent::\+status}



MIDI Status byte. 



The documentation for this struct was generated from the following file:\+\begin{DoxyCompactItemize}
\item 
src/\+engine/\+\doxymbox{\hyperlink{midi__event_8hpp}{midi\+\_\+event.\+hpp}}\end{DoxyCompactItemize}

\doxysection{Beam::\+Mouse\+Event Class Reference}
\hypertarget{class_beam_1_1_mouse_event}{}\label{class_beam_1_1_mouse_event}\index{Beam::MouseEvent@{Beam::MouseEvent}}


{\ttfamily \+\#include $<$gui\+\_\+component.\+hpp$>$}

\doxysubsubsection*{Public Member Functions}
\begin{DoxyCompactItemize}
\item 
\doxymbox{\hyperlink{class_beam_1_1_mouse_event_a2357c0cf0cb23d8e8eb9e096868e8a20}{Mouse\+Event}} (float x\+\_\+pos, float y\+\_\+pos, int event\+\_\+num, bool dragged=false)
\end{DoxyCompactItemize}
\doxysubsubsection*{Public Attributes}
\begin{DoxyCompactItemize}
\item 
float \doxymbox{\hyperlink{class_beam_1_1_mouse_event_ac992c524932d4138e691c9dc6c54d993}{x}}
\item 
float \doxymbox{\hyperlink{class_beam_1_1_mouse_event_abdf88fe0f48b090fa74563f8b171a834}{y}}
\item 
int \doxymbox{\hyperlink{class_beam_1_1_mouse_event_a1fab365659dbda0e7233ca3f2044ac85}{event\+Number}}
\item 
bool \doxymbox{\hyperlink{class_beam_1_1_mouse_event_ad9f9be2218a68836d9b23ef384bf6f2e}{mouse\+Was\+Dragged}}
\end{DoxyCompactItemize}


\label{doc-constructors}
\Hypertarget{class_beam_1_1_mouse_event_doc-constructors}
\doxysubsection{Constructor \& Destructor Documentation}
\Hypertarget{class_beam_1_1_mouse_event_a2357c0cf0cb23d8e8eb9e096868e8a20}\index{Beam::MouseEvent@{Beam::MouseEvent}!MouseEvent@{MouseEvent}}
\index{MouseEvent@{MouseEvent}!Beam::MouseEvent@{Beam::MouseEvent}}
\doxysubsubsection{\texorpdfstring{MouseEvent()}{MouseEvent()}}
{\footnotesize\ttfamily \label{class_beam_1_1_mouse_event_a2357c0cf0cb23d8e8eb9e096868e8a20} 
Beam::\+\+Mouse\+Event::\+\+Mouse\+Event (\begin{DoxyParamCaption}\item[{float}]{x\+\_\+pos}{, }\item[{float}]{y\+\_\+pos}{, }\item[{int}]{event\+\_\+num}{, }\item[{bool}]{dragged}{ = {\ttfamily false}}\end{DoxyParamCaption})\hspace{0.3cm}{\ttfamily [inline]}}



\label{doc-variable-members}
\Hypertarget{class_beam_1_1_mouse_event_doc-variable-members}
\doxysubsection{Member Data Documentation}
\Hypertarget{class_beam_1_1_mouse_event_a1fab365659dbda0e7233ca3f2044ac85}\index{Beam::MouseEvent@{Beam::MouseEvent}!eventNumber@{eventNumber}}
\index{eventNumber@{eventNumber}!Beam::MouseEvent@{Beam::MouseEvent}}
\doxysubsubsection{\texorpdfstring{eventNumber}{eventNumber}}
{\footnotesize\ttfamily \label{class_beam_1_1_mouse_event_a1fab365659dbda0e7233ca3f2044ac85} 
int Beam::\+\+Mouse\+Event::\+event\+Number}

\Hypertarget{class_beam_1_1_mouse_event_ad9f9be2218a68836d9b23ef384bf6f2e}\index{Beam::MouseEvent@{Beam::MouseEvent}!mouseWasDragged@{mouseWasDragged}}
\index{mouseWasDragged@{mouseWasDragged}!Beam::MouseEvent@{Beam::MouseEvent}}
\doxysubsubsection{\texorpdfstring{mouseWasDragged}{mouseWasDragged}}
{\footnotesize\ttfamily \label{class_beam_1_1_mouse_event_ad9f9be2218a68836d9b23ef384bf6f2e} 
bool Beam::\+\+Mouse\+Event::\+mouse\+Was\+Dragged}

\Hypertarget{class_beam_1_1_mouse_event_ac992c524932d4138e691c9dc6c54d993}\index{Beam::MouseEvent@{Beam::MouseEvent}!x@{x}}
\index{x@{x}!Beam::MouseEvent@{Beam::MouseEvent}}
\doxysubsubsection{\texorpdfstring{x}{x}}
{\footnotesize\ttfamily \label{class_beam_1_1_mouse_event_ac992c524932d4138e691c9dc6c54d993} 
float Beam::\+\+Mouse\+Event::\+x}

\Hypertarget{class_beam_1_1_mouse_event_abdf88fe0f48b090fa74563f8b171a834}\index{Beam::MouseEvent@{Beam::MouseEvent}!y@{y}}
\index{y@{y}!Beam::MouseEvent@{Beam::MouseEvent}}
\doxysubsubsection{\texorpdfstring{y}{y}}
{\footnotesize\ttfamily \label{class_beam_1_1_mouse_event_abdf88fe0f48b090fa74563f8b171a834} 
float Beam::\+\+Mouse\+Event::\+y}



The documentation for this class was generated from the following file:\+\begin{DoxyCompactItemize}
\item 
src/\+interface/\+\doxymbox{\hyperlink{gui__component_8hpp}{gui\+\_\+component.\+hpp}}\end{DoxyCompactItemize}

\doxysection{Beam::\+Node\+Execution Struct Reference}
\hypertarget{struct_beam_1_1_node_execution}{}\label{struct_beam_1_1_node_execution}\index{Beam::NodeExecution@{Beam::NodeExecution}}


{\ttfamily \+\#include $<$render\+\_\+plan.\+hpp$>$}

\doxysubsubsection*{Public Attributes}
\begin{DoxyCompactItemize}
\item 
std::\+shared\+\_\+ptr$<$ \doxymbox{\hyperlink{class_beam_1_1_flux_node}{Flux\+Node}} $>$ \doxymbox{\hyperlink{struct_beam_1_1_node_execution_a88f4b898b8ccd5ac9afb44306d78c93b}{node}}
\item 
std::\+vector$<$ \doxymbox{\hyperlink{struct_beam_1_1_signal_route}{Signal\+Route}} $>$ \doxymbox{\hyperlink{struct_beam_1_1_node_execution_aa9c5585184b096cfdc64dad13064f453}{outgoing\+Routes}}
\end{DoxyCompactItemize}


\label{doc-variable-members}
\Hypertarget{struct_beam_1_1_node_execution_doc-variable-members}
\doxysubsection{Member Data Documentation}
\Hypertarget{struct_beam_1_1_node_execution_a88f4b898b8ccd5ac9afb44306d78c93b}\index{Beam::NodeExecution@{Beam::NodeExecution}!node@{node}}
\index{node@{node}!Beam::NodeExecution@{Beam::NodeExecution}}
\doxysubsubsection{\texorpdfstring{node}{node}}
{\footnotesize\ttfamily \label{struct_beam_1_1_node_execution_a88f4b898b8ccd5ac9afb44306d78c93b} 
std::\+shared\+\_\+ptr$<$\doxymbox{\hyperlink{class_beam_1_1_flux_node}{Flux\+Node}}$>$ Beam::\+\+Node\+Execution::\+node}

\Hypertarget{struct_beam_1_1_node_execution_aa9c5585184b096cfdc64dad13064f453}\index{Beam::NodeExecution@{Beam::NodeExecution}!outgoingRoutes@{outgoingRoutes}}
\index{outgoingRoutes@{outgoingRoutes}!Beam::NodeExecution@{Beam::NodeExecution}}
\doxysubsubsection{\texorpdfstring{outgoingRoutes}{outgoingRoutes}}
{\footnotesize\ttfamily \label{struct_beam_1_1_node_execution_aa9c5585184b096cfdc64dad13064f453} 
std::\+vector$<$\doxymbox{\hyperlink{struct_beam_1_1_signal_route}{Signal\+Route}}$>$ Beam::\+\+Node\+Execution::\+outgoing\+Routes}



The documentation for this struct was generated from the following file:\+\begin{DoxyCompactItemize}
\item 
src/\+engine/\+\doxymbox{\hyperlink{render__plan_8hpp}{render\+\_\+plan.\+hpp}}\end{DoxyCompactItemize}

\doxysection{Beam::\+Offline\+Renderer Class Reference}
\hypertarget{class_beam_1_1_offline_renderer}{}\label{class_beam_1_1_offline_renderer}\index{Beam::OfflineRenderer@{Beam::OfflineRenderer}}


Renders the DSP graph to a file as fast as possible.  




{\ttfamily \+\#include $<$offline\+\_\+renderer.\+hpp$>$}

\doxysubsubsection*{Static Public Member Functions}
\begin{DoxyCompactItemize}
\item 
static bool \doxymbox{\hyperlink{class_beam_1_1_offline_renderer_a1818153c92b4aa390933cc7a3d8ed51b}{render\+To\+Wav}} (const std::\+string \&file\+Path, std::\+shared\+\_\+ptr$<$ \doxymbox{\hyperlink{class_beam_1_1_flux_graph}{Flux\+Graph}} $>$ graph, size\+\_\+t total\+Frames, int sample\+Rate=44100)
\end{DoxyCompactItemize}


\doxysubsection{Detailed Description}
Renders the DSP graph to a file as fast as possible. 

\label{doc-func-members}
\Hypertarget{class_beam_1_1_offline_renderer_doc-func-members}
\doxysubsection{Member Function Documentation}
\Hypertarget{class_beam_1_1_offline_renderer_a1818153c92b4aa390933cc7a3d8ed51b}\index{Beam::OfflineRenderer@{Beam::OfflineRenderer}!renderToWav@{renderToWav}}
\index{renderToWav@{renderToWav}!Beam::OfflineRenderer@{Beam::OfflineRenderer}}
\doxysubsubsection{\texorpdfstring{renderToWav()}{renderToWav()}}
{\footnotesize\ttfamily \label{class_beam_1_1_offline_renderer_a1818153c92b4aa390933cc7a3d8ed51b} 
bool Beam::\+\+Offline\+Renderer::\+render\+To\+Wav (\begin{DoxyParamCaption}\item[{const std::\+string \&}]{file\+Path}{, }\item[{std::\+shared\+\_\+ptr$<$ \doxymbox{\hyperlink{class_beam_1_1_flux_graph}{Flux\+Graph}} $>$}]{graph}{, }\item[{size\+\_\+t}]{total\+Frames}{, }\item[{int}]{sample\+Rate}{ = {\ttfamily 44100}}\end{DoxyParamCaption})\hspace{0.3cm}{\ttfamily [inline]}, {\ttfamily [static]}}



The documentation for this class was generated from the following file:\+\begin{DoxyCompactItemize}
\item 
src/\+engine/\+\doxymbox{\hyperlink{offline__renderer_8hpp}{offline\+\_\+renderer.\+hpp}}\end{DoxyCompactItemize}

\doxysection{Beam::\+Analog\+Base::\+One\+Pole\+Filter Class Reference}
\hypertarget{class_beam_1_1_analog_base_1_1_one_pole_filter}{}\label{class_beam_1_1_analog_base_1_1_one_pole_filter}\index{Beam::AnalogBase::OnePoleFilter@{Beam::AnalogBase::OnePoleFilter}}


Simple 6d\+B/\+oct filter for simulating cable capacitance or basic tone shaping.  




{\ttfamily \+\#include $<$analog\+\_\+base.\+hpp$>$}

\doxysubsubsection*{Public Member Functions}
\begin{DoxyCompactItemize}
\item 
void \doxymbox{\hyperlink{class_beam_1_1_analog_base_1_1_one_pole_filter_ad5c46d54019ff65ee1ec061abdb41dc5}{set\+Cutoff}} (float freq, float sample\+Rate)
\item 
float \doxymbox{\hyperlink{class_beam_1_1_analog_base_1_1_one_pole_filter_ad578c445671aaab83a8b9373b6f8f331}{process}} (float x)
\item 
void \doxymbox{\hyperlink{class_beam_1_1_analog_base_1_1_one_pole_filter_aed129437b0d6d53597daa4a79f29cb87}{reset}} ()
\end{DoxyCompactItemize}
\doxysubsubsection*{Private Attributes}
\begin{DoxyCompactItemize}
\item 
float \doxymbox{\hyperlink{class_beam_1_1_analog_base_1_1_one_pole_filter_a3900a50b36b06885c1597c689ab1cdf7}{m\+\_\+a0}} = 1.\+0f
\item 
float \doxymbox{\hyperlink{class_beam_1_1_analog_base_1_1_one_pole_filter_a7dc0974e70a5268b4a35552d27ebbdc1}{m\+\_\+b1}} = 0.\+0f
\item 
float \doxymbox{\hyperlink{class_beam_1_1_analog_base_1_1_one_pole_filter_aaf6d6623ac4884e68d496b3395ef6059}{m\+\_\+z1}} = 0.\+0f
\end{DoxyCompactItemize}


\doxysubsection{Detailed Description}
Simple 6d\+B/\+oct filter for simulating cable capacitance or basic tone shaping. 

\label{doc-func-members}
\Hypertarget{class_beam_1_1_analog_base_1_1_one_pole_filter_doc-func-members}
\doxysubsection{Member Function Documentation}
\Hypertarget{class_beam_1_1_analog_base_1_1_one_pole_filter_ad578c445671aaab83a8b9373b6f8f331}\index{Beam::AnalogBase::OnePoleFilter@{Beam::AnalogBase::OnePoleFilter}!process@{process}}
\index{process@{process}!Beam::AnalogBase::OnePoleFilter@{Beam::AnalogBase::OnePoleFilter}}
\doxysubsubsection{\texorpdfstring{process()}{process()}}
{\footnotesize\ttfamily \label{class_beam_1_1_analog_base_1_1_one_pole_filter_ad578c445671aaab83a8b9373b6f8f331} 
float Beam::\+\+Analog\+Base::\+\+One\+Pole\+Filter::\+process (\begin{DoxyParamCaption}\item[{float}]{x}{}\end{DoxyParamCaption})\hspace{0.3cm}{\ttfamily [inline]}}

\Hypertarget{class_beam_1_1_analog_base_1_1_one_pole_filter_aed129437b0d6d53597daa4a79f29cb87}\index{Beam::AnalogBase::OnePoleFilter@{Beam::AnalogBase::OnePoleFilter}!reset@{reset}}
\index{reset@{reset}!Beam::AnalogBase::OnePoleFilter@{Beam::AnalogBase::OnePoleFilter}}
\doxysubsubsection{\texorpdfstring{reset()}{reset()}}
{\footnotesize\ttfamily \label{class_beam_1_1_analog_base_1_1_one_pole_filter_aed129437b0d6d53597daa4a79f29cb87} 
void Beam::\+\+Analog\+Base::\+\+One\+Pole\+Filter::\+reset (\begin{DoxyParamCaption}{}{}\end{DoxyParamCaption})\hspace{0.3cm}{\ttfamily [inline]}}

\Hypertarget{class_beam_1_1_analog_base_1_1_one_pole_filter_ad5c46d54019ff65ee1ec061abdb41dc5}\index{Beam::AnalogBase::OnePoleFilter@{Beam::AnalogBase::OnePoleFilter}!setCutoff@{setCutoff}}
\index{setCutoff@{setCutoff}!Beam::AnalogBase::OnePoleFilter@{Beam::AnalogBase::OnePoleFilter}}
\doxysubsubsection{\texorpdfstring{setCutoff()}{setCutoff()}}
{\footnotesize\ttfamily \label{class_beam_1_1_analog_base_1_1_one_pole_filter_ad5c46d54019ff65ee1ec061abdb41dc5} 
void Beam::\+\+Analog\+Base::\+\+One\+Pole\+Filter::\+set\+Cutoff (\begin{DoxyParamCaption}\item[{float}]{freq}{, }\item[{float}]{sample\+Rate}{}\end{DoxyParamCaption})\hspace{0.3cm}{\ttfamily [inline]}}



\label{doc-variable-members}
\Hypertarget{class_beam_1_1_analog_base_1_1_one_pole_filter_doc-variable-members}
\doxysubsection{Member Data Documentation}
\Hypertarget{class_beam_1_1_analog_base_1_1_one_pole_filter_a3900a50b36b06885c1597c689ab1cdf7}\index{Beam::AnalogBase::OnePoleFilter@{Beam::AnalogBase::OnePoleFilter}!m\_a0@{m\_a0}}
\index{m\_a0@{m\_a0}!Beam::AnalogBase::OnePoleFilter@{Beam::AnalogBase::OnePoleFilter}}
\doxysubsubsection{\texorpdfstring{m\_a0}{m\_a0}}
{\footnotesize\ttfamily \label{class_beam_1_1_analog_base_1_1_one_pole_filter_a3900a50b36b06885c1597c689ab1cdf7} 
float Beam::\+\+Analog\+Base::\+\+One\+Pole\+Filter::\+m\+\_\+a0 = 1.\+0f\hspace{0.3cm}{\ttfamily [private]}}

\Hypertarget{class_beam_1_1_analog_base_1_1_one_pole_filter_a7dc0974e70a5268b4a35552d27ebbdc1}\index{Beam::AnalogBase::OnePoleFilter@{Beam::AnalogBase::OnePoleFilter}!m\_b1@{m\_b1}}
\index{m\_b1@{m\_b1}!Beam::AnalogBase::OnePoleFilter@{Beam::AnalogBase::OnePoleFilter}}
\doxysubsubsection{\texorpdfstring{m\_b1}{m\_b1}}
{\footnotesize\ttfamily \label{class_beam_1_1_analog_base_1_1_one_pole_filter_a7dc0974e70a5268b4a35552d27ebbdc1} 
float Beam::\+\+Analog\+Base::\+\+One\+Pole\+Filter::\+m\+\_\+b1 = 0.\+0f\hspace{0.3cm}{\ttfamily [private]}}

\Hypertarget{class_beam_1_1_analog_base_1_1_one_pole_filter_aaf6d6623ac4884e68d496b3395ef6059}\index{Beam::AnalogBase::OnePoleFilter@{Beam::AnalogBase::OnePoleFilter}!m\_z1@{m\_z1}}
\index{m\_z1@{m\_z1}!Beam::AnalogBase::OnePoleFilter@{Beam::AnalogBase::OnePoleFilter}}
\doxysubsubsection{\texorpdfstring{m\_z1}{m\_z1}}
{\footnotesize\ttfamily \label{class_beam_1_1_analog_base_1_1_one_pole_filter_aaf6d6623ac4884e68d496b3395ef6059} 
float Beam::\+\+Analog\+Base::\+\+One\+Pole\+Filter::\+m\+\_\+z1 = 0.\+0f\hspace{0.3cm}{\ttfamily [private]}}



The documentation for this class was generated from the following file:\+\begin{DoxyCompactItemize}
\item 
src/\+engine/\+\doxymbox{\hyperlink{analog__base_8hpp}{analog\+\_\+base.\+hpp}}\end{DoxyCompactItemize}

\input{class_beam_1_1_opto2_a}
\doxysection{Beam::\+Parameter Class Reference}
\hypertarget{class_beam_1_1_parameter}{}\label{class_beam_1_1_parameter}\index{Beam::Parameter@{Beam::Parameter}}


{\ttfamily \+\#include $<$parameter.\+hpp$>$}

\doxysubsubsection*{Public Member Functions}
\begin{DoxyCompactItemize}
\item 
\doxymbox{\hyperlink{class_beam_1_1_parameter_a87627221006382b93ff04ebee9ff2f07}{Parameter}} (const std::\+string \&name, float min, float max, float initial\+Value)
\item 
float \doxymbox{\hyperlink{class_beam_1_1_parameter_adda03f7957d39bc1be4af61d762bb29c}{get\+Value}} () const
\item 
void \doxymbox{\hyperlink{class_beam_1_1_parameter_a7a2812018fe946fe8b3655351ce0a596}{set\+Value}} (float new\+Value)
\item 
float \doxymbox{\hyperlink{class_beam_1_1_parameter_ad090e7765271fbdafe4d56e687ab2f42}{get\+Normalized\+Value}} () const
\item 
void \doxymbox{\hyperlink{class_beam_1_1_parameter_a85f9218e178dc71186ead961bc42a8ca}{set\+Normalized\+Value}} (float norm)
\item 
const std::\+string \& \doxymbox{\hyperlink{class_beam_1_1_parameter_a03eededba269a2e9bbf25d51e0d2f9ab}{get\+Name}} () const
\item 
float \doxymbox{\hyperlink{class_beam_1_1_parameter_a0f96da5bd70a75e70163d9caa56c67f2}{get\+Min}} () const
\item 
float \doxymbox{\hyperlink{class_beam_1_1_parameter_adb12778f9ec3ed858bfcf04a8dfe86b7}{get\+Max}} () const
\end{DoxyCompactItemize}
\doxysubsubsection*{Public Attributes}
\begin{DoxyCompactItemize}
\item 
std::\+function$<$ void(float)$>$ \doxymbox{\hyperlink{class_beam_1_1_parameter_a71877343120cac766e22a56c58171148}{on\+Changed}}
\end{DoxyCompactItemize}
\doxysubsubsection*{Private Attributes}
\begin{DoxyCompactItemize}
\item 
std::\+string \doxymbox{\hyperlink{class_beam_1_1_parameter_aedf631d8dea7d9f52477112f1dc500d7}{m\+\_\+name}}
\item 
float \doxymbox{\hyperlink{class_beam_1_1_parameter_a9214b7e40765ab530376cf972d3c554d}{m\+\_\+min}}
\item 
float \doxymbox{\hyperlink{class_beam_1_1_parameter_ad03d943caa393bf44e8862ee7827a983}{m\+\_\+max}}
\item 
std::\+atomic$<$ float $>$ \doxymbox{\hyperlink{class_beam_1_1_parameter_ad54f0df2c3d8e655c2e3d164adc11c2e}{m\+\_\+value}}
\end{DoxyCompactItemize}


\label{doc-constructors}
\Hypertarget{class_beam_1_1_parameter_doc-constructors}
\doxysubsection{Constructor \& Destructor Documentation}
\Hypertarget{class_beam_1_1_parameter_a87627221006382b93ff04ebee9ff2f07}\index{Beam::Parameter@{Beam::Parameter}!Parameter@{Parameter}}
\index{Parameter@{Parameter}!Beam::Parameter@{Beam::Parameter}}
\doxysubsubsection{\texorpdfstring{Parameter()}{Parameter()}}
{\footnotesize\ttfamily \label{class_beam_1_1_parameter_a87627221006382b93ff04ebee9ff2f07} 
Beam::\+\+Parameter::\+\+Parameter (\begin{DoxyParamCaption}\item[{const std::\+string \&}]{name}{, }\item[{float}]{min}{, }\item[{float}]{max}{, }\item[{float}]{initial\+Value}{}\end{DoxyParamCaption})\hspace{0.3cm}{\ttfamily [inline]}}



\label{doc-func-members}
\Hypertarget{class_beam_1_1_parameter_doc-func-members}
\doxysubsection{Member Function Documentation}
\Hypertarget{class_beam_1_1_parameter_adb12778f9ec3ed858bfcf04a8dfe86b7}\index{Beam::Parameter@{Beam::Parameter}!getMax@{getMax}}
\index{getMax@{getMax}!Beam::Parameter@{Beam::Parameter}}
\doxysubsubsection{\texorpdfstring{getMax()}{getMax()}}
{\footnotesize\ttfamily \label{class_beam_1_1_parameter_adb12778f9ec3ed858bfcf04a8dfe86b7} 
float Beam::\+\+Parameter::\+get\+Max (\begin{DoxyParamCaption}{}{}\end{DoxyParamCaption}) const\hspace{0.3cm}{\ttfamily [inline]}}

\Hypertarget{class_beam_1_1_parameter_a0f96da5bd70a75e70163d9caa56c67f2}\index{Beam::Parameter@{Beam::Parameter}!getMin@{getMin}}
\index{getMin@{getMin}!Beam::Parameter@{Beam::Parameter}}
\doxysubsubsection{\texorpdfstring{getMin()}{getMin()}}
{\footnotesize\ttfamily \label{class_beam_1_1_parameter_a0f96da5bd70a75e70163d9caa56c67f2} 
float Beam::\+\+Parameter::\+get\+Min (\begin{DoxyParamCaption}{}{}\end{DoxyParamCaption}) const\hspace{0.3cm}{\ttfamily [inline]}}

\Hypertarget{class_beam_1_1_parameter_a03eededba269a2e9bbf25d51e0d2f9ab}\index{Beam::Parameter@{Beam::Parameter}!getName@{getName}}
\index{getName@{getName}!Beam::Parameter@{Beam::Parameter}}
\doxysubsubsection{\texorpdfstring{getName()}{getName()}}
{\footnotesize\ttfamily \label{class_beam_1_1_parameter_a03eededba269a2e9bbf25d51e0d2f9ab} 
const std::\+string \& Beam::\+\+Parameter::\+get\+Name (\begin{DoxyParamCaption}{}{}\end{DoxyParamCaption}) const\hspace{0.3cm}{\ttfamily [inline]}}

\Hypertarget{class_beam_1_1_parameter_ad090e7765271fbdafe4d56e687ab2f42}\index{Beam::Parameter@{Beam::Parameter}!getNormalizedValue@{getNormalizedValue}}
\index{getNormalizedValue@{getNormalizedValue}!Beam::Parameter@{Beam::Parameter}}
\doxysubsubsection{\texorpdfstring{getNormalizedValue()}{getNormalizedValue()}}
{\footnotesize\ttfamily \label{class_beam_1_1_parameter_ad090e7765271fbdafe4d56e687ab2f42} 
float Beam::\+\+Parameter::\+get\+Normalized\+Value (\begin{DoxyParamCaption}{}{}\end{DoxyParamCaption}) const\hspace{0.3cm}{\ttfamily [inline]}}

\Hypertarget{class_beam_1_1_parameter_adda03f7957d39bc1be4af61d762bb29c}\index{Beam::Parameter@{Beam::Parameter}!getValue@{getValue}}
\index{getValue@{getValue}!Beam::Parameter@{Beam::Parameter}}
\doxysubsubsection{\texorpdfstring{getValue()}{getValue()}}
{\footnotesize\ttfamily \label{class_beam_1_1_parameter_adda03f7957d39bc1be4af61d762bb29c} 
float Beam::\+\+Parameter::\+get\+Value (\begin{DoxyParamCaption}{}{}\end{DoxyParamCaption}) const\hspace{0.3cm}{\ttfamily [inline]}}

\Hypertarget{class_beam_1_1_parameter_a85f9218e178dc71186ead961bc42a8ca}\index{Beam::Parameter@{Beam::Parameter}!setNormalizedValue@{setNormalizedValue}}
\index{setNormalizedValue@{setNormalizedValue}!Beam::Parameter@{Beam::Parameter}}
\doxysubsubsection{\texorpdfstring{setNormalizedValue()}{setNormalizedValue()}}
{\footnotesize\ttfamily \label{class_beam_1_1_parameter_a85f9218e178dc71186ead961bc42a8ca} 
void Beam::\+\+Parameter::\+set\+Normalized\+Value (\begin{DoxyParamCaption}\item[{float}]{norm}{}\end{DoxyParamCaption})\hspace{0.3cm}{\ttfamily [inline]}}

\Hypertarget{class_beam_1_1_parameter_a7a2812018fe946fe8b3655351ce0a596}\index{Beam::Parameter@{Beam::Parameter}!setValue@{setValue}}
\index{setValue@{setValue}!Beam::Parameter@{Beam::Parameter}}
\doxysubsubsection{\texorpdfstring{setValue()}{setValue()}}
{\footnotesize\ttfamily \label{class_beam_1_1_parameter_a7a2812018fe946fe8b3655351ce0a596} 
void Beam::\+\+Parameter::\+set\+Value (\begin{DoxyParamCaption}\item[{float}]{new\+Value}{}\end{DoxyParamCaption})\hspace{0.3cm}{\ttfamily [inline]}}



\label{doc-variable-members}
\Hypertarget{class_beam_1_1_parameter_doc-variable-members}
\doxysubsection{Member Data Documentation}
\Hypertarget{class_beam_1_1_parameter_ad03d943caa393bf44e8862ee7827a983}\index{Beam::Parameter@{Beam::Parameter}!m\_max@{m\_max}}
\index{m\_max@{m\_max}!Beam::Parameter@{Beam::Parameter}}
\doxysubsubsection{\texorpdfstring{m\_max}{m\_max}}
{\footnotesize\ttfamily \label{class_beam_1_1_parameter_ad03d943caa393bf44e8862ee7827a983} 
float Beam::\+\+Parameter::\+m\+\_\+max\hspace{0.3cm}{\ttfamily [private]}}

\Hypertarget{class_beam_1_1_parameter_a9214b7e40765ab530376cf972d3c554d}\index{Beam::Parameter@{Beam::Parameter}!m\_min@{m\_min}}
\index{m\_min@{m\_min}!Beam::Parameter@{Beam::Parameter}}
\doxysubsubsection{\texorpdfstring{m\_min}{m\_min}}
{\footnotesize\ttfamily \label{class_beam_1_1_parameter_a9214b7e40765ab530376cf972d3c554d} 
float Beam::\+\+Parameter::\+m\+\_\+min\hspace{0.3cm}{\ttfamily [private]}}

\Hypertarget{class_beam_1_1_parameter_aedf631d8dea7d9f52477112f1dc500d7}\index{Beam::Parameter@{Beam::Parameter}!m\_name@{m\_name}}
\index{m\_name@{m\_name}!Beam::Parameter@{Beam::Parameter}}
\doxysubsubsection{\texorpdfstring{m\_name}{m\_name}}
{\footnotesize\ttfamily \label{class_beam_1_1_parameter_aedf631d8dea7d9f52477112f1dc500d7} 
std::\+string Beam::\+\+Parameter::\+m\+\_\+name\hspace{0.3cm}{\ttfamily [private]}}

\Hypertarget{class_beam_1_1_parameter_ad54f0df2c3d8e655c2e3d164adc11c2e}\index{Beam::Parameter@{Beam::Parameter}!m\_value@{m\_value}}
\index{m\_value@{m\_value}!Beam::Parameter@{Beam::Parameter}}
\doxysubsubsection{\texorpdfstring{m\_value}{m\_value}}
{\footnotesize\ttfamily \label{class_beam_1_1_parameter_ad54f0df2c3d8e655c2e3d164adc11c2e} 
std::\+atomic$<$float$>$ Beam::\+\+Parameter::\+m\+\_\+value\hspace{0.3cm}{\ttfamily [private]}}

\Hypertarget{class_beam_1_1_parameter_a71877343120cac766e22a56c58171148}\index{Beam::Parameter@{Beam::Parameter}!onChanged@{onChanged}}
\index{onChanged@{onChanged}!Beam::Parameter@{Beam::Parameter}}
\doxysubsubsection{\texorpdfstring{onChanged}{onChanged}}
{\footnotesize\ttfamily \label{class_beam_1_1_parameter_a71877343120cac766e22a56c58171148} 
std::\+function$<$void(float)$>$ Beam::\+\+Parameter::\+on\+Changed}



The documentation for this class was generated from the following file:\+\begin{DoxyCompactItemize}
\item 
src/\+core/\+\doxymbox{\hyperlink{parameter_8hpp}{parameter.\+hpp}}\end{DoxyCompactItemize}

\doxysection{Beam::\+Flux\+Node::\+Port Struct Reference}
\hypertarget{struct_beam_1_1_flux_node_1_1_port}{}\label{struct_beam_1_1_flux_node_1_1_port}\index{Beam::FluxNode::Port@{Beam::FluxNode::Port}}


Describes an audio input or output port.  




{\ttfamily \+\#include $<$flux\+\_\+node.\+hpp$>$}

\doxysubsubsection*{Public Attributes}
\begin{DoxyCompactItemize}
\item 
std::\+string \doxymbox{\hyperlink{struct_beam_1_1_flux_node_1_1_port_a12305258fc537605704c363c82066950}{name}}
\item 
int \doxymbox{\hyperlink{struct_beam_1_1_flux_node_1_1_port_abd4bf86ac483a7ef33991aa1e402dc29}{channels}}
\end{DoxyCompactItemize}


\doxysubsection{Detailed Description}
Describes an audio input or output port. 

\label{doc-variable-members}
\Hypertarget{struct_beam_1_1_flux_node_1_1_port_doc-variable-members}
\doxysubsection{Member Data Documentation}
\Hypertarget{struct_beam_1_1_flux_node_1_1_port_abd4bf86ac483a7ef33991aa1e402dc29}\index{Beam::FluxNode::Port@{Beam::FluxNode::Port}!channels@{channels}}
\index{channels@{channels}!Beam::FluxNode::Port@{Beam::FluxNode::Port}}
\doxysubsubsection{\texorpdfstring{channels}{channels}}
{\footnotesize\ttfamily \label{struct_beam_1_1_flux_node_1_1_port_abd4bf86ac483a7ef33991aa1e402dc29} 
int Beam::\+\+Flux\+Node::\+\+Port::\+channels}

\Hypertarget{struct_beam_1_1_flux_node_1_1_port_a12305258fc537605704c363c82066950}\index{Beam::FluxNode::Port@{Beam::FluxNode::Port}!name@{name}}
\index{name@{name}!Beam::FluxNode::Port@{Beam::FluxNode::Port}}
\doxysubsubsection{\texorpdfstring{name}{name}}
{\footnotesize\ttfamily \label{struct_beam_1_1_flux_node_1_1_port_a12305258fc537605704c363c82066950} 
std::\+string Beam::\+\+Flux\+Node::\+\+Port::\+name}



The documentation for this struct was generated from the following file:\+\begin{DoxyCompactItemize}
\item 
src/\+engine/\+\doxymbox{\hyperlink{flux__node_8hpp}{flux\+\_\+node.\+hpp}}\end{DoxyCompactItemize}

\doxysection{Beam::\+Port Class Reference}
\hypertarget{class_beam_1_1_port}{}\label{class_beam_1_1_port}\index{Beam::Port@{Beam::Port}}


{\ttfamily \+\#include $<$port.\+hpp$>$}

Inheritance diagram for Beam::\+Port:\+\begin{figure}[H]
\begin{center}
\leavevmode
\includegraphics[height=2.000000cm]{class_beam_1_1_port}
\end{center}
\end{figure}
\doxysubsubsection*{Public Member Functions}
\begin{DoxyCompactItemize}
\item 
\doxymbox{\hyperlink{class_beam_1_1_port_a8a1f189383117d4772c37cc8b81f938b}{Port}} (\doxymbox{\hyperlink{namespace_beam_ae4c392a66fa6e2b6c86c6d5356ffe846}{Port\+Type}} type, \doxymbox{\hyperlink{class_beam_1_1_component}{Component}} \texorpdfstring{$\ast$}{*}parent)
\item 
void \doxymbox{\hyperlink{class_beam_1_1_port_ad7cba89612682a527635042473a3d4c0}{render}} (\doxymbox{\hyperlink{class_beam_1_1_quad_batcher}{Quad\+Batcher}} \&batcher) override
\item 
bool \doxymbox{\hyperlink{class_beam_1_1_port_af911fc850b8cf4bbe1491f7845d16cc9}{on\+Mouse\+Down}} (float x, float y, int button) override
\item 
\doxymbox{\hyperlink{namespace_beam_ae4c392a66fa6e2b6c86c6d5356ffe846}{Port\+Type}} \doxymbox{\hyperlink{class_beam_1_1_port_a275b6f61ad0282299769b94e98c70e95}{get\+Type}} () const
\item 
\doxymbox{\hyperlink{class_beam_1_1_component}{Component}} \texorpdfstring{$\ast$}{*} \doxymbox{\hyperlink{class_beam_1_1_port_a62b3dc3446556b1f5a8b725d8eef5377}{get\+Parent\+Module}} () const
\end{DoxyCompactItemize}
\doxysubsection*{Public Member Functions inherited from \doxymbox{\hyperlink{class_beam_1_1_component}{Beam::\+\+Component}}}
\begin{DoxyCompactItemize}
\item 
virtual \doxymbox{\hyperlink{class_beam_1_1_component_af9d734d649978e027412a87bc54362cd}{\texorpdfstring{$\sim$}{\string~}\+Component}} ()=default
\item 
virtual void \doxymbox{\hyperlink{class_beam_1_1_component_ad3d3fb19d25b4371d07620567970a158}{update}} (float dt)
\item 
virtual bool \doxymbox{\hyperlink{class_beam_1_1_component_ae36b8e9d70e8f9a1b9ba81c23c54d5c8}{on\+Mouse\+Up}} (float x, float y, int button)
\item 
virtual bool \doxymbox{\hyperlink{class_beam_1_1_component_a9d8e5970783d315044277a1228659e6c}{on\+Mouse\+Move}} (float x, float y)
\item 
virtual void \doxymbox{\hyperlink{class_beam_1_1_component_a6865b1f22388af467bf6c789120fac05}{set\+Bounds}} (float x, float y, float w, float h)
\item 
const \doxymbox{\hyperlink{struct_beam_1_1_rect}{Rect}} \& \doxymbox{\hyperlink{class_beam_1_1_component_a5746dbc69d5b0adb4cffbcf920936d00}{get\+Bounds}} () const
\item 
void \doxymbox{\hyperlink{class_beam_1_1_component_a00d4e2dfa7703e59d6486852321dbdf1}{set\+Draggable}} (bool draggable)
\item 
void \doxymbox{\hyperlink{class_beam_1_1_component_aca7b02d1dddf7cd20378db9e3242fb84}{start\+Dragging}} (float x, float y)
\end{DoxyCompactItemize}
\doxysubsubsection*{Public Attributes}
\begin{DoxyCompactItemize}
\item 
std::\+function$<$ void(\doxymbox{\hyperlink{class_beam_1_1_port_a8a1f189383117d4772c37cc8b81f938b}{Port}} \texorpdfstring{$\ast$}{*})$>$ \doxymbox{\hyperlink{class_beam_1_1_port_a8430d28d08e047868d1a09520b866c71}{on\+Connect\+Started}}
\end{DoxyCompactItemize}
\doxysubsubsection*{Private Attributes}
\begin{DoxyCompactItemize}
\item 
\doxymbox{\hyperlink{namespace_beam_ae4c392a66fa6e2b6c86c6d5356ffe846}{Port\+Type}} \doxymbox{\hyperlink{class_beam_1_1_port_ae595cb7a3eb36e2f54a1452ffee7aed9}{m\+\_\+type}}
\item 
\doxymbox{\hyperlink{class_beam_1_1_component}{Component}} \texorpdfstring{$\ast$}{*} \doxymbox{\hyperlink{class_beam_1_1_port_ad2754b207312bfe193330194d8998720}{m\+\_\+parent}}
\end{DoxyCompactItemize}
\doxysubsubsection*{Additional Inherited Members}
\doxysubsection*{Protected Attributes inherited from \doxymbox{\hyperlink{class_beam_1_1_component}{Beam::\+\+Component}}}
\begin{DoxyCompactItemize}
\item 
\doxymbox{\hyperlink{struct_beam_1_1_rect}{Rect}} \doxymbox{\hyperlink{class_beam_1_1_component_a4f1ec4a5fb168c39a6c18f958b2b1495}{m\+\_\+bounds}} \{0, 0, 0, 0\}
\item 
bool \doxymbox{\hyperlink{class_beam_1_1_component_adc07913aed6ddadf1c730e7b3bb599cf}{m\+\_\+is\+Visible}} = true
\item 
bool \doxymbox{\hyperlink{class_beam_1_1_component_a0bf77b204ae374a14b5a6d7e5a3c13c6}{m\+\_\+is\+Enabled}} = true
\item 
bool \doxymbox{\hyperlink{class_beam_1_1_component_a9646efcaa9540a26a387f5da9aae4bde}{m\+\_\+is\+Draggable}} = false
\item 
bool \doxymbox{\hyperlink{class_beam_1_1_component_ab03af9a9743acf040f38e3fb11f8dc14}{m\+\_\+is\+Dragging}} = false
\item 
float \doxymbox{\hyperlink{class_beam_1_1_component_a7110b2b9dc235f724bf4689569266a63}{m\+\_\+last\+MouseX}} = 0
\item 
float \doxymbox{\hyperlink{class_beam_1_1_component_a768931a0f51394bf011f821f6ed2efe9}{m\+\_\+last\+MouseY}} = 0
\end{DoxyCompactItemize}


\label{doc-constructors}
\Hypertarget{class_beam_1_1_port_doc-constructors}
\doxysubsection{Constructor \& Destructor Documentation}
\Hypertarget{class_beam_1_1_port_a8a1f189383117d4772c37cc8b81f938b}\index{Beam::Port@{Beam::Port}!Port@{Port}}
\index{Port@{Port}!Beam::Port@{Beam::Port}}
\doxysubsubsection{\texorpdfstring{Port()}{Port()}}
{\footnotesize\ttfamily \label{class_beam_1_1_port_a8a1f189383117d4772c37cc8b81f938b} 
Beam::\+\+Port::\+\+Port (\begin{DoxyParamCaption}\item[{\doxymbox{\hyperlink{namespace_beam_ae4c392a66fa6e2b6c86c6d5356ffe846}{Port\+Type}}}]{type}{, }\item[{\doxymbox{\hyperlink{class_beam_1_1_component}{Component}} \texorpdfstring{$\ast$}{*}}]{parent}{}\end{DoxyParamCaption})\hspace{0.3cm}{\ttfamily [inline]}}



\label{doc-func-members}
\Hypertarget{class_beam_1_1_port_doc-func-members}
\doxysubsection{Member Function Documentation}
\Hypertarget{class_beam_1_1_port_a62b3dc3446556b1f5a8b725d8eef5377}\index{Beam::Port@{Beam::Port}!getParentModule@{getParentModule}}
\index{getParentModule@{getParentModule}!Beam::Port@{Beam::Port}}
\doxysubsubsection{\texorpdfstring{getParentModule()}{getParentModule()}}
{\footnotesize\ttfamily \label{class_beam_1_1_port_a62b3dc3446556b1f5a8b725d8eef5377} 
\doxymbox{\hyperlink{class_beam_1_1_component}{Component}} \texorpdfstring{$\ast$}{*} Beam::\+\+Port::\+get\+Parent\+Module (\begin{DoxyParamCaption}{}{}\end{DoxyParamCaption}) const\hspace{0.3cm}{\ttfamily [inline]}}

\Hypertarget{class_beam_1_1_port_a275b6f61ad0282299769b94e98c70e95}\index{Beam::Port@{Beam::Port}!getType@{getType}}
\index{getType@{getType}!Beam::Port@{Beam::Port}}
\doxysubsubsection{\texorpdfstring{getType()}{getType()}}
{\footnotesize\ttfamily \label{class_beam_1_1_port_a275b6f61ad0282299769b94e98c70e95} 
\doxymbox{\hyperlink{namespace_beam_ae4c392a66fa6e2b6c86c6d5356ffe846}{Port\+Type}} Beam::\+\+Port::\+get\+Type (\begin{DoxyParamCaption}{}{}\end{DoxyParamCaption}) const\hspace{0.3cm}{\ttfamily [inline]}}

\Hypertarget{class_beam_1_1_port_af911fc850b8cf4bbe1491f7845d16cc9}\index{Beam::Port@{Beam::Port}!onMouseDown@{onMouseDown}}
\index{onMouseDown@{onMouseDown}!Beam::Port@{Beam::Port}}
\doxysubsubsection{\texorpdfstring{onMouseDown()}{onMouseDown()}}
{\footnotesize\ttfamily \label{class_beam_1_1_port_af911fc850b8cf4bbe1491f7845d16cc9} 
bool Beam::\+\+Port::\+on\+Mouse\+Down (\begin{DoxyParamCaption}\item[{float}]{x}{, }\item[{float}]{y}{, }\item[{int}]{button}{}\end{DoxyParamCaption})\hspace{0.3cm}{\ttfamily [inline]}, {\ttfamily [override]}, {\ttfamily [virtual]}}



Reimplemented from \doxymbox{\hyperlink{class_beam_1_1_component_aec1da33d2d6e3d4e7dd6708309264e76}{Beam::\+\+Component}}.

\Hypertarget{class_beam_1_1_port_ad7cba89612682a527635042473a3d4c0}\index{Beam::Port@{Beam::Port}!render@{render}}
\index{render@{render}!Beam::Port@{Beam::Port}}
\doxysubsubsection{\texorpdfstring{render()}{render()}}
{\footnotesize\ttfamily \label{class_beam_1_1_port_ad7cba89612682a527635042473a3d4c0} 
void Beam::\+\+Port::\+render (\begin{DoxyParamCaption}\item[{\doxymbox{\hyperlink{class_beam_1_1_quad_batcher}{Quad\+Batcher}} \&}]{batcher}{}\end{DoxyParamCaption})\hspace{0.3cm}{\ttfamily [inline]}, {\ttfamily [override]}, {\ttfamily [virtual]}}



Implements \doxymbox{\hyperlink{class_beam_1_1_component_a2ed6b7841a25bd992bd46b822311ef1d}{Beam::\+\+Component}}.



\label{doc-variable-members}
\Hypertarget{class_beam_1_1_port_doc-variable-members}
\doxysubsection{Member Data Documentation}
\Hypertarget{class_beam_1_1_port_ad2754b207312bfe193330194d8998720}\index{Beam::Port@{Beam::Port}!m\_parent@{m\_parent}}
\index{m\_parent@{m\_parent}!Beam::Port@{Beam::Port}}
\doxysubsubsection{\texorpdfstring{m\_parent}{m\_parent}}
{\footnotesize\ttfamily \label{class_beam_1_1_port_ad2754b207312bfe193330194d8998720} 
\doxymbox{\hyperlink{class_beam_1_1_component}{Component}}\texorpdfstring{$\ast$}{*} Beam::\+\+Port::\+m\+\_\+parent\hspace{0.3cm}{\ttfamily [private]}}

\Hypertarget{class_beam_1_1_port_ae595cb7a3eb36e2f54a1452ffee7aed9}\index{Beam::Port@{Beam::Port}!m\_type@{m\_type}}
\index{m\_type@{m\_type}!Beam::Port@{Beam::Port}}
\doxysubsubsection{\texorpdfstring{m\_type}{m\_type}}
{\footnotesize\ttfamily \label{class_beam_1_1_port_ae595cb7a3eb36e2f54a1452ffee7aed9} 
\doxymbox{\hyperlink{namespace_beam_ae4c392a66fa6e2b6c86c6d5356ffe846}{Port\+Type}} Beam::\+\+Port::\+m\+\_\+type\hspace{0.3cm}{\ttfamily [private]}}

\Hypertarget{class_beam_1_1_port_a8430d28d08e047868d1a09520b866c71}\index{Beam::Port@{Beam::Port}!onConnectStarted@{onConnectStarted}}
\index{onConnectStarted@{onConnectStarted}!Beam::Port@{Beam::Port}}
\doxysubsubsection{\texorpdfstring{onConnectStarted}{onConnectStarted}}
{\footnotesize\ttfamily \label{class_beam_1_1_port_a8430d28d08e047868d1a09520b866c71} 
std::\+function$<$void(\doxymbox{\hyperlink{class_beam_1_1_port_a8a1f189383117d4772c37cc8b81f938b}{Port}}\texorpdfstring{$\ast$}{*})$>$ Beam::\+\+Port::\+on\+Connect\+Started}



The documentation for this class was generated from the following file:\+\begin{DoxyCompactItemize}
\item 
src/\+ui/\+\doxymbox{\hyperlink{port_8hpp}{port.\+hpp}}\end{DoxyCompactItemize}

\doxysection{Beam::\+Project\+Manager Class Reference}
\hypertarget{class_beam_1_1_project_manager}{}\label{class_beam_1_1_project_manager}\index{Beam::ProjectManager@{Beam::ProjectManager}}


{\ttfamily \+\#include $<$project\+\_\+manager.\+hpp$>$}

\doxysubsubsection*{Static Public Member Functions}
\begin{DoxyCompactItemize}
\item 
static void \doxymbox{\hyperlink{class_beam_1_1_project_manager_a4edae2bd477512f4925e60b2e1bdcd37}{save\+Project}} (const std::\+string \&filename, const nlohmann::\+json \&data)
\item 
static nlohmann::\+json \doxymbox{\hyperlink{class_beam_1_1_project_manager_a18a6ab5b5e7f725aaed19ddc652b9206}{load\+Project}} (const std::\+string \&filename)
\end{DoxyCompactItemize}


\label{doc-func-members}
\Hypertarget{class_beam_1_1_project_manager_doc-func-members}
\doxysubsection{Member Function Documentation}
\Hypertarget{class_beam_1_1_project_manager_a18a6ab5b5e7f725aaed19ddc652b9206}\index{Beam::ProjectManager@{Beam::ProjectManager}!loadProject@{loadProject}}
\index{loadProject@{loadProject}!Beam::ProjectManager@{Beam::ProjectManager}}
\doxysubsubsection{\texorpdfstring{loadProject()}{loadProject()}}
{\footnotesize\ttfamily \label{class_beam_1_1_project_manager_a18a6ab5b5e7f725aaed19ddc652b9206} 
nlohmann::\+json Beam::\+\+Project\+Manager::\+load\+Project (\begin{DoxyParamCaption}\item[{const std::\+string \&}]{filename}{}\end{DoxyParamCaption})\hspace{0.3cm}{\ttfamily [inline]}, {\ttfamily [static]}}

\Hypertarget{class_beam_1_1_project_manager_a4edae2bd477512f4925e60b2e1bdcd37}\index{Beam::ProjectManager@{Beam::ProjectManager}!saveProject@{saveProject}}
\index{saveProject@{saveProject}!Beam::ProjectManager@{Beam::ProjectManager}}
\doxysubsubsection{\texorpdfstring{saveProject()}{saveProject()}}
{\footnotesize\ttfamily \label{class_beam_1_1_project_manager_a4edae2bd477512f4925e60b2e1bdcd37} 
void Beam::\+\+Project\+Manager::\+save\+Project (\begin{DoxyParamCaption}\item[{const std::\+string \&}]{filename}{, }\item[{const nlohmann::\+json \&}]{data}{}\end{DoxyParamCaption})\hspace{0.3cm}{\ttfamily [inline]}, {\ttfamily [static]}}



The documentation for this class was generated from the following file:\+\begin{DoxyCompactItemize}
\item 
src/\+session/\+\doxymbox{\hyperlink{project__manager_8hpp}{project\+\_\+manager.\+hpp}}\end{DoxyCompactItemize}

\doxysection{Beam::\+Rect Struct Reference}
\hypertarget{struct_beam_1_1_rect}{}\label{struct_beam_1_1_rect}\index{Beam::Rect@{Beam::Rect}}


{\ttfamily \+\#include $<$component.\+hpp$>$}

\doxysubsubsection*{Public Member Functions}
\begin{DoxyCompactItemize}
\item 
bool \doxymbox{\hyperlink{struct_beam_1_1_rect_a4b5a429163e8d87425b54e12ff6a1a80}{contains}} (float px, float py) const
\end{DoxyCompactItemize}
\doxysubsubsection*{Public Attributes}
\begin{DoxyCompactItemize}
\item 
float \doxymbox{\hyperlink{struct_beam_1_1_rect_a94d1d7f7501ecc84f202b7c4bfd61e08}{x}}
\item 
float \doxymbox{\hyperlink{struct_beam_1_1_rect_a396c73bfc44165cb8af22d271dbdfae6}{y}}
\item 
float \doxymbox{\hyperlink{struct_beam_1_1_rect_a69b99db01aa296ad7964be8fc0365a10}{w}}
\item 
float \doxymbox{\hyperlink{struct_beam_1_1_rect_a17a83604e8b45e52f9fdbcf4289f9851}{h}}
\end{DoxyCompactItemize}


\label{doc-func-members}
\Hypertarget{struct_beam_1_1_rect_doc-func-members}
\doxysubsection{Member Function Documentation}
\Hypertarget{struct_beam_1_1_rect_a4b5a429163e8d87425b54e12ff6a1a80}\index{Beam::Rect@{Beam::Rect}!contains@{contains}}
\index{contains@{contains}!Beam::Rect@{Beam::Rect}}
\doxysubsubsection{\texorpdfstring{contains()}{contains()}}
{\footnotesize\ttfamily \label{struct_beam_1_1_rect_a4b5a429163e8d87425b54e12ff6a1a80} 
bool Beam::\+\+Rect::\+contains (\begin{DoxyParamCaption}\item[{float}]{px}{, }\item[{float}]{py}{}\end{DoxyParamCaption}) const\hspace{0.3cm}{\ttfamily [inline]}}



\label{doc-variable-members}
\Hypertarget{struct_beam_1_1_rect_doc-variable-members}
\doxysubsection{Member Data Documentation}
\Hypertarget{struct_beam_1_1_rect_a17a83604e8b45e52f9fdbcf4289f9851}\index{Beam::Rect@{Beam::Rect}!h@{h}}
\index{h@{h}!Beam::Rect@{Beam::Rect}}
\doxysubsubsection{\texorpdfstring{h}{h}}
{\footnotesize\ttfamily \label{struct_beam_1_1_rect_a17a83604e8b45e52f9fdbcf4289f9851} 
float Beam::\+\+Rect::\+h}

\Hypertarget{struct_beam_1_1_rect_a69b99db01aa296ad7964be8fc0365a10}\index{Beam::Rect@{Beam::Rect}!w@{w}}
\index{w@{w}!Beam::Rect@{Beam::Rect}}
\doxysubsubsection{\texorpdfstring{w}{w}}
{\footnotesize\ttfamily \label{struct_beam_1_1_rect_a69b99db01aa296ad7964be8fc0365a10} 
float Beam::\+\+Rect::\+w}

\Hypertarget{struct_beam_1_1_rect_a94d1d7f7501ecc84f202b7c4bfd61e08}\index{Beam::Rect@{Beam::Rect}!x@{x}}
\index{x@{x}!Beam::Rect@{Beam::Rect}}
\doxysubsubsection{\texorpdfstring{x}{x}}
{\footnotesize\ttfamily \label{struct_beam_1_1_rect_a94d1d7f7501ecc84f202b7c4bfd61e08} 
float Beam::\+\+Rect::\+x}

\Hypertarget{struct_beam_1_1_rect_a396c73bfc44165cb8af22d271dbdfae6}\index{Beam::Rect@{Beam::Rect}!y@{y}}
\index{y@{y}!Beam::Rect@{Beam::Rect}}
\doxysubsubsection{\texorpdfstring{y}{y}}
{\footnotesize\ttfamily \label{struct_beam_1_1_rect_a396c73bfc44165cb8af22d271dbdfae6} 
float Beam::\+\+Rect::\+y}



The documentation for this struct was generated from the following file:\+\begin{DoxyCompactItemize}
\item 
src/\+interface/\+\doxymbox{\hyperlink{component_8hpp}{component.\+hpp}}\end{DoxyCompactItemize}

\doxysection{Beam::\+Region Struct Reference}
\hypertarget{struct_beam_1_1_region}{}\label{struct_beam_1_1_region}\index{Beam::Region@{Beam::Region}}


Represents a clip of audio on the timeline.  




{\ttfamily \+\#include $<$region.\+hpp$>$}

\doxysubsubsection*{Public Attributes}
\begin{DoxyCompactItemize}
\item 
std::\+string \doxymbox{\hyperlink{struct_beam_1_1_region_a4f363f9d4f297363dc9a00932f01b191}{name}}
\item 
size\+\_\+t \doxymbox{\hyperlink{struct_beam_1_1_region_a600e281a408fb3c8942fbd9b3e19bdf0}{start\+Frame}}
\begin{DoxyCompactList}\small\item\em Position on the timeline. \end{DoxyCompactList}\item 
size\+\_\+t \doxymbox{\hyperlink{struct_beam_1_1_region_a15154609c7c8a247b70b9831347f5a17}{duration}}
\begin{DoxyCompactList}\small\item\em Length in frames. \end{DoxyCompactList}\item 
size\+\_\+t \doxymbox{\hyperlink{struct_beam_1_1_region_aa8c0cd8326bdc79150b057c179af2663}{source\+Offset}}
\begin{DoxyCompactList}\small\item\em Offset into the source audio file. \end{DoxyCompactList}\item 
int \doxymbox{\hyperlink{struct_beam_1_1_region_a1e3980ad314e3f5779ce6de4331f5b4c}{track\+Index}}
\begin{DoxyCompactList}\small\item\em Vertical lane index. \end{DoxyCompactList}\item 
std::\+vector$<$ std::\+vector$<$ float $>$ $>$ \doxymbox{\hyperlink{struct_beam_1_1_region_a632fe1c9b80b4f1a5e33b371a269086c}{channel\+Peaks}}
\begin{DoxyCompactList}\small\item\em Cached waveform data per channel. \end{DoxyCompactList}\end{DoxyCompactItemize}


\doxysubsection{Detailed Description}
Represents a clip of audio on the timeline. 

\label{doc-variable-members}
\Hypertarget{struct_beam_1_1_region_doc-variable-members}
\doxysubsection{Member Data Documentation}
\Hypertarget{struct_beam_1_1_region_a632fe1c9b80b4f1a5e33b371a269086c}\index{Beam::Region@{Beam::Region}!channelPeaks@{channelPeaks}}
\index{channelPeaks@{channelPeaks}!Beam::Region@{Beam::Region}}
\doxysubsubsection{\texorpdfstring{channelPeaks}{channelPeaks}}
{\footnotesize\ttfamily \label{struct_beam_1_1_region_a632fe1c9b80b4f1a5e33b371a269086c} 
std::\+vector$<$std::\+vector$<$float$>$ $>$ Beam::\+\+Region::\+channel\+Peaks}



Cached waveform data per channel. 

\Hypertarget{struct_beam_1_1_region_a15154609c7c8a247b70b9831347f5a17}\index{Beam::Region@{Beam::Region}!duration@{duration}}
\index{duration@{duration}!Beam::Region@{Beam::Region}}
\doxysubsubsection{\texorpdfstring{duration}{duration}}
{\footnotesize\ttfamily \label{struct_beam_1_1_region_a15154609c7c8a247b70b9831347f5a17} 
size\+\_\+t Beam::\+\+Region::\+duration}



Length in frames. 

\Hypertarget{struct_beam_1_1_region_a4f363f9d4f297363dc9a00932f01b191}\index{Beam::Region@{Beam::Region}!name@{name}}
\index{name@{name}!Beam::Region@{Beam::Region}}
\doxysubsubsection{\texorpdfstring{name}{name}}
{\footnotesize\ttfamily \label{struct_beam_1_1_region_a4f363f9d4f297363dc9a00932f01b191} 
std::\+string Beam::\+\+Region::\+name}

\Hypertarget{struct_beam_1_1_region_aa8c0cd8326bdc79150b057c179af2663}\index{Beam::Region@{Beam::Region}!sourceOffset@{sourceOffset}}
\index{sourceOffset@{sourceOffset}!Beam::Region@{Beam::Region}}
\doxysubsubsection{\texorpdfstring{sourceOffset}{sourceOffset}}
{\footnotesize\ttfamily \label{struct_beam_1_1_region_aa8c0cd8326bdc79150b057c179af2663} 
size\+\_\+t Beam::\+\+Region::\+source\+Offset}



Offset into the source audio file. 

\Hypertarget{struct_beam_1_1_region_a600e281a408fb3c8942fbd9b3e19bdf0}\index{Beam::Region@{Beam::Region}!startFrame@{startFrame}}
\index{startFrame@{startFrame}!Beam::Region@{Beam::Region}}
\doxysubsubsection{\texorpdfstring{startFrame}{startFrame}}
{\footnotesize\ttfamily \label{struct_beam_1_1_region_a600e281a408fb3c8942fbd9b3e19bdf0} 
size\+\_\+t Beam::\+\+Region::\+start\+Frame}



Position on the timeline. 

\Hypertarget{struct_beam_1_1_region_a1e3980ad314e3f5779ce6de4331f5b4c}\index{Beam::Region@{Beam::Region}!trackIndex@{trackIndex}}
\index{trackIndex@{trackIndex}!Beam::Region@{Beam::Region}}
\doxysubsubsection{\texorpdfstring{trackIndex}{trackIndex}}
{\footnotesize\ttfamily \label{struct_beam_1_1_region_a1e3980ad314e3f5779ce6de4331f5b4c} 
int Beam::\+\+Region::\+track\+Index}



Vertical lane index. 



The documentation for this struct was generated from the following file:\+\begin{DoxyCompactItemize}
\item 
src/\+session/\+\doxymbox{\hyperlink{region_8hpp}{region.\+hpp}}\end{DoxyCompactItemize}

\doxysection{Beam::\+Render\+Plan Struct Reference}
\hypertarget{struct_beam_1_1_render_plan}{}\label{struct_beam_1_1_render_plan}\index{Beam::RenderPlan@{Beam::RenderPlan}}


{\ttfamily \+\#include $<$render\+\_\+plan.\+hpp$>$}

\doxysubsubsection*{Classes}
\begin{DoxyCompactItemize}
\item 
struct \doxymbox{\hyperlink{struct_beam_1_1_render_plan_1_1_buffer_clear_op}{Buffer\+Clear\+Op}}
\end{DoxyCompactItemize}
\doxysubsubsection*{Public Attributes}
\begin{DoxyCompactItemize}
\item 
std::\+vector$<$ \doxymbox{\hyperlink{struct_beam_1_1_node_execution}{Node\+Execution}} $>$ \doxymbox{\hyperlink{struct_beam_1_1_render_plan_ad1b24402e011edf0b5d4fda821574a9e}{sequence}}
\item 
std::\+vector$<$ \doxymbox{\hyperlink{struct_beam_1_1_render_plan_1_1_buffer_clear_op}{Buffer\+Clear\+Op}} $>$ \doxymbox{\hyperlink{struct_beam_1_1_render_plan_aa0ca486792dea6df1389f8baa52ab7b1}{clear\+Ops}}
\end{DoxyCompactItemize}


\label{doc-variable-members}
\Hypertarget{struct_beam_1_1_render_plan_doc-variable-members}
\doxysubsection{Member Data Documentation}
\Hypertarget{struct_beam_1_1_render_plan_aa0ca486792dea6df1389f8baa52ab7b1}\index{Beam::RenderPlan@{Beam::RenderPlan}!clearOps@{clearOps}}
\index{clearOps@{clearOps}!Beam::RenderPlan@{Beam::RenderPlan}}
\doxysubsubsection{\texorpdfstring{clearOps}{clearOps}}
{\footnotesize\ttfamily \label{struct_beam_1_1_render_plan_aa0ca486792dea6df1389f8baa52ab7b1} 
std::\+vector$<$\doxymbox{\hyperlink{struct_beam_1_1_render_plan_1_1_buffer_clear_op}{Buffer\+Clear\+Op}}$>$ Beam::\+\+Render\+Plan::\+clear\+Ops}

\Hypertarget{struct_beam_1_1_render_plan_ad1b24402e011edf0b5d4fda821574a9e}\index{Beam::RenderPlan@{Beam::RenderPlan}!sequence@{sequence}}
\index{sequence@{sequence}!Beam::RenderPlan@{Beam::RenderPlan}}
\doxysubsubsection{\texorpdfstring{sequence}{sequence}}
{\footnotesize\ttfamily \label{struct_beam_1_1_render_plan_ad1b24402e011edf0b5d4fda821574a9e} 
std::\+vector$<$\doxymbox{\hyperlink{struct_beam_1_1_node_execution}{Node\+Execution}}$>$ Beam::\+\+Render\+Plan::\+sequence}



The documentation for this struct was generated from the following file:\+\begin{DoxyCompactItemize}
\item 
src/\+engine/\+\doxymbox{\hyperlink{render__plan_8hpp}{render\+\_\+plan.\+hpp}}\end{DoxyCompactItemize}

\doxysection{Beam::\+Shader Class Reference}
\hypertarget{class_beam_1_1_shader}{}\label{class_beam_1_1_shader}\index{Beam::Shader@{Beam::Shader}}


{\ttfamily \+\#include $<$shader.\+hpp$>$}

\doxysubsubsection*{Public Member Functions}
\begin{DoxyCompactItemize}
\item 
\doxymbox{\hyperlink{class_beam_1_1_shader_acbabc393a8a2ae8eafaa62bcbc354af3}{Shader}} (const char \texorpdfstring{$\ast$}{*}vertex\+Source, const char \texorpdfstring{$\ast$}{*}fragment\+Source)
\item 
\doxymbox{\hyperlink{class_beam_1_1_shader_aeacb90a56be515de05b7733e7a123735}{\texorpdfstring{$\sim$}{\string~}\+Shader}} ()
\item 
void \doxymbox{\hyperlink{class_beam_1_1_shader_ace7ec7da00788442736ffc592f3acb49}{use}} () const
\item 
void \doxymbox{\hyperlink{class_beam_1_1_shader_ac32398d37191130501c1c2c4892bc98c}{set\+Int}} (const std::\+string \&name, int value) const
\item 
void \doxymbox{\hyperlink{class_beam_1_1_shader_a9f35b32e3dd725dc6d2048586fe6f78a}{set\+Float}} (const std::\+string \&name, float value) const
\item 
void \doxymbox{\hyperlink{class_beam_1_1_shader_aebd0a7cd99c1aadff3e1c4b43cec8c15}{set\+Mat4}} (const std::\+string \&name, const float \texorpdfstring{$\ast$}{*}matrix) const
\end{DoxyCompactItemize}
\doxysubsubsection*{Private Member Functions}
\begin{DoxyCompactItemize}
\item 
void \doxymbox{\hyperlink{class_beam_1_1_shader_a79baeb63a856ed225042a9cced169b96}{check\+Compile\+Errors}} (unsigned int shader, std::\+string type)
\end{DoxyCompactItemize}
\doxysubsubsection*{Private Attributes}
\begin{DoxyCompactItemize}
\item 
unsigned int \doxymbox{\hyperlink{class_beam_1_1_shader_a98fd87c26ac672d65aa674b9bfbc9c5f}{m\+\_\+id}}
\end{DoxyCompactItemize}


\label{doc-constructors}
\Hypertarget{class_beam_1_1_shader_doc-constructors}
\doxysubsection{Constructor \& Destructor Documentation}
\Hypertarget{class_beam_1_1_shader_acbabc393a8a2ae8eafaa62bcbc354af3}\index{Beam::Shader@{Beam::Shader}!Shader@{Shader}}
\index{Shader@{Shader}!Beam::Shader@{Beam::Shader}}
\doxysubsubsection{\texorpdfstring{Shader()}{Shader()}}
{\footnotesize\ttfamily \label{class_beam_1_1_shader_acbabc393a8a2ae8eafaa62bcbc354af3} 
Beam::\+\+Shader::\+\+Shader (\begin{DoxyParamCaption}\item[{const char \texorpdfstring{$\ast$}{*}}]{vertex\+Source}{, }\item[{const char \texorpdfstring{$\ast$}{*}}]{fragment\+Source}{}\end{DoxyParamCaption})}

\Hypertarget{class_beam_1_1_shader_aeacb90a56be515de05b7733e7a123735}\index{Beam::Shader@{Beam::Shader}!````~Shader@{\texorpdfstring{$\sim$}{\string~}Shader}}
\index{````~Shader@{\texorpdfstring{$\sim$}{\string~}Shader}!Beam::Shader@{Beam::Shader}}
\doxysubsubsection{\texorpdfstring{\texorpdfstring{$\sim$}{\string~}Shader()}{\string~Shader()}}
{\footnotesize\ttfamily \label{class_beam_1_1_shader_aeacb90a56be515de05b7733e7a123735} 
Beam::\+\+Shader::\+\texorpdfstring{$\sim$}{\string~}\+Shader (\begin{DoxyParamCaption}{}{}\end{DoxyParamCaption})}



\label{doc-func-members}
\Hypertarget{class_beam_1_1_shader_doc-func-members}
\doxysubsection{Member Function Documentation}
\Hypertarget{class_beam_1_1_shader_a79baeb63a856ed225042a9cced169b96}\index{Beam::Shader@{Beam::Shader}!checkCompileErrors@{checkCompileErrors}}
\index{checkCompileErrors@{checkCompileErrors}!Beam::Shader@{Beam::Shader}}
\doxysubsubsection{\texorpdfstring{checkCompileErrors()}{checkCompileErrors()}}
{\footnotesize\ttfamily \label{class_beam_1_1_shader_a79baeb63a856ed225042a9cced169b96} 
void Beam::\+\+Shader::\+check\+Compile\+Errors (\begin{DoxyParamCaption}\item[{unsigned int}]{shader}{, }\item[{std::\+string}]{type}{}\end{DoxyParamCaption})\hspace{0.3cm}{\ttfamily [private]}}

\Hypertarget{class_beam_1_1_shader_a9f35b32e3dd725dc6d2048586fe6f78a}\index{Beam::Shader@{Beam::Shader}!setFloat@{setFloat}}
\index{setFloat@{setFloat}!Beam::Shader@{Beam::Shader}}
\doxysubsubsection{\texorpdfstring{setFloat()}{setFloat()}}
{\footnotesize\ttfamily \label{class_beam_1_1_shader_a9f35b32e3dd725dc6d2048586fe6f78a} 
void Beam::\+\+Shader::\+set\+Float (\begin{DoxyParamCaption}\item[{const std::\+string \&}]{name}{, }\item[{float}]{value}{}\end{DoxyParamCaption}) const}

\Hypertarget{class_beam_1_1_shader_ac32398d37191130501c1c2c4892bc98c}\index{Beam::Shader@{Beam::Shader}!setInt@{setInt}}
\index{setInt@{setInt}!Beam::Shader@{Beam::Shader}}
\doxysubsubsection{\texorpdfstring{setInt()}{setInt()}}
{\footnotesize\ttfamily \label{class_beam_1_1_shader_ac32398d37191130501c1c2c4892bc98c} 
void Beam::\+\+Shader::\+set\+Int (\begin{DoxyParamCaption}\item[{const std::\+string \&}]{name}{, }\item[{int}]{value}{}\end{DoxyParamCaption}) const}

\Hypertarget{class_beam_1_1_shader_aebd0a7cd99c1aadff3e1c4b43cec8c15}\index{Beam::Shader@{Beam::Shader}!setMat4@{setMat4}}
\index{setMat4@{setMat4}!Beam::Shader@{Beam::Shader}}
\doxysubsubsection{\texorpdfstring{setMat4()}{setMat4()}}
{\footnotesize\ttfamily \label{class_beam_1_1_shader_aebd0a7cd99c1aadff3e1c4b43cec8c15} 
void Beam::\+\+Shader::\+set\+Mat4 (\begin{DoxyParamCaption}\item[{const std::\+string \&}]{name}{, }\item[{const float \texorpdfstring{$\ast$}{*}}]{matrix}{}\end{DoxyParamCaption}) const}

\Hypertarget{class_beam_1_1_shader_ace7ec7da00788442736ffc592f3acb49}\index{Beam::Shader@{Beam::Shader}!use@{use}}
\index{use@{use}!Beam::Shader@{Beam::Shader}}
\doxysubsubsection{\texorpdfstring{use()}{use()}}
{\footnotesize\ttfamily \label{class_beam_1_1_shader_ace7ec7da00788442736ffc592f3acb49} 
void Beam::\+\+Shader::\+use (\begin{DoxyParamCaption}{}{}\end{DoxyParamCaption}) const}



\label{doc-variable-members}
\Hypertarget{class_beam_1_1_shader_doc-variable-members}
\doxysubsection{Member Data Documentation}
\Hypertarget{class_beam_1_1_shader_a98fd87c26ac672d65aa674b9bfbc9c5f}\index{Beam::Shader@{Beam::Shader}!m\_id@{m\_id}}
\index{m\_id@{m\_id}!Beam::Shader@{Beam::Shader}}
\doxysubsubsection{\texorpdfstring{m\_id}{m\_id}}
{\footnotesize\ttfamily \label{class_beam_1_1_shader_a98fd87c26ac672d65aa674b9bfbc9c5f} 
unsigned int Beam::\+\+Shader::\+m\+\_\+id\hspace{0.3cm}{\ttfamily [private]}}



The documentation for this class was generated from the following files:\+\begin{DoxyCompactItemize}
\item 
src/\+graphics/\+\doxymbox{\hyperlink{shader_8hpp}{shader.\+hpp}}\item 
src/\+graphics/\+\doxymbox{\hyperlink{shader_8cpp}{shader.\+cpp}}\end{DoxyCompactItemize}

\doxysection{Beam::\+Sidebar Class Reference}
\hypertarget{class_beam_1_1_sidebar}{}\label{class_beam_1_1_sidebar}\index{Beam::Sidebar@{Beam::Sidebar}}


{\ttfamily \+\#include $<$sidebar.\+hpp$>$}

Inheritance diagram for Beam::\+Sidebar:\+\begin{figure}[H]
\begin{center}
\leavevmode
\includegraphics[height=2.000000cm]{class_beam_1_1_sidebar}
\end{center}
\end{figure}
\doxysubsubsection*{Public Types}
\begin{DoxyCompactItemize}
\item 
enum class \doxymbox{\hyperlink{class_beam_1_1_sidebar_a156ccc2325cb05049f4ed58d21667c75}{Side}} \{ \doxymbox{\hyperlink{class_beam_1_1_sidebar_a156ccc2325cb05049f4ed58d21667c75a945d5e233cf7d6240f6b783b36a374ff}{Left}}
, \doxymbox{\hyperlink{class_beam_1_1_sidebar_a156ccc2325cb05049f4ed58d21667c75a92b09c7c48c520c3c55e497875da437c}{Right}}
 \}
\end{DoxyCompactItemize}
\doxysubsubsection*{Public Member Functions}
\begin{DoxyCompactItemize}
\item 
\doxymbox{\hyperlink{class_beam_1_1_sidebar_a89f3319c51bcdf4c52b0fd31f5844c7d}{Sidebar}} (\doxymbox{\hyperlink{class_beam_1_1_sidebar_a156ccc2325cb05049f4ed58d21667c75}{Side}} side)
\item 
void \doxymbox{\hyperlink{class_beam_1_1_sidebar_ab53552812985613ba4cfd66428b49c04}{render}} (\doxymbox{\hyperlink{class_beam_1_1_quad_batcher}{Quad\+Batcher}} \&batcher) override
\item 
bool \doxymbox{\hyperlink{class_beam_1_1_sidebar_a7fa418614ff6c3f7ca87552af765026c}{on\+Mouse\+Down}} (float x, float y, int button) override
\end{DoxyCompactItemize}
\doxysubsection*{Public Member Functions inherited from \doxymbox{\hyperlink{class_beam_1_1_component}{Beam::\+\+Component}}}
\begin{DoxyCompactItemize}
\item 
virtual \doxymbox{\hyperlink{class_beam_1_1_component_af9d734d649978e027412a87bc54362cd}{\texorpdfstring{$\sim$}{\string~}\+Component}} ()=default
\item 
virtual void \doxymbox{\hyperlink{class_beam_1_1_component_ad3d3fb19d25b4371d07620567970a158}{update}} (float dt)
\item 
virtual bool \doxymbox{\hyperlink{class_beam_1_1_component_ae36b8e9d70e8f9a1b9ba81c23c54d5c8}{on\+Mouse\+Up}} (float x, float y, int button)
\item 
virtual bool \doxymbox{\hyperlink{class_beam_1_1_component_a9d8e5970783d315044277a1228659e6c}{on\+Mouse\+Move}} (float x, float y)
\item 
virtual void \doxymbox{\hyperlink{class_beam_1_1_component_a6865b1f22388af467bf6c789120fac05}{set\+Bounds}} (float x, float y, float w, float h)
\item 
const \doxymbox{\hyperlink{struct_beam_1_1_rect}{Rect}} \& \doxymbox{\hyperlink{class_beam_1_1_component_a5746dbc69d5b0adb4cffbcf920936d00}{get\+Bounds}} () const
\item 
void \doxymbox{\hyperlink{class_beam_1_1_component_a00d4e2dfa7703e59d6486852321dbdf1}{set\+Draggable}} (bool draggable)
\item 
void \doxymbox{\hyperlink{class_beam_1_1_component_aca7b02d1dddf7cd20378db9e3242fb84}{start\+Dragging}} (float x, float y)
\end{DoxyCompactItemize}
\doxysubsubsection*{Public Attributes}
\begin{DoxyCompactItemize}
\item 
std::\+function$<$ void(std::\+string)$>$ \doxymbox{\hyperlink{class_beam_1_1_sidebar_a90fb5915e2a42d282734c8ca1b0da207}{on\+Add\+FX}}
\end{DoxyCompactItemize}
\doxysubsubsection*{Private Attributes}
\begin{DoxyCompactItemize}
\item 
\doxymbox{\hyperlink{class_beam_1_1_sidebar_a156ccc2325cb05049f4ed58d21667c75}{Side}} \doxymbox{\hyperlink{class_beam_1_1_sidebar_a0a81e8a2eb43da84567a1aa23efa34c1}{m\+\_\+side}}
\end{DoxyCompactItemize}
\doxysubsubsection*{Additional Inherited Members}
\doxysubsection*{Protected Attributes inherited from \doxymbox{\hyperlink{class_beam_1_1_component}{Beam::\+\+Component}}}
\begin{DoxyCompactItemize}
\item 
\doxymbox{\hyperlink{struct_beam_1_1_rect}{Rect}} \doxymbox{\hyperlink{class_beam_1_1_component_a4f1ec4a5fb168c39a6c18f958b2b1495}{m\+\_\+bounds}} \{0, 0, 0, 0\}
\item 
bool \doxymbox{\hyperlink{class_beam_1_1_component_adc07913aed6ddadf1c730e7b3bb599cf}{m\+\_\+is\+Visible}} = true
\item 
bool \doxymbox{\hyperlink{class_beam_1_1_component_a0bf77b204ae374a14b5a6d7e5a3c13c6}{m\+\_\+is\+Enabled}} = true
\item 
bool \doxymbox{\hyperlink{class_beam_1_1_component_a9646efcaa9540a26a387f5da9aae4bde}{m\+\_\+is\+Draggable}} = false
\item 
bool \doxymbox{\hyperlink{class_beam_1_1_component_ab03af9a9743acf040f38e3fb11f8dc14}{m\+\_\+is\+Dragging}} = false
\item 
float \doxymbox{\hyperlink{class_beam_1_1_component_a7110b2b9dc235f724bf4689569266a63}{m\+\_\+last\+MouseX}} = 0
\item 
float \doxymbox{\hyperlink{class_beam_1_1_component_a768931a0f51394bf011f821f6ed2efe9}{m\+\_\+last\+MouseY}} = 0
\end{DoxyCompactItemize}


\label{doc-enum-members}
\Hypertarget{class_beam_1_1_sidebar_doc-enum-members}
\doxysubsection{Member Enumeration Documentation}
\Hypertarget{class_beam_1_1_sidebar_a156ccc2325cb05049f4ed58d21667c75}\index{Beam::Sidebar@{Beam::Sidebar}!Side@{Side}}
\index{Side@{Side}!Beam::Sidebar@{Beam::Sidebar}}
\doxysubsubsection{\texorpdfstring{Side}{Side}}
{\footnotesize\ttfamily \label{class_beam_1_1_sidebar_a156ccc2325cb05049f4ed58d21667c75} 
enum class \doxymbox{\hyperlink{class_beam_1_1_sidebar_a156ccc2325cb05049f4ed58d21667c75}{Beam::\+\+Sidebar::\+\+Side}}\hspace{0.3cm}{\ttfamily [strong]}}

\begin{DoxyEnumFields}[2]{Enumerator}
\raisebox{\heightof{T}}[0pt][0pt]{\index{Left@{Left}!Beam::Sidebar@{Beam::Sidebar}}
\index{Beam::Sidebar@{Beam::Sidebar}!Left@{Left}}
}\Hypertarget{class_beam_1_1_sidebar_a156ccc2325cb05049f4ed58d21667c75a945d5e233cf7d6240f6b783b36a374ff}\label{class_beam_1_1_sidebar_a156ccc2325cb05049f4ed58d21667c75a945d5e233cf7d6240f6b783b36a374ff} 
Left&\\
\hline

\raisebox{\heightof{T}}[0pt][0pt]{\index{Right@{Right}!Beam::Sidebar@{Beam::Sidebar}}
\index{Beam::Sidebar@{Beam::Sidebar}!Right@{Right}}
}\Hypertarget{class_beam_1_1_sidebar_a156ccc2325cb05049f4ed58d21667c75a92b09c7c48c520c3c55e497875da437c}\label{class_beam_1_1_sidebar_a156ccc2325cb05049f4ed58d21667c75a92b09c7c48c520c3c55e497875da437c} 
Right&\\
\hline

\end{DoxyEnumFields}


\label{doc-constructors}
\Hypertarget{class_beam_1_1_sidebar_doc-constructors}
\doxysubsection{Constructor \& Destructor Documentation}
\Hypertarget{class_beam_1_1_sidebar_a89f3319c51bcdf4c52b0fd31f5844c7d}\index{Beam::Sidebar@{Beam::Sidebar}!Sidebar@{Sidebar}}
\index{Sidebar@{Sidebar}!Beam::Sidebar@{Beam::Sidebar}}
\doxysubsubsection{\texorpdfstring{Sidebar()}{Sidebar()}}
{\footnotesize\ttfamily \label{class_beam_1_1_sidebar_a89f3319c51bcdf4c52b0fd31f5844c7d} 
Beam::\+\+Sidebar::\+\+Sidebar (\begin{DoxyParamCaption}\item[{\doxymbox{\hyperlink{class_beam_1_1_sidebar_a156ccc2325cb05049f4ed58d21667c75}{Side}}}]{side}{}\end{DoxyParamCaption})\hspace{0.3cm}{\ttfamily [inline]}}



\label{doc-func-members}
\Hypertarget{class_beam_1_1_sidebar_doc-func-members}
\doxysubsection{Member Function Documentation}
\Hypertarget{class_beam_1_1_sidebar_a7fa418614ff6c3f7ca87552af765026c}\index{Beam::Sidebar@{Beam::Sidebar}!onMouseDown@{onMouseDown}}
\index{onMouseDown@{onMouseDown}!Beam::Sidebar@{Beam::Sidebar}}
\doxysubsubsection{\texorpdfstring{onMouseDown()}{onMouseDown()}}
{\footnotesize\ttfamily \label{class_beam_1_1_sidebar_a7fa418614ff6c3f7ca87552af765026c} 
bool Beam::\+\+Sidebar::\+on\+Mouse\+Down (\begin{DoxyParamCaption}\item[{float}]{x}{, }\item[{float}]{y}{, }\item[{int}]{button}{}\end{DoxyParamCaption})\hspace{0.3cm}{\ttfamily [inline]}, {\ttfamily [override]}, {\ttfamily [virtual]}}



Reimplemented from \doxymbox{\hyperlink{class_beam_1_1_component_aec1da33d2d6e3d4e7dd6708309264e76}{Beam::\+\+Component}}.

\Hypertarget{class_beam_1_1_sidebar_ab53552812985613ba4cfd66428b49c04}\index{Beam::Sidebar@{Beam::Sidebar}!render@{render}}
\index{render@{render}!Beam::Sidebar@{Beam::Sidebar}}
\doxysubsubsection{\texorpdfstring{render()}{render()}}
{\footnotesize\ttfamily \label{class_beam_1_1_sidebar_ab53552812985613ba4cfd66428b49c04} 
void Beam::\+\+Sidebar::\+render (\begin{DoxyParamCaption}\item[{\doxymbox{\hyperlink{class_beam_1_1_quad_batcher}{Quad\+Batcher}} \&}]{batcher}{}\end{DoxyParamCaption})\hspace{0.3cm}{\ttfamily [inline]}, {\ttfamily [override]}, {\ttfamily [virtual]}}



Implements \doxymbox{\hyperlink{class_beam_1_1_component_a2ed6b7841a25bd992bd46b822311ef1d}{Beam::\+\+Component}}.



\label{doc-variable-members}
\Hypertarget{class_beam_1_1_sidebar_doc-variable-members}
\doxysubsection{Member Data Documentation}
\Hypertarget{class_beam_1_1_sidebar_a0a81e8a2eb43da84567a1aa23efa34c1}\index{Beam::Sidebar@{Beam::Sidebar}!m\_side@{m\_side}}
\index{m\_side@{m\_side}!Beam::Sidebar@{Beam::Sidebar}}
\doxysubsubsection{\texorpdfstring{m\_side}{m\_side}}
{\footnotesize\ttfamily \label{class_beam_1_1_sidebar_a0a81e8a2eb43da84567a1aa23efa34c1} 
\doxymbox{\hyperlink{class_beam_1_1_sidebar_a156ccc2325cb05049f4ed58d21667c75}{Side}} Beam::\+\+Sidebar::\+m\+\_\+side\hspace{0.3cm}{\ttfamily [private]}}

\Hypertarget{class_beam_1_1_sidebar_a90fb5915e2a42d282734c8ca1b0da207}\index{Beam::Sidebar@{Beam::Sidebar}!onAddFX@{onAddFX}}
\index{onAddFX@{onAddFX}!Beam::Sidebar@{Beam::Sidebar}}
\doxysubsubsection{\texorpdfstring{onAddFX}{onAddFX}}
{\footnotesize\ttfamily \label{class_beam_1_1_sidebar_a90fb5915e2a42d282734c8ca1b0da207} 
std::\+function$<$void(std::\+string)$>$ Beam::\+\+Sidebar::\+on\+Add\+FX}



The documentation for this class was generated from the following file:\+\begin{DoxyCompactItemize}
\item 
src/\+ui/\+\doxymbox{\hyperlink{sidebar_8hpp}{sidebar.\+hpp}}\end{DoxyCompactItemize}

\doxysection{Beam::\+Signal\+Route Struct Reference}
\hypertarget{struct_beam_1_1_signal_route}{}\label{struct_beam_1_1_signal_route}\index{Beam::SignalRoute@{Beam::SignalRoute}}


{\ttfamily \+\#include $<$render\+\_\+plan.\+hpp$>$}

\doxysubsubsection*{Public Attributes}
\begin{DoxyCompactItemize}
\item 
std::\+shared\+\_\+ptr$<$ \doxymbox{\hyperlink{class_beam_1_1_flux_node}{Flux\+Node}} $>$ \doxymbox{\hyperlink{struct_beam_1_1_signal_route_a83a38e58d6c0dd0620ddea7689a6cff0}{source\+Node}}
\item 
int \doxymbox{\hyperlink{struct_beam_1_1_signal_route_a2e59c2c878cf9d002800cdb63ff7f7cd}{source\+Port}}
\item 
std::\+shared\+\_\+ptr$<$ \doxymbox{\hyperlink{class_beam_1_1_flux_node}{Flux\+Node}} $>$ \doxymbox{\hyperlink{struct_beam_1_1_signal_route_adac6d195dd071aeca4142e3f9258b6c7}{dest\+Node}}
\item 
int \doxymbox{\hyperlink{struct_beam_1_1_signal_route_af114ab14e34fa10a72e4e773f879fdeb}{dest\+Port}}
\end{DoxyCompactItemize}


\label{doc-variable-members}
\Hypertarget{struct_beam_1_1_signal_route_doc-variable-members}
\doxysubsection{Member Data Documentation}
\Hypertarget{struct_beam_1_1_signal_route_adac6d195dd071aeca4142e3f9258b6c7}\index{Beam::SignalRoute@{Beam::SignalRoute}!destNode@{destNode}}
\index{destNode@{destNode}!Beam::SignalRoute@{Beam::SignalRoute}}
\doxysubsubsection{\texorpdfstring{destNode}{destNode}}
{\footnotesize\ttfamily \label{struct_beam_1_1_signal_route_adac6d195dd071aeca4142e3f9258b6c7} 
std::\+shared\+\_\+ptr$<$\doxymbox{\hyperlink{class_beam_1_1_flux_node}{Flux\+Node}}$>$ Beam::\+\+Signal\+Route::\+dest\+Node}

\Hypertarget{struct_beam_1_1_signal_route_af114ab14e34fa10a72e4e773f879fdeb}\index{Beam::SignalRoute@{Beam::SignalRoute}!destPort@{destPort}}
\index{destPort@{destPort}!Beam::SignalRoute@{Beam::SignalRoute}}
\doxysubsubsection{\texorpdfstring{destPort}{destPort}}
{\footnotesize\ttfamily \label{struct_beam_1_1_signal_route_af114ab14e34fa10a72e4e773f879fdeb} 
int Beam::\+\+Signal\+Route::\+dest\+Port}

\Hypertarget{struct_beam_1_1_signal_route_a83a38e58d6c0dd0620ddea7689a6cff0}\index{Beam::SignalRoute@{Beam::SignalRoute}!sourceNode@{sourceNode}}
\index{sourceNode@{sourceNode}!Beam::SignalRoute@{Beam::SignalRoute}}
\doxysubsubsection{\texorpdfstring{sourceNode}{sourceNode}}
{\footnotesize\ttfamily \label{struct_beam_1_1_signal_route_a83a38e58d6c0dd0620ddea7689a6cff0} 
std::\+shared\+\_\+ptr$<$\doxymbox{\hyperlink{class_beam_1_1_flux_node}{Flux\+Node}}$>$ Beam::\+\+Signal\+Route::\+source\+Node}

\Hypertarget{struct_beam_1_1_signal_route_a2e59c2c878cf9d002800cdb63ff7f7cd}\index{Beam::SignalRoute@{Beam::SignalRoute}!sourcePort@{sourcePort}}
\index{sourcePort@{sourcePort}!Beam::SignalRoute@{Beam::SignalRoute}}
\doxysubsubsection{\texorpdfstring{sourcePort}{sourcePort}}
{\footnotesize\ttfamily \label{struct_beam_1_1_signal_route_a2e59c2c878cf9d002800cdb63ff7f7cd} 
int Beam::\+\+Signal\+Route::\+source\+Port}



The documentation for this struct was generated from the following file:\+\begin{DoxyCompactItemize}
\item 
src/\+engine/\+\doxymbox{\hyperlink{render__plan_8hpp}{render\+\_\+plan.\+hpp}}\end{DoxyCompactItemize}

\doxysection{Beam::\+Simple\+Gain\+Processor Class Reference}
\hypertarget{class_beam_1_1_simple_gain_processor}{}\label{class_beam_1_1_simple_gain_processor}\index{Beam::SimpleGainProcessor@{Beam::SimpleGainProcessor}}


A simple gain processor demonstrating the new API.  




{\ttfamily \+\#include $<$simple\+\_\+gain\+\_\+processor.\+hpp$>$}

Inheritance diagram for Beam::\+Simple\+Gain\+Processor:\+\begin{figure}[H]
\begin{center}
\leavevmode
\includegraphics[height=2.000000cm]{class_beam_1_1_simple_gain_processor}
\end{center}
\end{figure}
\doxysubsubsection*{Public Member Functions}
\begin{DoxyCompactItemize}
\item 
\doxymbox{\hyperlink{class_beam_1_1_simple_gain_processor_add45d2b5c7ff6109e8dd0d7b1b282b90}{Simple\+Gain\+Processor}} ()
\item 
\doxymbox{\hyperlink{class_beam_1_1_simple_gain_processor_a0aedd2a7595e4d97e1c090449266e7e2}{\texorpdfstring{$\sim$}{\string~}\+Simple\+Gain\+Processor}} () override
\item 
void \doxymbox{\hyperlink{class_beam_1_1_simple_gain_processor_a29edd229c91eb66476785fe38c983718}{process}} (int frames) override
\begin{DoxyCompactList}\small\item\em Main audio processing method. Must be implemented by subclasses. \end{DoxyCompactList}\item 
std::\+string \doxymbox{\hyperlink{class_beam_1_1_simple_gain_processor_a73bd1595872e04e5abac05d7d17ca336}{get\+Name}} () const override
\item 
std::\+vector$<$ \doxymbox{\hyperlink{struct_beam_1_1_flux_node_1_1_port}{Port}} $>$ \doxymbox{\hyperlink{class_beam_1_1_simple_gain_processor_a26ab1fae0c646a95dd158babd134af91}{get\+Input\+Ports}} () const override
\item 
std::\+vector$<$ \doxymbox{\hyperlink{struct_beam_1_1_flux_node_1_1_port}{Port}} $>$ \doxymbox{\hyperlink{class_beam_1_1_simple_gain_processor_a225eb5f513d1dd5138d020155c0978ed}{get\+Output\+Ports}} () const override
\item 
float \doxymbox{\hyperlink{class_beam_1_1_simple_gain_processor_a17d80f27c22f909f859e5441d25c0a4d}{get\+Gain}} () const
\end{DoxyCompactItemize}
\doxysubsection*{Public Member Functions inherited from \doxymbox{\hyperlink{class_beam_1_1_flux_node}{Beam::\+\+Flux\+Node}}}
\begin{DoxyCompactItemize}
\item 
virtual \doxymbox{\hyperlink{class_beam_1_1_flux_node_a708c135cdb61e8838469998cd8a84e65}{\texorpdfstring{$\sim$}{\string~}\+Flux\+Node}} ()=default
\item 
virtual void \doxymbox{\hyperlink{class_beam_1_1_flux_node_ae9d1e151eff5166de969f45de06d5596}{process\+MIDI}} (const \doxymbox{\hyperlink{class_beam_1_1_m_i_d_i_buffer}{MIDIBuffer}} \&midi)
\begin{DoxyCompactList}\small\item\em Optional MIDI processing. Called before \doxylink{class_beam_1_1_flux_node_a3c263446753fa7ae5ff6928ee57bcd4d}{process()} in the engine loop. \end{DoxyCompactList}\item 
virtual void \doxymbox{\hyperlink{class_beam_1_1_flux_node_ace8cc49479d8924d44bca5fd4cd955e2}{on\+Transport\+State\+Changed}} (bool playing)
\begin{DoxyCompactList}\small\item\em Responds to global transport changes (Play/\+\+Pause). \end{DoxyCompactList}\item 
virtual void \doxymbox{\hyperlink{class_beam_1_1_flux_node_adc7c4e979bf27de5bfca66815ae97a67}{on\+Transport\+Seek}} (size\+\_\+t frame)
\begin{DoxyCompactList}\small\item\em Responds to timeline seeking. \end{DoxyCompactList}\item 
void \doxymbox{\hyperlink{class_beam_1_1_flux_node_aa579ec06608fd776987bbb089f27fd94}{set\+Current\+Frame}} (size\+\_\+t frame)
\begin{DoxyCompactList}\small\item\em Sets the current playhead position for this block. \end{DoxyCompactList}\item 
float \texorpdfstring{$\ast$}{*} \doxymbox{\hyperlink{class_beam_1_1_flux_node_ac90bd1a05b5bed3d68978f532386ed29}{get\+Input\+Buffer}} (int port\+Idx)
\item 
float \texorpdfstring{$\ast$}{*} \doxymbox{\hyperlink{class_beam_1_1_flux_node_abf11cfd4f2346ee0cd46d4345f1ed7d4}{get\+Output\+Buffer}} (int port\+Idx)
\item 
void \doxymbox{\hyperlink{class_beam_1_1_flux_node_af37f8c1b6b825da2ce7e35011d6f8253}{set\+Bypass}} (bool bypass)
\item 
bool \doxymbox{\hyperlink{class_beam_1_1_flux_node_a4bd30f3c8d311afdcd5c0d208e3bbf0f}{is\+Bypassed}} () const
\item 
void \doxymbox{\hyperlink{class_beam_1_1_flux_node_ad53f3fcaa5737f46d88530f40dbfbe32}{add\+Parameter}} (std::\+shared\+\_\+ptr$<$ \doxymbox{\hyperlink{class_beam_1_1_parameter}{Parameter}} $>$ \doxymbox{\hyperlink{texture_8cpp_aaded45152436a99bb4f9bda081df9f69}{param}})
\item 
std::\+shared\+\_\+ptr$<$ \doxymbox{\hyperlink{class_beam_1_1_parameter}{Parameter}} $>$ \doxymbox{\hyperlink{class_beam_1_1_flux_node_a59a32442eec144010741b9f2086c516e}{get\+Parameter}} (const std::\+string \&name)
\item 
const std::\+map$<$ std::\+string, std::\+shared\+\_\+ptr$<$ \doxymbox{\hyperlink{class_beam_1_1_parameter}{Parameter}} $>$ $>$ \& \doxymbox{\hyperlink{class_beam_1_1_flux_node_a6296c79b1ba77aa8b9526ace4a109529}{get\+Parameters}} () const
\end{DoxyCompactItemize}
\doxysubsubsection*{Private Attributes}
\begin{DoxyCompactItemize}
\item 
\doxymbox{\hyperlink{class_beam_1_1_audio_processor_value_tree_state}{Audio\+Processor\+Value\+Tree\+State}} \doxymbox{\hyperlink{class_beam_1_1_simple_gain_processor_a3620b00b6a0d607ec70e50a7edf9df97}{m\+\_\+value\+Tree\+State}}
\item 
std::\+shared\+\_\+ptr$<$ \doxymbox{\hyperlink{class_beam_1_1_parameter}{Parameter}} $>$ \doxymbox{\hyperlink{class_beam_1_1_simple_gain_processor_a423c6462ae7d81afe3a1b460166fa413}{m\+\_\+gain\+Parameter}}
\end{DoxyCompactItemize}
\doxysubsubsection*{Additional Inherited Members}
\doxysubsection*{Protected Member Functions inherited from \doxymbox{\hyperlink{class_beam_1_1_flux_node}{Beam::\+\+Flux\+Node}}}
\begin{DoxyCompactItemize}
\item 
void \doxymbox{\hyperlink{class_beam_1_1_flux_node_ae3bafc1c5a1aa545167256172b3d3688}{setup\+Buffers}} (int num\+Inputs, int num\+Outputs, int buffer\+Size, int channels)
\begin{DoxyCompactList}\small\item\em Pre-\/allocates buffers for inputs and outputs. \end{DoxyCompactList}\end{DoxyCompactItemize}
\doxysubsection*{Protected Attributes inherited from \doxymbox{\hyperlink{class_beam_1_1_flux_node}{Beam::\+\+Flux\+Node}}}
\begin{DoxyCompactItemize}
\item 
std::\+vector$<$ std::\+vector$<$ float $>$ $>$ \doxymbox{\hyperlink{class_beam_1_1_flux_node_a8edab1c9ebd83e73bbfd92af29d6e92c}{m\+\_\+inputs}}
\item 
std::\+vector$<$ std::\+vector$<$ float $>$ $>$ \doxymbox{\hyperlink{class_beam_1_1_flux_node_a496905f0ff42c432eb38e19bd6135383}{m\+\_\+outputs}}
\item 
std::\+map$<$ std::\+string, std::\+shared\+\_\+ptr$<$ \doxymbox{\hyperlink{class_beam_1_1_parameter}{Parameter}} $>$ $>$ \doxymbox{\hyperlink{class_beam_1_1_flux_node_a65628a37cd2dd2832eda60e74ec1aed3}{m\+\_\+parameters}}
\item 
std::\+atomic$<$ bool $>$ \doxymbox{\hyperlink{class_beam_1_1_flux_node_a6116dcdcfa20998fe90dc75a74f25d9b}{m\+\_\+bypassed}} \{false\}
\item 
size\+\_\+t \doxymbox{\hyperlink{class_beam_1_1_flux_node_a7d8556ddb1482f997cda7749d737668b}{m\+\_\+current\+Frame}} = 0
\end{DoxyCompactItemize}


\doxysubsection{Detailed Description}
A simple gain processor demonstrating the new API. 

\label{doc-constructors}
\Hypertarget{class_beam_1_1_simple_gain_processor_doc-constructors}
\doxysubsection{Constructor \& Destructor Documentation}
\Hypertarget{class_beam_1_1_simple_gain_processor_add45d2b5c7ff6109e8dd0d7b1b282b90}\index{Beam::SimpleGainProcessor@{Beam::SimpleGainProcessor}!SimpleGainProcessor@{SimpleGainProcessor}}
\index{SimpleGainProcessor@{SimpleGainProcessor}!Beam::SimpleGainProcessor@{Beam::SimpleGainProcessor}}
\doxysubsubsection{\texorpdfstring{SimpleGainProcessor()}{SimpleGainProcessor()}}
{\footnotesize\ttfamily \label{class_beam_1_1_simple_gain_processor_add45d2b5c7ff6109e8dd0d7b1b282b90} 
Beam::\+\+Simple\+Gain\+Processor::\+\+Simple\+Gain\+Processor (\begin{DoxyParamCaption}{}{}\end{DoxyParamCaption})}

\Hypertarget{class_beam_1_1_simple_gain_processor_a0aedd2a7595e4d97e1c090449266e7e2}\index{Beam::SimpleGainProcessor@{Beam::SimpleGainProcessor}!````~SimpleGainProcessor@{\texorpdfstring{$\sim$}{\string~}SimpleGainProcessor}}
\index{````~SimpleGainProcessor@{\texorpdfstring{$\sim$}{\string~}SimpleGainProcessor}!Beam::SimpleGainProcessor@{Beam::SimpleGainProcessor}}
\doxysubsubsection{\texorpdfstring{\texorpdfstring{$\sim$}{\string~}SimpleGainProcessor()}{\string~SimpleGainProcessor()}}
{\footnotesize\ttfamily \label{class_beam_1_1_simple_gain_processor_a0aedd2a7595e4d97e1c090449266e7e2} 
Beam::\+\+Simple\+Gain\+Processor::\+\texorpdfstring{$\sim$}{\string~}\+Simple\+Gain\+Processor (\begin{DoxyParamCaption}{}{}\end{DoxyParamCaption})\hspace{0.3cm}{\ttfamily [override]}}



\label{doc-func-members}
\Hypertarget{class_beam_1_1_simple_gain_processor_doc-func-members}
\doxysubsection{Member Function Documentation}
\Hypertarget{class_beam_1_1_simple_gain_processor_a17d80f27c22f909f859e5441d25c0a4d}\index{Beam::SimpleGainProcessor@{Beam::SimpleGainProcessor}!getGain@{getGain}}
\index{getGain@{getGain}!Beam::SimpleGainProcessor@{Beam::SimpleGainProcessor}}
\doxysubsubsection{\texorpdfstring{getGain()}{getGain()}}
{\footnotesize\ttfamily \label{class_beam_1_1_simple_gain_processor_a17d80f27c22f909f859e5441d25c0a4d} 
float Beam::\+\+Simple\+Gain\+Processor::\+get\+Gain (\begin{DoxyParamCaption}{}{}\end{DoxyParamCaption}) const}

\Hypertarget{class_beam_1_1_simple_gain_processor_a26ab1fae0c646a95dd158babd134af91}\index{Beam::SimpleGainProcessor@{Beam::SimpleGainProcessor}!getInputPorts@{getInputPorts}}
\index{getInputPorts@{getInputPorts}!Beam::SimpleGainProcessor@{Beam::SimpleGainProcessor}}
\doxysubsubsection{\texorpdfstring{getInputPorts()}{getInputPorts()}}
{\footnotesize\ttfamily \label{class_beam_1_1_simple_gain_processor_a26ab1fae0c646a95dd158babd134af91} 
std::\+vector$<$ \doxymbox{\hyperlink{struct_beam_1_1_flux_node_1_1_port}{Flux\+Node::\+\+Port}} $>$ Beam::\+\+Simple\+Gain\+Processor::\+get\+Input\+Ports (\begin{DoxyParamCaption}{}{}\end{DoxyParamCaption}) const\hspace{0.3cm}{\ttfamily [override]}, {\ttfamily [virtual]}}



Implements \doxymbox{\hyperlink{class_beam_1_1_flux_node_a17eb02187925b52bf8e53fa3ebe3da66}{Beam::\+\+Flux\+Node}}.

\Hypertarget{class_beam_1_1_simple_gain_processor_a73bd1595872e04e5abac05d7d17ca336}\index{Beam::SimpleGainProcessor@{Beam::SimpleGainProcessor}!getName@{getName}}
\index{getName@{getName}!Beam::SimpleGainProcessor@{Beam::SimpleGainProcessor}}
\doxysubsubsection{\texorpdfstring{getName()}{getName()}}
{\footnotesize\ttfamily \label{class_beam_1_1_simple_gain_processor_a73bd1595872e04e5abac05d7d17ca336} 
std::\+string Beam::\+\+Simple\+Gain\+Processor::\+get\+Name (\begin{DoxyParamCaption}{}{}\end{DoxyParamCaption}) const\hspace{0.3cm}{\ttfamily [inline]}, {\ttfamily [override]}, {\ttfamily [virtual]}}



Implements \doxymbox{\hyperlink{class_beam_1_1_flux_node_ac638d3d9bb1050d658294bc5470abeba}{Beam::\+\+Flux\+Node}}.

\Hypertarget{class_beam_1_1_simple_gain_processor_a225eb5f513d1dd5138d020155c0978ed}\index{Beam::SimpleGainProcessor@{Beam::SimpleGainProcessor}!getOutputPorts@{getOutputPorts}}
\index{getOutputPorts@{getOutputPorts}!Beam::SimpleGainProcessor@{Beam::SimpleGainProcessor}}
\doxysubsubsection{\texorpdfstring{getOutputPorts()}{getOutputPorts()}}
{\footnotesize\ttfamily \label{class_beam_1_1_simple_gain_processor_a225eb5f513d1dd5138d020155c0978ed} 
std::\+vector$<$ \doxymbox{\hyperlink{struct_beam_1_1_flux_node_1_1_port}{Flux\+Node::\+\+Port}} $>$ Beam::\+\+Simple\+Gain\+Processor::\+get\+Output\+Ports (\begin{DoxyParamCaption}{}{}\end{DoxyParamCaption}) const\hspace{0.3cm}{\ttfamily [override]}, {\ttfamily [virtual]}}



Implements \doxymbox{\hyperlink{class_beam_1_1_flux_node_a034f59d236afd7901ed84090422e3279}{Beam::\+\+Flux\+Node}}.

\Hypertarget{class_beam_1_1_simple_gain_processor_a29edd229c91eb66476785fe38c983718}\index{Beam::SimpleGainProcessor@{Beam::SimpleGainProcessor}!process@{process}}
\index{process@{process}!Beam::SimpleGainProcessor@{Beam::SimpleGainProcessor}}
\doxysubsubsection{\texorpdfstring{process()}{process()}}
{\footnotesize\ttfamily \label{class_beam_1_1_simple_gain_processor_a29edd229c91eb66476785fe38c983718} 
void Beam::\+\+Simple\+Gain\+Processor::\+process (\begin{DoxyParamCaption}\item[{int}]{frames}{}\end{DoxyParamCaption})\hspace{0.3cm}{\ttfamily [override]}, {\ttfamily [virtual]}}



Main audio processing method. Must be implemented by subclasses. 


\begin{DoxyParams}{Parameters}
{\em frames} & Number of frames to process in the current block. \\
\hline
\end{DoxyParams}


Implements \doxymbox{\hyperlink{class_beam_1_1_flux_node_a3c263446753fa7ae5ff6928ee57bcd4d}{Beam::\+\+Flux\+Node}}.



\label{doc-variable-members}
\Hypertarget{class_beam_1_1_simple_gain_processor_doc-variable-members}
\doxysubsection{Member Data Documentation}
\Hypertarget{class_beam_1_1_simple_gain_processor_a423c6462ae7d81afe3a1b460166fa413}\index{Beam::SimpleGainProcessor@{Beam::SimpleGainProcessor}!m\_gainParameter@{m\_gainParameter}}
\index{m\_gainParameter@{m\_gainParameter}!Beam::SimpleGainProcessor@{Beam::SimpleGainProcessor}}
\doxysubsubsection{\texorpdfstring{m\_gainParameter}{m\_gainParameter}}
{\footnotesize\ttfamily \label{class_beam_1_1_simple_gain_processor_a423c6462ae7d81afe3a1b460166fa413} 
std::\+shared\+\_\+ptr$<$\doxymbox{\hyperlink{class_beam_1_1_parameter}{Parameter}}$>$ Beam::\+\+Simple\+Gain\+Processor::\+m\+\_\+gain\+Parameter\hspace{0.3cm}{\ttfamily [private]}}

\Hypertarget{class_beam_1_1_simple_gain_processor_a3620b00b6a0d607ec70e50a7edf9df97}\index{Beam::SimpleGainProcessor@{Beam::SimpleGainProcessor}!m\_valueTreeState@{m\_valueTreeState}}
\index{m\_valueTreeState@{m\_valueTreeState}!Beam::SimpleGainProcessor@{Beam::SimpleGainProcessor}}
\doxysubsubsection{\texorpdfstring{m\_valueTreeState}{m\_valueTreeState}}
{\footnotesize\ttfamily \label{class_beam_1_1_simple_gain_processor_a3620b00b6a0d607ec70e50a7edf9df97} 
\doxymbox{\hyperlink{class_beam_1_1_audio_processor_value_tree_state}{Audio\+Processor\+Value\+Tree\+State}} Beam::\+\+Simple\+Gain\+Processor::\+m\+\_\+value\+Tree\+State\hspace{0.3cm}{\ttfamily [private]}}



The documentation for this class was generated from the following files:\+\begin{DoxyCompactItemize}
\item 
src/\+engine/\+\doxymbox{\hyperlink{simple__gain__processor_8hpp}{simple\+\_\+gain\+\_\+processor.\+hpp}}\item 
src/\+engine/\+\doxymbox{\hyperlink{simple__gain__processor_8cpp}{simple\+\_\+gain\+\_\+processor.\+cpp}}\end{DoxyCompactItemize}

\doxysection{Beam::\+Sine\+Synth\+Node Class Reference}
\hypertarget{class_beam_1_1_sine_synth_node}{}\label{class_beam_1_1_sine_synth_node}\index{Beam::SineSynthNode@{Beam::SineSynthNode}}


A basic monophonic sine wave synthesizer that responds to MIDI.  




{\ttfamily \+\#include $<$sine\+\_\+synth\+\_\+node.\+hpp$>$}

Inheritance diagram for Beam::\+Sine\+Synth\+Node:\+\begin{figure}[H]
\begin{center}
\leavevmode
\includegraphics[height=2.000000cm]{class_beam_1_1_sine_synth_node}
\end{center}
\end{figure}
\doxysubsubsection*{Public Member Functions}
\begin{DoxyCompactItemize}
\item 
\doxymbox{\hyperlink{class_beam_1_1_sine_synth_node_a46897a3b738719199b26351203ae9d22}{Sine\+Synth\+Node}} (int buffer\+Size, float sample\+Rate)
\item 
void \doxymbox{\hyperlink{class_beam_1_1_sine_synth_node_a3440774582a70b64fa5cd67cec5ffceb}{process\+MIDI}} (const \doxymbox{\hyperlink{class_beam_1_1_m_i_d_i_buffer}{MIDIBuffer}} \&midi) override
\begin{DoxyCompactList}\small\item\em Optional MIDI processing. Called before \doxylink{class_beam_1_1_sine_synth_node_aa39837475aad06ad5ae86a8cddfd425f}{process()} in the engine loop. \end{DoxyCompactList}\item 
void \doxymbox{\hyperlink{class_beam_1_1_sine_synth_node_aa39837475aad06ad5ae86a8cddfd425f}{process}} (int frames) override
\begin{DoxyCompactList}\small\item\em Main audio processing method. Must be implemented by subclasses. \end{DoxyCompactList}\item 
std::\+string \doxymbox{\hyperlink{class_beam_1_1_sine_synth_node_a745b626198ff61dc6196df3ff08b1de9}{get\+Name}} () const override
\item 
std::\+vector$<$ \doxymbox{\hyperlink{struct_beam_1_1_flux_node_1_1_port}{Port}} $>$ \doxymbox{\hyperlink{class_beam_1_1_sine_synth_node_a5bdac8c5ac96a5413cc233af8918f29b}{get\+Input\+Ports}} () const override
\item 
std::\+vector$<$ \doxymbox{\hyperlink{struct_beam_1_1_flux_node_1_1_port}{Port}} $>$ \doxymbox{\hyperlink{class_beam_1_1_sine_synth_node_ac3bde3e31ac67687fd6470b72a0d7b6f}{get\+Output\+Ports}} () const override
\end{DoxyCompactItemize}
\doxysubsection*{Public Member Functions inherited from \doxymbox{\hyperlink{class_beam_1_1_flux_node}{Beam::\+\+Flux\+Node}}}
\begin{DoxyCompactItemize}
\item 
virtual \doxymbox{\hyperlink{class_beam_1_1_flux_node_a708c135cdb61e8838469998cd8a84e65}{\texorpdfstring{$\sim$}{\string~}\+Flux\+Node}} ()=default
\item 
virtual void \doxymbox{\hyperlink{class_beam_1_1_flux_node_ace8cc49479d8924d44bca5fd4cd955e2}{on\+Transport\+State\+Changed}} (bool playing)
\begin{DoxyCompactList}\small\item\em Responds to global transport changes (Play/\+\+Pause). \end{DoxyCompactList}\item 
virtual void \doxymbox{\hyperlink{class_beam_1_1_flux_node_adc7c4e979bf27de5bfca66815ae97a67}{on\+Transport\+Seek}} (size\+\_\+t frame)
\begin{DoxyCompactList}\small\item\em Responds to timeline seeking. \end{DoxyCompactList}\item 
void \doxymbox{\hyperlink{class_beam_1_1_flux_node_aa579ec06608fd776987bbb089f27fd94}{set\+Current\+Frame}} (size\+\_\+t frame)
\begin{DoxyCompactList}\small\item\em Sets the current playhead position for this block. \end{DoxyCompactList}\item 
float \texorpdfstring{$\ast$}{*} \doxymbox{\hyperlink{class_beam_1_1_flux_node_ac90bd1a05b5bed3d68978f532386ed29}{get\+Input\+Buffer}} (int port\+Idx)
\item 
float \texorpdfstring{$\ast$}{*} \doxymbox{\hyperlink{class_beam_1_1_flux_node_abf11cfd4f2346ee0cd46d4345f1ed7d4}{get\+Output\+Buffer}} (int port\+Idx)
\item 
void \doxymbox{\hyperlink{class_beam_1_1_flux_node_af37f8c1b6b825da2ce7e35011d6f8253}{set\+Bypass}} (bool bypass)
\item 
bool \doxymbox{\hyperlink{class_beam_1_1_flux_node_a4bd30f3c8d311afdcd5c0d208e3bbf0f}{is\+Bypassed}} () const
\item 
void \doxymbox{\hyperlink{class_beam_1_1_flux_node_ad53f3fcaa5737f46d88530f40dbfbe32}{add\+Parameter}} (std::\+shared\+\_\+ptr$<$ \doxymbox{\hyperlink{class_beam_1_1_parameter}{Parameter}} $>$ \doxymbox{\hyperlink{texture_8cpp_aaded45152436a99bb4f9bda081df9f69}{param}})
\item 
std::\+shared\+\_\+ptr$<$ \doxymbox{\hyperlink{class_beam_1_1_parameter}{Parameter}} $>$ \doxymbox{\hyperlink{class_beam_1_1_flux_node_a59a32442eec144010741b9f2086c516e}{get\+Parameter}} (const std::\+string \&name)
\item 
const std::\+map$<$ std::\+string, std::\+shared\+\_\+ptr$<$ \doxymbox{\hyperlink{class_beam_1_1_parameter}{Parameter}} $>$ $>$ \& \doxymbox{\hyperlink{class_beam_1_1_flux_node_a6296c79b1ba77aa8b9526ace4a109529}{get\+Parameters}} () const
\end{DoxyCompactItemize}
\doxysubsubsection*{Private Attributes}
\begin{DoxyCompactItemize}
\item 
float \doxymbox{\hyperlink{class_beam_1_1_sine_synth_node_a0dcc51f18341e2293a20dfe16557df43}{m\+\_\+sample\+Rate}}
\item 
float \doxymbox{\hyperlink{class_beam_1_1_sine_synth_node_a4abc2fc473844135c565eb64590d8463}{m\+\_\+frequency}} = 440.\+0f
\item 
float \doxymbox{\hyperlink{class_beam_1_1_sine_synth_node_ace44a74c34e39fff03b1d0ba1733550b}{m\+\_\+phase}}
\item 
bool \doxymbox{\hyperlink{class_beam_1_1_sine_synth_node_a9b584e682c6ff2268bb38a29840a0189}{m\+\_\+active}}
\end{DoxyCompactItemize}
\doxysubsubsection*{Additional Inherited Members}
\doxysubsection*{Protected Member Functions inherited from \doxymbox{\hyperlink{class_beam_1_1_flux_node}{Beam::\+\+Flux\+Node}}}
\begin{DoxyCompactItemize}
\item 
void \doxymbox{\hyperlink{class_beam_1_1_flux_node_ae3bafc1c5a1aa545167256172b3d3688}{setup\+Buffers}} (int num\+Inputs, int num\+Outputs, int buffer\+Size, int channels)
\begin{DoxyCompactList}\small\item\em Pre-\/allocates buffers for inputs and outputs. \end{DoxyCompactList}\end{DoxyCompactItemize}
\doxysubsection*{Protected Attributes inherited from \doxymbox{\hyperlink{class_beam_1_1_flux_node}{Beam::\+\+Flux\+Node}}}
\begin{DoxyCompactItemize}
\item 
std::\+vector$<$ std::\+vector$<$ float $>$ $>$ \doxymbox{\hyperlink{class_beam_1_1_flux_node_a8edab1c9ebd83e73bbfd92af29d6e92c}{m\+\_\+inputs}}
\item 
std::\+vector$<$ std::\+vector$<$ float $>$ $>$ \doxymbox{\hyperlink{class_beam_1_1_flux_node_a496905f0ff42c432eb38e19bd6135383}{m\+\_\+outputs}}
\item 
std::\+map$<$ std::\+string, std::\+shared\+\_\+ptr$<$ \doxymbox{\hyperlink{class_beam_1_1_parameter}{Parameter}} $>$ $>$ \doxymbox{\hyperlink{class_beam_1_1_flux_node_a65628a37cd2dd2832eda60e74ec1aed3}{m\+\_\+parameters}}
\item 
std::\+atomic$<$ bool $>$ \doxymbox{\hyperlink{class_beam_1_1_flux_node_a6116dcdcfa20998fe90dc75a74f25d9b}{m\+\_\+bypassed}} \{false\}
\item 
size\+\_\+t \doxymbox{\hyperlink{class_beam_1_1_flux_node_a7d8556ddb1482f997cda7749d737668b}{m\+\_\+current\+Frame}} = 0
\end{DoxyCompactItemize}


\doxysubsection{Detailed Description}
A basic monophonic sine wave synthesizer that responds to MIDI. 

\label{doc-constructors}
\Hypertarget{class_beam_1_1_sine_synth_node_doc-constructors}
\doxysubsection{Constructor \& Destructor Documentation}
\Hypertarget{class_beam_1_1_sine_synth_node_a46897a3b738719199b26351203ae9d22}\index{Beam::SineSynthNode@{Beam::SineSynthNode}!SineSynthNode@{SineSynthNode}}
\index{SineSynthNode@{SineSynthNode}!Beam::SineSynthNode@{Beam::SineSynthNode}}
\doxysubsubsection{\texorpdfstring{SineSynthNode()}{SineSynthNode()}}
{\footnotesize\ttfamily \label{class_beam_1_1_sine_synth_node_a46897a3b738719199b26351203ae9d22} 
Beam::\+\+Sine\+Synth\+Node::\+\+Sine\+Synth\+Node (\begin{DoxyParamCaption}\item[{int}]{buffer\+Size}{, }\item[{float}]{sample\+Rate}{}\end{DoxyParamCaption})\hspace{0.3cm}{\ttfamily [inline]}}



\label{doc-func-members}
\Hypertarget{class_beam_1_1_sine_synth_node_doc-func-members}
\doxysubsection{Member Function Documentation}
\Hypertarget{class_beam_1_1_sine_synth_node_a5bdac8c5ac96a5413cc233af8918f29b}\index{Beam::SineSynthNode@{Beam::SineSynthNode}!getInputPorts@{getInputPorts}}
\index{getInputPorts@{getInputPorts}!Beam::SineSynthNode@{Beam::SineSynthNode}}
\doxysubsubsection{\texorpdfstring{getInputPorts()}{getInputPorts()}}
{\footnotesize\ttfamily \label{class_beam_1_1_sine_synth_node_a5bdac8c5ac96a5413cc233af8918f29b} 
std::\+vector$<$ \doxymbox{\hyperlink{struct_beam_1_1_flux_node_1_1_port}{Port}} $>$ Beam::\+\+Sine\+Synth\+Node::\+get\+Input\+Ports (\begin{DoxyParamCaption}{}{}\end{DoxyParamCaption}) const\hspace{0.3cm}{\ttfamily [inline]}, {\ttfamily [override]}, {\ttfamily [virtual]}}



Implements \doxymbox{\hyperlink{class_beam_1_1_flux_node_a17eb02187925b52bf8e53fa3ebe3da66}{Beam::\+\+Flux\+Node}}.

\Hypertarget{class_beam_1_1_sine_synth_node_a745b626198ff61dc6196df3ff08b1de9}\index{Beam::SineSynthNode@{Beam::SineSynthNode}!getName@{getName}}
\index{getName@{getName}!Beam::SineSynthNode@{Beam::SineSynthNode}}
\doxysubsubsection{\texorpdfstring{getName()}{getName()}}
{\footnotesize\ttfamily \label{class_beam_1_1_sine_synth_node_a745b626198ff61dc6196df3ff08b1de9} 
std::\+string Beam::\+\+Sine\+Synth\+Node::\+get\+Name (\begin{DoxyParamCaption}{}{}\end{DoxyParamCaption}) const\hspace{0.3cm}{\ttfamily [inline]}, {\ttfamily [override]}, {\ttfamily [virtual]}}



Implements \doxymbox{\hyperlink{class_beam_1_1_flux_node_ac638d3d9bb1050d658294bc5470abeba}{Beam::\+\+Flux\+Node}}.

\Hypertarget{class_beam_1_1_sine_synth_node_ac3bde3e31ac67687fd6470b72a0d7b6f}\index{Beam::SineSynthNode@{Beam::SineSynthNode}!getOutputPorts@{getOutputPorts}}
\index{getOutputPorts@{getOutputPorts}!Beam::SineSynthNode@{Beam::SineSynthNode}}
\doxysubsubsection{\texorpdfstring{getOutputPorts()}{getOutputPorts()}}
{\footnotesize\ttfamily \label{class_beam_1_1_sine_synth_node_ac3bde3e31ac67687fd6470b72a0d7b6f} 
std::\+vector$<$ \doxymbox{\hyperlink{struct_beam_1_1_flux_node_1_1_port}{Port}} $>$ Beam::\+\+Sine\+Synth\+Node::\+get\+Output\+Ports (\begin{DoxyParamCaption}{}{}\end{DoxyParamCaption}) const\hspace{0.3cm}{\ttfamily [inline]}, {\ttfamily [override]}, {\ttfamily [virtual]}}



Implements \doxymbox{\hyperlink{class_beam_1_1_flux_node_a034f59d236afd7901ed84090422e3279}{Beam::\+\+Flux\+Node}}.

\Hypertarget{class_beam_1_1_sine_synth_node_aa39837475aad06ad5ae86a8cddfd425f}\index{Beam::SineSynthNode@{Beam::SineSynthNode}!process@{process}}
\index{process@{process}!Beam::SineSynthNode@{Beam::SineSynthNode}}
\doxysubsubsection{\texorpdfstring{process()}{process()}}
{\footnotesize\ttfamily \label{class_beam_1_1_sine_synth_node_aa39837475aad06ad5ae86a8cddfd425f} 
void Beam::\+\+Sine\+Synth\+Node::\+process (\begin{DoxyParamCaption}\item[{int}]{frames}{}\end{DoxyParamCaption})\hspace{0.3cm}{\ttfamily [inline]}, {\ttfamily [override]}, {\ttfamily [virtual]}}



Main audio processing method. Must be implemented by subclasses. 


\begin{DoxyParams}{Parameters}
{\em frames} & Number of frames to process in the current block. \\
\hline
\end{DoxyParams}


Implements \doxymbox{\hyperlink{class_beam_1_1_flux_node_a3c263446753fa7ae5ff6928ee57bcd4d}{Beam::\+\+Flux\+Node}}.

\Hypertarget{class_beam_1_1_sine_synth_node_a3440774582a70b64fa5cd67cec5ffceb}\index{Beam::SineSynthNode@{Beam::SineSynthNode}!processMIDI@{processMIDI}}
\index{processMIDI@{processMIDI}!Beam::SineSynthNode@{Beam::SineSynthNode}}
\doxysubsubsection{\texorpdfstring{processMIDI()}{processMIDI()}}
{\footnotesize\ttfamily \label{class_beam_1_1_sine_synth_node_a3440774582a70b64fa5cd67cec5ffceb} 
void Beam::\+\+Sine\+Synth\+Node::\+process\+MIDI (\begin{DoxyParamCaption}\item[{const \doxymbox{\hyperlink{class_beam_1_1_m_i_d_i_buffer}{MIDIBuffer}} \&}]{midi}{}\end{DoxyParamCaption})\hspace{0.3cm}{\ttfamily [inline]}, {\ttfamily [override]}, {\ttfamily [virtual]}}



Optional MIDI processing. Called before \doxylink{class_beam_1_1_sine_synth_node_aa39837475aad06ad5ae86a8cddfd425f}{process()} in the engine loop. 



Reimplemented from \doxymbox{\hyperlink{class_beam_1_1_flux_node_ae9d1e151eff5166de969f45de06d5596}{Beam::\+\+Flux\+Node}}.



\label{doc-variable-members}
\Hypertarget{class_beam_1_1_sine_synth_node_doc-variable-members}
\doxysubsection{Member Data Documentation}
\Hypertarget{class_beam_1_1_sine_synth_node_a9b584e682c6ff2268bb38a29840a0189}\index{Beam::SineSynthNode@{Beam::SineSynthNode}!m\_active@{m\_active}}
\index{m\_active@{m\_active}!Beam::SineSynthNode@{Beam::SineSynthNode}}
\doxysubsubsection{\texorpdfstring{m\_active}{m\_active}}
{\footnotesize\ttfamily \label{class_beam_1_1_sine_synth_node_a9b584e682c6ff2268bb38a29840a0189} 
bool Beam::\+\+Sine\+Synth\+Node::\+m\+\_\+active\hspace{0.3cm}{\ttfamily [private]}}

\Hypertarget{class_beam_1_1_sine_synth_node_a4abc2fc473844135c565eb64590d8463}\index{Beam::SineSynthNode@{Beam::SineSynthNode}!m\_frequency@{m\_frequency}}
\index{m\_frequency@{m\_frequency}!Beam::SineSynthNode@{Beam::SineSynthNode}}
\doxysubsubsection{\texorpdfstring{m\_frequency}{m\_frequency}}
{\footnotesize\ttfamily \label{class_beam_1_1_sine_synth_node_a4abc2fc473844135c565eb64590d8463} 
float Beam::\+\+Sine\+Synth\+Node::\+m\+\_\+frequency = 440.\+0f\hspace{0.3cm}{\ttfamily [private]}}

\Hypertarget{class_beam_1_1_sine_synth_node_ace44a74c34e39fff03b1d0ba1733550b}\index{Beam::SineSynthNode@{Beam::SineSynthNode}!m\_phase@{m\_phase}}
\index{m\_phase@{m\_phase}!Beam::SineSynthNode@{Beam::SineSynthNode}}
\doxysubsubsection{\texorpdfstring{m\_phase}{m\_phase}}
{\footnotesize\ttfamily \label{class_beam_1_1_sine_synth_node_ace44a74c34e39fff03b1d0ba1733550b} 
float Beam::\+\+Sine\+Synth\+Node::\+m\+\_\+phase\hspace{0.3cm}{\ttfamily [private]}}

\Hypertarget{class_beam_1_1_sine_synth_node_a0dcc51f18341e2293a20dfe16557df43}\index{Beam::SineSynthNode@{Beam::SineSynthNode}!m\_sampleRate@{m\_sampleRate}}
\index{m\_sampleRate@{m\_sampleRate}!Beam::SineSynthNode@{Beam::SineSynthNode}}
\doxysubsubsection{\texorpdfstring{m\_sampleRate}{m\_sampleRate}}
{\footnotesize\ttfamily \label{class_beam_1_1_sine_synth_node_a0dcc51f18341e2293a20dfe16557df43} 
float Beam::\+\+Sine\+Synth\+Node::\+m\+\_\+sample\+Rate\hspace{0.3cm}{\ttfamily [private]}}



The documentation for this class was generated from the following file:\+\begin{DoxyCompactItemize}
\item 
src/\+engine/\+\doxymbox{\hyperlink{sine__synth__node_8hpp}{sine\+\_\+synth\+\_\+node.\+hpp}}\end{DoxyCompactItemize}

\doxysection{Beam::\+Slider Class Reference}
\hypertarget{class_beam_1_1_slider}{}\label{class_beam_1_1_slider}\index{Beam::Slider@{Beam::Slider}}


A slider component, similar to JUCE\textquotesingle{}s \doxylink{class_beam_1_1_slider}{Slider}.  




{\ttfamily \+\#include $<$slider.\+hpp$>$}

Inheritance diagram for Beam::\+Slider:\+\begin{figure}[H]
\begin{center}
\leavevmode
\includegraphics[height=2.000000cm]{class_beam_1_1_slider}
\end{center}
\end{figure}
\doxysubsubsection*{Public Member Functions}
\begin{DoxyCompactItemize}
\item 
\doxymbox{\hyperlink{class_beam_1_1_slider_a5958527c0734fb54aef065227a9a80f6}{Slider}} ()
\item 
\doxymbox{\hyperlink{class_beam_1_1_slider_ab054dd6f9e6e84e9e98227a30b785084}{Slider}} (std::\+shared\+\_\+ptr$<$ \doxymbox{\hyperlink{class_beam_1_1_parameter}{Parameter}} $>$ parameter)
\item 
\doxymbox{\hyperlink{class_beam_1_1_slider_a0f59043f59f0087c5eb7e79bd5216edf}{\texorpdfstring{$\sim$}{\string~}\+Slider}} () override
\item 
void \doxymbox{\hyperlink{class_beam_1_1_slider_a91ee4b48c9ba471bf5198fac14b5453e}{set\+Slider\+Style}} (\doxymbox{\hyperlink{namespace_beam_a0f94fc63f922f04ab49fd1f60f79ef54}{Slider\+Style}} style)
\begin{DoxyCompactList}\small\item\em Sets the slider style. \end{DoxyCompactList}\item 
\doxymbox{\hyperlink{namespace_beam_a0f94fc63f922f04ab49fd1f60f79ef54}{Slider\+Style}} \doxymbox{\hyperlink{class_beam_1_1_slider_a4918a748bf46c078f7e1cd1ab8a62826}{get\+Slider\+Style}} () const
\begin{DoxyCompactList}\small\item\em Gets the slider style. \end{DoxyCompactList}\item 
void \doxymbox{\hyperlink{class_beam_1_1_slider_a724ec07492a3f048e8e46808998ad6dd}{set\+Range}} (double min, double max, double interval=0.\+0)
\begin{DoxyCompactList}\small\item\em Sets the range of values. \end{DoxyCompactList}\item 
void \doxymbox{\hyperlink{class_beam_1_1_slider_ad23c25d65be5a2d93c76e5c8ff14021d}{set\+Value}} (double new\+Value, bool notify=true)
\begin{DoxyCompactList}\small\item\em Sets the current value. \end{DoxyCompactList}\item 
double \doxymbox{\hyperlink{class_beam_1_1_slider_af9238992dbbbe5bd064d2b991b00d0e6}{get\+Value}} () const
\begin{DoxyCompactList}\small\item\em Gets the current value. \end{DoxyCompactList}\item 
void \doxymbox{\hyperlink{class_beam_1_1_slider_af127d4fb0160359c837b7750244922cd}{set\+Text\+Value\+Suffix}} (const std::\+string \&suffix)
\begin{DoxyCompactList}\small\item\em Sets the text value to display. \end{DoxyCompactList}\item 
void \doxymbox{\hyperlink{class_beam_1_1_slider_a6be4631112646615e8d4f7f722fea46b}{paint}} (\doxymbox{\hyperlink{class_beam_1_1_quad_batcher}{Quad\+Batcher}} \&g) override
\begin{DoxyCompactList}\small\item\em Paints the slider. \end{DoxyCompactList}\item 
void \doxymbox{\hyperlink{class_beam_1_1_slider_a4a2744e42c43c6568720baa27ecdffeb}{mouse\+Down}} (const \doxymbox{\hyperlink{class_beam_1_1_mouse_event}{Mouse\+Event}} \&event) override
\begin{DoxyCompactList}\small\item\em Called when the mouse is pressed. \end{DoxyCompactList}\item 
void \doxymbox{\hyperlink{class_beam_1_1_slider_adb038e50182312cdb6d1d96e60f6dfea}{mouse\+Drag}} (const \doxymbox{\hyperlink{class_beam_1_1_mouse_event}{Mouse\+Event}} \&event) override
\begin{DoxyCompactList}\small\item\em Called when the mouse is dragged. \end{DoxyCompactList}\item 
void \doxymbox{\hyperlink{class_beam_1_1_slider_a454bd346772535c3fa29e7240fc2166e}{mouse\+Up}} (const \doxymbox{\hyperlink{class_beam_1_1_mouse_event}{Mouse\+Event}} \&event) override
\begin{DoxyCompactList}\small\item\em Called when the mouse is released. \end{DoxyCompactList}\item 
void \doxymbox{\hyperlink{class_beam_1_1_slider_a064fa51ef1e0c7d3d5ec9f9b7906ccb0}{set\+Parameter}} (std::\+shared\+\_\+ptr$<$ \doxymbox{\hyperlink{class_beam_1_1_parameter}{Parameter}} $>$ parameter)
\begin{DoxyCompactList}\small\item\em Sets the parameter to control. \end{DoxyCompactList}\item 
std::\+shared\+\_\+ptr$<$ \doxymbox{\hyperlink{class_beam_1_1_parameter}{Parameter}} $>$ \doxymbox{\hyperlink{class_beam_1_1_slider_ac38e23e5169013f677b612baa9ca23cf}{get\+Parameter}} () const
\begin{DoxyCompactList}\small\item\em Gets the controlled parameter. \end{DoxyCompactList}\end{DoxyCompactItemize}
\doxysubsection*{Public Member Functions inherited from \doxymbox{\hyperlink{class_beam_1_1_gui_component}{Beam::\+\+Gui\+Component}}}
\begin{DoxyCompactItemize}
\item 
\doxymbox{\hyperlink{class_beam_1_1_gui_component_a3a2d448a1f30616384f16a41102c613a}{Gui\+Component}} ()
\item 
virtual \doxymbox{\hyperlink{class_beam_1_1_gui_component_a255a25fc9e1a2abdfdfcb7c11da1e66e}{\texorpdfstring{$\sim$}{\string~}\+Gui\+Component}} ()
\item 
virtual void \doxymbox{\hyperlink{class_beam_1_1_gui_component_a6ecaca5ca83fc45c232093a1012191d2}{set\+Bounds}} (float x, float y, float \doxymbox{\hyperlink{texture_8cpp_a8710f3c5c66c09e158c8619b3fca614a}{width}}, float \doxymbox{\hyperlink{texture_8cpp_a1055637f17e35a0ca82b396bb94914e5}{height}})
\begin{DoxyCompactList}\small\item\em Sets the bounds of this component. \end{DoxyCompactList}\item 
const \doxymbox{\hyperlink{struct_beam_1_1_rect}{Rect}} \& \doxymbox{\hyperlink{class_beam_1_1_gui_component_a2de58576d7e55fd82acbdb2f18207982}{get\+Bounds}} () const
\begin{DoxyCompactList}\small\item\em Gets the bounds of this component. \end{DoxyCompactList}\item 
void \doxymbox{\hyperlink{class_beam_1_1_gui_component_abc37535b6bef8b996c41849df8c95e96}{set\+Visible}} (bool should\+Be\+Visible)
\begin{DoxyCompactList}\small\item\em Sets the component\textquotesingle{}s visibility. \end{DoxyCompactList}\item 
bool \doxymbox{\hyperlink{class_beam_1_1_gui_component_ae4752b8375293063779e88a83c1a799a}{is\+Visible}} () const
\begin{DoxyCompactList}\small\item\em Checks if the component is visible. \end{DoxyCompactList}\item 
virtual void \doxymbox{\hyperlink{class_beam_1_1_gui_component_a683989837c3ab83ffb1148e0b473e573}{resized}} ()
\begin{DoxyCompactList}\small\item\em Called when the component\textquotesingle{}s size changes. \end{DoxyCompactList}\item 
virtual void \doxymbox{\hyperlink{class_beam_1_1_gui_component_a357d8829f546299e6fda6ff119c582c1}{mouse\+Enter}} (const \doxymbox{\hyperlink{class_beam_1_1_mouse_event}{Mouse\+Event}} \&event)
\begin{DoxyCompactList}\small\item\em Called when the mouse enters the component. \end{DoxyCompactList}\item 
virtual void \doxymbox{\hyperlink{class_beam_1_1_gui_component_a9c580600fa3396a3085b2c7ca9bd6869}{mouse\+Exit}} (const \doxymbox{\hyperlink{class_beam_1_1_mouse_event}{Mouse\+Event}} \&event)
\begin{DoxyCompactList}\small\item\em Called when the mouse exits the component. \end{DoxyCompactList}\item 
virtual void \doxymbox{\hyperlink{class_beam_1_1_gui_component_abb1547968352dd7ed45ec761e4ceac07}{mouse\+Move}} (const \doxymbox{\hyperlink{class_beam_1_1_mouse_event}{Mouse\+Event}} \&event)
\begin{DoxyCompactList}\small\item\em Called when the mouse is moved. \end{DoxyCompactList}\item 
virtual bool \doxymbox{\hyperlink{class_beam_1_1_gui_component_a2459c6228fcbc914614e6252a43b016f}{key\+Pressed}} (const \doxymbox{\hyperlink{class_beam_1_1_key_press}{Key\+Press}} \&key)
\begin{DoxyCompactList}\small\item\em Called when a key is pressed. \end{DoxyCompactList}\item 
void \doxymbox{\hyperlink{class_beam_1_1_gui_component_af85677b7e220a2b1bf69659d4167bce5}{add\+Child\+Component}} (std::\+shared\+\_\+ptr$<$ \doxymbox{\hyperlink{class_beam_1_1_gui_component_a3a2d448a1f30616384f16a41102c613a}{Gui\+Component}} $>$ child)
\begin{DoxyCompactList}\small\item\em Adds a child component. \end{DoxyCompactList}\item 
void \doxymbox{\hyperlink{class_beam_1_1_gui_component_a2b68395b5a1ebdfe673053b2aa83122b}{remove\+Child\+Component}} (std::\+shared\+\_\+ptr$<$ \doxymbox{\hyperlink{class_beam_1_1_gui_component_a3a2d448a1f30616384f16a41102c613a}{Gui\+Component}} $>$ child)
\begin{DoxyCompactList}\small\item\em Removes a child component. \end{DoxyCompactList}\item 
void \doxymbox{\hyperlink{class_beam_1_1_gui_component_a7dbd4e5e0b8955748fe0962bcb476089}{paint\+Entire\+Component}} (\doxymbox{\hyperlink{class_beam_1_1_quad_batcher}{Quad\+Batcher}} \&g)
\begin{DoxyCompactList}\small\item\em Paints this component and all its children. \end{DoxyCompactList}\item 
bool \doxymbox{\hyperlink{class_beam_1_1_gui_component_abad300b88267731a68afd9a94e5225c4}{contains}} (float x, float y) const
\begin{DoxyCompactList}\small\item\em Checks if a point is inside this component. \end{DoxyCompactList}\item 
void \doxymbox{\hyperlink{class_beam_1_1_gui_component_ae7e5d9dde8c3bae347fdaaac29b5d96a}{set\+Name}} (const std::\+string \&name)
\begin{DoxyCompactList}\small\item\em Sets the component\textquotesingle{}s name. \end{DoxyCompactList}\item 
const std::\+string \& \doxymbox{\hyperlink{class_beam_1_1_gui_component_ac21fae6abb5616da7b0882c924961ae9}{get\+Name}} () const
\begin{DoxyCompactList}\small\item\em Gets the component\textquotesingle{}s name. \end{DoxyCompactList}\end{DoxyCompactItemize}
\doxysubsubsection*{Private Attributes}
\begin{DoxyCompactItemize}
\item 
\doxymbox{\hyperlink{namespace_beam_a0f94fc63f922f04ab49fd1f60f79ef54}{Slider\+Style}} \doxymbox{\hyperlink{class_beam_1_1_slider_addc7759649fff42b885cef58a37448b2}{m\+\_\+style}} = \doxymbox{\hyperlink{namespace_beam_a0f94fc63f922f04ab49fd1f60f79ef54a27d814af3beb23f25ed8d70819270a9b}{Slider\+Style::\+\+Linear\+Horizontal}}
\item 
double \doxymbox{\hyperlink{class_beam_1_1_slider_a68cfb1023f6aba6e2a750cc2cae4e212}{m\+\_\+min}} = 0.\+0
\item 
double \doxymbox{\hyperlink{class_beam_1_1_slider_a6ef074c99976f2ea263c784fd9a3d8db}{m\+\_\+max}} = 1.\+0
\item 
double \doxymbox{\hyperlink{class_beam_1_1_slider_ae53982bc1ad1ea8fb9ceae230f5501b0}{m\+\_\+interval}} = 0.\+0
\item 
double \doxymbox{\hyperlink{class_beam_1_1_slider_a1b74d44a5f99d21a967852c984a92fb2}{m\+\_\+value}} = 0.\+0
\item 
std::\+string \doxymbox{\hyperlink{class_beam_1_1_slider_a96c9d52ffd0c65b179d0dce80ed8171e}{m\+\_\+text\+Suffix}}
\item 
std::\+shared\+\_\+ptr$<$ \doxymbox{\hyperlink{class_beam_1_1_parameter}{Parameter}} $>$ \doxymbox{\hyperlink{class_beam_1_1_slider_a69a9c19aabdc09653ec86c26e27de3f2}{m\+\_\+parameter}}
\item 
bool \doxymbox{\hyperlink{class_beam_1_1_slider_a8968f4d20f4c6e44bfb8148f5f8e881a}{m\+\_\+is\+Dragging}} = false
\item 
float \doxymbox{\hyperlink{class_beam_1_1_slider_a5921921547e4f92fe4ae302d6b4f01fb}{m\+\_\+drag\+StartX}}
\item 
float \doxymbox{\hyperlink{class_beam_1_1_slider_a9b0d97c3eddfc07f075634da191a25ab}{m\+\_\+drag\+StartY}}
\item 
double \doxymbox{\hyperlink{class_beam_1_1_slider_ae6a11cb61bef6ad7fd3c6496b67efde5}{m\+\_\+drag\+Start\+Value}}
\end{DoxyCompactItemize}
\doxysubsubsection*{Additional Inherited Members}
\doxysubsection*{Protected Attributes inherited from \doxymbox{\hyperlink{class_beam_1_1_gui_component}{Beam::\+\+Gui\+Component}}}
\begin{DoxyCompactItemize}
\item 
\doxymbox{\hyperlink{struct_beam_1_1_rect}{Rect}} \doxymbox{\hyperlink{class_beam_1_1_gui_component_ad3c42f55d6a7e47f65bd1d0f3ffb5291}{m\+\_\+bounds}} \{0, 0, 0, 0\}
\item 
bool \doxymbox{\hyperlink{class_beam_1_1_gui_component_afbfe066e00bfffc064d80082c839ebe6}{m\+\_\+visible}} = true
\item 
std::\+vector$<$ std::\+shared\+\_\+ptr$<$ \doxymbox{\hyperlink{class_beam_1_1_gui_component_a3a2d448a1f30616384f16a41102c613a}{Gui\+Component}} $>$ $>$ \doxymbox{\hyperlink{class_beam_1_1_gui_component_ada191fb579394c85ec4fa977b1b108b0}{m\+\_\+children}}
\item 
std::\+string \doxymbox{\hyperlink{class_beam_1_1_gui_component_a5e462cb9a299d364cb53b1a768369246}{m\+\_\+name}}
\item 
std::\+function$<$ void()$>$ \doxymbox{\hyperlink{class_beam_1_1_gui_component_a46e1692a3cd91b4464cce1680c759a1e}{m\+\_\+paint\+Callback}}
\item 
std::\+function$<$ void()$>$ \doxymbox{\hyperlink{class_beam_1_1_gui_component_a5ea91ef234ce12ce8dc7fcc143178072}{m\+\_\+resized\+Callback}}
\end{DoxyCompactItemize}


\doxysubsection{Detailed Description}
A slider component, similar to JUCE\textquotesingle{}s \doxylink{class_beam_1_1_slider}{Slider}. 

\label{doc-constructors}
\Hypertarget{class_beam_1_1_slider_doc-constructors}
\doxysubsection{Constructor \& Destructor Documentation}
\Hypertarget{class_beam_1_1_slider_a5958527c0734fb54aef065227a9a80f6}\index{Beam::Slider@{Beam::Slider}!Slider@{Slider}}
\index{Slider@{Slider}!Beam::Slider@{Beam::Slider}}
\doxysubsubsection{\texorpdfstring{Slider()}{Slider()}\hspace{0.1cm}{\footnotesize\ttfamily [1/2]}}
{\footnotesize\ttfamily \label{class_beam_1_1_slider_a5958527c0734fb54aef065227a9a80f6} 
Beam::\+\+Slider::\+\+Slider (\begin{DoxyParamCaption}{}{}\end{DoxyParamCaption})}

\Hypertarget{class_beam_1_1_slider_ab054dd6f9e6e84e9e98227a30b785084}\index{Beam::Slider@{Beam::Slider}!Slider@{Slider}}
\index{Slider@{Slider}!Beam::Slider@{Beam::Slider}}
\doxysubsubsection{\texorpdfstring{Slider()}{Slider()}\hspace{0.1cm}{\footnotesize\ttfamily [2/2]}}
{\footnotesize\ttfamily \label{class_beam_1_1_slider_ab054dd6f9e6e84e9e98227a30b785084} 
Beam::\+\+Slider::\+\+Slider (\begin{DoxyParamCaption}\item[{std::\+shared\+\_\+ptr$<$ \doxymbox{\hyperlink{class_beam_1_1_parameter}{Parameter}} $>$}]{parameter}{}\end{DoxyParamCaption})\hspace{0.3cm}{\ttfamily [explicit]}}

\Hypertarget{class_beam_1_1_slider_a0f59043f59f0087c5eb7e79bd5216edf}\index{Beam::Slider@{Beam::Slider}!````~Slider@{\texorpdfstring{$\sim$}{\string~}Slider}}
\index{````~Slider@{\texorpdfstring{$\sim$}{\string~}Slider}!Beam::Slider@{Beam::Slider}}
\doxysubsubsection{\texorpdfstring{\texorpdfstring{$\sim$}{\string~}Slider()}{\string~Slider()}}
{\footnotesize\ttfamily \label{class_beam_1_1_slider_a0f59043f59f0087c5eb7e79bd5216edf} 
Beam::\+\+Slider::\+\texorpdfstring{$\sim$}{\string~}\+Slider (\begin{DoxyParamCaption}{}{}\end{DoxyParamCaption})\hspace{0.3cm}{\ttfamily [override]}}



\label{doc-func-members}
\Hypertarget{class_beam_1_1_slider_doc-func-members}
\doxysubsection{Member Function Documentation}
\Hypertarget{class_beam_1_1_slider_ac38e23e5169013f677b612baa9ca23cf}\index{Beam::Slider@{Beam::Slider}!getParameter@{getParameter}}
\index{getParameter@{getParameter}!Beam::Slider@{Beam::Slider}}
\doxysubsubsection{\texorpdfstring{getParameter()}{getParameter()}}
{\footnotesize\ttfamily \label{class_beam_1_1_slider_ac38e23e5169013f677b612baa9ca23cf} 
std::\+shared\+\_\+ptr$<$ \doxymbox{\hyperlink{class_beam_1_1_parameter}{Parameter}} $>$ Beam::\+\+Slider::\+get\+Parameter (\begin{DoxyParamCaption}{}{}\end{DoxyParamCaption}) const\hspace{0.3cm}{\ttfamily [inline]}}



Gets the controlled parameter. 

\Hypertarget{class_beam_1_1_slider_a4918a748bf46c078f7e1cd1ab8a62826}\index{Beam::Slider@{Beam::Slider}!getSliderStyle@{getSliderStyle}}
\index{getSliderStyle@{getSliderStyle}!Beam::Slider@{Beam::Slider}}
\doxysubsubsection{\texorpdfstring{getSliderStyle()}{getSliderStyle()}}
{\footnotesize\ttfamily \label{class_beam_1_1_slider_a4918a748bf46c078f7e1cd1ab8a62826} 
\doxymbox{\hyperlink{namespace_beam_a0f94fc63f922f04ab49fd1f60f79ef54}{Slider\+Style}} Beam::\+\+Slider::\+get\+Slider\+Style (\begin{DoxyParamCaption}{}{}\end{DoxyParamCaption}) const\hspace{0.3cm}{\ttfamily [inline]}}



Gets the slider style. 

\Hypertarget{class_beam_1_1_slider_af9238992dbbbe5bd064d2b991b00d0e6}\index{Beam::Slider@{Beam::Slider}!getValue@{getValue}}
\index{getValue@{getValue}!Beam::Slider@{Beam::Slider}}
\doxysubsubsection{\texorpdfstring{getValue()}{getValue()}}
{\footnotesize\ttfamily \label{class_beam_1_1_slider_af9238992dbbbe5bd064d2b991b00d0e6} 
double Beam::\+\+Slider::\+get\+Value (\begin{DoxyParamCaption}{}{}\end{DoxyParamCaption}) const}



Gets the current value. 

\Hypertarget{class_beam_1_1_slider_a4a2744e42c43c6568720baa27ecdffeb}\index{Beam::Slider@{Beam::Slider}!mouseDown@{mouseDown}}
\index{mouseDown@{mouseDown}!Beam::Slider@{Beam::Slider}}
\doxysubsubsection{\texorpdfstring{mouseDown()}{mouseDown()}}
{\footnotesize\ttfamily \label{class_beam_1_1_slider_a4a2744e42c43c6568720baa27ecdffeb} 
void Beam::\+\+Slider::\+mouse\+Down (\begin{DoxyParamCaption}\item[{const \doxymbox{\hyperlink{class_beam_1_1_mouse_event}{Mouse\+Event}} \&}]{event}{}\end{DoxyParamCaption})\hspace{0.3cm}{\ttfamily [override]}, {\ttfamily [virtual]}}



Called when the mouse is pressed. 



Reimplemented from \doxymbox{\hyperlink{class_beam_1_1_gui_component_a3f1bac930389048f3ab5217ece24e032}{Beam::\+\+Gui\+Component}}.

\Hypertarget{class_beam_1_1_slider_adb038e50182312cdb6d1d96e60f6dfea}\index{Beam::Slider@{Beam::Slider}!mouseDrag@{mouseDrag}}
\index{mouseDrag@{mouseDrag}!Beam::Slider@{Beam::Slider}}
\doxysubsubsection{\texorpdfstring{mouseDrag()}{mouseDrag()}}
{\footnotesize\ttfamily \label{class_beam_1_1_slider_adb038e50182312cdb6d1d96e60f6dfea} 
void Beam::\+\+Slider::\+mouse\+Drag (\begin{DoxyParamCaption}\item[{const \doxymbox{\hyperlink{class_beam_1_1_mouse_event}{Mouse\+Event}} \&}]{event}{}\end{DoxyParamCaption})\hspace{0.3cm}{\ttfamily [override]}, {\ttfamily [virtual]}}



Called when the mouse is dragged. 



Reimplemented from \doxymbox{\hyperlink{class_beam_1_1_gui_component_a5f9e28ec7aea30422982fcb748bd54c2}{Beam::\+\+Gui\+Component}}.

\Hypertarget{class_beam_1_1_slider_a454bd346772535c3fa29e7240fc2166e}\index{Beam::Slider@{Beam::Slider}!mouseUp@{mouseUp}}
\index{mouseUp@{mouseUp}!Beam::Slider@{Beam::Slider}}
\doxysubsubsection{\texorpdfstring{mouseUp()}{mouseUp()}}
{\footnotesize\ttfamily \label{class_beam_1_1_slider_a454bd346772535c3fa29e7240fc2166e} 
void Beam::\+\+Slider::\+mouse\+Up (\begin{DoxyParamCaption}\item[{const \doxymbox{\hyperlink{class_beam_1_1_mouse_event}{Mouse\+Event}} \&}]{event}{}\end{DoxyParamCaption})\hspace{0.3cm}{\ttfamily [override]}, {\ttfamily [virtual]}}



Called when the mouse is released. 



Reimplemented from \doxymbox{\hyperlink{class_beam_1_1_gui_component_aeaa0f5b76f80ee669de3a9bf06158a04}{Beam::\+\+Gui\+Component}}.

\Hypertarget{class_beam_1_1_slider_a6be4631112646615e8d4f7f722fea46b}\index{Beam::Slider@{Beam::Slider}!paint@{paint}}
\index{paint@{paint}!Beam::Slider@{Beam::Slider}}
\doxysubsubsection{\texorpdfstring{paint()}{paint()}}
{\footnotesize\ttfamily \label{class_beam_1_1_slider_a6be4631112646615e8d4f7f722fea46b} 
void Beam::\+\+Slider::\+paint (\begin{DoxyParamCaption}\item[{\doxymbox{\hyperlink{class_beam_1_1_quad_batcher}{Quad\+Batcher}} \&}]{g}{}\end{DoxyParamCaption})\hspace{0.3cm}{\ttfamily [override]}, {\ttfamily [virtual]}}



Paints the slider. 



Reimplemented from \doxymbox{\hyperlink{class_beam_1_1_gui_component_afd0b01a0cf776f3e5746622e7a4e7c5c}{Beam::\+\+Gui\+Component}}.

\Hypertarget{class_beam_1_1_slider_a064fa51ef1e0c7d3d5ec9f9b7906ccb0}\index{Beam::Slider@{Beam::Slider}!setParameter@{setParameter}}
\index{setParameter@{setParameter}!Beam::Slider@{Beam::Slider}}
\doxysubsubsection{\texorpdfstring{setParameter()}{setParameter()}}
{\footnotesize\ttfamily \label{class_beam_1_1_slider_a064fa51ef1e0c7d3d5ec9f9b7906ccb0} 
void Beam::\+\+Slider::\+set\+Parameter (\begin{DoxyParamCaption}\item[{std::\+shared\+\_\+ptr$<$ \doxymbox{\hyperlink{class_beam_1_1_parameter}{Parameter}} $>$}]{parameter}{}\end{DoxyParamCaption})}



Sets the parameter to control. 

\Hypertarget{class_beam_1_1_slider_a724ec07492a3f048e8e46808998ad6dd}\index{Beam::Slider@{Beam::Slider}!setRange@{setRange}}
\index{setRange@{setRange}!Beam::Slider@{Beam::Slider}}
\doxysubsubsection{\texorpdfstring{setRange()}{setRange()}}
{\footnotesize\ttfamily \label{class_beam_1_1_slider_a724ec07492a3f048e8e46808998ad6dd} 
void Beam::\+\+Slider::\+set\+Range (\begin{DoxyParamCaption}\item[{double}]{min}{, }\item[{double}]{max}{, }\item[{double}]{interval}{ = {\ttfamily 0.0}}\end{DoxyParamCaption})}



Sets the range of values. 

\Hypertarget{class_beam_1_1_slider_a91ee4b48c9ba471bf5198fac14b5453e}\index{Beam::Slider@{Beam::Slider}!setSliderStyle@{setSliderStyle}}
\index{setSliderStyle@{setSliderStyle}!Beam::Slider@{Beam::Slider}}
\doxysubsubsection{\texorpdfstring{setSliderStyle()}{setSliderStyle()}}
{\footnotesize\ttfamily \label{class_beam_1_1_slider_a91ee4b48c9ba471bf5198fac14b5453e} 
void Beam::\+\+Slider::\+set\+Slider\+Style (\begin{DoxyParamCaption}\item[{\doxymbox{\hyperlink{namespace_beam_a0f94fc63f922f04ab49fd1f60f79ef54}{Slider\+Style}}}]{style}{}\end{DoxyParamCaption})}



Sets the slider style. 

\Hypertarget{class_beam_1_1_slider_af127d4fb0160359c837b7750244922cd}\index{Beam::Slider@{Beam::Slider}!setTextValueSuffix@{setTextValueSuffix}}
\index{setTextValueSuffix@{setTextValueSuffix}!Beam::Slider@{Beam::Slider}}
\doxysubsubsection{\texorpdfstring{setTextValueSuffix()}{setTextValueSuffix()}}
{\footnotesize\ttfamily \label{class_beam_1_1_slider_af127d4fb0160359c837b7750244922cd} 
void Beam::\+\+Slider::\+set\+Text\+Value\+Suffix (\begin{DoxyParamCaption}\item[{const std::\+string \&}]{suffix}{}\end{DoxyParamCaption})}



Sets the text value to display. 

\Hypertarget{class_beam_1_1_slider_ad23c25d65be5a2d93c76e5c8ff14021d}\index{Beam::Slider@{Beam::Slider}!setValue@{setValue}}
\index{setValue@{setValue}!Beam::Slider@{Beam::Slider}}
\doxysubsubsection{\texorpdfstring{setValue()}{setValue()}}
{\footnotesize\ttfamily \label{class_beam_1_1_slider_ad23c25d65be5a2d93c76e5c8ff14021d} 
void Beam::\+\+Slider::\+set\+Value (\begin{DoxyParamCaption}\item[{double}]{new\+Value}{, }\item[{bool}]{notify}{ = {\ttfamily true}}\end{DoxyParamCaption})}



Sets the current value. 



\label{doc-variable-members}
\Hypertarget{class_beam_1_1_slider_doc-variable-members}
\doxysubsection{Member Data Documentation}
\Hypertarget{class_beam_1_1_slider_ae6a11cb61bef6ad7fd3c6496b67efde5}\index{Beam::Slider@{Beam::Slider}!m\_dragStartValue@{m\_dragStartValue}}
\index{m\_dragStartValue@{m\_dragStartValue}!Beam::Slider@{Beam::Slider}}
\doxysubsubsection{\texorpdfstring{m\_dragStartValue}{m\_dragStartValue}}
{\footnotesize\ttfamily \label{class_beam_1_1_slider_ae6a11cb61bef6ad7fd3c6496b67efde5} 
double Beam::\+\+Slider::\+m\+\_\+drag\+Start\+Value\hspace{0.3cm}{\ttfamily [private]}}

\Hypertarget{class_beam_1_1_slider_a5921921547e4f92fe4ae302d6b4f01fb}\index{Beam::Slider@{Beam::Slider}!m\_dragStartX@{m\_dragStartX}}
\index{m\_dragStartX@{m\_dragStartX}!Beam::Slider@{Beam::Slider}}
\doxysubsubsection{\texorpdfstring{m\_dragStartX}{m\_dragStartX}}
{\footnotesize\ttfamily \label{class_beam_1_1_slider_a5921921547e4f92fe4ae302d6b4f01fb} 
float Beam::\+\+Slider::\+m\+\_\+drag\+StartX\hspace{0.3cm}{\ttfamily [private]}}

\Hypertarget{class_beam_1_1_slider_a9b0d97c3eddfc07f075634da191a25ab}\index{Beam::Slider@{Beam::Slider}!m\_dragStartY@{m\_dragStartY}}
\index{m\_dragStartY@{m\_dragStartY}!Beam::Slider@{Beam::Slider}}
\doxysubsubsection{\texorpdfstring{m\_dragStartY}{m\_dragStartY}}
{\footnotesize\ttfamily \label{class_beam_1_1_slider_a9b0d97c3eddfc07f075634da191a25ab} 
float Beam::\+\+Slider::\+m\+\_\+drag\+StartY\hspace{0.3cm}{\ttfamily [private]}}

\Hypertarget{class_beam_1_1_slider_ae53982bc1ad1ea8fb9ceae230f5501b0}\index{Beam::Slider@{Beam::Slider}!m\_interval@{m\_interval}}
\index{m\_interval@{m\_interval}!Beam::Slider@{Beam::Slider}}
\doxysubsubsection{\texorpdfstring{m\_interval}{m\_interval}}
{\footnotesize\ttfamily \label{class_beam_1_1_slider_ae53982bc1ad1ea8fb9ceae230f5501b0} 
double Beam::\+\+Slider::\+m\+\_\+interval = 0.\+0\hspace{0.3cm}{\ttfamily [private]}}

\Hypertarget{class_beam_1_1_slider_a8968f4d20f4c6e44bfb8148f5f8e881a}\index{Beam::Slider@{Beam::Slider}!m\_isDragging@{m\_isDragging}}
\index{m\_isDragging@{m\_isDragging}!Beam::Slider@{Beam::Slider}}
\doxysubsubsection{\texorpdfstring{m\_isDragging}{m\_isDragging}}
{\footnotesize\ttfamily \label{class_beam_1_1_slider_a8968f4d20f4c6e44bfb8148f5f8e881a} 
bool Beam::\+\+Slider::\+m\+\_\+is\+Dragging = false\hspace{0.3cm}{\ttfamily [private]}}

\Hypertarget{class_beam_1_1_slider_a6ef074c99976f2ea263c784fd9a3d8db}\index{Beam::Slider@{Beam::Slider}!m\_max@{m\_max}}
\index{m\_max@{m\_max}!Beam::Slider@{Beam::Slider}}
\doxysubsubsection{\texorpdfstring{m\_max}{m\_max}}
{\footnotesize\ttfamily \label{class_beam_1_1_slider_a6ef074c99976f2ea263c784fd9a3d8db} 
double Beam::\+\+Slider::\+m\+\_\+max = 1.\+0\hspace{0.3cm}{\ttfamily [private]}}

\Hypertarget{class_beam_1_1_slider_a68cfb1023f6aba6e2a750cc2cae4e212}\index{Beam::Slider@{Beam::Slider}!m\_min@{m\_min}}
\index{m\_min@{m\_min}!Beam::Slider@{Beam::Slider}}
\doxysubsubsection{\texorpdfstring{m\_min}{m\_min}}
{\footnotesize\ttfamily \label{class_beam_1_1_slider_a68cfb1023f6aba6e2a750cc2cae4e212} 
double Beam::\+\+Slider::\+m\+\_\+min = 0.\+0\hspace{0.3cm}{\ttfamily [private]}}

\Hypertarget{class_beam_1_1_slider_a69a9c19aabdc09653ec86c26e27de3f2}\index{Beam::Slider@{Beam::Slider}!m\_parameter@{m\_parameter}}
\index{m\_parameter@{m\_parameter}!Beam::Slider@{Beam::Slider}}
\doxysubsubsection{\texorpdfstring{m\_parameter}{m\_parameter}}
{\footnotesize\ttfamily \label{class_beam_1_1_slider_a69a9c19aabdc09653ec86c26e27de3f2} 
std::\+shared\+\_\+ptr$<$\doxymbox{\hyperlink{class_beam_1_1_parameter}{Parameter}}$>$ Beam::\+\+Slider::\+m\+\_\+parameter\hspace{0.3cm}{\ttfamily [private]}}

\Hypertarget{class_beam_1_1_slider_addc7759649fff42b885cef58a37448b2}\index{Beam::Slider@{Beam::Slider}!m\_style@{m\_style}}
\index{m\_style@{m\_style}!Beam::Slider@{Beam::Slider}}
\doxysubsubsection{\texorpdfstring{m\_style}{m\_style}}
{\footnotesize\ttfamily \label{class_beam_1_1_slider_addc7759649fff42b885cef58a37448b2} 
\doxymbox{\hyperlink{namespace_beam_a0f94fc63f922f04ab49fd1f60f79ef54}{Slider\+Style}} Beam::\+\+Slider::\+m\+\_\+style = \doxymbox{\hyperlink{namespace_beam_a0f94fc63f922f04ab49fd1f60f79ef54a27d814af3beb23f25ed8d70819270a9b}{Slider\+Style::\+\+Linear\+Horizontal}}\hspace{0.3cm}{\ttfamily [private]}}

\Hypertarget{class_beam_1_1_slider_a96c9d52ffd0c65b179d0dce80ed8171e}\index{Beam::Slider@{Beam::Slider}!m\_textSuffix@{m\_textSuffix}}
\index{m\_textSuffix@{m\_textSuffix}!Beam::Slider@{Beam::Slider}}
\doxysubsubsection{\texorpdfstring{m\_textSuffix}{m\_textSuffix}}
{\footnotesize\ttfamily \label{class_beam_1_1_slider_a96c9d52ffd0c65b179d0dce80ed8171e} 
std::\+string Beam::\+\+Slider::\+m\+\_\+text\+Suffix\hspace{0.3cm}{\ttfamily [private]}}

\Hypertarget{class_beam_1_1_slider_a1b74d44a5f99d21a967852c984a92fb2}\index{Beam::Slider@{Beam::Slider}!m\_value@{m\_value}}
\index{m\_value@{m\_value}!Beam::Slider@{Beam::Slider}}
\doxysubsubsection{\texorpdfstring{m\_value}{m\_value}}
{\footnotesize\ttfamily \label{class_beam_1_1_slider_a1b74d44a5f99d21a967852c984a92fb2} 
double Beam::\+\+Slider::\+m\+\_\+value = 0.\+0\hspace{0.3cm}{\ttfamily [private]}}



The documentation for this class was generated from the following files:\+\begin{DoxyCompactItemize}
\item 
src/\+interface/\+\doxymbox{\hyperlink{slider_8hpp}{slider.\+hpp}}\item 
src/\+interface/\+\doxymbox{\hyperlink{slider_8cpp}{slider.\+cpp}}\end{DoxyCompactItemize}

\input{class_beam_1_1_steel_plate}
\doxysection{Beam::\+Tape\+Reel Class Reference}
\hypertarget{class_beam_1_1_tape_reel}{}\label{class_beam_1_1_tape_reel}\index{Beam::TapeReel@{Beam::TapeReel}}


{\ttfamily \+\#include $<$tape\+\_\+reel.\+hpp$>$}

Inheritance diagram for Beam::\+Tape\+Reel:\+\begin{figure}[H]
\begin{center}
\leavevmode
\includegraphics[height=3.000000cm]{class_beam_1_1_tape_reel}
\end{center}
\end{figure}
\doxysubsubsection*{Public Member Functions}
\begin{DoxyCompactItemize}
\item 
\doxymbox{\hyperlink{class_beam_1_1_tape_reel_a5d72930147553615cde3a6357ad77f37}{Tape\+Reel}} (std::\+shared\+\_\+ptr$<$ \doxymbox{\hyperlink{class_beam_1_1_flux_track_node}{Flux\+Track\+Node}} $>$ track, size\+\_\+t node\+Id, float x, float y)
\item 
void \doxymbox{\hyperlink{class_beam_1_1_tape_reel_a92ce3fb77345df4a6a2a0aebf04d8bd8}{update}} (float dt) override
\item 
void \doxymbox{\hyperlink{class_beam_1_1_tape_reel_a3c7a82b43aed2a8f48528bf64c2cde8e}{render}} (\doxymbox{\hyperlink{class_beam_1_1_quad_batcher}{Quad\+Batcher}} \&batcher, float dt, float screenW, float screenH) override
\item 
bool \doxymbox{\hyperlink{class_beam_1_1_tape_reel_ad2fe7a4986075d4a3c6047f2e745460e}{on\+Mouse\+Down}} (float x, float y, int button) override
\end{DoxyCompactItemize}
\doxysubsection*{Public Member Functions inherited from \doxymbox{\hyperlink{class_beam_1_1_audio_module}{Beam::\+\+Audio\+Module}}}
\begin{DoxyCompactItemize}
\item 
\doxymbox{\hyperlink{class_beam_1_1_audio_module_a409c75189798146d7f556b1d50f4ba98}{Audio\+Module}} (std::\+shared\+\_\+ptr$<$ \doxymbox{\hyperlink{class_beam_1_1_flux_node}{Flux\+Node}} $>$ node, size\+\_\+t node\+Id, float x, float y)
\item 
size\+\_\+t \doxymbox{\hyperlink{class_beam_1_1_audio_module_a7fc32b3bfadec2badbfd067d2f37da96}{get\+Node\+Id}} () const
\item 
void \doxymbox{\hyperlink{class_beam_1_1_audio_module_a655aa14548b3d4e977294a9bdaafa879}{auto\+Generate\+UI}} ()
\item 
void \doxymbox{\hyperlink{class_beam_1_1_audio_module_a75c0758091a1cec0871134babb541135}{set\+Bounds}} (float x, float y, float w, float h) override
\item 
bool \doxymbox{\hyperlink{class_beam_1_1_audio_module_a4ac204a52c61e603ae6a51daa610bc19}{on\+Mouse\+Up}} (float x, float y, int button) override
\item 
bool \doxymbox{\hyperlink{class_beam_1_1_audio_module_aff90c84092de907409b84eed2c46cbe2}{on\+Mouse\+Move}} (float x, float y) override
\item 
void \doxymbox{\hyperlink{class_beam_1_1_audio_module_a68a3d4bef3787a290f9a313b975f7925}{add\+Child}} (std::\+shared\+\_\+ptr$<$ \doxymbox{\hyperlink{class_beam_1_1_component}{Component}} $>$ child)
\item 
std::\+shared\+\_\+ptr$<$ \doxymbox{\hyperlink{class_beam_1_1_port}{Port}} $>$ \doxymbox{\hyperlink{class_beam_1_1_audio_module_a0d0da8bdbfb3a2355994bba086e0e721}{get\+Input\+Port}} ()
\item 
std::\+shared\+\_\+ptr$<$ \doxymbox{\hyperlink{class_beam_1_1_port}{Port}} $>$ \doxymbox{\hyperlink{class_beam_1_1_audio_module_a75e91aa2e7da1c9a6d31f4c90915403a}{get\+Output\+Port}} ()
\end{DoxyCompactItemize}
\doxysubsection*{Public Member Functions inherited from \doxymbox{\hyperlink{class_beam_1_1_component}{Beam::\+\+Component}}}
\begin{DoxyCompactItemize}
\item 
virtual \doxymbox{\hyperlink{class_beam_1_1_component_af9d734d649978e027412a87bc54362cd}{\texorpdfstring{$\sim$}{\string~}\+Component}} ()=default
\item 
virtual bool \doxymbox{\hyperlink{class_beam_1_1_component_ab92e884903f8a621fcd57bc00a24b041}{on\+Mouse\+Wheel}} (float x, float y, float delta)
\item 
const \doxymbox{\hyperlink{struct_beam_1_1_rect}{Rect}} \& \doxymbox{\hyperlink{class_beam_1_1_component_a5746dbc69d5b0adb4cffbcf920936d00}{get\+Bounds}} () const
\item 
void \doxymbox{\hyperlink{class_beam_1_1_component_a00d4e2dfa7703e59d6486852321dbdf1}{set\+Draggable}} (bool draggable)
\item 
void \doxymbox{\hyperlink{class_beam_1_1_component_aca7b02d1dddf7cd20378db9e3242fb84}{start\+Dragging}} (float x, float y)
\end{DoxyCompactItemize}
\doxysubsubsection*{Private Attributes}
\begin{DoxyCompactItemize}
\item 
std::\+shared\+\_\+ptr$<$ \doxymbox{\hyperlink{class_beam_1_1_flux_track_node}{Flux\+Track\+Node}} $>$ \doxymbox{\hyperlink{class_beam_1_1_tape_reel_ab1b3f8dc62427ee16843f9c65fb687da}{m\+\_\+track\+Node}}
\item 
float \doxymbox{\hyperlink{class_beam_1_1_tape_reel_a9a6d8ed12ec3daa3d6470f74064a2fc5}{m\+\_\+rotation}} = 0.\+0f
\item 
float \doxymbox{\hyperlink{class_beam_1_1_tape_reel_a095f4ba7d1f97b8e89619f6b4bd6e38c}{m\+\_\+scroll\+Timer}} = 0.\+0f
\end{DoxyCompactItemize}
\doxysubsubsection*{Additional Inherited Members}
\doxysubsection*{Public Attributes inherited from \doxymbox{\hyperlink{class_beam_1_1_audio_module}{Beam::\+\+Audio\+Module}}}
\begin{DoxyCompactItemize}
\item 
std::\+function$<$ void(\doxymbox{\hyperlink{class_beam_1_1_audio_module_a409c75189798146d7f556b1d50f4ba98}{Audio\+Module}} \texorpdfstring{$\ast$}{*})$>$ \doxymbox{\hyperlink{class_beam_1_1_audio_module_af95229e9824037c0c9916cc04bf67e90}{on\+Delete\+Requested}}
\end{DoxyCompactItemize}
\doxysubsection*{Protected Attributes inherited from \doxymbox{\hyperlink{class_beam_1_1_audio_module}{Beam::\+\+Audio\+Module}}}
\begin{DoxyCompactItemize}
\item 
std::\+vector$<$ std::\+shared\+\_\+ptr$<$ \doxymbox{\hyperlink{class_beam_1_1_component}{Component}} $>$ $>$ \doxymbox{\hyperlink{class_beam_1_1_audio_module_a1af200892e351f910e6447a50913a560}{m\+\_\+children}}
\end{DoxyCompactItemize}
\doxysubsection*{Protected Attributes inherited from \doxymbox{\hyperlink{class_beam_1_1_component}{Beam::\+\+Component}}}
\begin{DoxyCompactItemize}
\item 
\doxymbox{\hyperlink{struct_beam_1_1_rect}{Rect}} \doxymbox{\hyperlink{class_beam_1_1_component_a4f1ec4a5fb168c39a6c18f958b2b1495}{m\+\_\+bounds}} \{0, 0, 0, 0\}
\item 
bool \doxymbox{\hyperlink{class_beam_1_1_component_adc07913aed6ddadf1c730e7b3bb599cf}{m\+\_\+is\+Visible}} = true
\item 
bool \doxymbox{\hyperlink{class_beam_1_1_component_a0bf77b204ae374a14b5a6d7e5a3c13c6}{m\+\_\+is\+Enabled}} = true
\item 
bool \doxymbox{\hyperlink{class_beam_1_1_component_a9646efcaa9540a26a387f5da9aae4bde}{m\+\_\+is\+Draggable}} = false
\item 
bool \doxymbox{\hyperlink{class_beam_1_1_component_ab03af9a9743acf040f38e3fb11f8dc14}{m\+\_\+is\+Dragging}} = false
\item 
float \doxymbox{\hyperlink{class_beam_1_1_component_a7110b2b9dc235f724bf4689569266a63}{m\+\_\+last\+MouseX}} = 0
\item 
float \doxymbox{\hyperlink{class_beam_1_1_component_a768931a0f51394bf011f821f6ed2efe9}{m\+\_\+last\+MouseY}} = 0
\end{DoxyCompactItemize}


\label{doc-constructors}
\Hypertarget{class_beam_1_1_tape_reel_doc-constructors}
\doxysubsection{Constructor \& Destructor Documentation}
\Hypertarget{class_beam_1_1_tape_reel_a5d72930147553615cde3a6357ad77f37}\index{Beam::TapeReel@{Beam::TapeReel}!TapeReel@{TapeReel}}
\index{TapeReel@{TapeReel}!Beam::TapeReel@{Beam::TapeReel}}
\doxysubsubsection{\texorpdfstring{TapeReel()}{TapeReel()}}
{\footnotesize\ttfamily \label{class_beam_1_1_tape_reel_a5d72930147553615cde3a6357ad77f37} 
Beam::\+\+Tape\+Reel::\+\+Tape\+Reel (\begin{DoxyParamCaption}\item[{std::\+shared\+\_\+ptr$<$ \doxymbox{\hyperlink{class_beam_1_1_flux_track_node}{Flux\+Track\+Node}} $>$}]{track}{, }\item[{size\+\_\+t}]{node\+Id}{, }\item[{float}]{x}{, }\item[{float}]{y}{}\end{DoxyParamCaption})\hspace{0.3cm}{\ttfamily [inline]}}



\label{doc-func-members}
\Hypertarget{class_beam_1_1_tape_reel_doc-func-members}
\doxysubsection{Member Function Documentation}
\Hypertarget{class_beam_1_1_tape_reel_ad2fe7a4986075d4a3c6047f2e745460e}\index{Beam::TapeReel@{Beam::TapeReel}!onMouseDown@{onMouseDown}}
\index{onMouseDown@{onMouseDown}!Beam::TapeReel@{Beam::TapeReel}}
\doxysubsubsection{\texorpdfstring{onMouseDown()}{onMouseDown()}}
{\footnotesize\ttfamily \label{class_beam_1_1_tape_reel_ad2fe7a4986075d4a3c6047f2e745460e} 
bool Beam::\+\+Tape\+Reel::\+on\+Mouse\+Down (\begin{DoxyParamCaption}\item[{float}]{x}{, }\item[{float}]{y}{, }\item[{int}]{button}{}\end{DoxyParamCaption})\hspace{0.3cm}{\ttfamily [inline]}, {\ttfamily [override]}, {\ttfamily [virtual]}}



Reimplemented from \doxymbox{\hyperlink{class_beam_1_1_audio_module_adddaa58e40512d782c2d902917491499}{Beam::\+\+Audio\+Module}}.

\Hypertarget{class_beam_1_1_tape_reel_a3c7a82b43aed2a8f48528bf64c2cde8e}\index{Beam::TapeReel@{Beam::TapeReel}!render@{render}}
\index{render@{render}!Beam::TapeReel@{Beam::TapeReel}}
\doxysubsubsection{\texorpdfstring{render()}{render()}}
{\footnotesize\ttfamily \label{class_beam_1_1_tape_reel_a3c7a82b43aed2a8f48528bf64c2cde8e} 
void Beam::\+\+Tape\+Reel::\+render (\begin{DoxyParamCaption}\item[{\doxymbox{\hyperlink{class_beam_1_1_quad_batcher}{Quad\+Batcher}} \&}]{batcher}{, }\item[{float}]{dt}{, }\item[{float}]{screenW}{, }\item[{float}]{screenH}{}\end{DoxyParamCaption})\hspace{0.3cm}{\ttfamily [inline]}, {\ttfamily [override]}, {\ttfamily [virtual]}}



Reimplemented from \doxymbox{\hyperlink{class_beam_1_1_audio_module_a0e43bace4dcd9eb159ff2237b0c7c3fd}{Beam::\+\+Audio\+Module}}.

\Hypertarget{class_beam_1_1_tape_reel_a92ce3fb77345df4a6a2a0aebf04d8bd8}\index{Beam::TapeReel@{Beam::TapeReel}!update@{update}}
\index{update@{update}!Beam::TapeReel@{Beam::TapeReel}}
\doxysubsubsection{\texorpdfstring{update()}{update()}}
{\footnotesize\ttfamily \label{class_beam_1_1_tape_reel_a92ce3fb77345df4a6a2a0aebf04d8bd8} 
void Beam::\+\+Tape\+Reel::\+update (\begin{DoxyParamCaption}\item[{float}]{dt}{}\end{DoxyParamCaption})\hspace{0.3cm}{\ttfamily [inline]}, {\ttfamily [override]}, {\ttfamily [virtual]}}



Reimplemented from \doxymbox{\hyperlink{class_beam_1_1_component_ad3d3fb19d25b4371d07620567970a158}{Beam::\+\+Component}}.



\label{doc-variable-members}
\Hypertarget{class_beam_1_1_tape_reel_doc-variable-members}
\doxysubsection{Member Data Documentation}
\Hypertarget{class_beam_1_1_tape_reel_a9a6d8ed12ec3daa3d6470f74064a2fc5}\index{Beam::TapeReel@{Beam::TapeReel}!m\_rotation@{m\_rotation}}
\index{m\_rotation@{m\_rotation}!Beam::TapeReel@{Beam::TapeReel}}
\doxysubsubsection{\texorpdfstring{m\_rotation}{m\_rotation}}
{\footnotesize\ttfamily \label{class_beam_1_1_tape_reel_a9a6d8ed12ec3daa3d6470f74064a2fc5} 
float Beam::\+\+Tape\+Reel::\+m\+\_\+rotation = 0.\+0f\hspace{0.3cm}{\ttfamily [private]}}

\Hypertarget{class_beam_1_1_tape_reel_a095f4ba7d1f97b8e89619f6b4bd6e38c}\index{Beam::TapeReel@{Beam::TapeReel}!m\_scrollTimer@{m\_scrollTimer}}
\index{m\_scrollTimer@{m\_scrollTimer}!Beam::TapeReel@{Beam::TapeReel}}
\doxysubsubsection{\texorpdfstring{m\_scrollTimer}{m\_scrollTimer}}
{\footnotesize\ttfamily \label{class_beam_1_1_tape_reel_a095f4ba7d1f97b8e89619f6b4bd6e38c} 
float Beam::\+\+Tape\+Reel::\+m\+\_\+scroll\+Timer = 0.\+0f\hspace{0.3cm}{\ttfamily [private]}}

\Hypertarget{class_beam_1_1_tape_reel_ab1b3f8dc62427ee16843f9c65fb687da}\index{Beam::TapeReel@{Beam::TapeReel}!m\_trackNode@{m\_trackNode}}
\index{m\_trackNode@{m\_trackNode}!Beam::TapeReel@{Beam::TapeReel}}
\doxysubsubsection{\texorpdfstring{m\_trackNode}{m\_trackNode}}
{\footnotesize\ttfamily \label{class_beam_1_1_tape_reel_ab1b3f8dc62427ee16843f9c65fb687da} 
std::\+shared\+\_\+ptr$<$\doxymbox{\hyperlink{class_beam_1_1_flux_track_node}{Flux\+Track\+Node}}$>$ Beam::\+\+Tape\+Reel::\+m\+\_\+track\+Node\hspace{0.3cm}{\ttfamily [private]}}



The documentation for this class was generated from the following file:\+\begin{DoxyCompactItemize}
\item 
src/\+interface/\+\doxymbox{\hyperlink{tape__reel_8hpp}{tape\+\_\+reel.\+hpp}}\end{DoxyCompactItemize}

\doxysection{Beam::\+Texture Class Reference}
\hypertarget{class_beam_1_1_texture}{}\label{class_beam_1_1_texture}\index{Beam::Texture@{Beam::Texture}}


Represents an Open\+GL texture resource.  




{\ttfamily \+\#include $<$texture.\+hpp$>$}

\doxysubsubsection*{Public Member Functions}
\begin{DoxyCompactItemize}
\item 
\doxymbox{\hyperlink{class_beam_1_1_texture_ad993180e8e13efd7c5f7696ef5a47ac1}{Texture}} (const std::\+string \&path)
\begin{DoxyCompactList}\small\item\em Constructs a texture from a file path. \end{DoxyCompactList}\item 
\doxymbox{\hyperlink{class_beam_1_1_texture_a5ff8292b7ea1c8756d3ff56c46480084}{\texorpdfstring{$\sim$}{\string~}\+Texture}} ()
\item 
bool \doxymbox{\hyperlink{class_beam_1_1_texture_a8cc952b3897c329f1978addd6a825463}{load}} (const std::\+string \&path)
\begin{DoxyCompactList}\small\item\em Loads a texture from disk. \end{DoxyCompactList}\item 
void \doxymbox{\hyperlink{class_beam_1_1_texture_a21e8c9ca7639d6182fe4a0de25e0f8b6}{create\+RGB}} (int \doxymbox{\hyperlink{texture_8cpp_a8710f3c5c66c09e158c8619b3fca614a}{width}}, int \doxymbox{\hyperlink{texture_8cpp_a1055637f17e35a0ca82b396bb94914e5}{height}}, const unsigned char \texorpdfstring{$\ast$}{*}data)
\begin{DoxyCompactList}\small\item\em Creates a texture from raw RGB data in memory. \end{DoxyCompactList}\item 
void \doxymbox{\hyperlink{class_beam_1_1_texture_a8ecd0759254402be3bddeda2d6290974}{bind}} (unsigned int slot=0) const
\begin{DoxyCompactList}\small\item\em Binds the texture to a specific Open\+GL slot. \end{DoxyCompactList}\item 
void \doxymbox{\hyperlink{class_beam_1_1_texture_a6a6a9bd3f6bd747db6666909407cf4c9}{unbind}} () const
\item 
int \doxymbox{\hyperlink{class_beam_1_1_texture_a3925eac509482480ec2ae37def3bc219}{get\+Width}} () const
\item 
int \doxymbox{\hyperlink{class_beam_1_1_texture_ae400bfa948ad596e6b5f80fa0ab53867}{get\+Height}} () const
\item 
unsigned int \doxymbox{\hyperlink{class_beam_1_1_texture_ac5927d13c8e8ca703e876052061e1c2b}{get\+ID}} () const
\end{DoxyCompactItemize}
\doxysubsubsection*{Private Attributes}
\begin{DoxyCompactItemize}
\item 
unsigned int \doxymbox{\hyperlink{class_beam_1_1_texture_a70236a70e165be0798248c871c8ebc96}{m\+\_\+id}} = 0
\item 
int \doxymbox{\hyperlink{class_beam_1_1_texture_a4492043ee1b7012dd9455df590aa069c}{m\+\_\+width}} = 0
\item 
int \doxymbox{\hyperlink{class_beam_1_1_texture_a4d1c0e68aab6cc04e5764dc34ca0aaed}{m\+\_\+height}} = 0
\item 
std::\+string \doxymbox{\hyperlink{class_beam_1_1_texture_a0177bed5f5a00dfe6e960dbb2116b42b}{m\+\_\+path}}
\end{DoxyCompactItemize}


\doxysubsection{Detailed Description}
Represents an Open\+GL texture resource. 

\label{doc-constructors}
\Hypertarget{class_beam_1_1_texture_doc-constructors}
\doxysubsection{Constructor \& Destructor Documentation}
\Hypertarget{class_beam_1_1_texture_ad993180e8e13efd7c5f7696ef5a47ac1}\index{Beam::Texture@{Beam::Texture}!Texture@{Texture}}
\index{Texture@{Texture}!Beam::Texture@{Beam::Texture}}
\doxysubsubsection{\texorpdfstring{Texture()}{Texture()}}
{\footnotesize\ttfamily \label{class_beam_1_1_texture_ad993180e8e13efd7c5f7696ef5a47ac1} 
Beam::\+\+Texture::\+\+Texture (\begin{DoxyParamCaption}\item[{const std::\+string \&}]{path}{}\end{DoxyParamCaption})}



Constructs a texture from a file path. 


\begin{DoxyParams}{Parameters}
{\em path} & Path to the image file (BMP supported). \\
\hline
\end{DoxyParams}
\Hypertarget{class_beam_1_1_texture_a5ff8292b7ea1c8756d3ff56c46480084}\index{Beam::Texture@{Beam::Texture}!````~Texture@{\texorpdfstring{$\sim$}{\string~}Texture}}
\index{````~Texture@{\texorpdfstring{$\sim$}{\string~}Texture}!Beam::Texture@{Beam::Texture}}
\doxysubsubsection{\texorpdfstring{\texorpdfstring{$\sim$}{\string~}Texture()}{\string~Texture()}}
{\footnotesize\ttfamily \label{class_beam_1_1_texture_a5ff8292b7ea1c8756d3ff56c46480084} 
Beam::\+\+Texture::\+\texorpdfstring{$\sim$}{\string~}\+Texture (\begin{DoxyParamCaption}{}{}\end{DoxyParamCaption})}



\label{doc-func-members}
\Hypertarget{class_beam_1_1_texture_doc-func-members}
\doxysubsection{Member Function Documentation}
\Hypertarget{class_beam_1_1_texture_a8ecd0759254402be3bddeda2d6290974}\index{Beam::Texture@{Beam::Texture}!bind@{bind}}
\index{bind@{bind}!Beam::Texture@{Beam::Texture}}
\doxysubsubsection{\texorpdfstring{bind()}{bind()}}
{\footnotesize\ttfamily \label{class_beam_1_1_texture_a8ecd0759254402be3bddeda2d6290974} 
void Beam::\+\+Texture::\+bind (\begin{DoxyParamCaption}\item[{unsigned int}]{slot}{ = {\ttfamily 0}}\end{DoxyParamCaption}) const}



Binds the texture to a specific Open\+GL slot. 


\begin{DoxyParams}{Parameters}
{\em slot} & \doxylink{class_beam_1_1_texture}{Texture} unit (0-\/15). \\
\hline
\end{DoxyParams}
\Hypertarget{class_beam_1_1_texture_a21e8c9ca7639d6182fe4a0de25e0f8b6}\index{Beam::Texture@{Beam::Texture}!createRGB@{createRGB}}
\index{createRGB@{createRGB}!Beam::Texture@{Beam::Texture}}
\doxysubsubsection{\texorpdfstring{createRGB()}{createRGB()}}
{\footnotesize\ttfamily \label{class_beam_1_1_texture_a21e8c9ca7639d6182fe4a0de25e0f8b6} 
void Beam::\+\+Texture::\+create\+RGB (\begin{DoxyParamCaption}\item[{int}]{width}{, }\item[{int}]{height}{, }\item[{const unsigned char \texorpdfstring{$\ast$}{*}}]{data}{}\end{DoxyParamCaption})}



Creates a texture from raw RGB data in memory. 


\begin{DoxyParams}{Parameters}
{\em \doxylink{texture_8cpp_a8710f3c5c66c09e158c8619b3fca614a}{width}} & Width in pixels. \\
\hline
{\em \doxylink{texture_8cpp_a1055637f17e35a0ca82b396bb94914e5}{height}} & Height in pixels. \\
\hline
{\em data} & Raw byte array. \\
\hline
\end{DoxyParams}
\Hypertarget{class_beam_1_1_texture_ae400bfa948ad596e6b5f80fa0ab53867}\index{Beam::Texture@{Beam::Texture}!getHeight@{getHeight}}
\index{getHeight@{getHeight}!Beam::Texture@{Beam::Texture}}
\doxysubsubsection{\texorpdfstring{getHeight()}{getHeight()}}
{\footnotesize\ttfamily \label{class_beam_1_1_texture_ae400bfa948ad596e6b5f80fa0ab53867} 
int Beam::\+\+Texture::\+get\+Height (\begin{DoxyParamCaption}{}{}\end{DoxyParamCaption}) const\hspace{0.3cm}{\ttfamily [inline]}}

\Hypertarget{class_beam_1_1_texture_ac5927d13c8e8ca703e876052061e1c2b}\index{Beam::Texture@{Beam::Texture}!getID@{getID}}
\index{getID@{getID}!Beam::Texture@{Beam::Texture}}
\doxysubsubsection{\texorpdfstring{getID()}{getID()}}
{\footnotesize\ttfamily \label{class_beam_1_1_texture_ac5927d13c8e8ca703e876052061e1c2b} 
unsigned int Beam::\+\+Texture::\+get\+ID (\begin{DoxyParamCaption}{}{}\end{DoxyParamCaption}) const\hspace{0.3cm}{\ttfamily [inline]}}

\Hypertarget{class_beam_1_1_texture_a3925eac509482480ec2ae37def3bc219}\index{Beam::Texture@{Beam::Texture}!getWidth@{getWidth}}
\index{getWidth@{getWidth}!Beam::Texture@{Beam::Texture}}
\doxysubsubsection{\texorpdfstring{getWidth()}{getWidth()}}
{\footnotesize\ttfamily \label{class_beam_1_1_texture_a3925eac509482480ec2ae37def3bc219} 
int Beam::\+\+Texture::\+get\+Width (\begin{DoxyParamCaption}{}{}\end{DoxyParamCaption}) const\hspace{0.3cm}{\ttfamily [inline]}}

\Hypertarget{class_beam_1_1_texture_a8cc952b3897c329f1978addd6a825463}\index{Beam::Texture@{Beam::Texture}!load@{load}}
\index{load@{load}!Beam::Texture@{Beam::Texture}}
\doxysubsubsection{\texorpdfstring{load()}{load()}}
{\footnotesize\ttfamily \label{class_beam_1_1_texture_a8cc952b3897c329f1978addd6a825463} 
bool Beam::\+\+Texture::\+load (\begin{DoxyParamCaption}\item[{const std::\+string \&}]{path}{}\end{DoxyParamCaption})}



Loads a texture from disk. 


\begin{DoxyParams}{Parameters}
{\em path} & Path to the BMP file. \\
\hline
\end{DoxyParams}
\begin{DoxyReturn}{Returns}
true if successful. 
\end{DoxyReturn}
\Hypertarget{class_beam_1_1_texture_a6a6a9bd3f6bd747db6666909407cf4c9}\index{Beam::Texture@{Beam::Texture}!unbind@{unbind}}
\index{unbind@{unbind}!Beam::Texture@{Beam::Texture}}
\doxysubsubsection{\texorpdfstring{unbind()}{unbind()}}
{\footnotesize\ttfamily \label{class_beam_1_1_texture_a6a6a9bd3f6bd747db6666909407cf4c9} 
void Beam::\+\+Texture::\+unbind (\begin{DoxyParamCaption}{}{}\end{DoxyParamCaption}) const}



\label{doc-variable-members}
\Hypertarget{class_beam_1_1_texture_doc-variable-members}
\doxysubsection{Member Data Documentation}
\Hypertarget{class_beam_1_1_texture_a4d1c0e68aab6cc04e5764dc34ca0aaed}\index{Beam::Texture@{Beam::Texture}!m\_height@{m\_height}}
\index{m\_height@{m\_height}!Beam::Texture@{Beam::Texture}}
\doxysubsubsection{\texorpdfstring{m\_height}{m\_height}}
{\footnotesize\ttfamily \label{class_beam_1_1_texture_a4d1c0e68aab6cc04e5764dc34ca0aaed} 
int Beam::\+\+Texture::\+m\+\_\+height = 0\hspace{0.3cm}{\ttfamily [private]}}

\Hypertarget{class_beam_1_1_texture_a70236a70e165be0798248c871c8ebc96}\index{Beam::Texture@{Beam::Texture}!m\_id@{m\_id}}
\index{m\_id@{m\_id}!Beam::Texture@{Beam::Texture}}
\doxysubsubsection{\texorpdfstring{m\_id}{m\_id}}
{\footnotesize\ttfamily \label{class_beam_1_1_texture_a70236a70e165be0798248c871c8ebc96} 
unsigned int Beam::\+\+Texture::\+m\+\_\+id = 0\hspace{0.3cm}{\ttfamily [private]}}

\Hypertarget{class_beam_1_1_texture_a0177bed5f5a00dfe6e960dbb2116b42b}\index{Beam::Texture@{Beam::Texture}!m\_path@{m\_path}}
\index{m\_path@{m\_path}!Beam::Texture@{Beam::Texture}}
\doxysubsubsection{\texorpdfstring{m\_path}{m\_path}}
{\footnotesize\ttfamily \label{class_beam_1_1_texture_a0177bed5f5a00dfe6e960dbb2116b42b} 
std::\+string Beam::\+\+Texture::\+m\+\_\+path\hspace{0.3cm}{\ttfamily [private]}}

\Hypertarget{class_beam_1_1_texture_a4492043ee1b7012dd9455df590aa069c}\index{Beam::Texture@{Beam::Texture}!m\_width@{m\_width}}
\index{m\_width@{m\_width}!Beam::Texture@{Beam::Texture}}
\doxysubsubsection{\texorpdfstring{m\_width}{m\_width}}
{\footnotesize\ttfamily \label{class_beam_1_1_texture_a4492043ee1b7012dd9455df590aa069c} 
int Beam::\+\+Texture::\+m\+\_\+width = 0\hspace{0.3cm}{\ttfamily [private]}}



The documentation for this class was generated from the following files:\+\begin{DoxyCompactItemize}
\item 
src/\+render/\+\doxymbox{\hyperlink{texture_8hpp}{texture.\+hpp}}\item 
src/\+render/\+\doxymbox{\hyperlink{texture_8cpp}{texture.\+cpp}}\end{DoxyCompactItemize}

\doxysection{Beam::\+Timeline Class Reference}
\hypertarget{class_beam_1_1_timeline}{}\label{class_beam_1_1_timeline}\index{Beam::Timeline@{Beam::Timeline}}


{\ttfamily \+\#include $<$timeline.\+hpp$>$}

Inheritance diagram for Beam::\+Timeline:\+\begin{figure}[H]
\begin{center}
\leavevmode
\includegraphics[height=2.000000cm]{class_beam_1_1_timeline}
\end{center}
\end{figure}
\doxysubsubsection*{Public Member Functions}
\begin{DoxyCompactItemize}
\item 
\doxymbox{\hyperlink{class_beam_1_1_timeline_ab24201fb4781455aa0496eea0f173aa6}{Timeline}} ()
\item 
void \doxymbox{\hyperlink{class_beam_1_1_timeline_a10a7b70340a61dcfb1ebead1a02d6b5a}{render}} (\doxymbox{\hyperlink{class_beam_1_1_quad_batcher}{Quad\+Batcher}} \&batcher) override
\item 
void \doxymbox{\hyperlink{class_beam_1_1_timeline_ad21aa847e01f197a5b6eecdc4cb7f2f4}{set\+Visible}} (bool visible)
\end{DoxyCompactItemize}
\doxysubsection*{Public Member Functions inherited from \doxymbox{\hyperlink{class_beam_1_1_component}{Beam::\+\+Component}}}
\begin{DoxyCompactItemize}
\item 
virtual \doxymbox{\hyperlink{class_beam_1_1_component_af9d734d649978e027412a87bc54362cd}{\texorpdfstring{$\sim$}{\string~}\+Component}} ()=default
\item 
virtual void \doxymbox{\hyperlink{class_beam_1_1_component_ad3d3fb19d25b4371d07620567970a158}{update}} (float dt)
\item 
virtual bool \doxymbox{\hyperlink{class_beam_1_1_component_aec1da33d2d6e3d4e7dd6708309264e76}{on\+Mouse\+Down}} (float x, float y, int button)
\item 
virtual bool \doxymbox{\hyperlink{class_beam_1_1_component_ae36b8e9d70e8f9a1b9ba81c23c54d5c8}{on\+Mouse\+Up}} (float x, float y, int button)
\item 
virtual bool \doxymbox{\hyperlink{class_beam_1_1_component_a9d8e5970783d315044277a1228659e6c}{on\+Mouse\+Move}} (float x, float y)
\item 
virtual void \doxymbox{\hyperlink{class_beam_1_1_component_a6865b1f22388af467bf6c789120fac05}{set\+Bounds}} (float x, float y, float w, float h)
\item 
const \doxymbox{\hyperlink{struct_beam_1_1_rect}{Rect}} \& \doxymbox{\hyperlink{class_beam_1_1_component_a5746dbc69d5b0adb4cffbcf920936d00}{get\+Bounds}} () const
\item 
void \doxymbox{\hyperlink{class_beam_1_1_component_a00d4e2dfa7703e59d6486852321dbdf1}{set\+Draggable}} (bool draggable)
\item 
void \doxymbox{\hyperlink{class_beam_1_1_component_aca7b02d1dddf7cd20378db9e3242fb84}{start\+Dragging}} (float x, float y)
\end{DoxyCompactItemize}
\doxysubsubsection*{Private Attributes}
\begin{DoxyCompactItemize}
\item 
bool \doxymbox{\hyperlink{class_beam_1_1_timeline_a0021e8c656305015e2ad30a791338c09}{m\+\_\+is\+Visible}} = false
\end{DoxyCompactItemize}
\doxysubsubsection*{Additional Inherited Members}
\doxysubsection*{Protected Attributes inherited from \doxymbox{\hyperlink{class_beam_1_1_component}{Beam::\+\+Component}}}
\begin{DoxyCompactItemize}
\item 
\doxymbox{\hyperlink{struct_beam_1_1_rect}{Rect}} \doxymbox{\hyperlink{class_beam_1_1_component_a4f1ec4a5fb168c39a6c18f958b2b1495}{m\+\_\+bounds}} \{0, 0, 0, 0\}
\item 
bool \doxymbox{\hyperlink{class_beam_1_1_component_adc07913aed6ddadf1c730e7b3bb599cf}{m\+\_\+is\+Visible}} = true
\item 
bool \doxymbox{\hyperlink{class_beam_1_1_component_a0bf77b204ae374a14b5a6d7e5a3c13c6}{m\+\_\+is\+Enabled}} = true
\item 
bool \doxymbox{\hyperlink{class_beam_1_1_component_a9646efcaa9540a26a387f5da9aae4bde}{m\+\_\+is\+Draggable}} = false
\item 
bool \doxymbox{\hyperlink{class_beam_1_1_component_ab03af9a9743acf040f38e3fb11f8dc14}{m\+\_\+is\+Dragging}} = false
\item 
float \doxymbox{\hyperlink{class_beam_1_1_component_a7110b2b9dc235f724bf4689569266a63}{m\+\_\+last\+MouseX}} = 0
\item 
float \doxymbox{\hyperlink{class_beam_1_1_component_a768931a0f51394bf011f821f6ed2efe9}{m\+\_\+last\+MouseY}} = 0
\end{DoxyCompactItemize}


\label{doc-constructors}
\Hypertarget{class_beam_1_1_timeline_doc-constructors}
\doxysubsection{Constructor \& Destructor Documentation}
\Hypertarget{class_beam_1_1_timeline_ab24201fb4781455aa0496eea0f173aa6}\index{Beam::Timeline@{Beam::Timeline}!Timeline@{Timeline}}
\index{Timeline@{Timeline}!Beam::Timeline@{Beam::Timeline}}
\doxysubsubsection{\texorpdfstring{Timeline()}{Timeline()}}
{\footnotesize\ttfamily \label{class_beam_1_1_timeline_ab24201fb4781455aa0496eea0f173aa6} 
Beam::\+\+Timeline::\+\+Timeline (\begin{DoxyParamCaption}{}{}\end{DoxyParamCaption})\hspace{0.3cm}{\ttfamily [inline]}}



\label{doc-func-members}
\Hypertarget{class_beam_1_1_timeline_doc-func-members}
\doxysubsection{Member Function Documentation}
\Hypertarget{class_beam_1_1_timeline_a10a7b70340a61dcfb1ebead1a02d6b5a}\index{Beam::Timeline@{Beam::Timeline}!render@{render}}
\index{render@{render}!Beam::Timeline@{Beam::Timeline}}
\doxysubsubsection{\texorpdfstring{render()}{render()}}
{\footnotesize\ttfamily \label{class_beam_1_1_timeline_a10a7b70340a61dcfb1ebead1a02d6b5a} 
void Beam::\+\+Timeline::\+render (\begin{DoxyParamCaption}\item[{\doxymbox{\hyperlink{class_beam_1_1_quad_batcher}{Quad\+Batcher}} \&}]{batcher}{}\end{DoxyParamCaption})\hspace{0.3cm}{\ttfamily [inline]}, {\ttfamily [override]}, {\ttfamily [virtual]}}



Implements \doxymbox{\hyperlink{class_beam_1_1_component_a2ed6b7841a25bd992bd46b822311ef1d}{Beam::\+\+Component}}.

\Hypertarget{class_beam_1_1_timeline_ad21aa847e01f197a5b6eecdc4cb7f2f4}\index{Beam::Timeline@{Beam::Timeline}!setVisible@{setVisible}}
\index{setVisible@{setVisible}!Beam::Timeline@{Beam::Timeline}}
\doxysubsubsection{\texorpdfstring{setVisible()}{setVisible()}}
{\footnotesize\ttfamily \label{class_beam_1_1_timeline_ad21aa847e01f197a5b6eecdc4cb7f2f4} 
void Beam::\+\+Timeline::\+set\+Visible (\begin{DoxyParamCaption}\item[{bool}]{visible}{}\end{DoxyParamCaption})\hspace{0.3cm}{\ttfamily [inline]}}



\label{doc-variable-members}
\Hypertarget{class_beam_1_1_timeline_doc-variable-members}
\doxysubsection{Member Data Documentation}
\Hypertarget{class_beam_1_1_timeline_a0021e8c656305015e2ad30a791338c09}\index{Beam::Timeline@{Beam::Timeline}!m\_isVisible@{m\_isVisible}}
\index{m\_isVisible@{m\_isVisible}!Beam::Timeline@{Beam::Timeline}}
\doxysubsubsection{\texorpdfstring{m\_isVisible}{m\_isVisible}}
{\footnotesize\ttfamily \label{class_beam_1_1_timeline_a0021e8c656305015e2ad30a791338c09} 
bool Beam::\+\+Timeline::\+m\+\_\+is\+Visible = false\hspace{0.3cm}{\ttfamily [private]}}



The documentation for this class was generated from the following file:\+\begin{DoxyCompactItemize}
\item 
src/\+ui/\+\doxymbox{\hyperlink{timeline_8hpp}{timeline.\+hpp}}\end{DoxyCompactItemize}

\doxysection{Beam::\+Top\+Bar Class Reference}
\hypertarget{class_beam_1_1_top_bar}{}\label{class_beam_1_1_top_bar}\index{Beam::TopBar@{Beam::TopBar}}


{\ttfamily \+\#include $<$top\+\_\+bar.\+hpp$>$}

Inheritance diagram for Beam::\+Top\+Bar:\+\begin{figure}[H]
\begin{center}
\leavevmode
\includegraphics[height=2.000000cm]{class_beam_1_1_top_bar}
\end{center}
\end{figure}
\doxysubsubsection*{Public Member Functions}
\begin{DoxyCompactItemize}
\item 
\doxymbox{\hyperlink{class_beam_1_1_top_bar_a264f67bfc7519f1538c6728d8cd1e11e}{Top\+Bar}} (int width)
\item 
void \doxymbox{\hyperlink{class_beam_1_1_top_bar_afea870bc1388aee09948c8d5b0dd014f}{render}} (\doxymbox{\hyperlink{class_beam_1_1_quad_batcher}{Quad\+Batcher}} \&batcher) override
\item 
bool \doxymbox{\hyperlink{class_beam_1_1_top_bar_a00fbc22478eb3256b0e11a75cf78cec9}{on\+Mouse\+Down}} (float x, float y, int button) override
\item 
void \doxymbox{\hyperlink{class_beam_1_1_top_bar_a6d2bf867d1ee7d413b377cda942beb58}{set\+Playing}} (bool playing)
\end{DoxyCompactItemize}
\doxysubsection*{Public Member Functions inherited from \doxymbox{\hyperlink{class_beam_1_1_component}{Beam::\+\+Component}}}
\begin{DoxyCompactItemize}
\item 
virtual \doxymbox{\hyperlink{class_beam_1_1_component_af9d734d649978e027412a87bc54362cd}{\texorpdfstring{$\sim$}{\string~}\+Component}} ()=default
\item 
virtual void \doxymbox{\hyperlink{class_beam_1_1_component_ad3d3fb19d25b4371d07620567970a158}{update}} (float dt)
\item 
virtual bool \doxymbox{\hyperlink{class_beam_1_1_component_ae36b8e9d70e8f9a1b9ba81c23c54d5c8}{on\+Mouse\+Up}} (float x, float y, int button)
\item 
virtual bool \doxymbox{\hyperlink{class_beam_1_1_component_a9d8e5970783d315044277a1228659e6c}{on\+Mouse\+Move}} (float x, float y)
\item 
virtual void \doxymbox{\hyperlink{class_beam_1_1_component_a6865b1f22388af467bf6c789120fac05}{set\+Bounds}} (float x, float y, float w, float h)
\item 
const \doxymbox{\hyperlink{struct_beam_1_1_rect}{Rect}} \& \doxymbox{\hyperlink{class_beam_1_1_component_a5746dbc69d5b0adb4cffbcf920936d00}{get\+Bounds}} () const
\item 
void \doxymbox{\hyperlink{class_beam_1_1_component_a00d4e2dfa7703e59d6486852321dbdf1}{set\+Draggable}} (bool draggable)
\item 
void \doxymbox{\hyperlink{class_beam_1_1_component_aca7b02d1dddf7cd20378db9e3242fb84}{start\+Dragging}} (float x, float y)
\end{DoxyCompactItemize}
\doxysubsubsection*{Public Attributes}
\begin{DoxyCompactItemize}
\item 
std::\+function$<$ void(int)$>$ \doxymbox{\hyperlink{class_beam_1_1_top_bar_a49378791fe51f2dff50e1124024ca73d}{on\+Mode\+Changed}}
\item 
std::\+function$<$ void()$>$ \doxymbox{\hyperlink{class_beam_1_1_top_bar_ae6da30fe543f4a6a0f54cf2ef1663238}{on\+Save\+Requested}}
\item 
std::\+function$<$ void()$>$ \doxymbox{\hyperlink{class_beam_1_1_top_bar_aaa84af8dc8b326e0460006cf8364e41c}{on\+Load\+Requested}}
\item 
std::\+function$<$ void()$>$ \doxymbox{\hyperlink{class_beam_1_1_top_bar_a83dc6b9a4e8c9b9328bb0266ee2df6f6}{on\+Play\+Requested}}
\item 
std::\+function$<$ void()$>$ \doxymbox{\hyperlink{class_beam_1_1_top_bar_a6b7d5c0b5d176965a0c1b99fb4b61b86}{on\+Pause\+Requested}}
\item 
std::\+function$<$ void()$>$ \doxymbox{\hyperlink{class_beam_1_1_top_bar_a21063e55ca6ba9c9f7c9391750f26571}{on\+Rewind\+Requested}}
\end{DoxyCompactItemize}
\doxysubsubsection*{Private Attributes}
\begin{DoxyCompactItemize}
\item 
bool \doxymbox{\hyperlink{class_beam_1_1_top_bar_aacfd7a81b28a1cdcf7d9cfef3278a937}{m\+\_\+is\+Playing}} = false
\end{DoxyCompactItemize}
\doxysubsubsection*{Additional Inherited Members}
\doxysubsection*{Protected Attributes inherited from \doxymbox{\hyperlink{class_beam_1_1_component}{Beam::\+\+Component}}}
\begin{DoxyCompactItemize}
\item 
\doxymbox{\hyperlink{struct_beam_1_1_rect}{Rect}} \doxymbox{\hyperlink{class_beam_1_1_component_a4f1ec4a5fb168c39a6c18f958b2b1495}{m\+\_\+bounds}} \{0, 0, 0, 0\}
\item 
bool \doxymbox{\hyperlink{class_beam_1_1_component_adc07913aed6ddadf1c730e7b3bb599cf}{m\+\_\+is\+Visible}} = true
\item 
bool \doxymbox{\hyperlink{class_beam_1_1_component_a0bf77b204ae374a14b5a6d7e5a3c13c6}{m\+\_\+is\+Enabled}} = true
\item 
bool \doxymbox{\hyperlink{class_beam_1_1_component_a9646efcaa9540a26a387f5da9aae4bde}{m\+\_\+is\+Draggable}} = false
\item 
bool \doxymbox{\hyperlink{class_beam_1_1_component_ab03af9a9743acf040f38e3fb11f8dc14}{m\+\_\+is\+Dragging}} = false
\item 
float \doxymbox{\hyperlink{class_beam_1_1_component_a7110b2b9dc235f724bf4689569266a63}{m\+\_\+last\+MouseX}} = 0
\item 
float \doxymbox{\hyperlink{class_beam_1_1_component_a768931a0f51394bf011f821f6ed2efe9}{m\+\_\+last\+MouseY}} = 0
\end{DoxyCompactItemize}


\label{doc-constructors}
\Hypertarget{class_beam_1_1_top_bar_doc-constructors}
\doxysubsection{Constructor \& Destructor Documentation}
\Hypertarget{class_beam_1_1_top_bar_a264f67bfc7519f1538c6728d8cd1e11e}\index{Beam::TopBar@{Beam::TopBar}!TopBar@{TopBar}}
\index{TopBar@{TopBar}!Beam::TopBar@{Beam::TopBar}}
\doxysubsubsection{\texorpdfstring{TopBar()}{TopBar()}}
{\footnotesize\ttfamily \label{class_beam_1_1_top_bar_a264f67bfc7519f1538c6728d8cd1e11e} 
Beam::\+\+Top\+Bar::\+\+Top\+Bar (\begin{DoxyParamCaption}\item[{int}]{width}{}\end{DoxyParamCaption})\hspace{0.3cm}{\ttfamily [inline]}}



\label{doc-func-members}
\Hypertarget{class_beam_1_1_top_bar_doc-func-members}
\doxysubsection{Member Function Documentation}
\Hypertarget{class_beam_1_1_top_bar_a00fbc22478eb3256b0e11a75cf78cec9}\index{Beam::TopBar@{Beam::TopBar}!onMouseDown@{onMouseDown}}
\index{onMouseDown@{onMouseDown}!Beam::TopBar@{Beam::TopBar}}
\doxysubsubsection{\texorpdfstring{onMouseDown()}{onMouseDown()}}
{\footnotesize\ttfamily \label{class_beam_1_1_top_bar_a00fbc22478eb3256b0e11a75cf78cec9} 
bool Beam::\+\+Top\+Bar::\+on\+Mouse\+Down (\begin{DoxyParamCaption}\item[{float}]{x}{, }\item[{float}]{y}{, }\item[{int}]{button}{}\end{DoxyParamCaption})\hspace{0.3cm}{\ttfamily [inline]}, {\ttfamily [override]}, {\ttfamily [virtual]}}



Reimplemented from \doxymbox{\hyperlink{class_beam_1_1_component_aec1da33d2d6e3d4e7dd6708309264e76}{Beam::\+\+Component}}.

\Hypertarget{class_beam_1_1_top_bar_afea870bc1388aee09948c8d5b0dd014f}\index{Beam::TopBar@{Beam::TopBar}!render@{render}}
\index{render@{render}!Beam::TopBar@{Beam::TopBar}}
\doxysubsubsection{\texorpdfstring{render()}{render()}}
{\footnotesize\ttfamily \label{class_beam_1_1_top_bar_afea870bc1388aee09948c8d5b0dd014f} 
void Beam::\+\+Top\+Bar::\+render (\begin{DoxyParamCaption}\item[{\doxymbox{\hyperlink{class_beam_1_1_quad_batcher}{Quad\+Batcher}} \&}]{batcher}{}\end{DoxyParamCaption})\hspace{0.3cm}{\ttfamily [inline]}, {\ttfamily [override]}, {\ttfamily [virtual]}}



Implements \doxymbox{\hyperlink{class_beam_1_1_component_a2ed6b7841a25bd992bd46b822311ef1d}{Beam::\+\+Component}}.

\Hypertarget{class_beam_1_1_top_bar_a6d2bf867d1ee7d413b377cda942beb58}\index{Beam::TopBar@{Beam::TopBar}!setPlaying@{setPlaying}}
\index{setPlaying@{setPlaying}!Beam::TopBar@{Beam::TopBar}}
\doxysubsubsection{\texorpdfstring{setPlaying()}{setPlaying()}}
{\footnotesize\ttfamily \label{class_beam_1_1_top_bar_a6d2bf867d1ee7d413b377cda942beb58} 
void Beam::\+\+Top\+Bar::\+set\+Playing (\begin{DoxyParamCaption}\item[{bool}]{playing}{}\end{DoxyParamCaption})\hspace{0.3cm}{\ttfamily [inline]}}



\label{doc-variable-members}
\Hypertarget{class_beam_1_1_top_bar_doc-variable-members}
\doxysubsection{Member Data Documentation}
\Hypertarget{class_beam_1_1_top_bar_aacfd7a81b28a1cdcf7d9cfef3278a937}\index{Beam::TopBar@{Beam::TopBar}!m\_isPlaying@{m\_isPlaying}}
\index{m\_isPlaying@{m\_isPlaying}!Beam::TopBar@{Beam::TopBar}}
\doxysubsubsection{\texorpdfstring{m\_isPlaying}{m\_isPlaying}}
{\footnotesize\ttfamily \label{class_beam_1_1_top_bar_aacfd7a81b28a1cdcf7d9cfef3278a937} 
bool Beam::\+\+Top\+Bar::\+m\+\_\+is\+Playing = false\hspace{0.3cm}{\ttfamily [private]}}

\Hypertarget{class_beam_1_1_top_bar_aaa84af8dc8b326e0460006cf8364e41c}\index{Beam::TopBar@{Beam::TopBar}!onLoadRequested@{onLoadRequested}}
\index{onLoadRequested@{onLoadRequested}!Beam::TopBar@{Beam::TopBar}}
\doxysubsubsection{\texorpdfstring{onLoadRequested}{onLoadRequested}}
{\footnotesize\ttfamily \label{class_beam_1_1_top_bar_aaa84af8dc8b326e0460006cf8364e41c} 
std::\+function$<$void()$>$ Beam::\+\+Top\+Bar::\+on\+Load\+Requested}

\Hypertarget{class_beam_1_1_top_bar_a49378791fe51f2dff50e1124024ca73d}\index{Beam::TopBar@{Beam::TopBar}!onModeChanged@{onModeChanged}}
\index{onModeChanged@{onModeChanged}!Beam::TopBar@{Beam::TopBar}}
\doxysubsubsection{\texorpdfstring{onModeChanged}{onModeChanged}}
{\footnotesize\ttfamily \label{class_beam_1_1_top_bar_a49378791fe51f2dff50e1124024ca73d} 
std::\+function$<$void(int)$>$ Beam::\+\+Top\+Bar::\+on\+Mode\+Changed}

\Hypertarget{class_beam_1_1_top_bar_a6b7d5c0b5d176965a0c1b99fb4b61b86}\index{Beam::TopBar@{Beam::TopBar}!onPauseRequested@{onPauseRequested}}
\index{onPauseRequested@{onPauseRequested}!Beam::TopBar@{Beam::TopBar}}
\doxysubsubsection{\texorpdfstring{onPauseRequested}{onPauseRequested}}
{\footnotesize\ttfamily \label{class_beam_1_1_top_bar_a6b7d5c0b5d176965a0c1b99fb4b61b86} 
std::\+function$<$void()$>$ Beam::\+\+Top\+Bar::\+on\+Pause\+Requested}

\Hypertarget{class_beam_1_1_top_bar_a83dc6b9a4e8c9b9328bb0266ee2df6f6}\index{Beam::TopBar@{Beam::TopBar}!onPlayRequested@{onPlayRequested}}
\index{onPlayRequested@{onPlayRequested}!Beam::TopBar@{Beam::TopBar}}
\doxysubsubsection{\texorpdfstring{onPlayRequested}{onPlayRequested}}
{\footnotesize\ttfamily \label{class_beam_1_1_top_bar_a83dc6b9a4e8c9b9328bb0266ee2df6f6} 
std::\+function$<$void()$>$ Beam::\+\+Top\+Bar::\+on\+Play\+Requested}

\Hypertarget{class_beam_1_1_top_bar_a21063e55ca6ba9c9f7c9391750f26571}\index{Beam::TopBar@{Beam::TopBar}!onRewindRequested@{onRewindRequested}}
\index{onRewindRequested@{onRewindRequested}!Beam::TopBar@{Beam::TopBar}}
\doxysubsubsection{\texorpdfstring{onRewindRequested}{onRewindRequested}}
{\footnotesize\ttfamily \label{class_beam_1_1_top_bar_a21063e55ca6ba9c9f7c9391750f26571} 
std::\+function$<$void()$>$ Beam::\+\+Top\+Bar::\+on\+Rewind\+Requested}

\Hypertarget{class_beam_1_1_top_bar_ae6da30fe543f4a6a0f54cf2ef1663238}\index{Beam::TopBar@{Beam::TopBar}!onSaveRequested@{onSaveRequested}}
\index{onSaveRequested@{onSaveRequested}!Beam::TopBar@{Beam::TopBar}}
\doxysubsubsection{\texorpdfstring{onSaveRequested}{onSaveRequested}}
{\footnotesize\ttfamily \label{class_beam_1_1_top_bar_ae6da30fe543f4a6a0f54cf2ef1663238} 
std::\+function$<$void()$>$ Beam::\+\+Top\+Bar::\+on\+Save\+Requested}



The documentation for this class was generated from the following file:\+\begin{DoxyCompactItemize}
\item 
src/\+ui/\+\doxymbox{\hyperlink{top__bar_8hpp}{top\+\_\+bar.\+hpp}}\end{DoxyCompactItemize}

\doxysection{Beam::\+Track\+Data Struct Reference}
\hypertarget{struct_beam_1_1_track_data}{}\label{struct_beam_1_1_track_data}\index{Beam::TrackData@{Beam::TrackData}}


Represents a single track which has a DSP node and multiple timeline regions.  




{\ttfamily \+\#include $<$flux\+\_\+project.\+hpp$>$}

\doxysubsubsection*{Public Attributes}
\begin{DoxyCompactItemize}
\item 
std::\+shared\+\_\+ptr$<$ \doxymbox{\hyperlink{class_beam_1_1_flux_track_node}{Flux\+Track\+Node}} $>$ \doxymbox{\hyperlink{struct_beam_1_1_track_data_a0fe815f1f9877e2a29a039dca1a65e75}{node}}
\item 
size\+\_\+t \doxymbox{\hyperlink{struct_beam_1_1_track_data_a1a2ccea5bb7f8839502f360ea58e03fb}{node\+Id}}
\item 
std::\+vector$<$ \doxymbox{\hyperlink{struct_beam_1_1_region}{Region}} $>$ \doxymbox{\hyperlink{struct_beam_1_1_track_data_a1eef8d30fc0ac37d80bd581f566ff790}{regions}}
\item 
int \doxymbox{\hyperlink{struct_beam_1_1_track_data_a6b1694e4a50f04a1f044fe85ad4bd823}{track\+Index}}
\end{DoxyCompactItemize}


\doxysubsection{Detailed Description}
Represents a single track which has a DSP node and multiple timeline regions. 

\label{doc-variable-members}
\Hypertarget{struct_beam_1_1_track_data_doc-variable-members}
\doxysubsection{Member Data Documentation}
\Hypertarget{struct_beam_1_1_track_data_a0fe815f1f9877e2a29a039dca1a65e75}\index{Beam::TrackData@{Beam::TrackData}!node@{node}}
\index{node@{node}!Beam::TrackData@{Beam::TrackData}}
\doxysubsubsection{\texorpdfstring{node}{node}}
{\footnotesize\ttfamily \label{struct_beam_1_1_track_data_a0fe815f1f9877e2a29a039dca1a65e75} 
std::\+shared\+\_\+ptr$<$\doxymbox{\hyperlink{class_beam_1_1_flux_track_node}{Flux\+Track\+Node}}$>$ Beam::\+\+Track\+Data::\+node}

\Hypertarget{struct_beam_1_1_track_data_a1a2ccea5bb7f8839502f360ea58e03fb}\index{Beam::TrackData@{Beam::TrackData}!nodeId@{nodeId}}
\index{nodeId@{nodeId}!Beam::TrackData@{Beam::TrackData}}
\doxysubsubsection{\texorpdfstring{nodeId}{nodeId}}
{\footnotesize\ttfamily \label{struct_beam_1_1_track_data_a1a2ccea5bb7f8839502f360ea58e03fb} 
size\+\_\+t Beam::\+\+Track\+Data::\+node\+Id}

\Hypertarget{struct_beam_1_1_track_data_a1eef8d30fc0ac37d80bd581f566ff790}\index{Beam::TrackData@{Beam::TrackData}!regions@{regions}}
\index{regions@{regions}!Beam::TrackData@{Beam::TrackData}}
\doxysubsubsection{\texorpdfstring{regions}{regions}}
{\footnotesize\ttfamily \label{struct_beam_1_1_track_data_a1eef8d30fc0ac37d80bd581f566ff790} 
std::\+vector$<$\doxymbox{\hyperlink{struct_beam_1_1_region}{Region}}$>$ Beam::\+\+Track\+Data::\+regions}

\Hypertarget{struct_beam_1_1_track_data_a6b1694e4a50f04a1f044fe85ad4bd823}\index{Beam::TrackData@{Beam::TrackData}!trackIndex@{trackIndex}}
\index{trackIndex@{trackIndex}!Beam::TrackData@{Beam::TrackData}}
\doxysubsubsection{\texorpdfstring{trackIndex}{trackIndex}}
{\footnotesize\ttfamily \label{struct_beam_1_1_track_data_a6b1694e4a50f04a1f044fe85ad4bd823} 
int Beam::\+\+Track\+Data::\+track\+Index}



The documentation for this struct was generated from the following file:\+\begin{DoxyCompactItemize}
\item 
src/\+session/\+\doxymbox{\hyperlink{flux__project_8hpp}{flux\+\_\+project.\+hpp}}\end{DoxyCompactItemize}

\doxysection{Beam::\+Track\+Node Class Reference}
\hypertarget{class_beam_1_1_track_node}{}\label{class_beam_1_1_track_node}\index{Beam::TrackNode@{Beam::TrackNode}}


{\ttfamily \+\#include $<$track\+\_\+node.\+hpp$>$}

Inheritance diagram for Beam::\+Track\+Node:\+\begin{figure}[H]
\begin{center}
\leavevmode
\includegraphics[height=2.000000cm]{class_beam_1_1_track_node}
\end{center}
\end{figure}
\doxysubsubsection*{Public Member Functions}
\begin{DoxyCompactItemize}
\item 
\doxymbox{\hyperlink{class_beam_1_1_track_node_a300b73669c49d022a1dd10f7bedb546c}{Track\+Node}} (const std::\+string \&name)
\item 
bool \doxymbox{\hyperlink{class_beam_1_1_track_node_aacfc7312076b50be2cf6d4403df4ca95}{load}} (const std::\+string \&file\+Path)
\item 
void \doxymbox{\hyperlink{class_beam_1_1_track_node_a2e2bffa8b286946f8e32db31e672727e}{process}} (float \texorpdfstring{$\ast$}{*}buffer, int frames, int channels) override
\item 
void \doxymbox{\hyperlink{class_beam_1_1_track_node_a973d30146b430d0680b526740364a45c}{set\+State}} (\doxymbox{\hyperlink{namespace_beam_aaab526c0becfd931c9f29361daaf7e9f}{Track\+State}} state)
\item 
\doxymbox{\hyperlink{namespace_beam_aaab526c0becfd931c9f29361daaf7e9f}{Track\+State}} \doxymbox{\hyperlink{class_beam_1_1_track_node_a8bc557d0c5fe519e79fa335c695b1cf8}{get\+State}} () const
\item 
void \doxymbox{\hyperlink{class_beam_1_1_track_node_acdff1b83754ab94892dc1c1d6ffe5255}{seek}} (size\+\_\+t frame)
\item 
std::\+string \doxymbox{\hyperlink{class_beam_1_1_track_node_a8c233bb481da69fd335c86c286ae9606}{get\+Name}} () const override
\end{DoxyCompactItemize}
\doxysubsection*{Public Member Functions inherited from \doxymbox{\hyperlink{class_beam_1_1_audio_node}{Beam::\+\+Audio\+Node}}}
\begin{DoxyCompactItemize}
\item 
virtual \doxymbox{\hyperlink{class_beam_1_1_audio_node_afbea31954b50918131d31fc0d1f6de8c}{\texorpdfstring{$\sim$}{\string~}\+Audio\+Node}} ()=default
\item 
void \doxymbox{\hyperlink{class_beam_1_1_audio_node_a3fcc68eab5b1adf547a4205f258b212c}{set\+Bypass}} (bool bypass)
\item 
bool \doxymbox{\hyperlink{class_beam_1_1_audio_node_a6ba1724cff34b5bc0f811ee2537caae5}{is\+Bypassed}} () const
\end{DoxyCompactItemize}
\doxysubsubsection*{Private Attributes}
\begin{DoxyCompactItemize}
\item 
std::\+string \doxymbox{\hyperlink{class_beam_1_1_track_node_a6ac8f5d6a13f367ac52c01b8c933679f}{m\+\_\+name}}
\item 
std::\+unique\+\_\+ptr$<$ \doxymbox{\hyperlink{class_beam_1_1_disk_streamer}{Disk\+Streamer}} $>$ \doxymbox{\hyperlink{class_beam_1_1_track_node_abb11e4728816724f604713a672ad841d}{m\+\_\+streamer}}
\item 
std::\+atomic$<$ \doxymbox{\hyperlink{namespace_beam_aaab526c0becfd931c9f29361daaf7e9f}{Track\+State}} $>$ \doxymbox{\hyperlink{class_beam_1_1_track_node_a2d4f72820d3d3d7dc00f1ec2320613a3}{m\+\_\+state}}
\end{DoxyCompactItemize}
\doxysubsubsection*{Additional Inherited Members}
\doxysubsection*{Protected Attributes inherited from \doxymbox{\hyperlink{class_beam_1_1_audio_node}{Beam::\+\+Audio\+Node}}}
\begin{DoxyCompactItemize}
\item 
bool \doxymbox{\hyperlink{class_beam_1_1_audio_node_ac5ad81de4a5d0abe555fe9f06219b09f}{m\+\_\+is\+Bypassed}} = false
\end{DoxyCompactItemize}


\label{doc-constructors}
\Hypertarget{class_beam_1_1_track_node_doc-constructors}
\doxysubsection{Constructor \& Destructor Documentation}
\Hypertarget{class_beam_1_1_track_node_a300b73669c49d022a1dd10f7bedb546c}\index{Beam::TrackNode@{Beam::TrackNode}!TrackNode@{TrackNode}}
\index{TrackNode@{TrackNode}!Beam::TrackNode@{Beam::TrackNode}}
\doxysubsubsection{\texorpdfstring{TrackNode()}{TrackNode()}}
{\footnotesize\ttfamily \label{class_beam_1_1_track_node_a300b73669c49d022a1dd10f7bedb546c} 
Beam::\+\+Track\+Node::\+\+Track\+Node (\begin{DoxyParamCaption}\item[{const std::\+string \&}]{name}{}\end{DoxyParamCaption})\hspace{0.3cm}{\ttfamily [inline]}}



\label{doc-func-members}
\Hypertarget{class_beam_1_1_track_node_doc-func-members}
\doxysubsection{Member Function Documentation}
\Hypertarget{class_beam_1_1_track_node_a8c233bb481da69fd335c86c286ae9606}\index{Beam::TrackNode@{Beam::TrackNode}!getName@{getName}}
\index{getName@{getName}!Beam::TrackNode@{Beam::TrackNode}}
\doxysubsubsection{\texorpdfstring{getName()}{getName()}}
{\footnotesize\ttfamily \label{class_beam_1_1_track_node_a8c233bb481da69fd335c86c286ae9606} 
std::\+string Beam::\+\+Track\+Node::\+get\+Name (\begin{DoxyParamCaption}{}{}\end{DoxyParamCaption}) const\hspace{0.3cm}{\ttfamily [inline]}, {\ttfamily [override]}, {\ttfamily [virtual]}}



Implements \doxymbox{\hyperlink{class_beam_1_1_audio_node_a864b3bf9095638e43ad334b7b3706bec}{Beam::\+\+Audio\+Node}}.

\Hypertarget{class_beam_1_1_track_node_a8bc557d0c5fe519e79fa335c695b1cf8}\index{Beam::TrackNode@{Beam::TrackNode}!getState@{getState}}
\index{getState@{getState}!Beam::TrackNode@{Beam::TrackNode}}
\doxysubsubsection{\texorpdfstring{getState()}{getState()}}
{\footnotesize\ttfamily \label{class_beam_1_1_track_node_a8bc557d0c5fe519e79fa335c695b1cf8} 
\doxymbox{\hyperlink{namespace_beam_aaab526c0becfd931c9f29361daaf7e9f}{Track\+State}} Beam::\+\+Track\+Node::\+get\+State (\begin{DoxyParamCaption}{}{}\end{DoxyParamCaption}) const\hspace{0.3cm}{\ttfamily [inline]}}

\Hypertarget{class_beam_1_1_track_node_aacfc7312076b50be2cf6d4403df4ca95}\index{Beam::TrackNode@{Beam::TrackNode}!load@{load}}
\index{load@{load}!Beam::TrackNode@{Beam::TrackNode}}
\doxysubsubsection{\texorpdfstring{load()}{load()}}
{\footnotesize\ttfamily \label{class_beam_1_1_track_node_aacfc7312076b50be2cf6d4403df4ca95} 
bool Beam::\+\+Track\+Node::\+load (\begin{DoxyParamCaption}\item[{const std::\+string \&}]{file\+Path}{}\end{DoxyParamCaption})\hspace{0.3cm}{\ttfamily [inline]}}

\Hypertarget{class_beam_1_1_track_node_a2e2bffa8b286946f8e32db31e672727e}\index{Beam::TrackNode@{Beam::TrackNode}!process@{process}}
\index{process@{process}!Beam::TrackNode@{Beam::TrackNode}}
\doxysubsubsection{\texorpdfstring{process()}{process()}}
{\footnotesize\ttfamily \label{class_beam_1_1_track_node_a2e2bffa8b286946f8e32db31e672727e} 
void Beam::\+\+Track\+Node::\+process (\begin{DoxyParamCaption}\item[{float \texorpdfstring{$\ast$}{*}}]{buffer}{, }\item[{int}]{frames}{, }\item[{int}]{channels}{}\end{DoxyParamCaption})\hspace{0.3cm}{\ttfamily [inline]}, {\ttfamily [override]}, {\ttfamily [virtual]}}



Implements \doxymbox{\hyperlink{class_beam_1_1_audio_node_a6954940f3822298db5c2224e921432f6}{Beam::\+\+Audio\+Node}}.

\Hypertarget{class_beam_1_1_track_node_acdff1b83754ab94892dc1c1d6ffe5255}\index{Beam::TrackNode@{Beam::TrackNode}!seek@{seek}}
\index{seek@{seek}!Beam::TrackNode@{Beam::TrackNode}}
\doxysubsubsection{\texorpdfstring{seek()}{seek()}}
{\footnotesize\ttfamily \label{class_beam_1_1_track_node_acdff1b83754ab94892dc1c1d6ffe5255} 
void Beam::\+\+Track\+Node::\+seek (\begin{DoxyParamCaption}\item[{size\+\_\+t}]{frame}{}\end{DoxyParamCaption})\hspace{0.3cm}{\ttfamily [inline]}}

\Hypertarget{class_beam_1_1_track_node_a973d30146b430d0680b526740364a45c}\index{Beam::TrackNode@{Beam::TrackNode}!setState@{setState}}
\index{setState@{setState}!Beam::TrackNode@{Beam::TrackNode}}
\doxysubsubsection{\texorpdfstring{setState()}{setState()}}
{\footnotesize\ttfamily \label{class_beam_1_1_track_node_a973d30146b430d0680b526740364a45c} 
void Beam::\+\+Track\+Node::\+set\+State (\begin{DoxyParamCaption}\item[{\doxymbox{\hyperlink{namespace_beam_aaab526c0becfd931c9f29361daaf7e9f}{Track\+State}}}]{state}{}\end{DoxyParamCaption})\hspace{0.3cm}{\ttfamily [inline]}}



\label{doc-variable-members}
\Hypertarget{class_beam_1_1_track_node_doc-variable-members}
\doxysubsection{Member Data Documentation}
\Hypertarget{class_beam_1_1_track_node_a6ac8f5d6a13f367ac52c01b8c933679f}\index{Beam::TrackNode@{Beam::TrackNode}!m\_name@{m\_name}}
\index{m\_name@{m\_name}!Beam::TrackNode@{Beam::TrackNode}}
\doxysubsubsection{\texorpdfstring{m\_name}{m\_name}}
{\footnotesize\ttfamily \label{class_beam_1_1_track_node_a6ac8f5d6a13f367ac52c01b8c933679f} 
std::\+string Beam::\+\+Track\+Node::\+m\+\_\+name\hspace{0.3cm}{\ttfamily [private]}}

\Hypertarget{class_beam_1_1_track_node_a2d4f72820d3d3d7dc00f1ec2320613a3}\index{Beam::TrackNode@{Beam::TrackNode}!m\_state@{m\_state}}
\index{m\_state@{m\_state}!Beam::TrackNode@{Beam::TrackNode}}
\doxysubsubsection{\texorpdfstring{m\_state}{m\_state}}
{\footnotesize\ttfamily \label{class_beam_1_1_track_node_a2d4f72820d3d3d7dc00f1ec2320613a3} 
std::\+atomic$<$\doxymbox{\hyperlink{namespace_beam_aaab526c0becfd931c9f29361daaf7e9f}{Track\+State}}$>$ Beam::\+\+Track\+Node::\+m\+\_\+state\hspace{0.3cm}{\ttfamily [private]}}

\Hypertarget{class_beam_1_1_track_node_abb11e4728816724f604713a672ad841d}\index{Beam::TrackNode@{Beam::TrackNode}!m\_streamer@{m\_streamer}}
\index{m\_streamer@{m\_streamer}!Beam::TrackNode@{Beam::TrackNode}}
\doxysubsubsection{\texorpdfstring{m\_streamer}{m\_streamer}}
{\footnotesize\ttfamily \label{class_beam_1_1_track_node_abb11e4728816724f604713a672ad841d} 
std::\+unique\+\_\+ptr$<$\doxymbox{\hyperlink{class_beam_1_1_disk_streamer}{Disk\+Streamer}}$>$ Beam::\+\+Track\+Node::\+m\+\_\+streamer\hspace{0.3cm}{\ttfamily [private]}}



The documentation for this class was generated from the following file:\+\begin{DoxyCompactItemize}
\item 
src/\+dsp/\+\doxymbox{\hyperlink{track__node_8hpp}{track\+\_\+node.\+hpp}}\end{DoxyCompactItemize}

\input{class_beam_1_1_tube_compressor_node}
\doxysection{Beam::\+Tube\+Compressor\+UI Class Reference}
\hypertarget{class_beam_1_1_tube_compressor_u_i}{}\label{class_beam_1_1_tube_compressor_u_i}\index{Beam::TubeCompressorUI@{Beam::TubeCompressorUI}}


{\ttfamily \+\#include $<$tube\+\_\+compressor\+\_\+ui.\+hpp$>$}

Inheritance diagram for Beam::\+Tube\+Compressor\+UI:\+\begin{figure}[H]
\begin{center}
\leavevmode
\includegraphics[height=3.000000cm]{class_beam_1_1_tube_compressor_u_i}
\end{center}
\end{figure}
\doxysubsubsection*{Public Member Functions}
\begin{DoxyCompactItemize}
\item 
\doxymbox{\hyperlink{class_beam_1_1_tube_compressor_u_i_ab9f12c34c25e5381d6ec9db8faa13e16}{Tube\+Compressor\+UI}} (std::\+shared\+\_\+ptr$<$ \doxymbox{\hyperlink{class_beam_1_1_flux_node}{Flux\+Node}} $>$ node, size\+\_\+t id, float x, float y)
\item 
void \doxymbox{\hyperlink{class_beam_1_1_tube_compressor_u_i_a03af59494f87e40910fb9a38380def0a}{render}} (\doxymbox{\hyperlink{class_beam_1_1_quad_batcher}{Quad\+Batcher}} \&batcher, float dt, float screenW, float screenH) override
\end{DoxyCompactItemize}
\doxysubsection*{Public Member Functions inherited from \doxymbox{\hyperlink{class_beam_1_1_audio_module}{Beam::\+\+Audio\+Module}}}
\begin{DoxyCompactItemize}
\item 
\doxymbox{\hyperlink{class_beam_1_1_audio_module_a409c75189798146d7f556b1d50f4ba98}{Audio\+Module}} (std::\+shared\+\_\+ptr$<$ \doxymbox{\hyperlink{class_beam_1_1_flux_node}{Flux\+Node}} $>$ node, size\+\_\+t node\+Id, float x, float y)
\item 
size\+\_\+t \doxymbox{\hyperlink{class_beam_1_1_audio_module_a7fc32b3bfadec2badbfd067d2f37da96}{get\+Node\+Id}} () const
\item 
void \doxymbox{\hyperlink{class_beam_1_1_audio_module_a655aa14548b3d4e977294a9bdaafa879}{auto\+Generate\+UI}} ()
\item 
void \doxymbox{\hyperlink{class_beam_1_1_audio_module_a75c0758091a1cec0871134babb541135}{set\+Bounds}} (float x, float y, float w, float h) override
\item 
bool \doxymbox{\hyperlink{class_beam_1_1_audio_module_adddaa58e40512d782c2d902917491499}{on\+Mouse\+Down}} (float x, float y, int button) override
\item 
bool \doxymbox{\hyperlink{class_beam_1_1_audio_module_a4ac204a52c61e603ae6a51daa610bc19}{on\+Mouse\+Up}} (float x, float y, int button) override
\item 
bool \doxymbox{\hyperlink{class_beam_1_1_audio_module_aff90c84092de907409b84eed2c46cbe2}{on\+Mouse\+Move}} (float x, float y) override
\item 
void \doxymbox{\hyperlink{class_beam_1_1_audio_module_a68a3d4bef3787a290f9a313b975f7925}{add\+Child}} (std::\+shared\+\_\+ptr$<$ \doxymbox{\hyperlink{class_beam_1_1_component}{Component}} $>$ child)
\item 
std::\+shared\+\_\+ptr$<$ \doxymbox{\hyperlink{class_beam_1_1_port}{Port}} $>$ \doxymbox{\hyperlink{class_beam_1_1_audio_module_a0d0da8bdbfb3a2355994bba086e0e721}{get\+Input\+Port}} ()
\item 
std::\+shared\+\_\+ptr$<$ \doxymbox{\hyperlink{class_beam_1_1_port}{Port}} $>$ \doxymbox{\hyperlink{class_beam_1_1_audio_module_a75e91aa2e7da1c9a6d31f4c90915403a}{get\+Output\+Port}} ()
\end{DoxyCompactItemize}
\doxysubsection*{Public Member Functions inherited from \doxymbox{\hyperlink{class_beam_1_1_component}{Beam::\+\+Component}}}
\begin{DoxyCompactItemize}
\item 
virtual \doxymbox{\hyperlink{class_beam_1_1_component_af9d734d649978e027412a87bc54362cd}{\texorpdfstring{$\sim$}{\string~}\+Component}} ()=default
\item 
virtual void \doxymbox{\hyperlink{class_beam_1_1_component_ad3d3fb19d25b4371d07620567970a158}{update}} (float dt)
\item 
virtual bool \doxymbox{\hyperlink{class_beam_1_1_component_ab92e884903f8a621fcd57bc00a24b041}{on\+Mouse\+Wheel}} (float x, float y, float delta)
\item 
const \doxymbox{\hyperlink{struct_beam_1_1_rect}{Rect}} \& \doxymbox{\hyperlink{class_beam_1_1_component_a5746dbc69d5b0adb4cffbcf920936d00}{get\+Bounds}} () const
\item 
void \doxymbox{\hyperlink{class_beam_1_1_component_a00d4e2dfa7703e59d6486852321dbdf1}{set\+Draggable}} (bool draggable)
\item 
void \doxymbox{\hyperlink{class_beam_1_1_component_aca7b02d1dddf7cd20378db9e3242fb84}{start\+Dragging}} (float x, float y)
\end{DoxyCompactItemize}
\doxysubsubsection*{Private Member Functions}
\begin{DoxyCompactItemize}
\item 
void \doxymbox{\hyperlink{class_beam_1_1_tube_compressor_u_i_aad8a9acafa4f6892f535d6f6dab2dacc}{init\+Metal\+Texture}} ()
\end{DoxyCompactItemize}
\doxysubsubsection*{Private Attributes}
\begin{DoxyCompactItemize}
\item 
std::\+shared\+\_\+ptr$<$ \doxymbox{\hyperlink{class_beam_1_1_texture}{Texture}} $>$ \doxymbox{\hyperlink{class_beam_1_1_tube_compressor_u_i_aced7e438bc26fed70eaf3bdee5b39771}{m\+\_\+faceplate}}
\end{DoxyCompactItemize}
\doxysubsubsection*{Additional Inherited Members}
\doxysubsection*{Public Attributes inherited from \doxymbox{\hyperlink{class_beam_1_1_audio_module}{Beam::\+\+Audio\+Module}}}
\begin{DoxyCompactItemize}
\item 
std::\+function$<$ void(\doxymbox{\hyperlink{class_beam_1_1_audio_module_a409c75189798146d7f556b1d50f4ba98}{Audio\+Module}} \texorpdfstring{$\ast$}{*})$>$ \doxymbox{\hyperlink{class_beam_1_1_audio_module_af95229e9824037c0c9916cc04bf67e90}{on\+Delete\+Requested}}
\end{DoxyCompactItemize}
\doxysubsection*{Protected Attributes inherited from \doxymbox{\hyperlink{class_beam_1_1_audio_module}{Beam::\+\+Audio\+Module}}}
\begin{DoxyCompactItemize}
\item 
std::\+vector$<$ std::\+shared\+\_\+ptr$<$ \doxymbox{\hyperlink{class_beam_1_1_component}{Component}} $>$ $>$ \doxymbox{\hyperlink{class_beam_1_1_audio_module_a1af200892e351f910e6447a50913a560}{m\+\_\+children}}
\end{DoxyCompactItemize}
\doxysubsection*{Protected Attributes inherited from \doxymbox{\hyperlink{class_beam_1_1_component}{Beam::\+\+Component}}}
\begin{DoxyCompactItemize}
\item 
\doxymbox{\hyperlink{struct_beam_1_1_rect}{Rect}} \doxymbox{\hyperlink{class_beam_1_1_component_a4f1ec4a5fb168c39a6c18f958b2b1495}{m\+\_\+bounds}} \{0, 0, 0, 0\}
\item 
bool \doxymbox{\hyperlink{class_beam_1_1_component_adc07913aed6ddadf1c730e7b3bb599cf}{m\+\_\+is\+Visible}} = true
\item 
bool \doxymbox{\hyperlink{class_beam_1_1_component_a0bf77b204ae374a14b5a6d7e5a3c13c6}{m\+\_\+is\+Enabled}} = true
\item 
bool \doxymbox{\hyperlink{class_beam_1_1_component_a9646efcaa9540a26a387f5da9aae4bde}{m\+\_\+is\+Draggable}} = false
\item 
bool \doxymbox{\hyperlink{class_beam_1_1_component_ab03af9a9743acf040f38e3fb11f8dc14}{m\+\_\+is\+Dragging}} = false
\item 
float \doxymbox{\hyperlink{class_beam_1_1_component_a7110b2b9dc235f724bf4689569266a63}{m\+\_\+last\+MouseX}} = 0
\item 
float \doxymbox{\hyperlink{class_beam_1_1_component_a768931a0f51394bf011f821f6ed2efe9}{m\+\_\+last\+MouseY}} = 0
\end{DoxyCompactItemize}


\label{doc-constructors}
\Hypertarget{class_beam_1_1_tube_compressor_u_i_doc-constructors}
\doxysubsection{Constructor \& Destructor Documentation}
\Hypertarget{class_beam_1_1_tube_compressor_u_i_ab9f12c34c25e5381d6ec9db8faa13e16}\index{Beam::TubeCompressorUI@{Beam::TubeCompressorUI}!TubeCompressorUI@{TubeCompressorUI}}
\index{TubeCompressorUI@{TubeCompressorUI}!Beam::TubeCompressorUI@{Beam::TubeCompressorUI}}
\doxysubsubsection{\texorpdfstring{TubeCompressorUI()}{TubeCompressorUI()}}
{\footnotesize\ttfamily \label{class_beam_1_1_tube_compressor_u_i_ab9f12c34c25e5381d6ec9db8faa13e16} 
Beam::\+\+Tube\+Compressor\+UI::\+\+Tube\+Compressor\+UI (\begin{DoxyParamCaption}\item[{std::\+shared\+\_\+ptr$<$ \doxymbox{\hyperlink{class_beam_1_1_flux_node}{Flux\+Node}} $>$}]{node}{, }\item[{size\+\_\+t}]{id}{, }\item[{float}]{x}{, }\item[{float}]{y}{}\end{DoxyParamCaption})\hspace{0.3cm}{\ttfamily [inline]}}



\label{doc-func-members}
\Hypertarget{class_beam_1_1_tube_compressor_u_i_doc-func-members}
\doxysubsection{Member Function Documentation}
\Hypertarget{class_beam_1_1_tube_compressor_u_i_aad8a9acafa4f6892f535d6f6dab2dacc}\index{Beam::TubeCompressorUI@{Beam::TubeCompressorUI}!initMetalTexture@{initMetalTexture}}
\index{initMetalTexture@{initMetalTexture}!Beam::TubeCompressorUI@{Beam::TubeCompressorUI}}
\doxysubsubsection{\texorpdfstring{initMetalTexture()}{initMetalTexture()}}
{\footnotesize\ttfamily \label{class_beam_1_1_tube_compressor_u_i_aad8a9acafa4f6892f535d6f6dab2dacc} 
void Beam::\+\+Tube\+Compressor\+UI::\+init\+Metal\+Texture (\begin{DoxyParamCaption}{}{}\end{DoxyParamCaption})\hspace{0.3cm}{\ttfamily [inline]}, {\ttfamily [private]}}

\Hypertarget{class_beam_1_1_tube_compressor_u_i_a03af59494f87e40910fb9a38380def0a}\index{Beam::TubeCompressorUI@{Beam::TubeCompressorUI}!render@{render}}
\index{render@{render}!Beam::TubeCompressorUI@{Beam::TubeCompressorUI}}
\doxysubsubsection{\texorpdfstring{render()}{render()}}
{\footnotesize\ttfamily \label{class_beam_1_1_tube_compressor_u_i_a03af59494f87e40910fb9a38380def0a} 
void Beam::\+\+Tube\+Compressor\+UI::\+render (\begin{DoxyParamCaption}\item[{\doxymbox{\hyperlink{class_beam_1_1_quad_batcher}{Quad\+Batcher}} \&}]{batcher}{, }\item[{float}]{dt}{, }\item[{float}]{screenW}{, }\item[{float}]{screenH}{}\end{DoxyParamCaption})\hspace{0.3cm}{\ttfamily [inline]}, {\ttfamily [override]}, {\ttfamily [virtual]}}



Reimplemented from \doxymbox{\hyperlink{class_beam_1_1_audio_module_a0e43bace4dcd9eb159ff2237b0c7c3fd}{Beam::\+\+Audio\+Module}}.



\label{doc-variable-members}
\Hypertarget{class_beam_1_1_tube_compressor_u_i_doc-variable-members}
\doxysubsection{Member Data Documentation}
\Hypertarget{class_beam_1_1_tube_compressor_u_i_aced7e438bc26fed70eaf3bdee5b39771}\index{Beam::TubeCompressorUI@{Beam::TubeCompressorUI}!m\_faceplate@{m\_faceplate}}
\index{m\_faceplate@{m\_faceplate}!Beam::TubeCompressorUI@{Beam::TubeCompressorUI}}
\doxysubsubsection{\texorpdfstring{m\_faceplate}{m\_faceplate}}
{\footnotesize\ttfamily \label{class_beam_1_1_tube_compressor_u_i_aced7e438bc26fed70eaf3bdee5b39771} 
std::\+shared\+\_\+ptr$<$\doxymbox{\hyperlink{class_beam_1_1_texture}{Texture}}$>$ Beam::\+\+Tube\+Compressor\+UI::\+m\+\_\+faceplate\hspace{0.3cm}{\ttfamily [private]}}



The documentation for this class was generated from the following file:\+\begin{DoxyCompactItemize}
\item 
src/\+interface/\+\doxymbox{\hyperlink{tube__compressor__ui_8hpp}{tube\+\_\+compressor\+\_\+ui.\+hpp}}\end{DoxyCompactItemize}

\input{class_beam_1_1_tube_limiter}
\input{class_beam_1_1_tube_p___e_q}
\doxysection{Beam::\+Vertex Struct Reference}
\hypertarget{struct_beam_1_1_vertex}{}\label{struct_beam_1_1_vertex}\index{Beam::Vertex@{Beam::Vertex}}


{\ttfamily \+\#include $<$quad\+\_\+batcher.\+hpp$>$}

\doxysubsubsection*{Public Attributes}
\begin{DoxyCompactItemize}
\item 
float \doxymbox{\hyperlink{struct_beam_1_1_vertex_aee7d2780bc5f94c3d4e4fde33caf883f}{position}} \+[2]\+
\item 
float \doxymbox{\hyperlink{struct_beam_1_1_vertex_abcafa4e4b169e54d5983f2b59d19e7a6}{tex\+Coord}} \+[2]\+
\item 
float \doxymbox{\hyperlink{struct_beam_1_1_vertex_ad2bae330663b26bfd422f9d0b0361399}{color}} \+[4]\+
\end{DoxyCompactItemize}


\label{doc-variable-members}
\Hypertarget{struct_beam_1_1_vertex_doc-variable-members}
\doxysubsection{Member Data Documentation}
\Hypertarget{struct_beam_1_1_vertex_ad2bae330663b26bfd422f9d0b0361399}\index{Beam::Vertex@{Beam::Vertex}!color@{color}}
\index{color@{color}!Beam::Vertex@{Beam::Vertex}}
\doxysubsubsection{\texorpdfstring{color}{color}}
{\footnotesize\ttfamily \label{struct_beam_1_1_vertex_ad2bae330663b26bfd422f9d0b0361399} 
float Beam::\+\+Vertex::\+color\+[4]\+}

\Hypertarget{struct_beam_1_1_vertex_aee7d2780bc5f94c3d4e4fde33caf883f}\index{Beam::Vertex@{Beam::Vertex}!position@{position}}
\index{position@{position}!Beam::Vertex@{Beam::Vertex}}
\doxysubsubsection{\texorpdfstring{position}{position}}
{\footnotesize\ttfamily \label{struct_beam_1_1_vertex_aee7d2780bc5f94c3d4e4fde33caf883f} 
float Beam::\+\+Vertex::\+position\+[2]\+}

\Hypertarget{struct_beam_1_1_vertex_abcafa4e4b169e54d5983f2b59d19e7a6}\index{Beam::Vertex@{Beam::Vertex}!texCoord@{texCoord}}
\index{texCoord@{texCoord}!Beam::Vertex@{Beam::Vertex}}
\doxysubsubsection{\texorpdfstring{texCoord}{texCoord}}
{\footnotesize\ttfamily \label{struct_beam_1_1_vertex_abcafa4e4b169e54d5983f2b59d19e7a6} 
float Beam::\+\+Vertex::\+tex\+Coord\+[2]\+}



The documentation for this struct was generated from the following file:\+\begin{DoxyCompactItemize}
\item 
src/\+graphics/\+\doxymbox{\hyperlink{quad__batcher_8hpp}{quad\+\_\+batcher.\+hpp}}\end{DoxyCompactItemize}

\doxysection{Beam::\+VUMeter Class Reference}
\hypertarget{class_beam_1_1_v_u_meter}{}\label{class_beam_1_1_v_u_meter}\index{Beam::VUMeter@{Beam::VUMeter}}


{\ttfamily \+\#include $<$vu\+\_\+meter.\+hpp$>$}

Inheritance diagram for Beam::\+VUMeter:\+\begin{figure}[H]
\begin{center}
\leavevmode
\includegraphics[height=2.000000cm]{class_beam_1_1_v_u_meter}
\end{center}
\end{figure}
\doxysubsubsection*{Public Member Functions}
\begin{DoxyCompactItemize}
\item 
\doxymbox{\hyperlink{class_beam_1_1_v_u_meter_adc79b72cf10c8ef6446160b03e785933}{VUMeter}} ()
\item 
void \doxymbox{\hyperlink{class_beam_1_1_v_u_meter_ae9e38024ab79334ead7cdd4f642f9336}{set\+Level}} (float \doxymbox{\hyperlink{texture_8cpp_aac6e8d3b3465804e2be1198b2a2a3143}{level}})
\item 
void \doxymbox{\hyperlink{class_beam_1_1_v_u_meter_ac8eccd498a096225826a9a219f7816ed}{render}} (\doxymbox{\hyperlink{class_beam_1_1_quad_batcher}{Quad\+Batcher}} \&batcher, float dt, float screenW, float screenH) override
\end{DoxyCompactItemize}
\doxysubsection*{Public Member Functions inherited from \doxymbox{\hyperlink{class_beam_1_1_component}{Beam::\+\+Component}}}
\begin{DoxyCompactItemize}
\item 
virtual \doxymbox{\hyperlink{class_beam_1_1_component_af9d734d649978e027412a87bc54362cd}{\texorpdfstring{$\sim$}{\string~}\+Component}} ()=default
\item 
virtual void \doxymbox{\hyperlink{class_beam_1_1_component_ad3d3fb19d25b4371d07620567970a158}{update}} (float dt)
\item 
virtual bool \doxymbox{\hyperlink{class_beam_1_1_component_aec1da33d2d6e3d4e7dd6708309264e76}{on\+Mouse\+Down}} (float x, float y, int button)
\item 
virtual bool \doxymbox{\hyperlink{class_beam_1_1_component_ae36b8e9d70e8f9a1b9ba81c23c54d5c8}{on\+Mouse\+Up}} (float x, float y, int button)
\item 
virtual bool \doxymbox{\hyperlink{class_beam_1_1_component_a9d8e5970783d315044277a1228659e6c}{on\+Mouse\+Move}} (float x, float y)
\item 
virtual bool \doxymbox{\hyperlink{class_beam_1_1_component_ab92e884903f8a621fcd57bc00a24b041}{on\+Mouse\+Wheel}} (float x, float y, float delta)
\item 
virtual void \doxymbox{\hyperlink{class_beam_1_1_component_a6865b1f22388af467bf6c789120fac05}{set\+Bounds}} (float x, float y, float w, float h)
\item 
const \doxymbox{\hyperlink{struct_beam_1_1_rect}{Rect}} \& \doxymbox{\hyperlink{class_beam_1_1_component_a5746dbc69d5b0adb4cffbcf920936d00}{get\+Bounds}} () const
\item 
void \doxymbox{\hyperlink{class_beam_1_1_component_a00d4e2dfa7703e59d6486852321dbdf1}{set\+Draggable}} (bool draggable)
\item 
void \doxymbox{\hyperlink{class_beam_1_1_component_aca7b02d1dddf7cd20378db9e3242fb84}{start\+Dragging}} (float x, float y)
\end{DoxyCompactItemize}
\doxysubsubsection*{Private Attributes}
\begin{DoxyCompactItemize}
\item 
float \doxymbox{\hyperlink{class_beam_1_1_v_u_meter_a90cbe34e19c343b57b5256d7186fbf1c}{m\+\_\+level}}
\end{DoxyCompactItemize}
\doxysubsubsection*{Additional Inherited Members}
\doxysubsection*{Protected Attributes inherited from \doxymbox{\hyperlink{class_beam_1_1_component}{Beam::\+\+Component}}}
\begin{DoxyCompactItemize}
\item 
\doxymbox{\hyperlink{struct_beam_1_1_rect}{Rect}} \doxymbox{\hyperlink{class_beam_1_1_component_a4f1ec4a5fb168c39a6c18f958b2b1495}{m\+\_\+bounds}} \{0, 0, 0, 0\}
\item 
bool \doxymbox{\hyperlink{class_beam_1_1_component_adc07913aed6ddadf1c730e7b3bb599cf}{m\+\_\+is\+Visible}} = true
\item 
bool \doxymbox{\hyperlink{class_beam_1_1_component_a0bf77b204ae374a14b5a6d7e5a3c13c6}{m\+\_\+is\+Enabled}} = true
\item 
bool \doxymbox{\hyperlink{class_beam_1_1_component_a9646efcaa9540a26a387f5da9aae4bde}{m\+\_\+is\+Draggable}} = false
\item 
bool \doxymbox{\hyperlink{class_beam_1_1_component_ab03af9a9743acf040f38e3fb11f8dc14}{m\+\_\+is\+Dragging}} = false
\item 
float \doxymbox{\hyperlink{class_beam_1_1_component_a7110b2b9dc235f724bf4689569266a63}{m\+\_\+last\+MouseX}} = 0
\item 
float \doxymbox{\hyperlink{class_beam_1_1_component_a768931a0f51394bf011f821f6ed2efe9}{m\+\_\+last\+MouseY}} = 0
\end{DoxyCompactItemize}


\label{doc-constructors}
\Hypertarget{class_beam_1_1_v_u_meter_doc-constructors}
\doxysubsection{Constructor \& Destructor Documentation}
\Hypertarget{class_beam_1_1_v_u_meter_adc79b72cf10c8ef6446160b03e785933}\index{Beam::VUMeter@{Beam::VUMeter}!VUMeter@{VUMeter}}
\index{VUMeter@{VUMeter}!Beam::VUMeter@{Beam::VUMeter}}
\doxysubsubsection{\texorpdfstring{VUMeter()}{VUMeter()}}
{\footnotesize\ttfamily \label{class_beam_1_1_v_u_meter_adc79b72cf10c8ef6446160b03e785933} 
Beam::\+\+VUMeter::\+\+VUMeter (\begin{DoxyParamCaption}{}{}\end{DoxyParamCaption})\hspace{0.3cm}{\ttfamily [inline]}}



\label{doc-func-members}
\Hypertarget{class_beam_1_1_v_u_meter_doc-func-members}
\doxysubsection{Member Function Documentation}
\Hypertarget{class_beam_1_1_v_u_meter_ac8eccd498a096225826a9a219f7816ed}\index{Beam::VUMeter@{Beam::VUMeter}!render@{render}}
\index{render@{render}!Beam::VUMeter@{Beam::VUMeter}}
\doxysubsubsection{\texorpdfstring{render()}{render()}}
{\footnotesize\ttfamily \label{class_beam_1_1_v_u_meter_ac8eccd498a096225826a9a219f7816ed} 
void Beam::\+\+VUMeter::\+render (\begin{DoxyParamCaption}\item[{\doxymbox{\hyperlink{class_beam_1_1_quad_batcher}{Quad\+Batcher}} \&}]{batcher}{, }\item[{float}]{dt}{, }\item[{float}]{screenW}{, }\item[{float}]{screenH}{}\end{DoxyParamCaption})\hspace{0.3cm}{\ttfamily [inline]}, {\ttfamily [override]}, {\ttfamily [virtual]}}



Implements \doxymbox{\hyperlink{class_beam_1_1_component_acef3496a55f0d94c8678f6049dbaa7cd}{Beam::\+\+Component}}.

\Hypertarget{class_beam_1_1_v_u_meter_ae9e38024ab79334ead7cdd4f642f9336}\index{Beam::VUMeter@{Beam::VUMeter}!setLevel@{setLevel}}
\index{setLevel@{setLevel}!Beam::VUMeter@{Beam::VUMeter}}
\doxysubsubsection{\texorpdfstring{setLevel()}{setLevel()}}
{\footnotesize\ttfamily \label{class_beam_1_1_v_u_meter_ae9e38024ab79334ead7cdd4f642f9336} 
void Beam::\+\+VUMeter::\+set\+Level (\begin{DoxyParamCaption}\item[{float}]{level}{}\end{DoxyParamCaption})\hspace{0.3cm}{\ttfamily [inline]}}



\label{doc-variable-members}
\Hypertarget{class_beam_1_1_v_u_meter_doc-variable-members}
\doxysubsection{Member Data Documentation}
\Hypertarget{class_beam_1_1_v_u_meter_a90cbe34e19c343b57b5256d7186fbf1c}\index{Beam::VUMeter@{Beam::VUMeter}!m\_level@{m\_level}}
\index{m\_level@{m\_level}!Beam::VUMeter@{Beam::VUMeter}}
\doxysubsubsection{\texorpdfstring{m\_level}{m\_level}}
{\footnotesize\ttfamily \label{class_beam_1_1_v_u_meter_a90cbe34e19c343b57b5256d7186fbf1c} 
float Beam::\+\+VUMeter::\+m\+\_\+level\hspace{0.3cm}{\ttfamily [private]}}



The documentation for this class was generated from the following file:\+\begin{DoxyCompactItemize}
\item 
src/\+interface/\+\doxymbox{\hyperlink{vu__meter_8hpp}{vu\+\_\+meter.\+hpp}}\end{DoxyCompactItemize}

\doxysection{Beam::\+Wav\+Reader Class Reference}
\hypertarget{class_beam_1_1_wav_reader}{}\label{class_beam_1_1_wav_reader}\index{Beam::WavReader@{Beam::WavReader}}


{\ttfamily \+\#include $<$wav\+\_\+reader.\+hpp$>$}

\doxysubsubsection*{Public Member Functions}
\begin{DoxyCompactItemize}
\item 
\doxymbox{\hyperlink{class_beam_1_1_wav_reader_a495edcb749c0ed8d9337adea76f3f3ff}{Wav\+Reader}} ()
\item 
bool \doxymbox{\hyperlink{class_beam_1_1_wav_reader_a54dc2bfd8c1b1773df1f444211da7f6e}{open}} (const std::\+string \&file\+Path)
\item 
size\+\_\+t \doxymbox{\hyperlink{class_beam_1_1_wav_reader_a48fbbce631692e26e952393c21a379bd}{read\+Frames}} (float \texorpdfstring{$\ast$}{*}buffer, size\+\_\+t frames, int dest\+Channels)
\item 
void \doxymbox{\hyperlink{class_beam_1_1_wav_reader_a86c1c481afc06a74ee5e24613b420e7d}{seek}} (size\+\_\+t frame)
\item 
uint32\+\_\+t \doxymbox{\hyperlink{class_beam_1_1_wav_reader_a4e25dd9b2efd3e0a047ede3e7a5ae980}{get\+Sample\+Rate}} () const
\item 
uint16\+\_\+t \doxymbox{\hyperlink{class_beam_1_1_wav_reader_a7d216a1914a2c3d94f25dd728103ef8c}{get\+Channels}} () const
\end{DoxyCompactItemize}
\doxysubsubsection*{Private Attributes}
\begin{DoxyCompactItemize}
\item 
std::\+ifstream \doxymbox{\hyperlink{class_beam_1_1_wav_reader_a907cb2f0a02cb35404b64c83bc255af1}{m\+\_\+file}}
\item 
uint32\+\_\+t \doxymbox{\hyperlink{class_beam_1_1_wav_reader_ad6b1bf507beb872bc88d8653f5e97fc7}{m\+\_\+sample\+Rate}}
\item 
uint16\+\_\+t \doxymbox{\hyperlink{class_beam_1_1_wav_reader_a9d338f1e92fd6deff4ba9001dfe36e6f}{m\+\_\+channels}}
\item 
uint16\+\_\+t \doxymbox{\hyperlink{class_beam_1_1_wav_reader_a6a8e0388b157c1e105decfeb68bd21c2}{m\+\_\+bits\+Per\+Sample}}
\item 
uint16\+\_\+t \doxymbox{\hyperlink{class_beam_1_1_wav_reader_a1ef1515c74e66c5a9ff543639f9ed11e}{m\+\_\+format\+Tag}}
\item 
uint32\+\_\+t \doxymbox{\hyperlink{class_beam_1_1_wav_reader_ab412037c65d5e7c1a28cd5c195d3b87a}{m\+\_\+data\+Size}}
\item 
uint32\+\_\+t \doxymbox{\hyperlink{class_beam_1_1_wav_reader_ae11c806f1f18bb1b151294d158124596}{m\+\_\+data\+Offset}}
\end{DoxyCompactItemize}


\label{doc-constructors}
\Hypertarget{class_beam_1_1_wav_reader_doc-constructors}
\doxysubsection{Constructor \& Destructor Documentation}
\Hypertarget{class_beam_1_1_wav_reader_a495edcb749c0ed8d9337adea76f3f3ff}\index{Beam::WavReader@{Beam::WavReader}!WavReader@{WavReader}}
\index{WavReader@{WavReader}!Beam::WavReader@{Beam::WavReader}}
\doxysubsubsection{\texorpdfstring{WavReader()}{WavReader()}}
{\footnotesize\ttfamily \label{class_beam_1_1_wav_reader_a495edcb749c0ed8d9337adea76f3f3ff} 
Beam::\+\+Wav\+Reader::\+\+Wav\+Reader (\begin{DoxyParamCaption}{}{}\end{DoxyParamCaption})\hspace{0.3cm}{\ttfamily [inline]}}



\label{doc-func-members}
\Hypertarget{class_beam_1_1_wav_reader_doc-func-members}
\doxysubsection{Member Function Documentation}
\Hypertarget{class_beam_1_1_wav_reader_a7d216a1914a2c3d94f25dd728103ef8c}\index{Beam::WavReader@{Beam::WavReader}!getChannels@{getChannels}}
\index{getChannels@{getChannels}!Beam::WavReader@{Beam::WavReader}}
\doxysubsubsection{\texorpdfstring{getChannels()}{getChannels()}}
{\footnotesize\ttfamily \label{class_beam_1_1_wav_reader_a7d216a1914a2c3d94f25dd728103ef8c} 
uint16\+\_\+t Beam::\+\+Wav\+Reader::\+get\+Channels (\begin{DoxyParamCaption}{}{}\end{DoxyParamCaption}) const\hspace{0.3cm}{\ttfamily [inline]}}

\Hypertarget{class_beam_1_1_wav_reader_a4e25dd9b2efd3e0a047ede3e7a5ae980}\index{Beam::WavReader@{Beam::WavReader}!getSampleRate@{getSampleRate}}
\index{getSampleRate@{getSampleRate}!Beam::WavReader@{Beam::WavReader}}
\doxysubsubsection{\texorpdfstring{getSampleRate()}{getSampleRate()}}
{\footnotesize\ttfamily \label{class_beam_1_1_wav_reader_a4e25dd9b2efd3e0a047ede3e7a5ae980} 
uint32\+\_\+t Beam::\+\+Wav\+Reader::\+get\+Sample\+Rate (\begin{DoxyParamCaption}{}{}\end{DoxyParamCaption}) const\hspace{0.3cm}{\ttfamily [inline]}}

\Hypertarget{class_beam_1_1_wav_reader_a54dc2bfd8c1b1773df1f444211da7f6e}\index{Beam::WavReader@{Beam::WavReader}!open@{open}}
\index{open@{open}!Beam::WavReader@{Beam::WavReader}}
\doxysubsubsection{\texorpdfstring{open()}{open()}}
{\footnotesize\ttfamily \label{class_beam_1_1_wav_reader_a54dc2bfd8c1b1773df1f444211da7f6e} 
bool Beam::\+\+Wav\+Reader::\+open (\begin{DoxyParamCaption}\item[{const std::\+string \&}]{file\+Path}{}\end{DoxyParamCaption})\hspace{0.3cm}{\ttfamily [inline]}}

\Hypertarget{class_beam_1_1_wav_reader_a48fbbce631692e26e952393c21a379bd}\index{Beam::WavReader@{Beam::WavReader}!readFrames@{readFrames}}
\index{readFrames@{readFrames}!Beam::WavReader@{Beam::WavReader}}
\doxysubsubsection{\texorpdfstring{readFrames()}{readFrames()}}
{\footnotesize\ttfamily \label{class_beam_1_1_wav_reader_a48fbbce631692e26e952393c21a379bd} 
size\+\_\+t Beam::\+\+Wav\+Reader::\+read\+Frames (\begin{DoxyParamCaption}\item[{float \texorpdfstring{$\ast$}{*}}]{buffer}{, }\item[{size\+\_\+t}]{frames}{, }\item[{int}]{dest\+Channels}{}\end{DoxyParamCaption})\hspace{0.3cm}{\ttfamily [inline]}}

\Hypertarget{class_beam_1_1_wav_reader_a86c1c481afc06a74ee5e24613b420e7d}\index{Beam::WavReader@{Beam::WavReader}!seek@{seek}}
\index{seek@{seek}!Beam::WavReader@{Beam::WavReader}}
\doxysubsubsection{\texorpdfstring{seek()}{seek()}}
{\footnotesize\ttfamily \label{class_beam_1_1_wav_reader_a86c1c481afc06a74ee5e24613b420e7d} 
void Beam::\+\+Wav\+Reader::\+seek (\begin{DoxyParamCaption}\item[{size\+\_\+t}]{frame}{}\end{DoxyParamCaption})\hspace{0.3cm}{\ttfamily [inline]}}



\label{doc-variable-members}
\Hypertarget{class_beam_1_1_wav_reader_doc-variable-members}
\doxysubsection{Member Data Documentation}
\Hypertarget{class_beam_1_1_wav_reader_a6a8e0388b157c1e105decfeb68bd21c2}\index{Beam::WavReader@{Beam::WavReader}!m\_bitsPerSample@{m\_bitsPerSample}}
\index{m\_bitsPerSample@{m\_bitsPerSample}!Beam::WavReader@{Beam::WavReader}}
\doxysubsubsection{\texorpdfstring{m\_bitsPerSample}{m\_bitsPerSample}}
{\footnotesize\ttfamily \label{class_beam_1_1_wav_reader_a6a8e0388b157c1e105decfeb68bd21c2} 
uint16\+\_\+t Beam::\+\+Wav\+Reader::\+m\+\_\+bits\+Per\+Sample\hspace{0.3cm}{\ttfamily [private]}}

\Hypertarget{class_beam_1_1_wav_reader_a9d338f1e92fd6deff4ba9001dfe36e6f}\index{Beam::WavReader@{Beam::WavReader}!m\_channels@{m\_channels}}
\index{m\_channels@{m\_channels}!Beam::WavReader@{Beam::WavReader}}
\doxysubsubsection{\texorpdfstring{m\_channels}{m\_channels}}
{\footnotesize\ttfamily \label{class_beam_1_1_wav_reader_a9d338f1e92fd6deff4ba9001dfe36e6f} 
uint16\+\_\+t Beam::\+\+Wav\+Reader::\+m\+\_\+channels\hspace{0.3cm}{\ttfamily [private]}}

\Hypertarget{class_beam_1_1_wav_reader_ae11c806f1f18bb1b151294d158124596}\index{Beam::WavReader@{Beam::WavReader}!m\_dataOffset@{m\_dataOffset}}
\index{m\_dataOffset@{m\_dataOffset}!Beam::WavReader@{Beam::WavReader}}
\doxysubsubsection{\texorpdfstring{m\_dataOffset}{m\_dataOffset}}
{\footnotesize\ttfamily \label{class_beam_1_1_wav_reader_ae11c806f1f18bb1b151294d158124596} 
uint32\+\_\+t Beam::\+\+Wav\+Reader::\+m\+\_\+data\+Offset\hspace{0.3cm}{\ttfamily [private]}}

\Hypertarget{class_beam_1_1_wav_reader_ab412037c65d5e7c1a28cd5c195d3b87a}\index{Beam::WavReader@{Beam::WavReader}!m\_dataSize@{m\_dataSize}}
\index{m\_dataSize@{m\_dataSize}!Beam::WavReader@{Beam::WavReader}}
\doxysubsubsection{\texorpdfstring{m\_dataSize}{m\_dataSize}}
{\footnotesize\ttfamily \label{class_beam_1_1_wav_reader_ab412037c65d5e7c1a28cd5c195d3b87a} 
uint32\+\_\+t Beam::\+\+Wav\+Reader::\+m\+\_\+data\+Size\hspace{0.3cm}{\ttfamily [private]}}

\Hypertarget{class_beam_1_1_wav_reader_a907cb2f0a02cb35404b64c83bc255af1}\index{Beam::WavReader@{Beam::WavReader}!m\_file@{m\_file}}
\index{m\_file@{m\_file}!Beam::WavReader@{Beam::WavReader}}
\doxysubsubsection{\texorpdfstring{m\_file}{m\_file}}
{\footnotesize\ttfamily \label{class_beam_1_1_wav_reader_a907cb2f0a02cb35404b64c83bc255af1} 
std::\+ifstream Beam::\+\+Wav\+Reader::\+m\+\_\+file\hspace{0.3cm}{\ttfamily [private]}}

\Hypertarget{class_beam_1_1_wav_reader_a1ef1515c74e66c5a9ff543639f9ed11e}\index{Beam::WavReader@{Beam::WavReader}!m\_formatTag@{m\_formatTag}}
\index{m\_formatTag@{m\_formatTag}!Beam::WavReader@{Beam::WavReader}}
\doxysubsubsection{\texorpdfstring{m\_formatTag}{m\_formatTag}}
{\footnotesize\ttfamily \label{class_beam_1_1_wav_reader_a1ef1515c74e66c5a9ff543639f9ed11e} 
uint16\+\_\+t Beam::\+\+Wav\+Reader::\+m\+\_\+format\+Tag\hspace{0.3cm}{\ttfamily [private]}}

\Hypertarget{class_beam_1_1_wav_reader_ad6b1bf507beb872bc88d8653f5e97fc7}\index{Beam::WavReader@{Beam::WavReader}!m\_sampleRate@{m\_sampleRate}}
\index{m\_sampleRate@{m\_sampleRate}!Beam::WavReader@{Beam::WavReader}}
\doxysubsubsection{\texorpdfstring{m\_sampleRate}{m\_sampleRate}}
{\footnotesize\ttfamily \label{class_beam_1_1_wav_reader_ad6b1bf507beb872bc88d8653f5e97fc7} 
uint32\+\_\+t Beam::\+\+Wav\+Reader::\+m\+\_\+sample\+Rate\hspace{0.3cm}{\ttfamily [private]}}



The documentation for this class was generated from the following file:\+\begin{DoxyCompactItemize}
\item 
src/\+dsp/\+\doxymbox{\hyperlink{wav__reader_8hpp}{wav\+\_\+reader.\+hpp}}\end{DoxyCompactItemize}

\doxysection{Beam::\+Wav\+Writer Class Reference}
\hypertarget{class_beam_1_1_wav_writer}{}\label{class_beam_1_1_wav_writer}\index{Beam::WavWriter@{Beam::WavWriter}}


{\ttfamily \+\#include $<$wav\+\_\+writer.\+hpp$>$}

\doxysubsubsection*{Static Public Member Functions}
\begin{DoxyCompactItemize}
\item 
static bool \doxymbox{\hyperlink{class_beam_1_1_wav_writer_a907da59c59b79311cbe891d78c12ce6e}{write}} (const std::\+string \&filename, const float \texorpdfstring{$\ast$}{*}buffer, size\+\_\+t num\+Samples, int sample\+Rate, int channels)
\end{DoxyCompactItemize}


\label{doc-func-members}
\Hypertarget{class_beam_1_1_wav_writer_doc-func-members}
\doxysubsection{Member Function Documentation}
\Hypertarget{class_beam_1_1_wav_writer_a907da59c59b79311cbe891d78c12ce6e}\index{Beam::WavWriter@{Beam::WavWriter}!write@{write}}
\index{write@{write}!Beam::WavWriter@{Beam::WavWriter}}
\doxysubsubsection{\texorpdfstring{write()}{write()}}
{\footnotesize\ttfamily \label{class_beam_1_1_wav_writer_a907da59c59b79311cbe891d78c12ce6e} 
bool Beam::\+\+Wav\+Writer::\+write (\begin{DoxyParamCaption}\item[{const std::\+string \&}]{filename}{, }\item[{const float \texorpdfstring{$\ast$}{*}}]{buffer}{, }\item[{size\+\_\+t}]{num\+Samples}{, }\item[{int}]{sample\+Rate}{, }\item[{int}]{channels}{}\end{DoxyParamCaption})\hspace{0.3cm}{\ttfamily [inline]}, {\ttfamily [static]}}



The documentation for this class was generated from the following file:\+\begin{DoxyCompactItemize}
\item 
src/\+dsp/\+\doxymbox{\hyperlink{wav__writer_8hpp}{wav\+\_\+writer.\+hpp}}\end{DoxyCompactItemize}

\doxysection{Beam::\+Workspace Class Reference}
\hypertarget{class_beam_1_1_workspace}{}\label{class_beam_1_1_workspace}\index{Beam::Workspace@{Beam::Workspace}}


{\ttfamily \+\#include $<$workspace.\+hpp$>$}

Inheritance diagram for Beam::\+Workspace:\+\begin{figure}[H]
\begin{center}
\leavevmode
\includegraphics[height=2.000000cm]{class_beam_1_1_workspace}
\end{center}
\end{figure}
\doxysubsubsection*{Public Member Functions}
\begin{DoxyCompactItemize}
\item 
\doxymbox{\hyperlink{class_beam_1_1_workspace_a1cb1b67c1754b9082dde84b7db95729e}{Workspace}} (std::\+shared\+\_\+ptr$<$ \doxymbox{\hyperlink{class_beam_1_1_flux_project}{Flux\+Project}} $>$ project)
\item 
nlohmann::\+json \doxymbox{\hyperlink{class_beam_1_1_workspace_a5f75e8e118db827ac6e2b132ce3c511d}{serialize}} () const
\item 
void \doxymbox{\hyperlink{class_beam_1_1_workspace_a8959768175035442b92bda55f7ba1c2d}{deserialize}} (const nlohmann::\+json \&data)
\item 
void \doxymbox{\hyperlink{class_beam_1_1_workspace_a021920fd8b910cadf5286ce92b75ac79}{set\+Visible}} (bool visible)
\item 
void \doxymbox{\hyperlink{class_beam_1_1_workspace_a6be84daa0a4def0d2adfc162c45d9331}{render}} (\doxymbox{\hyperlink{class_beam_1_1_quad_batcher}{Quad\+Batcher}} \&batcher) override
\item 
void \doxymbox{\hyperlink{class_beam_1_1_workspace_acca5d578980d2463c579161ab6e52b29}{add\+Track}} (const std::\+string \&file\+Path, float x, float y, \doxymbox{\hyperlink{class_beam_1_1_audio_engine}{Audio\+Engine}} \&engine)
\item 
void \doxymbox{\hyperlink{class_beam_1_1_workspace_a02d36df76f94244013d15221d195f5a2}{add\+FX}} (const std::\+string \&type, float x, float y)
\item 
void \doxymbox{\hyperlink{class_beam_1_1_workspace_a45597a1c3ca82eb3848628fb36cdd461}{start\+Cable\+Drag}} (\doxymbox{\hyperlink{class_beam_1_1_port}{Port}} \texorpdfstring{$\ast$}{*}p)
\item 
void \doxymbox{\hyperlink{class_beam_1_1_workspace_a4ac39a74b5b2b887a520ff7eb0fdec0c}{connect\+Ports}} (\doxymbox{\hyperlink{class_beam_1_1_port}{Port}} \texorpdfstring{$\ast$}{*}p1, \doxymbox{\hyperlink{class_beam_1_1_port}{Port}} \texorpdfstring{$\ast$}{*}p2)
\item 
bool \doxymbox{\hyperlink{class_beam_1_1_workspace_a2d287ac679c39e3bb74a6806c627c610}{on\+Mouse\+Down}} (float x, float y, int button) override
\item 
bool \doxymbox{\hyperlink{class_beam_1_1_workspace_aa04ebe3cc67dbd9675a6a6eca5506547}{on\+Mouse\+Up}} (float x, float y, int button) override
\item 
bool \doxymbox{\hyperlink{class_beam_1_1_workspace_ae030281b49f7c11a28517b4deaf69b99}{on\+Mouse\+Move}} (float x, float y) override
\end{DoxyCompactItemize}
\doxysubsection*{Public Member Functions inherited from \doxymbox{\hyperlink{class_beam_1_1_component}{Beam::\+\+Component}}}
\begin{DoxyCompactItemize}
\item 
virtual \doxymbox{\hyperlink{class_beam_1_1_component_af9d734d649978e027412a87bc54362cd}{\texorpdfstring{$\sim$}{\string~}\+Component}} ()=default
\item 
virtual void \doxymbox{\hyperlink{class_beam_1_1_component_ad3d3fb19d25b4371d07620567970a158}{update}} (float dt)
\item 
virtual void \doxymbox{\hyperlink{class_beam_1_1_component_a6865b1f22388af467bf6c789120fac05}{set\+Bounds}} (float x, float y, float w, float h)
\item 
const \doxymbox{\hyperlink{struct_beam_1_1_rect}{Rect}} \& \doxymbox{\hyperlink{class_beam_1_1_component_a5746dbc69d5b0adb4cffbcf920936d00}{get\+Bounds}} () const
\item 
void \doxymbox{\hyperlink{class_beam_1_1_component_a00d4e2dfa7703e59d6486852321dbdf1}{set\+Draggable}} (bool draggable)
\item 
void \doxymbox{\hyperlink{class_beam_1_1_component_aca7b02d1dddf7cd20378db9e3242fb84}{start\+Dragging}} (float x, float y)
\end{DoxyCompactItemize}
\doxysubsubsection*{Private Attributes}
\begin{DoxyCompactItemize}
\item 
std::\+shared\+\_\+ptr$<$ \doxymbox{\hyperlink{class_beam_1_1_flux_project}{Flux\+Project}} $>$ \doxymbox{\hyperlink{class_beam_1_1_workspace_aaad17331a0dbbe3ce854b2bac8d97f2c}{m\+\_\+project}}
\item 
std::\+vector$<$ std::\+shared\+\_\+ptr$<$ \doxymbox{\hyperlink{class_beam_1_1_component}{Component}} $>$ $>$ \doxymbox{\hyperlink{class_beam_1_1_workspace_aab79b91f836a67545e1432d3f77af9e8}{m\+\_\+modules}}
\item 
std::\+vector$<$ \doxymbox{\hyperlink{struct_beam_1_1_cable}{Cable}} $>$ \doxymbox{\hyperlink{class_beam_1_1_workspace_a9f2831c5c89695ea1f4805d39c777415}{m\+\_\+cables}}
\item 
float \doxymbox{\hyperlink{class_beam_1_1_workspace_a3003006c89fd162e8649d664f1347edb}{m\+\_\+panX}} = 0
\item 
float \doxymbox{\hyperlink{class_beam_1_1_workspace_a2b0b9919abb15280141142e9bbf2ec70}{m\+\_\+panY}} = 0
\item 
bool \doxymbox{\hyperlink{class_beam_1_1_workspace_ab41c4919edcced005a0a399ee1e89796}{m\+\_\+is\+Panning}} = false
\item 
float \doxymbox{\hyperlink{class_beam_1_1_workspace_a28711dedfa0ddea4f3286b1c8ca026c6}{m\+\_\+last\+MouseX}} = 0
\item 
float \doxymbox{\hyperlink{class_beam_1_1_workspace_af5e49785a110e6bbf05c74f8ee859140}{m\+\_\+last\+MouseY}} = 0
\item 
bool \doxymbox{\hyperlink{class_beam_1_1_workspace_aa0d7f1088677837479f81fb8a84af8a1}{m\+\_\+is\+Dragging\+Cable}} = false
\item 
\doxymbox{\hyperlink{class_beam_1_1_port}{Port}} \texorpdfstring{$\ast$}{*} \doxymbox{\hyperlink{class_beam_1_1_workspace_a038f69a869c9cfaf59a882a960aa812f}{m\+\_\+active\+Port}} = nullptr
\end{DoxyCompactItemize}
\doxysubsubsection*{Additional Inherited Members}
\doxysubsection*{Protected Attributes inherited from \doxymbox{\hyperlink{class_beam_1_1_component}{Beam::\+\+Component}}}
\begin{DoxyCompactItemize}
\item 
\doxymbox{\hyperlink{struct_beam_1_1_rect}{Rect}} \doxymbox{\hyperlink{class_beam_1_1_component_a4f1ec4a5fb168c39a6c18f958b2b1495}{m\+\_\+bounds}} \{0, 0, 0, 0\}
\item 
bool \doxymbox{\hyperlink{class_beam_1_1_component_adc07913aed6ddadf1c730e7b3bb599cf}{m\+\_\+is\+Visible}} = true
\item 
bool \doxymbox{\hyperlink{class_beam_1_1_component_a0bf77b204ae374a14b5a6d7e5a3c13c6}{m\+\_\+is\+Enabled}} = true
\item 
bool \doxymbox{\hyperlink{class_beam_1_1_component_a9646efcaa9540a26a387f5da9aae4bde}{m\+\_\+is\+Draggable}} = false
\item 
bool \doxymbox{\hyperlink{class_beam_1_1_component_ab03af9a9743acf040f38e3fb11f8dc14}{m\+\_\+is\+Dragging}} = false
\item 
float \doxymbox{\hyperlink{class_beam_1_1_component_a7110b2b9dc235f724bf4689569266a63}{m\+\_\+last\+MouseX}} = 0
\item 
float \doxymbox{\hyperlink{class_beam_1_1_component_a768931a0f51394bf011f821f6ed2efe9}{m\+\_\+last\+MouseY}} = 0
\end{DoxyCompactItemize}


\label{doc-constructors}
\Hypertarget{class_beam_1_1_workspace_doc-constructors}
\doxysubsection{Constructor \& Destructor Documentation}
\Hypertarget{class_beam_1_1_workspace_a1cb1b67c1754b9082dde84b7db95729e}\index{Beam::Workspace@{Beam::Workspace}!Workspace@{Workspace}}
\index{Workspace@{Workspace}!Beam::Workspace@{Beam::Workspace}}
\doxysubsubsection{\texorpdfstring{Workspace()}{Workspace()}}
{\footnotesize\ttfamily \label{class_beam_1_1_workspace_a1cb1b67c1754b9082dde84b7db95729e} 
Beam::\+\+Workspace::\+\+Workspace (\begin{DoxyParamCaption}\item[{std::\+shared\+\_\+ptr$<$ \doxymbox{\hyperlink{class_beam_1_1_flux_project}{Flux\+Project}} $>$}]{project}{}\end{DoxyParamCaption})\hspace{0.3cm}{\ttfamily [inline]}}



\label{doc-func-members}
\Hypertarget{class_beam_1_1_workspace_doc-func-members}
\doxysubsection{Member Function Documentation}
\Hypertarget{class_beam_1_1_workspace_a02d36df76f94244013d15221d195f5a2}\index{Beam::Workspace@{Beam::Workspace}!addFX@{addFX}}
\index{addFX@{addFX}!Beam::Workspace@{Beam::Workspace}}
\doxysubsubsection{\texorpdfstring{addFX()}{addFX()}}
{\footnotesize\ttfamily \label{class_beam_1_1_workspace_a02d36df76f94244013d15221d195f5a2} 
void Beam::\+\+Workspace::\+add\+FX (\begin{DoxyParamCaption}\item[{const std::\+string \&}]{type}{, }\item[{float}]{x}{, }\item[{float}]{y}{}\end{DoxyParamCaption})\hspace{0.3cm}{\ttfamily [inline]}}

\Hypertarget{class_beam_1_1_workspace_acca5d578980d2463c579161ab6e52b29}\index{Beam::Workspace@{Beam::Workspace}!addTrack@{addTrack}}
\index{addTrack@{addTrack}!Beam::Workspace@{Beam::Workspace}}
\doxysubsubsection{\texorpdfstring{addTrack()}{addTrack()}}
{\footnotesize\ttfamily \label{class_beam_1_1_workspace_acca5d578980d2463c579161ab6e52b29} 
void Beam::\+\+Workspace::\+add\+Track (\begin{DoxyParamCaption}\item[{const std::\+string \&}]{file\+Path}{, }\item[{float}]{x}{, }\item[{float}]{y}{, }\item[{\doxymbox{\hyperlink{class_beam_1_1_audio_engine}{Audio\+Engine}} \&}]{engine}{}\end{DoxyParamCaption})\hspace{0.3cm}{\ttfamily [inline]}}

\Hypertarget{class_beam_1_1_workspace_a4ac39a74b5b2b887a520ff7eb0fdec0c}\index{Beam::Workspace@{Beam::Workspace}!connectPorts@{connectPorts}}
\index{connectPorts@{connectPorts}!Beam::Workspace@{Beam::Workspace}}
\doxysubsubsection{\texorpdfstring{connectPorts()}{connectPorts()}}
{\footnotesize\ttfamily \label{class_beam_1_1_workspace_a4ac39a74b5b2b887a520ff7eb0fdec0c} 
void Beam::\+\+Workspace::\+connect\+Ports (\begin{DoxyParamCaption}\item[{\doxymbox{\hyperlink{class_beam_1_1_port}{Port}} \texorpdfstring{$\ast$}{*}}]{p1}{, }\item[{\doxymbox{\hyperlink{class_beam_1_1_port}{Port}} \texorpdfstring{$\ast$}{*}}]{p2}{}\end{DoxyParamCaption})\hspace{0.3cm}{\ttfamily [inline]}}

\Hypertarget{class_beam_1_1_workspace_a8959768175035442b92bda55f7ba1c2d}\index{Beam::Workspace@{Beam::Workspace}!deserialize@{deserialize}}
\index{deserialize@{deserialize}!Beam::Workspace@{Beam::Workspace}}
\doxysubsubsection{\texorpdfstring{deserialize()}{deserialize()}}
{\footnotesize\ttfamily \label{class_beam_1_1_workspace_a8959768175035442b92bda55f7ba1c2d} 
void Beam::\+\+Workspace::\+deserialize (\begin{DoxyParamCaption}\item[{const nlohmann::\+json \&}]{data}{}\end{DoxyParamCaption})\hspace{0.3cm}{\ttfamily [inline]}}

\Hypertarget{class_beam_1_1_workspace_a2d287ac679c39e3bb74a6806c627c610}\index{Beam::Workspace@{Beam::Workspace}!onMouseDown@{onMouseDown}}
\index{onMouseDown@{onMouseDown}!Beam::Workspace@{Beam::Workspace}}
\doxysubsubsection{\texorpdfstring{onMouseDown()}{onMouseDown()}}
{\footnotesize\ttfamily \label{class_beam_1_1_workspace_a2d287ac679c39e3bb74a6806c627c610} 
bool Beam::\+\+Workspace::\+on\+Mouse\+Down (\begin{DoxyParamCaption}\item[{float}]{x}{, }\item[{float}]{y}{, }\item[{int}]{button}{}\end{DoxyParamCaption})\hspace{0.3cm}{\ttfamily [inline]}, {\ttfamily [override]}, {\ttfamily [virtual]}}



Reimplemented from \doxymbox{\hyperlink{class_beam_1_1_component_aec1da33d2d6e3d4e7dd6708309264e76}{Beam::\+\+Component}}.

\Hypertarget{class_beam_1_1_workspace_ae030281b49f7c11a28517b4deaf69b99}\index{Beam::Workspace@{Beam::Workspace}!onMouseMove@{onMouseMove}}
\index{onMouseMove@{onMouseMove}!Beam::Workspace@{Beam::Workspace}}
\doxysubsubsection{\texorpdfstring{onMouseMove()}{onMouseMove()}}
{\footnotesize\ttfamily \label{class_beam_1_1_workspace_ae030281b49f7c11a28517b4deaf69b99} 
bool Beam::\+\+Workspace::\+on\+Mouse\+Move (\begin{DoxyParamCaption}\item[{float}]{x}{, }\item[{float}]{y}{}\end{DoxyParamCaption})\hspace{0.3cm}{\ttfamily [inline]}, {\ttfamily [override]}, {\ttfamily [virtual]}}



Reimplemented from \doxymbox{\hyperlink{class_beam_1_1_component_a9d8e5970783d315044277a1228659e6c}{Beam::\+\+Component}}.

\Hypertarget{class_beam_1_1_workspace_aa04ebe3cc67dbd9675a6a6eca5506547}\index{Beam::Workspace@{Beam::Workspace}!onMouseUp@{onMouseUp}}
\index{onMouseUp@{onMouseUp}!Beam::Workspace@{Beam::Workspace}}
\doxysubsubsection{\texorpdfstring{onMouseUp()}{onMouseUp()}}
{\footnotesize\ttfamily \label{class_beam_1_1_workspace_aa04ebe3cc67dbd9675a6a6eca5506547} 
bool Beam::\+\+Workspace::\+on\+Mouse\+Up (\begin{DoxyParamCaption}\item[{float}]{x}{, }\item[{float}]{y}{, }\item[{int}]{button}{}\end{DoxyParamCaption})\hspace{0.3cm}{\ttfamily [inline]}, {\ttfamily [override]}, {\ttfamily [virtual]}}



Reimplemented from \doxymbox{\hyperlink{class_beam_1_1_component_ae36b8e9d70e8f9a1b9ba81c23c54d5c8}{Beam::\+\+Component}}.

\Hypertarget{class_beam_1_1_workspace_a6be84daa0a4def0d2adfc162c45d9331}\index{Beam::Workspace@{Beam::Workspace}!render@{render}}
\index{render@{render}!Beam::Workspace@{Beam::Workspace}}
\doxysubsubsection{\texorpdfstring{render()}{render()}}
{\footnotesize\ttfamily \label{class_beam_1_1_workspace_a6be84daa0a4def0d2adfc162c45d9331} 
void Beam::\+\+Workspace::\+render (\begin{DoxyParamCaption}\item[{\doxymbox{\hyperlink{class_beam_1_1_quad_batcher}{Quad\+Batcher}} \&}]{batcher}{}\end{DoxyParamCaption})\hspace{0.3cm}{\ttfamily [inline]}, {\ttfamily [override]}, {\ttfamily [virtual]}}



Implements \doxymbox{\hyperlink{class_beam_1_1_component_a2ed6b7841a25bd992bd46b822311ef1d}{Beam::\+\+Component}}.

\Hypertarget{class_beam_1_1_workspace_a5f75e8e118db827ac6e2b132ce3c511d}\index{Beam::Workspace@{Beam::Workspace}!serialize@{serialize}}
\index{serialize@{serialize}!Beam::Workspace@{Beam::Workspace}}
\doxysubsubsection{\texorpdfstring{serialize()}{serialize()}}
{\footnotesize\ttfamily \label{class_beam_1_1_workspace_a5f75e8e118db827ac6e2b132ce3c511d} 
nlohmann::\+json Beam::\+\+Workspace::\+serialize (\begin{DoxyParamCaption}{}{}\end{DoxyParamCaption}) const\hspace{0.3cm}{\ttfamily [inline]}}

\Hypertarget{class_beam_1_1_workspace_a021920fd8b910cadf5286ce92b75ac79}\index{Beam::Workspace@{Beam::Workspace}!setVisible@{setVisible}}
\index{setVisible@{setVisible}!Beam::Workspace@{Beam::Workspace}}
\doxysubsubsection{\texorpdfstring{setVisible()}{setVisible()}}
{\footnotesize\ttfamily \label{class_beam_1_1_workspace_a021920fd8b910cadf5286ce92b75ac79} 
void Beam::\+\+Workspace::\+set\+Visible (\begin{DoxyParamCaption}\item[{bool}]{visible}{}\end{DoxyParamCaption})\hspace{0.3cm}{\ttfamily [inline]}}

\Hypertarget{class_beam_1_1_workspace_a45597a1c3ca82eb3848628fb36cdd461}\index{Beam::Workspace@{Beam::Workspace}!startCableDrag@{startCableDrag}}
\index{startCableDrag@{startCableDrag}!Beam::Workspace@{Beam::Workspace}}
\doxysubsubsection{\texorpdfstring{startCableDrag()}{startCableDrag()}}
{\footnotesize\ttfamily \label{class_beam_1_1_workspace_a45597a1c3ca82eb3848628fb36cdd461} 
void Beam::\+\+Workspace::\+start\+Cable\+Drag (\begin{DoxyParamCaption}\item[{\doxymbox{\hyperlink{class_beam_1_1_port}{Port}} \texorpdfstring{$\ast$}{*}}]{p}{}\end{DoxyParamCaption})\hspace{0.3cm}{\ttfamily [inline]}}



\label{doc-variable-members}
\Hypertarget{class_beam_1_1_workspace_doc-variable-members}
\doxysubsection{Member Data Documentation}
\Hypertarget{class_beam_1_1_workspace_a038f69a869c9cfaf59a882a960aa812f}\index{Beam::Workspace@{Beam::Workspace}!m\_activePort@{m\_activePort}}
\index{m\_activePort@{m\_activePort}!Beam::Workspace@{Beam::Workspace}}
\doxysubsubsection{\texorpdfstring{m\_activePort}{m\_activePort}}
{\footnotesize\ttfamily \label{class_beam_1_1_workspace_a038f69a869c9cfaf59a882a960aa812f} 
\doxymbox{\hyperlink{class_beam_1_1_port}{Port}}\texorpdfstring{$\ast$}{*} Beam::\+\+Workspace::\+m\+\_\+active\+Port = nullptr\hspace{0.3cm}{\ttfamily [private]}}

\Hypertarget{class_beam_1_1_workspace_a9f2831c5c89695ea1f4805d39c777415}\index{Beam::Workspace@{Beam::Workspace}!m\_cables@{m\_cables}}
\index{m\_cables@{m\_cables}!Beam::Workspace@{Beam::Workspace}}
\doxysubsubsection{\texorpdfstring{m\_cables}{m\_cables}}
{\footnotesize\ttfamily \label{class_beam_1_1_workspace_a9f2831c5c89695ea1f4805d39c777415} 
std::\+vector$<$\doxymbox{\hyperlink{struct_beam_1_1_cable}{Cable}}$>$ Beam::\+\+Workspace::\+m\+\_\+cables\hspace{0.3cm}{\ttfamily [private]}}

\Hypertarget{class_beam_1_1_workspace_aa0d7f1088677837479f81fb8a84af8a1}\index{Beam::Workspace@{Beam::Workspace}!m\_isDraggingCable@{m\_isDraggingCable}}
\index{m\_isDraggingCable@{m\_isDraggingCable}!Beam::Workspace@{Beam::Workspace}}
\doxysubsubsection{\texorpdfstring{m\_isDraggingCable}{m\_isDraggingCable}}
{\footnotesize\ttfamily \label{class_beam_1_1_workspace_aa0d7f1088677837479f81fb8a84af8a1} 
bool Beam::\+\+Workspace::\+m\+\_\+is\+Dragging\+Cable = false\hspace{0.3cm}{\ttfamily [private]}}

\Hypertarget{class_beam_1_1_workspace_ab41c4919edcced005a0a399ee1e89796}\index{Beam::Workspace@{Beam::Workspace}!m\_isPanning@{m\_isPanning}}
\index{m\_isPanning@{m\_isPanning}!Beam::Workspace@{Beam::Workspace}}
\doxysubsubsection{\texorpdfstring{m\_isPanning}{m\_isPanning}}
{\footnotesize\ttfamily \label{class_beam_1_1_workspace_ab41c4919edcced005a0a399ee1e89796} 
bool Beam::\+\+Workspace::\+m\+\_\+is\+Panning = false\hspace{0.3cm}{\ttfamily [private]}}

\Hypertarget{class_beam_1_1_workspace_a28711dedfa0ddea4f3286b1c8ca026c6}\index{Beam::Workspace@{Beam::Workspace}!m\_lastMouseX@{m\_lastMouseX}}
\index{m\_lastMouseX@{m\_lastMouseX}!Beam::Workspace@{Beam::Workspace}}
\doxysubsubsection{\texorpdfstring{m\_lastMouseX}{m\_lastMouseX}}
{\footnotesize\ttfamily \label{class_beam_1_1_workspace_a28711dedfa0ddea4f3286b1c8ca026c6} 
float Beam::\+\+Workspace::\+m\+\_\+last\+MouseX = 0\hspace{0.3cm}{\ttfamily [private]}}

\Hypertarget{class_beam_1_1_workspace_af5e49785a110e6bbf05c74f8ee859140}\index{Beam::Workspace@{Beam::Workspace}!m\_lastMouseY@{m\_lastMouseY}}
\index{m\_lastMouseY@{m\_lastMouseY}!Beam::Workspace@{Beam::Workspace}}
\doxysubsubsection{\texorpdfstring{m\_lastMouseY}{m\_lastMouseY}}
{\footnotesize\ttfamily \label{class_beam_1_1_workspace_af5e49785a110e6bbf05c74f8ee859140} 
float Beam::\+\+Workspace::\+m\+\_\+last\+MouseY = 0\hspace{0.3cm}{\ttfamily [private]}}

\Hypertarget{class_beam_1_1_workspace_aab79b91f836a67545e1432d3f77af9e8}\index{Beam::Workspace@{Beam::Workspace}!m\_modules@{m\_modules}}
\index{m\_modules@{m\_modules}!Beam::Workspace@{Beam::Workspace}}
\doxysubsubsection{\texorpdfstring{m\_modules}{m\_modules}}
{\footnotesize\ttfamily \label{class_beam_1_1_workspace_aab79b91f836a67545e1432d3f77af9e8} 
std::\+vector$<$std::\+shared\+\_\+ptr$<$\doxymbox{\hyperlink{class_beam_1_1_component}{Component}}$>$ $>$ Beam::\+\+Workspace::\+m\+\_\+modules\hspace{0.3cm}{\ttfamily [private]}}

\Hypertarget{class_beam_1_1_workspace_a3003006c89fd162e8649d664f1347edb}\index{Beam::Workspace@{Beam::Workspace}!m\_panX@{m\_panX}}
\index{m\_panX@{m\_panX}!Beam::Workspace@{Beam::Workspace}}
\doxysubsubsection{\texorpdfstring{m\_panX}{m\_panX}}
{\footnotesize\ttfamily \label{class_beam_1_1_workspace_a3003006c89fd162e8649d664f1347edb} 
float Beam::\+\+Workspace::\+m\+\_\+panX = 0\hspace{0.3cm}{\ttfamily [private]}}

\Hypertarget{class_beam_1_1_workspace_a2b0b9919abb15280141142e9bbf2ec70}\index{Beam::Workspace@{Beam::Workspace}!m\_panY@{m\_panY}}
\index{m\_panY@{m\_panY}!Beam::Workspace@{Beam::Workspace}}
\doxysubsubsection{\texorpdfstring{m\_panY}{m\_panY}}
{\footnotesize\ttfamily \label{class_beam_1_1_workspace_a2b0b9919abb15280141142e9bbf2ec70} 
float Beam::\+\+Workspace::\+m\+\_\+panY = 0\hspace{0.3cm}{\ttfamily [private]}}

\Hypertarget{class_beam_1_1_workspace_aaad17331a0dbbe3ce854b2bac8d97f2c}\index{Beam::Workspace@{Beam::Workspace}!m\_project@{m\_project}}
\index{m\_project@{m\_project}!Beam::Workspace@{Beam::Workspace}}
\doxysubsubsection{\texorpdfstring{m\_project}{m\_project}}
{\footnotesize\ttfamily \label{class_beam_1_1_workspace_aaad17331a0dbbe3ce854b2bac8d97f2c} 
std::\+shared\+\_\+ptr$<$\doxymbox{\hyperlink{class_beam_1_1_flux_project}{Flux\+Project}}$>$ Beam::\+\+Workspace::\+m\+\_\+project\hspace{0.3cm}{\ttfamily [private]}}



The documentation for this class was generated from the following file:\+\begin{DoxyCompactItemize}
\item 
src/\+ui/\+\doxymbox{\hyperlink{workspace_8hpp}{workspace.\+hpp}}\end{DoxyCompactItemize}

\doxysection{Beam::\+Analog\+Base::\+Wow\+Flutter\+Generator Class Reference}
\hypertarget{class_beam_1_1_analog_base_1_1_wow_flutter_generator}{}\label{class_beam_1_1_analog_base_1_1_wow_flutter_generator}\index{Beam::AnalogBase::WowFlutterGenerator@{Beam::AnalogBase::WowFlutterGenerator}}


Generates physical tape speed fluctuations.  




{\ttfamily \+\#include $<$analog\+\_\+base.\+hpp$>$}

\doxysubsubsection*{Public Member Functions}
\begin{DoxyCompactItemize}
\item 
\doxymbox{\hyperlink{class_beam_1_1_analog_base_1_1_wow_flutter_generator_a21a4ea7345e590bd5ec7ff859d4098e5}{Wow\+Flutter\+Generator}} (float sample\+Rate)
\item 
void \doxymbox{\hyperlink{class_beam_1_1_analog_base_1_1_wow_flutter_generator_a4466d9a5a8b28e0710a2d1243b2d4cdd}{set\+Intensity}} (float wow, float flutter)
\item 
float \doxymbox{\hyperlink{class_beam_1_1_analog_base_1_1_wow_flutter_generator_ac2f900604d5757f79382940da5448765}{next}} ()
\end{DoxyCompactItemize}
\doxysubsubsection*{Private Attributes}
\begin{DoxyCompactItemize}
\item 
float \doxymbox{\hyperlink{class_beam_1_1_analog_base_1_1_wow_flutter_generator_a70eea959271d10f748eba7cba326fb75}{m\+\_\+sample\+Rate}}
\item 
float \doxymbox{\hyperlink{class_beam_1_1_analog_base_1_1_wow_flutter_generator_a27e7ed8f25a051a95d6bbcf5e183706e}{m\+\_\+wow\+Depth}} = 0.\+0f
\item 
float \doxymbox{\hyperlink{class_beam_1_1_analog_base_1_1_wow_flutter_generator_a3261a050317fd9ff64b63f0070f3c739}{m\+\_\+flutter\+Depth}} = 0.\+0f
\item 
float \doxymbox{\hyperlink{class_beam_1_1_analog_base_1_1_wow_flutter_generator_a8a795c38bb2b81427fe4d7b29ffa054e}{m\+\_\+wow\+Phase}} = 0.\+0f
\item 
\doxymbox{\hyperlink{class_beam_1_1_analog_base_1_1_one_pole_filter}{One\+Pole\+Filter}} \doxymbox{\hyperlink{class_beam_1_1_analog_base_1_1_wow_flutter_generator_ab99ec737da74f383a2c1d3b65f24d85b}{m\+\_\+flutter\+LP}}
\item 
std::\+mt19937 \doxymbox{\hyperlink{class_beam_1_1_analog_base_1_1_wow_flutter_generator_ad7381452afa60d0c5f2707fa7580c707}{m\+\_\+rng}}
\end{DoxyCompactItemize}


\doxysubsection{Detailed Description}
Generates physical tape speed fluctuations. 

\label{doc-constructors}
\Hypertarget{class_beam_1_1_analog_base_1_1_wow_flutter_generator_doc-constructors}
\doxysubsection{Constructor \& Destructor Documentation}
\Hypertarget{class_beam_1_1_analog_base_1_1_wow_flutter_generator_a21a4ea7345e590bd5ec7ff859d4098e5}\index{Beam::AnalogBase::WowFlutterGenerator@{Beam::AnalogBase::WowFlutterGenerator}!WowFlutterGenerator@{WowFlutterGenerator}}
\index{WowFlutterGenerator@{WowFlutterGenerator}!Beam::AnalogBase::WowFlutterGenerator@{Beam::AnalogBase::WowFlutterGenerator}}
\doxysubsubsection{\texorpdfstring{WowFlutterGenerator()}{WowFlutterGenerator()}}
{\footnotesize\ttfamily \label{class_beam_1_1_analog_base_1_1_wow_flutter_generator_a21a4ea7345e590bd5ec7ff859d4098e5} 
Beam::\+\+Analog\+Base::\+\+Wow\+Flutter\+Generator::\+\+Wow\+Flutter\+Generator (\begin{DoxyParamCaption}\item[{float}]{sample\+Rate}{}\end{DoxyParamCaption})\hspace{0.3cm}{\ttfamily [inline]}}



\label{doc-func-members}
\Hypertarget{class_beam_1_1_analog_base_1_1_wow_flutter_generator_doc-func-members}
\doxysubsection{Member Function Documentation}
\Hypertarget{class_beam_1_1_analog_base_1_1_wow_flutter_generator_ac2f900604d5757f79382940da5448765}\index{Beam::AnalogBase::WowFlutterGenerator@{Beam::AnalogBase::WowFlutterGenerator}!next@{next}}
\index{next@{next}!Beam::AnalogBase::WowFlutterGenerator@{Beam::AnalogBase::WowFlutterGenerator}}
\doxysubsubsection{\texorpdfstring{next()}{next()}}
{\footnotesize\ttfamily \label{class_beam_1_1_analog_base_1_1_wow_flutter_generator_ac2f900604d5757f79382940da5448765} 
float Beam::\+\+Analog\+Base::\+\+Wow\+Flutter\+Generator::\+next (\begin{DoxyParamCaption}{}{}\end{DoxyParamCaption})\hspace{0.3cm}{\ttfamily [inline]}}

\Hypertarget{class_beam_1_1_analog_base_1_1_wow_flutter_generator_a4466d9a5a8b28e0710a2d1243b2d4cdd}\index{Beam::AnalogBase::WowFlutterGenerator@{Beam::AnalogBase::WowFlutterGenerator}!setIntensity@{setIntensity}}
\index{setIntensity@{setIntensity}!Beam::AnalogBase::WowFlutterGenerator@{Beam::AnalogBase::WowFlutterGenerator}}
\doxysubsubsection{\texorpdfstring{setIntensity()}{setIntensity()}}
{\footnotesize\ttfamily \label{class_beam_1_1_analog_base_1_1_wow_flutter_generator_a4466d9a5a8b28e0710a2d1243b2d4cdd} 
void Beam::\+\+Analog\+Base::\+\+Wow\+Flutter\+Generator::\+set\+Intensity (\begin{DoxyParamCaption}\item[{float}]{wow}{, }\item[{float}]{flutter}{}\end{DoxyParamCaption})\hspace{0.3cm}{\ttfamily [inline]}}



\label{doc-variable-members}
\Hypertarget{class_beam_1_1_analog_base_1_1_wow_flutter_generator_doc-variable-members}
\doxysubsection{Member Data Documentation}
\Hypertarget{class_beam_1_1_analog_base_1_1_wow_flutter_generator_a3261a050317fd9ff64b63f0070f3c739}\index{Beam::AnalogBase::WowFlutterGenerator@{Beam::AnalogBase::WowFlutterGenerator}!m\_flutterDepth@{m\_flutterDepth}}
\index{m\_flutterDepth@{m\_flutterDepth}!Beam::AnalogBase::WowFlutterGenerator@{Beam::AnalogBase::WowFlutterGenerator}}
\doxysubsubsection{\texorpdfstring{m\_flutterDepth}{m\_flutterDepth}}
{\footnotesize\ttfamily \label{class_beam_1_1_analog_base_1_1_wow_flutter_generator_a3261a050317fd9ff64b63f0070f3c739} 
float Beam::\+\+Analog\+Base::\+\+Wow\+Flutter\+Generator::\+m\+\_\+flutter\+Depth = 0.\+0f\hspace{0.3cm}{\ttfamily [private]}}

\Hypertarget{class_beam_1_1_analog_base_1_1_wow_flutter_generator_ab99ec737da74f383a2c1d3b65f24d85b}\index{Beam::AnalogBase::WowFlutterGenerator@{Beam::AnalogBase::WowFlutterGenerator}!m\_flutterLP@{m\_flutterLP}}
\index{m\_flutterLP@{m\_flutterLP}!Beam::AnalogBase::WowFlutterGenerator@{Beam::AnalogBase::WowFlutterGenerator}}
\doxysubsubsection{\texorpdfstring{m\_flutterLP}{m\_flutterLP}}
{\footnotesize\ttfamily \label{class_beam_1_1_analog_base_1_1_wow_flutter_generator_ab99ec737da74f383a2c1d3b65f24d85b} 
\doxymbox{\hyperlink{class_beam_1_1_analog_base_1_1_one_pole_filter}{One\+Pole\+Filter}} Beam::\+\+Analog\+Base::\+\+Wow\+Flutter\+Generator::\+m\+\_\+flutter\+LP\hspace{0.3cm}{\ttfamily [private]}}

\Hypertarget{class_beam_1_1_analog_base_1_1_wow_flutter_generator_ad7381452afa60d0c5f2707fa7580c707}\index{Beam::AnalogBase::WowFlutterGenerator@{Beam::AnalogBase::WowFlutterGenerator}!m\_rng@{m\_rng}}
\index{m\_rng@{m\_rng}!Beam::AnalogBase::WowFlutterGenerator@{Beam::AnalogBase::WowFlutterGenerator}}
\doxysubsubsection{\texorpdfstring{m\_rng}{m\_rng}}
{\footnotesize\ttfamily \label{class_beam_1_1_analog_base_1_1_wow_flutter_generator_ad7381452afa60d0c5f2707fa7580c707} 
std::\+mt19937 Beam::\+\+Analog\+Base::\+\+Wow\+Flutter\+Generator::\+m\+\_\+rng\hspace{0.3cm}{\ttfamily [private]}}

\Hypertarget{class_beam_1_1_analog_base_1_1_wow_flutter_generator_a70eea959271d10f748eba7cba326fb75}\index{Beam::AnalogBase::WowFlutterGenerator@{Beam::AnalogBase::WowFlutterGenerator}!m\_sampleRate@{m\_sampleRate}}
\index{m\_sampleRate@{m\_sampleRate}!Beam::AnalogBase::WowFlutterGenerator@{Beam::AnalogBase::WowFlutterGenerator}}
\doxysubsubsection{\texorpdfstring{m\_sampleRate}{m\_sampleRate}}
{\footnotesize\ttfamily \label{class_beam_1_1_analog_base_1_1_wow_flutter_generator_a70eea959271d10f748eba7cba326fb75} 
float Beam::\+\+Analog\+Base::\+\+Wow\+Flutter\+Generator::\+m\+\_\+sample\+Rate\hspace{0.3cm}{\ttfamily [private]}}

\Hypertarget{class_beam_1_1_analog_base_1_1_wow_flutter_generator_a27e7ed8f25a051a95d6bbcf5e183706e}\index{Beam::AnalogBase::WowFlutterGenerator@{Beam::AnalogBase::WowFlutterGenerator}!m\_wowDepth@{m\_wowDepth}}
\index{m\_wowDepth@{m\_wowDepth}!Beam::AnalogBase::WowFlutterGenerator@{Beam::AnalogBase::WowFlutterGenerator}}
\doxysubsubsection{\texorpdfstring{m\_wowDepth}{m\_wowDepth}}
{\footnotesize\ttfamily \label{class_beam_1_1_analog_base_1_1_wow_flutter_generator_a27e7ed8f25a051a95d6bbcf5e183706e} 
float Beam::\+\+Analog\+Base::\+\+Wow\+Flutter\+Generator::\+m\+\_\+wow\+Depth = 0.\+0f\hspace{0.3cm}{\ttfamily [private]}}

\Hypertarget{class_beam_1_1_analog_base_1_1_wow_flutter_generator_a8a795c38bb2b81427fe4d7b29ffa054e}\index{Beam::AnalogBase::WowFlutterGenerator@{Beam::AnalogBase::WowFlutterGenerator}!m\_wowPhase@{m\_wowPhase}}
\index{m\_wowPhase@{m\_wowPhase}!Beam::AnalogBase::WowFlutterGenerator@{Beam::AnalogBase::WowFlutterGenerator}}
\doxysubsubsection{\texorpdfstring{m\_wowPhase}{m\_wowPhase}}
{\footnotesize\ttfamily \label{class_beam_1_1_analog_base_1_1_wow_flutter_generator_a8a795c38bb2b81427fe4d7b29ffa054e} 
float Beam::\+\+Analog\+Base::\+\+Wow\+Flutter\+Generator::\+m\+\_\+wow\+Phase = 0.\+0f\hspace{0.3cm}{\ttfamily [private]}}



The documentation for this class was generated from the following file:\+\begin{DoxyCompactItemize}
\item 
src/\+engine/\+\doxymbox{\hyperlink{analog__base_8hpp}{analog\+\_\+base.\+hpp}}\end{DoxyCompactItemize}


\chapter{File Documentation}
% I will selectively include file docs or all of them. 
% For a clean report, I'll assume the user might not want EVERY file detail, but 
% "Full Report" usually means everything. I will include the main ones.
\doxysection{src/\+core/\+beam\+\_\+host.hpp File Reference}
\hypertarget{beam__host_8hpp}{}\label{beam__host_8hpp}\index{src/core/beam\_host.hpp@{src/core/beam\_host.hpp}}
{\ttfamily \+\#include "{}../\+dsp/\+audio\+\_\+engine.\+hpp"{}}\newline
{\ttfamily \+\#include "{}../\+graphics/\+quad\+\_\+batcher.\+hpp"{}}\newline
{\ttfamily \+\#include "{}../\+graphics/\+shader.\+hpp"{}}\newline
{\ttfamily \+\#include "{}../\+ui/\+input\+\_\+handler.\+hpp"{}}\newline
{\ttfamily \+\#include "{}flux\+\_\+project.\+hpp"{}}\newline
\doxysubsubsection*{Classes}
\begin{DoxyCompactItemize}
\item 
class \doxymbox{\hyperlink{class_beam_1_1_beam_host}{Beam::\+\+Beam\+Host}}
\end{DoxyCompactItemize}
\doxysubsubsection*{Namespaces}
\begin{DoxyCompactItemize}
\item 
namespace \doxymbox{\hyperlink{namespace_beam}{Beam}}
\end{DoxyCompactItemize}
\doxysubsubsection*{Enumerations}
\begin{DoxyCompactItemize}
\item 
enum class \doxymbox{\hyperlink{namespace_beam_a857abf9a9358b461e5e0b3a25b0c3d88}{Beam::\+\+DAWMode}} \{ \doxymbox{\hyperlink{namespace_beam_a857abf9a9358b461e5e0b3a25b0c3d88a79cf326cd40869983c7c685989e6dde6}{Beam::\+\+Splicing}}
, \doxymbox{\hyperlink{namespace_beam_a857abf9a9358b461e5e0b3a25b0c3d88a1c30b35b12895df175ccd44dbb6f5ace}{Beam::\+\+Flux}}
 \}
\end{DoxyCompactItemize}

\doxysection{src/\+engine/\+flux\+\_\+graph.hpp File Reference}
\hypertarget{flux__graph_8hpp}{}\label{flux__graph_8hpp}\index{src/engine/flux\_graph.hpp@{src/engine/flux\_graph.hpp}}
{\ttfamily \+\#include "{}flux\+\_\+node.\+hpp"{}}\newline
{\ttfamily \+\#include "{}render\+\_\+plan.\+hpp"{}}\newline
{\ttfamily \+\#include $<$map$>$}\newline
{\ttfamily \+\#include $<$set$>$}\newline
{\ttfamily \+\#include $<$algorithm$>$}\newline
{\ttfamily \+\#include $<$iostream$>$}\newline
{\ttfamily \+\#include $<$mutex$>$}\newline
\doxysubsubsection*{Classes}
\begin{DoxyCompactItemize}
\item 
struct \doxymbox{\hyperlink{struct_beam_1_1_flux_connection}{Beam::\+\+Flux\+Connection}}
\item 
class \doxymbox{\hyperlink{class_beam_1_1_flux_graph}{Beam::\+\+Flux\+Graph}}
\end{DoxyCompactItemize}
\doxysubsubsection*{Namespaces}
\begin{DoxyCompactItemize}
\item 
namespace \doxymbox{\hyperlink{namespace_beam}{Beam}}
\end{DoxyCompactItemize}

\doxysection{src/\+graphics/\+quad\+\_\+batcher.hpp File Reference}
\hypertarget{quad__batcher_8hpp}{}\label{quad__batcher_8hpp}\index{src/graphics/quad\_batcher.hpp@{src/graphics/quad\_batcher.hpp}}
{\ttfamily \+\#include $<$vector$>$}\newline
{\ttfamily \+\#include $<$string$>$}\newline
{\ttfamily \+\#include "{}../\+../\+third\+\_\+party/\+glad.\+h"{}}\newline
\doxysubsubsection*{Classes}
\begin{DoxyCompactItemize}
\item 
struct \doxymbox{\hyperlink{struct_beam_1_1_vertex}{Beam::\+\+Vertex}}
\item 
class \doxymbox{\hyperlink{class_beam_1_1_quad_batcher}{Beam::\+\+Quad\+Batcher}}
\end{DoxyCompactItemize}
\doxysubsubsection*{Namespaces}
\begin{DoxyCompactItemize}
\item 
namespace \doxymbox{\hyperlink{namespace_beam}{Beam}}
\end{DoxyCompactItemize}

% (Adding the rest is tedious without a wildcard, I'll stick to the classes mostly as they contain the meat)

\printindex

\end{document}
